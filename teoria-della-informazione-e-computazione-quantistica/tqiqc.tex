\documentclass[a4paper, 12pt]{book}

% General Settings
\usepackage[paper = a4paper, margin = 1in]{geometry}
\usepackage[italian]{babel}
\usepackage[utf8]{inputenc}
\usepackage[T1]{fontenc}
\usepackage{comment}
\usepackage{hyperref}
\usepackage{graphicx}
\usepackage[small]{caption}
\usepackage{subfig}
\usepackage[usenames,dvipsnames]{xcolor}
\usepackage{graphicx,color,listings}
\usepackage{float}
\usepackage{booktabs}
\usepackage{hologo}
\usepackage{colortbl}

% Math and physic packages
\usepackage{physics}
\usepackage{amsmath,amssymb,amsthm}
\usepackage{mathrsfs}
\usepackage{mathtools}
\numberwithin{equation}{section}
\numberwithin{figure}{chapter}

% Quantum and classical circuits packages
\usepackage{circuitikz}
\usepackage{qcircuit}


% New or renewed commands
\newcommand{\lecture}[2]{{\scshape{Lezione #1 - #2}} \par}
\DeclareMathOperator{\eig}{eig}

\makeatletter
\providecommand{\leftsquigarrow}{%
  \mathrel{\mathpalette\reflect@squig\relax}%
}
\newcommand{\reflect@squig}[2]{%
  \reflectbox{$\m@th#1\rightsquigarrow$}%
}
\makeatother

% Josephson Junction
\usepackage{circuitikz}

\makeatletter
\pgfcircdeclarebipolescaled{instruments}
{
    \anchor{text}{\pgfextracty{\pgf@circ@res@up}{\northeast}
        \pgfpoint{-.7\wd\pgfnodeparttextbox}{
            \dimexpr.5\dp\pgfnodeparttextbox+.5\ht\pgfnodeparttextbox+\pgf@circ@res@up\relax
        }
    }
}
{\ctikzvalof{bipoles/oscope/height}}
{josephson}
{\ctikzvalof{bipoles/oscope/height}}
{\ctikzvalof{bipoles/oscope/width}}
{
    \pgf@circ@setlinewidth{bipoles}{\pgfstartlinewidth}
    \pgfextracty{\pgf@circ@res@up}{\northeast}
    \pgfextractx{\pgf@circ@res@right}{\northeast}
    \pgfextractx{\pgf@circ@res@left}{\southwest}
    \pgfextracty{\pgf@circ@res@down}{\southwest}
    \pgfmathsetlength{\pgf@circ@res@step}{0.25*\pgf@circ@res@up}
    \pgfscope
        \pgfpathrectanglecorners{\pgfpoint{\pgf@circ@res@left}{\pgf@circ@res@down}}{\pgfpoint{\pgf@circ@res@right}{\pgf@circ@res@up}}
        \pgf@circ@draworfill
    \endpgfscope
    \pgfscope
      \pgfpathmoveto{\pgfpoint{\pgf@circ@res@left}{\pgf@circ@res@up}}%
      \pgfpathlineto{\pgfpoint{\pgf@circ@res@right}{\pgf@circ@res@down}}%
      \pgfpathmoveto{\pgfpoint{\pgf@circ@res@right}{\pgf@circ@res@up}}%
      \pgfpathlineto{\pgfpoint{\pgf@circ@res@left}{\pgf@circ@res@down}}%
      \pgfusepath{draw}
    \endpgfscope
}
\def\pgf@circ@josephson@path#1{\pgf@circ@bipole@path{josephson}{#1}}
\tikzset{josephson/.style = {\circuitikzbasekey, /tikz/to path=\pgf@circ@josephson@path, l=#1}}

\newtheorem{definizione}{Definizione}[chapter]
\newtheorem{teorema}{Teorema}[chapter]
\newtheorem{esempio}{Esempio}[chapter]
\newtheorem{lemma}{Lemma}[chapter]
\renewcommand{\thefootnote}{\roman{footnote}}

\hypersetup{
    colorlinks=true,
    linkcolor=black,
    urlcolor=blue,
}
\urlstyle{same}

% Documents
\begin{document}
    \begin{titlepage}
        \begin{center}
            \vspace*{3cm}
            {\scshape\LARGE Università degli Studi di Milano - Bicocca \par}
            \vspace{1.0cm}
            \line(1,0){400} \\
            {\huge\bfseries Teoria della Informazione e \\ Computazione Quantistica \par}
            \line(1,0){400} \\
 	        \vspace{1.0cm}
            {\Large Raccolta di appunti, dispense e libri \par}
            \vspace{1.0cm}
            {Anno accademico 2021/2022 \par}
            \vspace{0.5cm}
            {\bfseries Marco Gobbo e Gabriele Morandi \par}
            \vspace{0.5cm}
            {\url{https://github.com/marcogobbo/tecnologie-quantistiche} \par}
            \vspace*{\fill}
            {\large \today \par}
        \end{center}
    \end{titlepage}
    \tableofcontents
    %%%%%%%%%%%%%
% LECTURE 1 %
%%%%%%%%%%%%%

\chapter{Meccanica quantistica}

\lecture{1}{07/10/2021}
\section{Stati e qubit}
Prima di addentrarci nello studio delle tecnologie quantistiche, risulta opportuno fare alcuni richiami di meccanica quantistica implementando alcuni concetti che ci saranno poi utili in futuro. In particolare iniziamo velocemente ricordando il primo postulato della meccanica quantistica
\begin{itemize}
    \item \textbf{I Postulato} (\textbf{Stato}): Che cos'è uno stato? Utilizziamo la notazione di Dirac per rappresentare un vettore $\ket{\psi}$ di uno spazio di Hilbert $\mathcal{H}$ (molto spesso uno spazio vettoriale finito dimensionale) e diremo che $\ket{\psi} \in \mathcal{H}$. Uno stato è un \textbf{raggio} tale che $\norm{\ket{\psi}} = 1$ (per la conservazione della probabilità) e $\ket{\psi} \cong e^{i \alpha} \ket{\psi}$ \footnote{La notazione $\cong$ significa "equivalente a".} con $\alpha \in \mathbb{R}$. Dato che la fase globale è irrilevante, quando due stati differiscono per una fase hanno il medesimo effetto fisico. 
\end{itemize}
Procediamo ora con il definire cosa sia un qubit
\begin{definizione}[\textbf{Qubit}]
    Un qubit è un qualsiasi sistema a due livelli. Ogni sistema quantomeccanico può essere un qubit, ad esempio si può creare utilizzando le due differenti polarizzazioni del fotone, utilizzando l’allineamento dello spin di un nucleo immerso in un campo magnetico uniforme, utilizzando la tecnica della trappola ionica, sistemi superconduttivi, \dots
\end{definizione}
\noindent Davide di Vincenzo, nel 2000, ha indicato cinque criteri necessari per la scelta di un sistema fisico adatto per la computazione quantistica:
\begin{enumerate}
    \item Un sistema fisico scalabile con qubit ben caratterizzati;
    \item La capacità di inizializzare lo stato dei qubit a un semplice stato fiduciale;
    \item Tempi di decoerenza lunghi e rilevanti;
    \item Un insieme "universale" di porte quantistiche;
    \item Una capacità di misurazione specifica per qubit.
\end{enumerate}
La meccanica quantistica si occupa di descrivere il comportamento del nostro sistema a due livelli mediante una hamiltoniana. Per fare ciò lavoriamo in spazi di Hilbert bidimensionali $\mathcal{H}=\mathbb{C}^2$, quindi le hamiltoniane di questi sistemi sono degli operatori definiti su $\mathbb{C}^2 \rightarrow \mathbb{C}^2$. Gli stati in cui si trova il nostro sistema sono descritti da funzioni d'onda generiche $\psi \in \mathbb{C}^2$, in particolar modo possono essere decomposte sulla base computazionale $\{\ket 0, \ket 1\}$. Avremo quindi che 
\begin{equation*}
    \begin{array}{l}
        \hat H \ket 0 = E_0 \ket 0 \\
        \hat H \ket 1 = E_1 \ket 1 \, ,
    \end{array}
\end{equation*}
dove
\begin{equation*}
    \begin{array}{l}
        \ip{0}{0}=\ip{1}{1}=1 \\
        \ip{0}{1}=\ip{1}{0}=0 \, .
    \end{array}
\end{equation*}
Per cui ogni stato generico $\ket \psi$ può essere scritto come combinazione lineare di $\{\ket 0, \ket 1\}$
\begin{equation*}
    \ket \psi = a \ket 0 + b \ket 1 \, ,
\end{equation*}
con $a,b \in \mathbb{C}$ e soddisfacenti la condizione di conservazione di probabilità
\begin{equation*}
    \abs{a}^2+\abs{b}^2=1 \, .
\end{equation*}
Osserviamo che, per come è definito, $\ket \psi$ è uno \textbf{stato puro}, ci dà la massima conoscenza che possiamo ottenere da questo sistema. Infatti abbiamo una probabilità pari a $\abs{a}^2$ di ottenere $\ket 0$ e una probabilità pari a $\abs{b}^2$ di ottenere $\ket 1$. Dobbiamo misurare un numero infinito di volte per poter ottenere queste distribuzioni di probabilità, tuttavia non possiamo eseguire una misura successiva per estrarre ulteriori informazioni sul nostro stato $\ket \psi$ poiché quest'ultimo sarà collassato in $\ket 0$ oppure $\ket 1$. Per determinare univocamente $\alpha$ e $\beta$ si necessiterebbe un'infinità di esperimenti su un'infinità di stati tutti preparati nel medesimo stato $\ket \psi$. La massima conoscenza che possiamo estrarre non è molta, questo fatto è stato oggetto di discussione per molti anni. In particolar modo ci si è chiesti se la teoria meccanica quantistica fosse una teoria completa o meno\footnote{Einstein, A., Podolsky, B., \& Rosen, N. (1935). Can Quantum-Mechanical Description of Physical Reality Be Considered Complete?. Phys. Rev., 47, 777–780.}.\\
Come abbiamo già accennato, $a$ e $b$ sono coefficienti complessi, attraverso la notazione esponenziale possiamo scriverli come
\begin{equation*}
    a=\abs{a}e^{i\theta_0} \qquad b=\abs{b}e^{i\theta_1}\, ,
\end{equation*}
in questo modo
\begin{equation*}
    \begin{aligned}
        \ket \psi &= \abs{a}e^{i\theta_0}\ket 0 + \abs{b}e^{i\theta_1}\ket 1 \\
                  &= \underbrace{e^{i\theta_0}}_{\mathclap{\text{Fase globale}}}\Big(\abs{a}\ket 0 + \abs{b}\underbrace{e^{i\left(\theta_1-\theta_0\right)}}_{\mathclap{\text{Fase relativa}}}\ket 1\Big) \, .
    \end{aligned}
\end{equation*}
Quando misuriamo uno stato, la \textit{fase globale} risulta essere irrilevante, ciò che conta è la \textit{fase relativa} perché può dar luogo a fenomeni come l'interferenza.
\begin{esempio}[Fase relativa]
    Consideriamo gli stati $\ket 0 e \ket 1$, per scrivere i seguenti stati
    \begin{equation*}
        \ket{\psi_1}=\frac{\ket 0 + \ket 1}{\sqrt 2}\, \qquad \ket{\psi_2}=\frac{\ket 0 - \ket 1}{\sqrt 2} 
    \end{equation*}
    In questo caso il segno meno proviene dalla fase relativa. $\ket{\psi_1}$ e $\ket{\psi_2}$ forniscono lo stesso risultato per una misura di energia (lo si può verificare calcolando $\mel{\psi_i}{\hat H}{\psi_i}$), tuttavia riusciamo a distinguerli se facciamo una misura diversa. Ad esempio possiamo considerare la matrice di Pauli
    \begin{equation*}
        \sigma_x = \begin{pmatrix}
                    0 & 1 \\
                    1 & 0
                   \end{pmatrix}\, ,
    \end{equation*}
    $\ket{\psi_1}$ e $\ket{\psi_2}$ sono autostati di $\sigma_x$ con autovalori, rispettivamente, $1$ e $-1$.
\end{esempio}
\noindent Uno dei problemi principali nell'aver a che fare con sistemi quantistici è trovare l'evoluto temporale di un certo stato, perché abbiamo delle hamiltoniane che descrivono ad esempio il rumore degli strumenti, la temperatura dell'ambiente, \dots L'equazione di Schrödinger si comporta bene nel descrivere l'evoluzione di \textbf{sistemi chiusi}, ma un qubit è, in generale, un \textbf{sistema aperto} che si lega a sistemi esterni e quindi la conoscenza sul suo stato tende a diminuire, finché non perdiamo completamente l'informazione che possedeva all'inizio. Questo fatto è legato al \textbf{tempo di coerenza}. Ci sono vari modi per tenere conto di queste interazioni così da poter descrivere al meglio il nostro sistema a due livelli.\\
Supponiamo di avere un sistema chiuso che evolve secondo l'equazione di Schrödinger
\begin{equation*}
    \hat H \ket{\psi(t)}=i\hbar \partialderivative{t}\ket{\psi(t)}\, ,
\end{equation*}
dove $\ket{\psi(t)}=\hat U(t)\ket{\psi(0)}$. $\hat U$ in questo caso è un operatore unitario che può essere espresso, se l'hamiltoniana è costante nel tempo, come
\begin{equation*}
    \hat U(t)=e^{-\frac{i}{\hbar}\hat H t}\, .
\end{equation*}
Pertanto, considerando gli autostati dell'hamiltoniana 
\begin{equation*}
    \hat H \ket{i}=E_i\ket{i}\, ,
\end{equation*}
e riscrivendo il nostro stato iniziale in termini di autostati dell'hamiltoniana
\begin{equation*}
    \ket{\psi(0)}=\sum_i a_i\ket i\, ,
\end{equation*}
possiamo valutare il nostro stato al tempo generico $t$ come
\begin{equation*}
    \ket{\psi(t)}=\sum_i a_ie^{-\frac i \hbar \hat H t}\ket i=\sum_i a_i e^{-\frac i \hbar E_i t}\ket i \qquad \text{dove} \quad a_i(t)=a_i(0)e^{-\frac i \hbar E_i t}\, .
\end{equation*}
Da questo caso generale possiamo trattare il nostro sistema a due livelli, in questo caso l'hamiltoniana sarà
\begin{equation*}
    \hat H = \begin{pmatrix}
        E_0 & 0 \\
        0 & E_1
       \end{pmatrix}\, ,
\end{equation*}
applicando l'equazione di Schrödinger sui coefficienti
\begin{equation*}
    i\hbar\derivative{a_0(t)}{t}=E_0a_0(t)\, ,
\end{equation*}
\begin{equation*}
    i\hbar\derivative{a_1(t)}{t}=E_1a_1(t)\, ,
\end{equation*}
troviamo che il nostro stato finale al tempo generico $t$ sarà
\begin{equation*}
    \ket{\psi(t)}=\underbrace{e^{-\frac{i}{\hbar}E_0t}}_{\text{Fase globale}}\Big(a_0(0)\ket 0 +\underbrace{e^{-\frac{i}{\hbar}(E_1-E_0)t}a_1(0)}_{\text{Fase relativa}}\ket 1\Big)\, .
\end{equation*}
Ancora una volta, la fase globale non produce alcun effetto, ciò che notiamo è che l'evoluzione temporale cambia la fase relativa tra gli stati $\ket 0$ e $\ket 1$. Questo spiega perché se abbiamo una interazione che disturba il nostro sistema possiamo avere un cambio nella fase relativa, questo è dato dal fatto che abbiamo una variazione in termini energetici. Questo disturbo è generato da tutto ciò che è esterno al sistema a due livelli. Se perdiamo il controllo su questa fase, perdiamo tutta l'informazione che abbiamo su $\ket{\psi(t)}$, e se questo accade, non abbiamo più uno stato puro. Per questo motivo necessitiamo qualcosa che vada oltre al concetto di funzione d'onda generica $\psi$.

\section{Matrice densità}
Vogliamo realizzare uno stato puro $\ket \psi$ che sia una combinazione pura di stati $\ket 0$ e $\ket 1$:
\begin{equation*}
    \ket \psi = a\ket 0 + b \ket 1\, ,
\end{equation*}
nella realtà quando cerchiamo di realizzare questo stato, abbiamo un'indeterminazione classica rappresentata da una distribuzione di probabilità di ottenere lo stato esatto oppure uno stato simile. Supponiamo di avere un insieme di stati che indichiamo con $\{p_i, \ket{\psi_i}\}$, dove $p_i$ è la probabilità classica di ottenere un generico stato. Questi stati $\ket{\psi_i}$ sono tutti stati puri, ma non sappiamo quale sia quello giusto e la sua conoscenza è persa. Tutte queste informazioni sono contenute nella \textbf{matrice densità} che rappresenta una distribuzione classica di probabilità.\\
Dal punto di vista della teoria della meccanica quantistica, esiste un altro modo per introdurre la teoria anziché sfruttare gli stati $\psi$. Quello che si fa è sfruttare la matrice densità che è un operatore che agisce nel seguente modo
\begin{equation*}
    \hat \rho \ket{\psi_i}=p_i\ket{\psi_i}\, ,
\end{equation*}
dove $p_i$ rappresenta la probabilità di ottenere lo stato i-esimo. La matrice densità è ora una miscela di stati puri
\begin{equation*}
    \hat \rho = \sum_i p_i \op{\psi_i}{\psi_i}
\end{equation*}
e descrive la mancanza di conoscenza sui sistemi quantistici che avevamo precedentemente. Se utilizzassimo lo stesso operatore $\hat U$ per descrivere l'evoluto temporale di $\ket{\psi_i} \overset{t}{\longrightarrow} \hat U\ket{\psi_i}$, come possiamo applicarlo a $\hat \rho$?
\begin{equation*}
    \hat \rho = \sum_i p_i \op{\psi_i}{\psi_i} \longrightarrow \sum_i p_i \hat U\op{\psi_i}{\psi_i}\hat U^\dagger \,
\end{equation*}
\begin{equation*}
    \hat U \hat \rho \hat U^\dagger = \hat \rho ' \, .
\end{equation*}
Vediamo se le distribuzioni di probabilità classiche, nel caso di stati ortonormali, vengono conservate:
\begin{proof}\mbox{}\\*
    \noindent A $t=0$ :
    \begin{equation*}
          \hat \rho \ket{\psi_i (0)} = p_i \ket{\psi_i (0)} \\
    \end{equation*}
    A $t>0$ :
    \begin{equation*}
        \begin{aligned}
            \hat \rho' \ket{\psi_i (t)} &= \hat U \hat \rho \hat U^\dagger \hat U \ket{\psi_i (0)} \\      
                                        &=\hat U \hat \rho \ket{\psi_i (0)} \\
                                        &=\hat U p_i \ket{\psi_i (0)} \\
                                        &=p_i U \ket{\psi_i (0)} \\
                                        &=p_i\ket{\psi_i (t)}
        \end{aligned}
    \end{equation*}

    \noindent La probabilità $p_i$ non è cambiata nel tempo, ma lo stato sì perché ora è $\ket{\psi_i (t)}$ che non è uguale a $\ket{\psi_i (0)}$.
\end{proof}
    %%%%%%%%%%%%%
% LECTURE 2 %
%%%%%%%%%%%%%
\newpage
\noindent \lecture{2}{08/10/2021}

\section{Osservabili}\label{sec:osservabili}

\begin{itemize}
    \item \textbf{II Postulato} (\textbf{Osservabili}): Che cosa si può misurare in QM? Vengono misurate le \textbf{osservabili}, ossia \textbf{operatori autoaggiunti} (o \textbf{hermitiani}) $\hat{A}$ tali che
    \begin{equation*}
        \hat{A}: \mathcal{H}\rightarrow \mathcal{H} \, \text{ con } \, \hat{A}^\dagger = \hat{A} \, ,
    \end{equation*}
    dove più precisamente $\hat{A}^\dagger \equiv (\hat{A}^t)^\ast$. Dal punto di vista degli elementi di matrice, calcolare l'aggiunto di $A_{ij}$ significa $A^\dagger_{ij} = A^\ast_{ji}$. Dunque le matrici autoaggiunte (hermitiane) sono tali che $A^\dagger \equiv (A^t)^\ast = A$.  
\end{itemize}

\noindent In base a ciò che abbiamo visto sulla notazione braket  ($\bra{\phi} = \ket{\phi}^\dagger$) abbiamo necessariamente che

\begin{equation*}
    \ket{\psi} = B \ket{\phi} \, , \quad \Rightarrow \quad  \bra{\psi} = \bra{\phi} B^\dagger \, .
\end{equation*}

\noindent Focalizzando la nostra attenzione sugli operatori autoaggiunti, richiamiamo un importante teorema di algebra lineare:
\begin{teorema}[\textbf{Teorema Spettrale}]
    Sia $\hat{A}$ un operatore autoaggiunto su uno spazio di Hilbert $\mathcal{H}$. Allora esiste una base ortonormale di $\mathcal{H}$ composta da autovettori di $\hat{A}$, ossia $\exists$ $\lbrace \ket{n} \rbrace \in \mathcal{H}$ tale che $\hat{A} \ket{n} = a_n \ket{n}$ dove gli autovalori $a_n \in \mathbb{R}$.
\end{teorema}

\noindent Si noti dal teorema che $\braket{n}{m} = \delta_{nm}$ dove $n,m = 1, \ldots, N$ con $N \equiv \dim \mathcal{H}$. Trattandosi di una base, qualsiasi vettore dello spazio di Hilbert può essere scritto come combinazione lineare di tali vettori: 

\begin{equation*}
    \ket{\psi} = \sum_{n=1}^N \alpha_n \ket n \, , \, \text{ dove } \, \alpha_n \equiv \braket{n}{\psi} \in \mathbb{C} \, .
\end{equation*}

\noindent Ritornando al nostro caso del sistema a due livelli, lo spazio di Hilbert in esame è $\mathbb{C}^2$, dove consideriamo la \textbf{base canonica} (o \textbf{base computazionale}) data dagli stati $\ket 0$ e $\ket 1$ (si vedano le definizioni in \eqref{computational_basis}). In questo spazio vettoriale gli operatori sono rappresentati da matrici $2 \times 2$. La più generale matrice $2 \times 2$ hermitiana contenente 4 parametri reali è

\begin{equation*}
    A = 
    \begin{pmatrix}
        a+b & c-id \\ 
        c+id & a-b
    \end{pmatrix} \, ,
\end{equation*}

\noindent dove $a, b, c, d \in \mathbb{R}$. Si noti che sulla diagonale le entrate sono puramente reali. Così come abbiamo decomposto uno stato generico $\ket{\psi}$ mediante combinazione lineare di autovettori $\ket{n}$, possiamo decomporre il generico operatore hermitiano di $\mathbb{C}^2$ come 

\begin{equation}\label{generical_matrix_C2}
    A = a \mathbb{I} + c \sigma_1 + d \sigma_2 + b \sigma_3 \, ,
\end{equation}

\noindent dove $\mathbb{I}$ è la matrice \textbf{identità} e $\sigma_1, \sigma_2, \sigma_3$ sono le \textbf{matrici Pauli}:

\begin{equation}\label{Pauli_matrices}
    \sigma_1=
    \begin{pmatrix}
        0 & 1 \\
        1 & 0
    \end{pmatrix} \, , \ \ \ \ \
    \sigma_2=
    \begin{pmatrix}
        0 & -i \\
        i & 0
    \end{pmatrix} \, , \ \ \ \ \
    \sigma_3=
    \begin{pmatrix}
        1 & 0 \\
        0 & -1
    \end{pmatrix} \, . \ \ \ \ \
\end{equation}

\noindent Si ricordi che le matrici di Pauli sono i generatori del momento angolare in QM e sono infatti utilizzate per descrivere l'operatore di spin $\hat{\vec{S}} = \frac{\hbar}{2} \hat{\vec{\sigma}}$. I relativi autovalori e autovettori sono mostrati nella Tabella \ref{tab:Pauli_eig}.

\begin{table}[!ht]
	\centering
    \begin{tabular}{ccc}
        \toprule
        \textbf{Matrice di Pauli} & \textbf{Autovettori} & \textbf{Autovalori}  \\
        \midrule
        $\sigma_1$ & $\ket + = \frac{1}{\sqrt 2} \begin{pmatrix} 1 \\ 1 \end{pmatrix}, \qquad \ket - = \frac{1}{\sqrt 2} \begin{pmatrix} 1 \\ -1 \end{pmatrix} $ & $\lbrace 1, -1 \rbrace$ \\
        $\sigma_2$ & $\ket i = \frac{1}{\sqrt 2} \begin{pmatrix} 1 \\ i \end{pmatrix}, \qquad \ket{-i} = \frac{1}{\sqrt 2} \begin{pmatrix} 1 \\ -i \end{pmatrix} $ & $\lbrace 1, -1 \rbrace$ \\
        $\sigma_3$ & $\ket 0 = \begin{pmatrix} 1 \\ 0 \end{pmatrix}, \qquad \ket 1 = \begin{pmatrix} 0 \\ 1 \end{pmatrix} $ & $\lbrace 1,-1 \rbrace$ \\
        \bottomrule
    \end{tabular}\\
    \caption{Autovettori e autovalori delle matrici di Pauli.}
    \label{tab:Pauli_eig}
\end{table}

\noindent Dato che in futuro ci tornerà utile, osserviamo che gli autovettori di $\sigma_1$ e $\sigma_2$ possono essere espressi mediante base computazionale come

\begin{equation}\label{basi_di_sigma_12}
    \ket + = \frac{\ket 0 + \ket 1}{\sqrt 2} \, , \quad \ket - = \frac{\ket 0 - \ket 1}{\sqrt 2} \, , \quad \ket i = \frac{\ket 0 + i \ket 1}{\sqrt 2} \, , \quad \ket{-i} = \frac{\ket 0 - i \ket 1}{\sqrt 2} \, ,
\end{equation}

\noindent Utilizzando la rappresentazione dei qubit tramite sfera di Bloch, questi autovettori sono mostrati in Figura \ref{fig:BlochSphere2}. 

\begin{figure}[!ht]
    \centering
    \includegraphics[scale=0.6]{images/bloch-hdr-440.png}
    \caption{Rappresentazione degli autovettori delle matrici di Pauli sulla sfera di Bloch. Il punto indicato dalla freccia rossa indica un generico qubit.}
    \label{fig:BlochSphere2}
\end{figure}

\noindent Come detto in precedenza, le 3 matrici di Pauli parametrizzano lo spin e i 3 assi della sfera di Bloch possono essere associati allo spin. Considerando lo stato generico $\ket{\psi}$ della \eqref{generic qubit}, possiamo definire lo spin lungo una direzione generica $\vec{\sigma} \cdot \vec{n}$ dove $\vec n = (\cos\phi\sin\theta, \sin\phi\sin\theta, \cos\theta)$:

\begin{equation*}
    \vec{\sigma} \cdot \vec n = \cos\phi\sin\theta \, \sigma_1 + \sin \phi \sin \theta \, \sigma_2 + \cos \theta \, \sigma_3 \, ;
\end{equation*}
così facendo è un semplice esercizio di QM dimostrare che $\ket{\psi}$ è autostato di $\vec{\sigma} \cdot \vec n$ con autovalore 1, ossia $\vec{\sigma} \cdot \vec n \ket{\psi} = \ket{\psi}$. Questo significa che dato uno stato sulla sfera di Bloch, allora esso è anche autostato di spin nella direzione individuata da tale qubit: infatti l'idea fisica alla base della sfera di Bloch è che la direzione arbitraria scelta non è altro che la direzione della quantizzazione dello spin. 

\section{Misurazioni}

\begin{itemize}
    \item \textbf{III Postulato} (\textbf{Regola di Born}):
    \begin{enumerate}
        \item \textbf{Misurazione}: sia $\hat{A}$ un osservabile con autostati $\ket{n}$, ossia $\hat{A} \ket{n} = a_n \ket{n}$. Prendiamo per semplicità $a_n \neq a_m \ \forall \, n \neq m$ (osservabile con autovalori distinti). Consideriamo uno stato generico espanso sugli autostati precedenti: $\ket \psi = \sum_n \alpha_n \ket n$. Allora una misura dell'osservabile $\hat{A}$ produce il valore $a_n$ con probabilità data da $\abs{\alpha_n}^2$ (assumendo lo stato correttamente normalizzato).
        
        \item \textbf{Collasso dello stato}: cosa succede allo stato del sistema dopo la misurazione? Istantaneamente lo stato $\ket \psi$ collassa sull'autostato associato all'autovalore risultante dalla misura. Ad esempio se misurando otteniamo $a_n$ allora $\ket{\psi} \rightarrow \ket{n}$. Effettuando delle misure successive sullo stato si ottiene sempre $\ket{n}$ con probabilità esattamente uguale a 1.  
    \end{enumerate}
\end{itemize}

\begin{esempio}
    Consideriamo per esempio il generico qubit in \eqref{generic qubit} e immaginiamo di voler effettuare delle misurazioni in differenti basi. Supponiamo di voler misurare lo spin lungo $z$ (base $\lbrace \ket{0}, \ket{1} \rbrace$ di $\sigma_3$) e lungo $x$ (base $\lbrace \ket{+}, \ket{-} \rbrace$ di $\sigma_1$). Essendo il qubit già decomposto sulla base computazionale, una misurazione lungo $z$ produrrà
    
    \begin{equation*}
        P(\ket{0}) = \abs{\cos \! \left( \frac{\theta}{2} \right)}^2 \, , \qquad P(\ket{1}) = \abs{\sin \! \left( \frac{\theta}{2} \right)}^2 \, .
    \end{equation*}
    
    \noindent Per capire il risultato della misurazione lungo $x$, invece, dobbiamo espandere $\ket{\psi}$ sulla base $\lbrace \ket{+}, \ket{-} \rbrace$: usando le \eqref{basi_di_sigma_12} per esprimere $\lbrace \ket{0}, \ket{1} \rbrace$ in termini di $\lbrace \ket{+}, \ket{-} \rbrace$ ricaviamo
    
    \begin{equation*}
        P(\ket{+}) = \frac{1}{2} \abs{\cos \! \left( \frac{\theta}{2} \right) + e^{i \phi} \sin \! \left( \frac{\theta}{2} \right)}^2 \, , \qquad P(\ket{-}) = \frac{1}{2} \abs{\cos \! \left( \frac{\theta}{2} \right) - e^{i \phi} \sin \! \left( \frac{\theta}{2} \right)}^2 \, .
    \end{equation*}
    
    \noindent Si noti come in entrambe le situazioni la probabilità risulta correttamente normalizzata: $P(\ket{0}) + P(\ket{1}) = P(\ket{+}) + P(\ket{-}) = 1$. 
\end{esempio}

\begin{esempio}
    Consideriamo lo stato $\ket{+}$ delle \eqref{basi_di_sigma_12}. Qual è l'interpretazione fisica di tale stato? Supponiamo che rappresenti lo spin di una particella: quando lo spin si trova in $\ket{+}$ allora sappiamo con certezza che punta lungo la direzione $x$, ossia $P(\ket{+}) = 1$. Al contrario, per una misurazione lungo $z$ sappiamo che $P(\ket{0}) = 1/2$ e $P(\ket{1}) = 1/2$: abbiamo la certezza del risultato lungo l'asse $x$, ma lungo l'asse $z$ si ha totale incertezza. Questo fenomeno è dovuto alla non commutatività degli operatori di spin nelle 3 direzioni:
    
    \begin{equation*}
        \comm{\hat{S}_i}{\hat{S}_j} = i \hbar \varepsilon_{ijk} \hat{S}_k \, .
    \end{equation*}
    
    \noindent Se consideriamo infatti il sistema preparato in $\ket{+}$ e supponiamo di effettuare una misura lungo $z$ ottenendo $\ket{0}$ allora lo stato collasserà in $\ket{0}$ e, d'ora in avanti, qualsiasi misurazione lungo $z$ produrrà sempre $\ket{0}$ con $P(\ket{0}) = 1$. Nonostante ciò, il fatto che $\hat{S}_z$ non commuti con $\hat{S}_x$ fa sì che una misura successiva lungo $x$ "rigeneri" dell'incertezza: $P(\ket{+}) = 1/2$ e $P(\ket{-}) = 1/2$ (si veda $\ket{0}$ espresso in termini di $\lbrace \ket{+}, \ket{-} \rbrace$ dalle \eqref{basi_di_sigma_12}). 
\end{esempio}

\noindent Discutiamo la generalizzazione del III postulato nel caso in cui alcuni autovalori associati ad autostati differenti siano uguali, cioè siamo in presenza di \textbf{degenerazione}. Per esempio supponiamo il caso $N = \dim \mathcal{H} = 6$: 

\begin{equation*}
    \ket \psi = \alpha_1 \ket 1 + \alpha_2 \ket 2 + \alpha_3 \ket 3 + \alpha_4 \ket 4 + \alpha_5 \ket 5 + \alpha_6 \ket 6 \, , 
\end{equation*}

\noindent dove supponiamo la degenerazione su $a_1 = a_2$ e $a_4 = a_5 = a_6$. Introduciamo gli operatori $\hat{P}_{a_i}$ che considerano solamente la parte di $\ket{\psi}$ corrispondente all'autospazio associato ad $a_i$:

\begin{equation*}
    \ket \psi = \underbrace{\alpha_1 \ket 1 + \alpha_2 \ket 2 }_{\hat P_{a_1} \ket \psi} + \underbrace{\alpha_3 \ket 3}_{\hat P_{a_3} \ket \psi} + \underbrace{\alpha_4 \ket 4 + \alpha_5 \ket 5 + \alpha_6 \ket 6}_{\hat P_{a_4 \ket \psi}} \, ;
\end{equation*}

\noindent tali operatori prendono il nome di \textbf{proiettori} e soddisfano le proprietà seguenti:  
\begin{enumerate}
    \item $\hat P_{a_i}^\dagger = \hat P_{a_i}$;
    \item $\hat P_{a_i}^2 = \hat P_{a_i}$;
    \item $\sum_i \hat P_{a_i} = \mathbb{I}$. 
\end{enumerate}

\noindent I proiettori sono utili per scrivere la \textbf{regola di Born} (III postulato) nel caso generale: dato uno stato $\ket{\psi}$ con degenerazione sugli autovalori $a_i$, la probabilità di ottenere il risultato $a_n$ è

\begin{equation*}
    P(a_n) = \norm{\hat{P}_{a_n} \ket{\psi}}^2 \, ;
\end{equation*}

\noindent dopo la misura, la funzione d'onda collassa nel seguente stato normalizzato:

\begin{equation*}
    \ket{\psi} \rightarrow \frac{\hat{P}_{a_n} \ket{\psi}}{\norm{\hat{P}_{a_n} \ket{\psi}}} \, .
\end{equation*}

\noindent Ad esempio, nel caso dello stato sopra scritto, la probabilità di ottenere $a_1 = a_2$ non è altro che 

\begin{equation*}
    P(a_1) = \norm{\hat{P}_{a_1} \ket{\psi}}^2 = \norm{\alpha_1 \ket{1} + \alpha_2 \ket{2}}^2 = \abs{\alpha_1}^2 + \abs{\alpha_2}^2 \, ,
\end{equation*}

\noindent e lo stato collassa in

\begin{equation*}
    \ket{\psi} \rightarrow \frac{\alpha_1 \ket{1} + \alpha_2 \ket{2}}{\sqrt{\abs{\alpha_1}^2 + \abs{\alpha_2}^2}} \, ;
\end{equation*}

\noindent si noti che si ha ancora incertezza su in quale stato si trovi $\ket{\psi}$, ma con un esperimento successivo \textbf{diverso} siamo in grado di risolvere la degenerazione e ottenere $\ket{1}$ o $\ket{2}$. 

\section{Evoluzione temporale}
Il postulato successivo riguarda l'evoluzione temporale degli stati:

\begin{itemize}
    \item \textbf{IV Postulato} (\textbf{Evoluzione temporale}): L'evoluzione temporale di uno stato generico $\ket{\psi(0)}$ è descritta dall'equazione di Schrödinger:
    
    \begin{equation*}
        i \hbar \dv{t} \ket{\psi(t)} = \hat H \ket{\psi(t)} \, ,
    \end{equation*}
    
    \noindent dove $\hat{H}$ è l'operatore (hermitiano) \textbf{hamiltoniana} del sistema. L'equazione di Schrödinger conserva le probabilità: $\braket{\psi(t)} = \braket{\psi(0)} = 1$. 
\end{itemize}

\noindent Solitamente si risolve questa equazione introducendo l'\textbf{operatore di evoluzione temporale} $\hat{U}(t)$:

\begin{equation*}
    \ket{\psi(t)} = \hat{U}(t) \ket{\psi(0)} \, ;
\end{equation*}

\noindent quando l'hamiltoniana è indipendente dal tempo, $\hat{U}(t)$ diventa semplicemente

\begin{equation*}
    \hat{U} (t) = e^{-\frac{i}{\hbar} \hat H t} \, ;
\end{equation*}

\noindent se invece $\hat H = \hat H(t)$, è necessario distinguere i casi di hamiltoniane commutanti o non commutanti a tempi differenti.\\
\noindent Come detto sopra, l'evoluzione temporale preserva le probabilità e ciò è una diretta conseguenza del fatto che $\hat{U}(t)$ sia \textbf{unitario}: 
\begin{itemize}
    \item $\hat U \hat U^\dagger = \hat U^\dagger U = \mathbb{I} \quad \Rightarrow \quad \hat{U}^\dagger = \hat{U}^{-1}$.
    \item Il prodotto scalare è conservato: $\braket{\hat U\phi}{\hat U\psi} = \braket{\phi}{\hat U^\dagger \hat U\psi} = \braket{\phi}{\psi}$. 
\end{itemize}
\noindent Notiamo che $\hat{U}(t)$ per hamiltoniane indipendenti da $t$ è effettivamente unitario:

\begin{equation*}
    \hat U^\dagger \hat U = \left( e^{-\frac{i}{\hbar} \hat H t} \right)^\dagger e^{-\frac{i}{\hbar} \hat H t} = e^{\frac{i}{\hbar} \hat H t} e^{-\frac{i}{\hbar} \hat H t} = \mathbb{I} \, .
\end{equation*}

\section{Gate}\label{sec:gate}

\begin{definizione}[\textbf{Porte quantistiche}]
    L'analogo quantistico delle porte (o gate) logiche classiche sono le \textbf{porte quantistiche} (o \textbf{gate quantistici}). Un gate quantistico è un operatore unitario che cambia lo stato del sistema.
\end{definizione}

\noindent Notiamo che una delle principali differenze che rendono di difficile implementazione le porte quantistiche risiede nel fatto che non possiamo implementare direttamente le più semplici operazioni classiche come \texttt{AND}, \texttt{OR} o \texttt{XOR}. 

\begin{definizione}[\textbf{Circuito Quantistico}]
    Un \textbf{circuito quantistico} è un modello di computazione quantistica in cui una sequenza ordinata di gate quantistici è applicata ai qubit.
\end{definizione}

\noindent In un circuito classico l'uso dei gate logici è banale. Supponiamo di considerare un bit che si trova in 0 o 1: un gate costituisce l'implementazione di un agente esterno che cambia lo stato del bit. Si pensi ad esempio al gate \texttt{NOT} per il quale $a \rightarrow$ \texttt{NOT} $a$:
\begin{center}
    \begin{circuitikz}
        \draw
        (0,4.5) node[not port] (mynot) {}
        (mynot.in) node[left = .4cm, anchor=east] (a) {$0$}
        (mynot.out) node[right = .4cm,anchor=west] (b) {$1 \, ,$}
        (mynot.in) -- (a)
        (mynot.out) -- (b);
    \end{circuitikz}
    $\qquad$
    \begin{circuitikz}
        \draw
        (0,4.5) node[not port] (mynot) {}
        (mynot.in) node[left = .4cm, anchor=east] (a) {$1$}
        (mynot.out) node[right = .4cm,anchor=west] (b) {$0 \, ,$}
        (mynot.in) -- (a)
        (mynot.out) -- (b);
    \end{circuitikz}
\end{center}

\noindent Nel caso invece di un qubit, i circuiti funzionano diversamente perché le porte agiscono su sistemi a due livelli. Immaginiamo che a causa di un agente esterno il qubit $\ket{\psi}$ subisca un evoluzione temporale $\hat{U}$: rappresentiamo questo fatto mediante il circuito seguente
\begin{center}
    \mbox{
        \Qcircuit @C=2em @R=2em {
            \lstick{\ket{\psi}} & \gate{\hat{U}} & \rstick{\hat{U} \ket{\psi} \, ,} \qw \\
        }
    }
\end{center}

\noindent Si ricordi che $\hat{U}$ è sempre un operatore unitario: ad esempio per un'hamiltoniana indipendente dal tempo si ha semplicemente $\hat{U}(t) = e^{-\frac{i}{\hbar}\hat H t}$. \\
\noindent Consideriamo le matrici di Pauli: sappiamo che sono hermitiane ($\sigma^\dagger_i = \sigma_i$) e che soddisfano la proprietà $\sigma_i^2 = \mathbb{I}$, ma questo significa che sono anche matrici unitarie. Questo fatto ci permette di costruire\footnote{Un tale sistema in natura è abbastanza semplice da implementare poiché, essendo $\hat{H} = \vec{\sigma} \cdot \vec B$ l'accoppiamento tra spin e campo magnetico, è facile costruire una tale evoluzione temporale.} dei gate in cui $\hat{U} = \hat{\sigma}_i$.  Ad esempio è possibile implementare dei gate come $\mathbb{I}$, $\sigma_1 \equiv X$, $\sigma_2 \equiv Y$ e $\sigma_3 \equiv Z$. Ricordando che $\sigma_i \sigma_j = 2i \varepsilon_{ijk} \sigma_k$, notiamo che $XZ = - i Y$ e inoltre anche la matrice $-iY$ è unitaria. Per tale ragione molto spesso, al posto di considerare i gate $\lbrace \mathbb{I}, X, Y, Z \rbrace$ si sceglie la base $\lbrace \mathbb{I}, X, Z, XZ \rbrace$: questo significa che possiamo implementare i gate seguenti
\begin{center}
    \mbox{
        \Qcircuit @C=2em @R=2em {
            & \gate{X} & \qw \\
        }
    } 
    , \ \ \ \ 
    \mbox{
        \Qcircuit @C=2em @R=2em {
            & \gate{Z} & \qw \\
        }
    }
    , \ \ \ \ 
    \mbox{
        \Qcircuit @C=2em @R=2em {
            & \gate{XZ} & \qw \\
        }
    }
    ,
\end{center}

\noindent Consideriamo l'\texttt{X-gate}: dalle \eqref{Pauli_matrices} è evidente che $X$ rappresenta una sorta di "quantum" \texttt{NOT} perché inverte semplicemente lo stato della base computazionale:
\begin{center}
    \mbox{
        \Qcircuit @C=2em @R=2em {
            \lstick{\ket{0}} & \gate{X} & \rstick{\ket{1} \, ,} \qw \\
        }
    } 
    \\
    \mbox{
        \Qcircuit @C=2em @R=2em {
            \lstick{\ket{1}} & \gate{X} & \rstick{\ket{0} \, ,} \qw \\
        }
    }
\end{center}

\noindent Consideriamo ora lo \texttt{Z-gate}: gli stati della base computazionale sono autovettori con autovalori 0 e 1 di $\sigma_3$, quindi questo gate inverte semplicemente il segno
\begin{center}
    \mbox{
        \Qcircuit @C=2em @R=2em {
            \lstick{\ket{0}} & \gate{Z} & \rstick{\ket{0} \, ,} \qw \\
        }
    } 
    \\
    \mbox{
        \Qcircuit @C=2em @R=2em {
            \lstick{\ket{1}} & \gate{Z} & \rstick{-\ket{1} \, ,} \qw \\
        }
    }
\end{center}
L'azione dello \texttt{Z-gate} su un generico qubit risulterà quindi in
\begin{center}
    \mbox{
        \Qcircuit @C=2em @R=2em {
            \lstick{a \ket{0} + b \ket{1}} & \gate{Z} & \rstick{a \ket{0} - b \ket{1} \, ,} \qw \\
        }
    } 
\end{center}
e questo significa che $Z$ aggiunge semplicemente una fase $e^{i \pi} = -1$ allo stato. Ricapitolando: l'\texttt{X-gate} implementa un'interferenza dall'esterno che inverte lo stato (ad esempio cambia segno dello spin lungo $z$) e lo \texttt{Z-gate} implementa l'introduzione di una fase. \\
\noindent Una matrice particolarmente importante per i nostri scopi è 
\begin{equation}\label{Hadamard_matrix}
    H = \frac{1}{\sqrt{2}} 
    \begin{pmatrix}
        1 & 1 \\ 1 & -1 
    \end{pmatrix} \, , 
\end{equation}
chiamata \textbf{matrice di Hadamard}. Notiamo che è unitaria in quanto $H^\dagger H = \mathbb{I}$. Essa può essere implementata nel cosiddetto \texttt{H-gate} o \textbf{gate di Hadamard}: si tratta di un gate particolarmente importante (lo useremo largamente durante tutto il corso) in quanto permette di cambiare base $\lbrace \ket{0}, \ket{1} \rbrace \leftrightarrow \lbrace \ket{+}, \ket{-} \rbrace$ 

\begin{center}
    \mbox{
        \Qcircuit @C=2em @R=2em {
            \lstick{\ket{0}} & \gate{H} & \rstick{\ket{+} \, ,} \qw \\
        }
        \ \ \ \ \ \ \ \ \ \ \ \ \ \ \ \ \ \ \ \ 
        \Qcircuit @C=2em @R=2em {
            \lstick{\ket{+}} & \gate{H} & \rstick{\ket{0}\, ,} \qw \\
        }
    }
    \\
    \mbox{
        \Qcircuit @C=2em @R=2em {
            \lstick{\ket{1}} & \gate{H} & \rstick{\ket{-} \, ,} \qw \\
        } 
        \ \ \ \ \ \ \ \ \ \ \ \ \ \ \ \ \ \ \ \ 
        \Qcircuit @C=2em @R=2em {
            \lstick{\ket{-}} & \gate{H} & \rstick{\ket{1} \, ,} \qw \\
        }
    }
\end{center}

\noindent Possiamo introdurre anche le matrici seguenti (ci serviranno più avanti)

\begin{equation}\label{S_T_matrices}
    S \equiv \sqrt{Z} =
\begin{pmatrix}
    1 & 0 \\ 0 & i
\end{pmatrix} \, , \qquad 
T \equiv \sqrt{S} =
\begin{pmatrix}
    1 & 0 \\ 0 & e^{i \frac{\pi}{4}}
\end{pmatrix} \, ,
\end{equation}

\noindent Le matrici introdotte in precedenza costituiscono gli oggetti base con cui andremo a implementare diversi gate e circuiti durante tutto il corso. Per costruire il gate più generale possiamo esponenziare scrivendo $U = e^{-\frac{i}{\hbar} H t}$ dove $H = a \mathbb{I} + b_i \sigma_i$ e $a, b_i \in \mathbb{R}$ con $i = 1,2,3$. In particolare esiste una particolare classe di operatori che utilizzeremo molto
\begin{equation*}
    R_{\vec{n}} = e^{-i \frac{\lambda}{2} (\vec n \cdot \vec \sigma)} \, ;
\end{equation*}
si tratta di un caso particolare dell'esponenziazione precedente in cui $a = 0$ e i coefficienti $b_i$ sono scelti lungo un particolare versore $\vec n$. Questo operatore unitario implementa una rotazione di angolo $\lambda$ attorno ad una direzione particolare individuata da $\vec n$:

\begin{equation}\label{rotation_n_lambda}
    R_{\vec{n}}(\lambda) = e^{-i \frac{\lambda}{2} (\vec n \cdot \vec \sigma)} = \cos \! \left( \frac{\lambda}{2} \right) \mathbb{I} - i \sin \! \left( \frac{\lambda}{2} \right) \vec \sigma \cdot \vec n \, ;
\end{equation}
(si espanda il LHS con la serie di Taylor dell'esponenziale e si usi $(\vec \sigma \cdot \vec n)^2 = \mathbb{I}$ per dimostrare l'uguaglianza con il RHS). È possibile dimostrare, inoltre, che qualsiasi matrice unitaria $2 \times 2$ può essere scritta nella forma seguente
\begin{equation}\label{general_2by2_matrix}
    U = e^{i \alpha}
    \begin{pmatrix}
        e^{-i \frac{\beta}{2}} & 0 \\ 0 & e^{i \frac{\beta}{2}}
    \end{pmatrix}
    \begin{pmatrix}
        \cos \frac{\gamma}{2} & - \sin \frac{\gamma}{2} \\ \sin  \frac{\gamma}{2} & \cos \frac{\gamma}{2}
    \end{pmatrix}
    \begin{pmatrix}
        e^{-i \frac{\delta}{2}} & 0 \\ 0 & e^{i \frac{\delta}{2}}
    \end{pmatrix}
    = e^{i \alpha} R_z(\beta) R_x(\gamma) R_z(\delta) \, ; 
\end{equation}
perciò il più generale operatore unitario presenta 4 parametri reali $\alpha, \beta, \gamma, \delta \in \mathbb{R}$ e può implementare un possibile gate in un computer quantistico. Appare subito evidente come la scelta di 4 possibili parametri reali (quindi continui) consenta di realizzare un numero nettamente maggiore di gate logici quantistici rispetto al caso dei gate logici classici. 
    %%%%%%%%%%%%%%%%%%%%%%%
%%%%%% Lezione 3 %%%%%%
%%%%%%%%%%%%%%%%%%%%%%%
\vspace{1.0cm}
\newline
\lecture{3}{11/10/2021}
\vspace{1.0cm}

\noindent Nel caso dei funzionali come $E\big[u_\alpha\big]$ con vincolo $\ip{u_\alpha}{u_\beta}-\delta_{\alpha\beta}=0$ utilizziamo $\Lambda_{\alpha\beta}$ come moltiplicatore di Lagrange:
\begin{equation*}
    \functionalderivative{u_\alpha^*(\overline x)}\bigg(E\big[\{u_\alpha\},\{u^*_\alpha\}\big]-\Lambda_{\alpha'\beta'}(\ip{u_{\alpha'}}{u_{\beta'}}-\delta_{\alpha'\beta'})\bigg)=0
\end{equation*}
\textbf{Termine} $\ip{u_{\alpha'}}{u_{\beta'}}$:
\begin{equation*}
    \functionalderivative{u_\alpha^*(\overline x)}\int d^3x'u_{\alpha'}^*(\overline{x}')u_{\beta'}(\overline{x}')=u_{\beta'}(\overline x)
\end{equation*}
\textbf{Termine} $\sum_{\mu}\mel{\mu}{\hat h}{\mu}$:
\begin{equation*}
    \functionalderivative{u_\alpha^*(\overline x)}\int d^3x'u_\mu^*(\overline{x}')\hat h u_\mu(\overline{x}')=\delta_{\alpha\mu}\hat h u_\alpha(\overline x)
\end{equation*}
\textbf{Termine} $\frac 12\sum_{\mu\nu}J_{\mu\nu}$:
\begin{equation*}
    \functionalderivative{u_\alpha^*(\overline x)}\frac 12 \sum_{\mu\nu}\int d^3x' d^3x''u_\mu^*(\overline{x}')u_\nu^*(\overline{x}'')\frac{e_0^2}{4\pi\varepsilon_0|\overline{x}'-\overline{x}''|}u_\mu({\overline{x}'})u_\nu(\overline{x}'')
\end{equation*}
Se $\mu=\alpha \vee \nu=\alpha$ ho due contributi, prendiamo $\mu=\alpha$ e semplifichiamo il fattore $\frac 12$
\begin{equation*}
    \functionalderivative{u_\alpha^*(\overline x)} \sum_{\nu}\int d^3x' u_\alpha^*(\overline{x}')u_\alpha(\overline{x}') \int d^3x''u_\nu^*(\overline{x}'')\frac{e_0^2}{4\pi\varepsilon_0|\overline{x}'-\overline{x}''|}u_\nu(\overline{x}'') =
\end{equation*}
\begin{equation*}
    = \underbrace{\sum_{\nu}\int d^3x''u_\nu^*(\overline{x}'')u_\nu(\overline{x}'')\frac{e_0^2}{4\pi\varepsilon_0|\overline{x}'-\overline{x}''|}}_{\text{Potenziale di Hartree}}u_\alpha(\overline x)=V_{\text{Hartree}}(\overline x)u_\alpha(\overline x)
\end{equation*}
\textbf{Termine} -$\frac 12\sum_{\mu\nu}K_{\mu\nu}$:
\begin{equation*}
    -\functionalderivative{u_\alpha^*(\overline x)}\frac 12 \sum_{\mu\nu}\int d^3x' d^3x''u_\mu^*(\overline{x}')u_\nu^*(\overline{x}'')\frac{e_0^2}{4\pi\varepsilon_0|\overline{x}'-\overline{x}''|}u_\nu({\overline{x}'})u_\mu(\overline{x}'')
\end{equation*}
Come prima, se $\mu=\alpha \vee \nu=\alpha$ ho due contributi, prendiamo $\mu=\alpha$ e semplifichiamo il fattore $\frac 12$
\begin{equation*}
    - \functionalderivative{u_\alpha^*(\overline x)} \sum_{\nu}\int d^3x' u_{\alpha}^*(\overline{x}')\int d^3x''u_\nu^*(\overline{x}')u_\nu(\overline{x}'')\frac{e_0^2}{4\pi\varepsilon_0|\overline{x}'-\overline{x}''|}u_\alpha(\overline{x}') =
\end{equation*}
\begin{equation*}
    = -\sum_\nu \int d^3x''u_\nu^*(\overline{x}'')u_\alpha(\overline{x}'')\frac{e_0^2}{4\pi\varepsilon_0|\overline{x}-\overline{x}''|}u_\nu(\overline{x})=-\hat V_{\text{Fock}}(\overline x)u_\alpha(\overline x)
\end{equation*}
Mettendo insieme tutti i termini abbiamo:
\begin{equation*}
    \hat h(\overline x) u_\alpha(\overline x)+\hat V_H(\overline x)u_\alpha(\overline x)-\hat V_{\text F}(\overline x) u_\alpha(\overline x) = \Lambda_{\alpha\beta'}u_{\beta'}(\overline x)
\end{equation*}
Specifichiamo meglio questi contributi:
\begin{itemize}
    \item $J_{\mu\nu}=\mel{\mu\nu}{V_{12}}{\mu\nu}$: senza perdere in generalità, distinguiamo la componente spaziale da quella di spin come: $\mu=\mu\sigma$, $\sigma=\pm\frac 12$, quindi
    \begin{equation*}
        J_{\mu\nu}=\mel{\mu\nu}{V_{12}}{\mu\nu}=\mel{\mu\sigma\nu\sigma'}{V_{12}}{\mu\sigma\nu\sigma'}=\mel{\mu\nu}{V_12}{\mu\nu}\ip{\sigma\sigma'}{\sigma\sigma'}
    \end{equation*}
    dal momento che $V_{12}$ non dipende dallo spin, rimane solo la componente spaziale. Pertanto nel valutare l'energia di Hartree, l'integrale è sì un integrale spaziale, ma la somma viene eseguita anche sullo spin:
    \begin{equation*}
        E_{\text H}=\frac 12\sum_{\mu\nu\sigma\sigma'}\mel{\mu\nu}{V_{12}}{\mu\nu}
    \end{equation*}
    \item $K_{\mu\nu}=\mel{\mu\nu}{V_{12}}{\nu\mu}$: stesso discorso vale per l'integrale di scambio, tuttavia, il suo valore è non nullo se gli spin hanno lo stesso verso:
    \begin{equation*}
        K_{\mu\nu}=\mel{\mu\nu}{V_{12}}{\nu\mu}=\mel{\mu\sigma\nu\sigma'}{V_{12}}{\nu\sigma'\mu\sigma}=-\frac 12 \sum_{\mu\nu\sigma}K_{\mu\sigma\nu\sigma}
    \end{equation*}
\end{itemize}
Possiamo fare un passo in più considerando una \textbf{proprietà del determinante di Slater}: esso è invariante rispetto a una transformazione unitaria del set di numeri quantici delle funzioni d'onda a singola particella $\{u_\alpha\}$. Questo significa che $E_{\text H}$ è invariante. Osserviamo inoltre che la \textbf{matrice dei moltiplicatori di Lagrange} $\Lambda_{\alpha\beta}$ è hermitiana (autovalori lineari) e simmetrica. Mettendo insieme queste informazioni abbiamo che l'equazione dei vincoli è invariante rispetto allo scambio tra particelle, quindi i moltplicatori di Lagrange non dipendono dalle funzioni d'onda e $\Lambda_{\alpha\beta}$ è una matrice hermitiana. Possiamo quindi scegliere $u_\alpha$ affinché $\Lambda_{\alpha\beta}$ sia \textbf{diagonale}. In questo caso avremo:
\begin{equation*}
    \hat h u_\alpha(\overline x)+\hat V_{\text H}u_\alpha(\overline x)-\hat V_{\text F}u_\alpha(\overline x)=\varepsilon_\alpha u_\alpha(\overline x)
\end{equation*}
dove i $\varepsilon_\alpha$ sono numeri reali e dipendono dalla funzione d'onda.
\begin{equation*}
    \underbrace{(\hat h + \hat V_{\text H} - \hat V_{\text F})}_{\hat h_{\text{HF}}}u_\alpha(\overline x)=\varepsilon_\alpha u_\alpha(\overline x)
\end{equation*}
se rimuoviamo il termine $\hat V_{\text F}$, rifiutiando quindi di considerare il termine di scambio, troviamo il risultato del \textbf{metodo di Hartree}. \\
Alla fine quello che dobbiamo risolvere è questa equazione in modo da ottenere soluzioni autoconsistenti $\{u_\alpha\}$. Possiamo calcolare:
\begin{equation*}
    E=\sum_{\mu\sigma}\mel{\mu\sigma}{\hat h}{\mu\sigma}+\frac 12 \sum_{\mu\nu\sigma\sigma'}J_{\mu\sigma\nu\sigma'}-\frac 12 \sum_{\mu\nu\sigma}K_{\mu\sigma\nu\sigma}
\end{equation*}
Un altro modo di esprimere l'\textbf{equazione di Hartree-Fock}:
\begin{equation*}
    \mel{u_\alpha}{\hat h + \hat V_{\text H} - \hat V_{\text F}}{u_\alpha}=\mel{u_\alpha}{\varepsilon_\alpha}{u_\alpha}
\end{equation*}
\begin{equation*}
    \sum_\alpha\mel{\alpha}{\hat h}{\alpha}+\sum_\alpha\mel{u_\alpha}{\hat V_{\text H}}{u_\alpha}-\sum_\alpha\mel{u_\alpha}{\hat V_{\text F}}{u_\alpha}=\sum_\alpha\varepsilon_\alpha
\end{equation*}
\begin{equation*}
    \sum_\mu\mel{\mu}{\hat h}{\mu}+\sum_{\mu\nu}(J_{\mu\nu}-K_{\mu\nu})=\sum_\alpha \varepsilon_\alpha
\end{equation*}
è simile all'energia esatta, ma non è esatta poiché stiamo contando due volte l'interazione, vista in un altro modo:
\begin{equation*}
    E_{\text{HF}}=\sum_\alpha\varepsilon_\alpha-\frac 12\sum_{\mu\nu}(J_{\mu\nu}-K_{\mu\nu})=\sum_\alpha\varepsilon_\alpha-E_{\text H}-E_{\text X}
\end{equation*}
Gli \textbf{autovalori} dell'energia di Hartree-Fock sono i \textbf{moltiplicatori di Lagrange} $\varepsilon_\alpha$, ma qual è il loro significato fisico? Come possono essere usati? La risposta viene dal seguente teorema
\begin{theorem}[\textbf{Teorema di Koopmans}]
    Si consideri $N$ elettroni con funzioni d'onda $\{u_\gamma\}$ e il determinante di Slater della soluzione dell'\textbf{equazione di Hartree-Fock}. Supponiamo di considerare N-1 elettroni, rimuovendo quindi un elettrone: un partifolare stato a energia $\varepsilon_\alpha$ descritto da $u_\alpha$. Il nuovo set di funzioni d'onda sarà $\{u_\beta\}\not\owns u_\alpha$. Il sistema a N particella è: $E_N=E_{N-1}+\varepsilon_\alpha$. Ciò significa che $\varepsilon_\alpha$ è l'energia da fornire al sistema per rimuovere l'elettrone. Se per esempio abbiamo un atomo e $\varepsilon_\alpha$ dell'elettrone è negativa, $\varepsilon_\alpha$ corrisponde all'energia da fornire per ionizzare il sistema.
\end{theorem}
\begin{prf}
    \begin{equation*}
        E_N=\sum_\mu\mel{\mu}{\hat h}{\mu}+\frac 12\sum_{\mu\nu}(J_{\mu\nu}-K_{\mu\nu})
    \end{equation*}
    \begin{equation*}
        E_{N-1}=\sum_{\mu\neq\alpha}\mel{\mu}{\hat h}{\mu}+\frac 12\sum_{\mu\nu\neq \alpha}(J_{\mu\nu}-K_{\mu\nu})
    \end{equation*}
    \begin{equation*}
        E_N=E_{N-1}+\underbrace{\mel{\alpha}{\hat h}{\alpha}+\sum_\mu(J_{\alpha\mu}-K_{\alpha\mu})}_{\varepsilon_\alpha}
    \end{equation*}
\end{prf}
\subsection*{Osservazioni}
Consideriamo $N$ elettroni e un sistema che passa dallo stato fondamentale a uno eccitato: $\varepsilon_\alpha \rightarrow \varepsilon_{\alpha'}$. Abbiamo un nuovo determinante di Slater e la differenza di energia sarà:
\begin{equation*}
    \Delta E = E_{\alpha'}-E_\alpha=\varepsilon_{\alpha'}-\varepsilon_\alpha - (J_{\alpha\alpha'}-K_{\alpha\alpha'})
\end{equation*}
Questo valore non è trascurabile quando parliamo di atomi, tuttavia quando parliamo di materiali, se eseguiamo gli integrali, questi valori sono molto piccoli soprattutto tra due elettroni separati da una grande distanza. Questi valori possono essere usati nell'equazione di Hartree-Fock per calcolare lo stato eccitato. Supponiamo di rimuovere in $\alpha$ un elettrone e di metterlo nello stato $\alpha'$, l'\textbf{energia di rilassamento}, $J_{\alpha\alpha}$ è trascurabile in questo caso.\\
Il \textbf{metodo di Hartree-Fock} fornisce una soluzione per sistemi a molti elettroni con un singolo determinante di Slater ed è molto utile per i \textbf{sistemi a shell chiusa}.\\
Consideriamo:
\begin{itemize}
    \item Atomo di He $1s^2$: è un sistema a shell chiusa ed è descritto bene dal modello di Hartree poiché non c'è energia di scambio.
    \item Atomo di Be $1s^22s^2$: è un sistema a shell chiusa e abbiamo un contributo da parte dell'energia di scambio.
    \item Atomo di He eccitato $1s2s$: in questo caso applicando la teoria delle perturbazioni abbiamo:
    \begin{itemize}
        \item Stato di singoletto: $S=0, m_s=0$
        \item Stato di tripletto: $S=1, m_s=\pm 1$ (di cui ne prendiamo il determinante a singola particella del Slater) e $S=1, m_s= 0$
    \end{itemize}
    L'idea è quindi di risolvere in maniera autoconsistente il \textbf{metodo di Hartree-Fock} con una funzione d'onda a singola particella per ottenere tutti gli stati. Per gli stati a singola particella l'energia sarà esatta, altrimenti sarà differente.
    \item Atomo di C $1s^22s^22p^2$: abbiamo differenti termini per questo atomo ${}^3P, {}^2D, {}^1S$, ma soltanto il secondo possiede un determinante di Slater a singola particella, gli altri no.
\end{itemize}

\section{Modello di Thomas-Fermi}
Un altro modello che possiamo prendere in considerazione è il \textbf{modello di Thomas-Fermi}, a differenza del \textbf{metodo di Hartree-Fock}, l'energia è scritta come funzionale della densità di elettroni:
\begin{equation*}
    E\big[n(\overline x)\big]=\int \dd[3]{x}V_{\text ext}(\overline x)n(\overline x)+E_{\text H}\big[n(\overline x)\big]+E_{\text K}\big[n(\overline x)\big]
\end{equation*}
dove
\begin{equation*}
    E_{\text H}\big[n(\overline x)\big]=\frac 12 \frac{e_0^2}{4\pi\varepsilon_0}\int \dd[3]{x'}\dd[3]{x''}\frac{n(\overline{x}')n(\overline{x}'')}{|\overline{x}'-\overline{x}''|}
\end{equation*}
e supponiamo che il funzionale dell'energia cinetica sia funzione di $n(\overline x)$.\\
L'approssimazione che si fa in questo modello consiste nel considerare un sistema di un grande numero N di elettroni liberi e confinati in una certa regione dello spazio con le opportune condizioni periodiche al contorno. Supponiamo di avere delle particelle indipendenti in una scatola di volume $V=L^3$:
\begin{equation*}
    u_{\overline k}(\overline x)=\frac{e^{i\overline k \cdot \overline x}}{\sqrt V} \ \ \ \ \ \ \ \ \ \  \overline k = \frac{2\pi}{L}(n_x,n_y,n_z) \ \ \ \ \ \ \ \ \ \ \varepsilon_{\overline k}=\frac{\hbar^2\overline{k}^2}{2m}
\end{equation*}
Nello stato fondamentale, possiamo riempire tutti i livelli energetici fino all'energia più alta: \textbf{energia di Fermi}, $\varepsilon_{\text F}$.
Per N particelle e un volume V, possiamo definire la \textbf{densità di particelle}: $n=\frac NV$. Cosicché l'energia di Fermi risulti definita come:
\begin{equation*}
    \varepsilon_{\text F}=\frac{\hbar^2k_{\text F}^2}{2m} \ \ \ \ \ k_{\text F}=(3\pi^2n)^{\frac 13}
\end{equation*}
Alla luce di ciò possiamo definire la densità di particelle come:
\begin{equation*}
    n=\int_0^{\varepsilon_F}d\varepsilon D(\varepsilon)
\end{equation*}
dove $D(\varepsilon)$ rappresenta la densità di stati, cioè stati per unità di volume ed energia, $D(\varepsilon)\propto\sqrt\varepsilon$.
\begin{equation*}
    D(\varepsilon)=\frac{1}{2\pi^2}\bigg(\frac{2m}{\hbar^2}\bigg)^{\frac 32}\sqrt\varepsilon
\end{equation*}
L'energia media per particella sarà:
\begin{equation*}
    \expval{\varepsilon}=\frac EN=\frac VN \frac EV = \frac 1 n \frac EV=\frac 1 n \int_{0}^{\varepsilon_{\text F}}d\varepsilon D(\varepsilon)\varepsilon = \frac 3 5 \varepsilon_{\text F}(n)
\end{equation*}
L'idea nel \textbf{modello di Thomas-Fermi} è di considerare che localmente non abbiamo più un sistema omogeneo, abbiamo in principio un sistema non omogeneo perché la densità di elettroni dipende da $\overline x$. Più precisamente l'approssimazione riguarda l'energia cinetica quantistica in cui si assume che localmente gli elettroni abbiano una energia cinetica media che corrisponde all'energia cinetica media del sistema omogeneo di particelle indipendenti a quella densità locale.
\begin{equation*}
    E_{\text K}=\int \dd[3]{x}n(\overline x)\underbrace{\frac 35 \varepsilon_{\text F}(n(\overline x))}_{\mathclap{\expval{\varepsilon} \text{ al punto } \overline x}}
\end{equation*}
Abbiamo usato quindi il risultato di particelle indipendenti in una scatola per costruire un'approssimazione per l'energia cinetica di un sistema non omogeneo di elettroni interagenti. A questo punto possiamo scrivere il funzionale dell'energia come:
\begin{equation*}
    E_{\text H}\big[n(\overline x)\big]=\int \dd[3]{x} V_{\text{ext}}(\overline x)n(\overline x)+E_{\text H}\big[n(\overline x)\big]+\int \dd[3]{x}n(\overline x)\frac 35\varepsilon_{\text F}(n(\overline x))
\end{equation*}
Ancora una volta vogliamo minimizzare questo funzionale con il seguente vincolo:
\begin{equation*}
    \int_V \dd[3]{x} n(\overline x)=N
\end{equation*}
A differenza del \textbf{metodo di Hartree-Fock}, qui abbiamo un solo moltiplicatore di Lagrange:
\begin{equation*}
    \functionalderivative{n(\overline x)}\Bigg(E\big[n(\overline x)\big]-\mu\bigg(\underbrace{\int d^3x'n(x')}_{=1}-\underbrace{N}_{=0}\bigg)\Bigg)=0
\end{equation*}
\begin{equation*}
    \functionalderivative{n(\overline x)}E\big[n(\overline x)\big]=\mu
\end{equation*}
Questa è l'equazione che dobbiamo risolvere e ci dà la densità dello stato fondamentale. Scriviamo esplicitamente:
\begin{equation*}
    V_{\text {ext}}(\overline x)+V_{\text H}(\overline x) + \functionalderivative{n(\overline x)}E_{\text K}=\mu
\end{equation*}
\begin{equation*}
    \begin{aligned}
        \functionalderivative{n(\overline x)}E_{K}
        & =\functionalderivative{n(\overline x)}\int d^3x n(\overline x)\frac 35 \frac{\hbar^2}{2m}(3\pi^2)^{\frac 23}n(\overline x)^{\frac 23} \\
        & = \functionalderivative{n(\overline x)}\int d^3x n(\overline x)^{\frac 53}\frac 35 \frac{\hbar^2}{2m}(3\pi^2)^{\frac 23} \\
        & = \frac 35\frac{\hbar^2}{2m}(3\pi^2)^{\frac 23}\frac 53n^{\frac 23}(\overline x)\\
        & = \varepsilon_{\text F}(n(\overline x))
    \end{aligned}
\end{equation*}
Ottenendo così l'equazione di Thomas-Fermi:
\begin{equation*}
    V_{\text {ext}}(\overline x)+V_{\text H}(\overline x) + \varepsilon_{\text F}(n(\overline x))=\mu
\end{equation*}
Risolvendo questa equazione si ottiene la densità dello stato fondamentale, da cui poi possiamo ricavare l'energia dello stato fondamentale.
    %%%%%%%%%%%%%
% LECTURE 4 %
%%%%%%%%%%%%%
\vspace{1cm}

\noindent \lecture{4}{15/10/2021}

\vspace{0.5cm}

\noindent Continuiamo la discussione riguardante il concetto di entanglement. Dato che questo fenomeno si manifesta in sistemi con almeno due qubit, possiamo utilizzare due differenti basi:
\begin{itemize}
    \item \textbf{Base computazionale} (o \textbf{standard}): formata dagli stati $\lbrace \ket{00}, \ket{01}, \ket{10}, \ket{11} \rbrace$ (la due entrate indicano gli stati del primo e secondo qubit rispettivamente).
    
    \item \textbf{Base di Bell} (o \textbf{EPR}): costituita dagli stati 
    \begin{align*}
        \ket{\beta_{00}} &= \frac{1}{\sqrt{2}} (\ket{00} + \ket{11}) \, , &\ket{\beta_{01}} &= \frac{1}{\sqrt{2}} (\ket{01} + \ket{10}) \, , \\
        \ket{\beta_{10}} &= \frac{1}{\sqrt{2}} (\ket{00} - \ket{11}) \, , &\ket{\beta_{11}} &= \frac{1}{\sqrt{2}} (\ket{01} - \ket{10}) \, ;
    \end{align*}
    notiamo che si trattano di stati entangled costruiti a partire da combinazioni lineari indipendenti degli stati della base computazionale.
\end{itemize}

\noindent Chiaramente, trattandosi di una base, possiamo espandere qualsiasi stato $\ket{\psi}$ nella base EPR, scrivendo
\begin{equation*}
    \ket{\psi} = \sum_{n,m=0}^1 \alpha_{nm} \ket{\beta_{nm}} \, .
\end{equation*}
Conseguentemente, se si cerca la probabilità di trovarsi in $\ket{\beta_{nm}}$ si può effettuare una misurazione nella base di Bell e ottenere $P(\ket{\beta_{nm}}) = \abs{\alpha_{nm}}^2$. 

\noindent Gli stati della base di Bell non sono difficili da costruire utilizzando i gate che abbiamo visto precedentemente. Supponiamo di poter utilizzare un computer quantistico i cui qubit si trovano nella base standard $\lbrace \ket{0}, \ket{1} \rbrace$. Utilizziamo l'\texttt{H-gate} e il \texttt{CNOT-gate}: 
\begin{itemize}
    \item Ricordiamo che $H\ket{0} = \ket{+}$ e $H \ket{1} = \ket{-}$, quindi il gate di Hadamard permette di passare da un qubit nella base computazionale ad un qubit in una combinazione lineare di elementi di questa base (si ricordi la matrice in \eqref{Hadamard_matrix} e le definizioni in \eqref{basi_di_sigma_12}). 
    
    \item Il \texttt{CNOT-gate}, invece, scambia il secondo qubit solamente se il primo si trova in $\ket{1}$:
    \begin{align*}
    \ket{00} &\overset{\texttt{CNOT}}{\longrightarrow} \ket{00} \, , &\ket{01} &\overset{\texttt{CNOT}}{\longrightarrow} \ket{01} \, , \\
    \ket{10} &\overset{\texttt{CNOT}}{\longrightarrow} \ket{11} \, , &\ket{11} &\overset{\texttt{CNOT}}{\longrightarrow} \ket{10} \, .
\end{align*}
\end{itemize}

\noindent Utilizzando questi due gate possiamo facilmente implementare il circuito seguente 
\begin{center}
    \mbox{
        \Qcircuit @C=2em @R=1.5em {
            \lstick{\ket{x}} & \gate{H} & \ctrl{1} & \qw \\
            \lstick{\ket{y}} & \qw & \targ & \qw
        }
    }
\end{center}
i cui output sono esattamente gli stati $\ket{\beta_{xy}}$ della base di Bell. Verifichiamolo:
\begin{align*}
    \ket{00} &\overset{H}{\rightarrow} \frac{1}{\sqrt{2}} (\ket{0} + \ket{1}) \otimes \ket{0} = \frac{1}{\sqrt{2}} \left( \ket{00} + \ket{10} \right) \overset{\texttt{CNOT}}{\rightarrow} \frac{1}{\sqrt{2}} \left( \ket{00} + \ket{11} \right) \equiv \ket{\beta_{00}} \, , \\
    \ket{01} &\overset{H}{\rightarrow} \frac{1}{\sqrt{2}} (\ket{0} + \ket{1}) \otimes \ket{1} = \frac{1}{\sqrt{2}} \left( \ket{01} + \ket{11} \right) \overset{\texttt{CNOT}}{\rightarrow} \frac{1}{\sqrt{2}} \left( \ket{01} + \ket{10} \right) \equiv \ket{\beta_{01}} \, , \\
    \ket{10} &\overset{H}{\rightarrow} \frac{1}{\sqrt{2}} (\ket{0} - \ket{1}) \otimes \ket{0} = \frac{1}{\sqrt{2}} \left( \ket{00} - \ket{10} \right) \overset{\texttt{CNOT}}{\rightarrow} \frac{1}{\sqrt{2}} \left( \ket{00} - \ket{11} \right) \equiv \ket{\beta_{10}} \, , \\
    \ket{11} &\overset{H}{\rightarrow} \frac{1}{\sqrt{2}} (\ket{0} - \ket{1}) \otimes \ket{1} = \frac{1}{\sqrt{2}} \left( \ket{01} - \ket{11} \right) \overset{\texttt{CNOT}}{\rightarrow} \frac{1}{\sqrt{2}} \left( \ket{01} - \ket{10} \right) \equiv \ket{\beta_{11}} \, ;
\end{align*}
si noti come l'azione di un gate (in questo caso l'\texttt{H-gate}) su un singolo qubit non basti per produrre uno stato entangled. Al contrario il \texttt{CNOT-gate}, invece, crea stati entangled poiché agisce su coppie di qubit. 

\noindent Vediamo ora due esplicite applicazioni dell'entanglement. 

\section{Superdense coding}
Si tratta del primo esempio esplicito delle potenzialità dei metodi quantistici contro i metodi classici. Il problema riguarda il come inviare informazioni di due bit classici (00, 01, 10, 11) ad un generico sperimentatore. Consideriamo la sperimentatrice Alice e supponiamo che abbia due bit di informazione (due numeri $xy$) e che voglia inviarli allo sperimentatore Bob. Dal punto di vista classico, Alice semplicemente utilizza un canale classico (un telefono ad esempio) per comunicare direttamente a Bob quale coppia di numeri possiede. Nel caso in cui Alice possieda due qubit, invece, può inviare solamente uno dei due sfruttando il fatto che siano entangled. Supponiamo che Alice e Bob condividano due qubit entangled, ad esempio
\begin{equation*}
    \ket{\psi} = \frac{1}{\sqrt{2}} \left( \ket{00} + \ket{11} \right) = \ket{\beta_{00}} \, ,
\end{equation*}
dove Alice possiede il primo qubit (prima entrata del ket) e Bob il secondo. Cosa deve fare Alice per inviare solamente un singolo "pezzo" di informazione? Ad esempio Alice può effettuare una qualche operazione sul suo qubit e, sfruttando l'entanglement, Bob sarà in grado di leggere l'informazione desiderata (la coppia di numeri $xy$ che Alice vuole inviare) facendo una singola misurazione. Più precisamente, supponiamo che Alice voglia inviare delle informazioni effettuando le seguenti operazioni sul proprio qubit di $\ket{\psi}$:
\begin{equation*}
    \text{Alice invia:} \; 
    \begin{cases}
        00 \, , &\text{non fa niente} \\
        10 \, , &\text{applica } Z \\
        01 \, , &\text{applica } X \\
        11 \, , &\text{applica } ZX
    \end{cases} \, .
\end{equation*}
Cosa succede allo stato condiviso quando applica queste operazioni? 
\begin{itemize}
    \item Quando vuole inviare 00 non effettua alcuna operazione quindi $\ket{\psi} \rightarrow \ket{\psi} = \ket{\beta_{00}}$. 
    
    \item Nel caso in cui decide di inviare 10 applica $Z$:
    \begin{equation*}
        \ket{\psi} \rightarrow Z \frac{1}{\sqrt{2}} \left( \ket{00} + \ket{11} \right) = \frac{1}{\sqrt{2}} \left( \ket{00} - \ket{11} \right) = \ket{\beta_{10}} \, .
    \end{equation*}
    
    \item Quando invece vuole inviare 01 applica $X$:
    \begin{equation*}
        \ket{\psi} \rightarrow X \frac{1}{\sqrt{2}} \left( \ket{00} + \ket{11} \right) = \frac{1}{\sqrt{2}} \left( \ket{10} + \ket{01} \right) = \ket{\beta_{01}} \, .
    \end{equation*}
    
    \item Infine se vuole inviare 11 applica $ZX$:
    \begin{equation*}
        \ket{\psi} \rightarrow ZX \frac{1}{\sqrt{2}} \left( \ket{00} + \ket{11} \right) = Z \frac{1}{\sqrt{2}} \left( \ket{10} + \ket{01} \right) = \frac{1}{\sqrt{2}} \left( \ket{01} - \ket{10} \right) = \ket{\beta_{11}} \, .
    \end{equation*}
\end{itemize}

\noindent Effettuando queste operazioni, Alice è in grado di spedire ciò che vuole: $xy \rightarrow \ket{\beta_{xy}}$. Bob può effettuare una (singola) misura nella base di Bell, stabilire quale dei 4 stati possiede, %Qui risiede l'idea di \textbf{non-località} della QM: sebbene Bob possa trovarsi molto lontano da Alice, il suo qubit è cambiato a seguito delle operazioni di lei e può 
e leggere quindi i bit corrispondenti $xy$. L'informazione classica corrispondente a due bit \`e stata inviata attraverso un solo qubit.

\noindent Questo esempio, nonostante sia un po' accademico, risulta particolarmente interessante perché mette in risalto come, grazie all'entanglement, sia possibile ridurre il numero di operazioni necessarie per inviare un'informazione rispetto al caso classico. 

\section{Teleportation}
Innanzitutto che cosa intendiamo con il termine \textit{teletrasporto}? In questo contesto viene inteso con il significato di ricostruire un qubit molto lontano da dove si trovava in origine: il qubit originale sparisce e una sua nuova copia viene creata altrove. L'idea è quella di effettuare questa particolare ricostruzione usando solamente operazioni classiche sui qubit.

\noindent Supponiamo che Alice (d'ora in avanti chiameremo sempre in questo modo i nostri due sperimentatori) abbia un generico qubit $\ket{\psi} = \alpha \ket{0} + \beta \ket{1}$ e che voglia inviarlo a Bob senza utilizzare alcun canale quantistico. Dalle leggi della QM sappiamo che non possiamo estrarre sia $\alpha$ che $\beta$ con una singola misura e inoltre non è possibile clonare questo stato generico. Inoltre, se volesse inviare direttamente questo stato con assoluta precisione (assumiamo $\alpha , \beta \in \mathbb{R}$) mediante un canale classico, allora necessiterebbe due stringhe infinite di bit e quindi del tempo infinito per inviarle. Come nel caso precedente, assumiamo che Alice e Bob condividano lo stato entangled $\ket{\beta_{00}} = \frac{1}{\sqrt{2}} \left( \ket{00} + \ket{11} \right)$, dove il primo qubit è di Alice e il secondo di Bob. Notiamo che Alice possiede due qubit: il qubit generico $\ket{\psi}$ che vuole teletrasportare e il qubit entangled con quello di Bob. Lo stato iniziale non è quindi altro che
\begin{equation}\label{teleportation_initial_state}
    \left( \alpha \ket{0} + \beta \ket{1} \right) \ket{\beta_{00}} = \frac{\alpha}{\sqrt{2}} \ket{0} \left( \ket{00} + \ket{11} \right) + \frac{\beta}{\sqrt{2}} \ket{1} \left( \ket{00} + \ket{11} \right) \, .
\end{equation}
Alice sottopone gli stati in suo possesso al seguente circuito:
\begin{center}
    \mbox{
        \Qcircuit @C=1em @R=1em {
            \lstick{\ket{\psi}} & \ctrl{1} & \gate{H} & \qw & \rstick{\text{Alice}} \\
            \lstick{} & \targ & \qw & \qw & \rstick{\text{Alice}} \\
            \lstick{} & \qw & \qw & \qw & \rstick{\text{Bob}}
            \inputgroupv{2}{3}{1.4em}{1em}{\ket{\beta_{00}}\; \; \; \;}{2em}
        }
    }
\end{center}
dove si è indicato in output a chi appartiene quel determinato qubit. Esplicitamente, si applica \texttt{CNOT-gate} ai due qubit di Alice in \eqref{teleportation_initial_state}, ottenendo
\begin{equation*}
    \frac{\alpha}{\sqrt{2}} \ket{0} \left( \ket{00} + \ket{11} \right) + \frac{\beta}{\sqrt{2}} \ket{1} \left( \ket{10} + \ket{01} \right)  \, ;
\end{equation*}
dopodiché viene applicato \texttt{H-gate} al primo qubit di Alice:
\begin{equation*}
    \frac{\alpha}{2} (\ket{0} + \ket{1}) \left( \ket{00} + \ket{11} \right) + \frac{\beta}{2} (\ket{0} - \ket{1}) \left( \ket{10} + \ket{01} \right) \, ;
\end{equation*}
infine possiamo riscrivere l'espressione nel modo seguente
\begin{equation*}
    \frac{1}{2} \bigg[ \ket{00} (\alpha \ket{0} + \beta \ket{1}) + \ket{01} (\alpha \ket{1} + \beta \ket{0}) + \ket{10} (\alpha \ket{0} - \beta \ket{1}) + \ket{11} (\alpha \ket{1} - \beta \ket{0}) \bigg] \, ,
\end{equation*}
dove in questo ultimo passaggio abbiamo svolto i conti e riordinato l'espressione focalizzandoci su ciò che è posseduto da Alice (i due qubit di fronte alle 4 parentesi tonde) e da Bob (stato nella parentesi tonda). Consideriamo ora la Tabella \ref{tab:teleportation}: Alice può effettuare una misura nella base computazionale e dire a Bob (mediante un canale classico) ciò che ha ottenuto; dopo la misura lo stato collassa e Bob, a seconda del risultato, può effettuare o meno un'opportuna operazione sul proprio stato per ricostruire precisamente ciò che si voleva teletrasportare. 

\begin{table}[!ht]
	\centering
    \begin{tabular}{ccc}
        \toprule
        \text{Alice misura} & \text{Bob trova} & \text{Bob applica}  \\
        \midrule
        $\ket{00}$ & $\alpha \ket{0} + \beta \ket{1}$ & \text{Nulla} \\
        $\ket{01}$ & $\alpha \ket{1} + \beta \ket{0}$ & $X$ \\
        $\ket{10}$ & $\alpha \ket{0} - \beta \ket{1}$ & $Z$ \\
        $\ket{11}$ & $\alpha \ket{1} - \beta \ket{0}$ & $ZX$ \\        \bottomrule
    \end{tabular}\\
    \caption{Una volta che Alice effettua la propria misura nella base computazionale e dice a Bob ciò che ha ottenuto, quest'ultimo può applicare una precisa operazione per ricostruire lo stato $\ket{\psi}$ che Alice voleva teletrasportare. Si noti che, in tutti e quattro i casi, lo stato finale che ha Bob è sempre $\ket{\psi}$ indipendentemente dal risultato di Alice.}
    \label{tab:teleportation}
\end{table}

\noindent Un fatto fondamentale da evidenziare è che solamente informazioni classiche sono state trasferite tra Bob e Alice poiché tutto il resto (misurazioni e operazioni sugli stati) viene svolto localmente nel laboratorio: non c'è né violazione della relatività speciale in quanto non avviene alcun trasferimento di informazioni più veloce della luce, né violazione del teorema di no-cloning perché, una volta che Bob ottiene $\ket{\psi}$, Alice non possiede più lo stato che voleva teletrasportare. Si tratta solamente di un modo ingegnoso per sfruttare l'entanglement. 

\section{Disuguaglianze di Bell}
L'argomento delle disuguaglianze di Bell è un tema molto vasto che comprende moltissime disuguaglianze testabili sperimentalmente: ciò che accomuna tutte le misurazioni è la profonda differenza tra il concetto di probabilità \textit{classica} e \textit{quantistica}. Nella prima metà del '900, dopo la nascita della QM e i conseguenti trionfi che tale teoria era in grado di riportare, molti fisici, tra cui lo stesso Einstein, erano profondamente insoddisfatti del concetto intrinseco ed inevitabile di probabilità che permea tale teoria. In particolare, coloro che non accettavano la QM come teoria completa, credevano che il suo comportamento fosse in realtà dovuto alla nostra ignoranza su teorie ancora più fondamentali. Questo gruppo di persone credevano che le \textbf{osservabili}, in fisica, dovessero sempre soddisfare 2 requisiti base:
\begin{itemize}
    \item \textbf{Realismo}: un'osservabile deve avere un valore definito anche prima che la misura sia effettuata.
    
    \item \textbf{Località}: un esperimento effettuato in un ben preciso luogo ha solamente un effetto locale perché non può in alcun modo modificare risultati e comportamenti di altri esperimenti effettuati in regioni causalmente disconnesse. L'entanglement, ad esempio, è in profondo contrasto con il concetto di località. 
\end{itemize}

\noindent Nel corso di quegli anni furono svolti numerosi tentativi di riscrivere la QM in maniera tale che soddisfacesse i requisiti precedenti. Ad esempio, furono utilizzate le cosiddette \textbf{teorie delle variabili nascoste}. Tali teorie si basano sull'assunto secondo cui quando si misura un valore $a$ di un'osservabile $A$, in realtà il risultato della misurazione è incompleto perché l'intera teoria prevede l'esistenza di un'altra variabile $\lambda$ \textit{nascosta} ed inaccessibile. Se si potesse conoscere $\lambda$ allora si potrebbe predire qualsiasi cosa in maniera del tutto deterministica. Nonostante ciò, il concetto di probabilità in QM vìola queste regole e, come vedremo tra poco, utilizzando le disuguaglianze di Bell è possibile rilevare sperimentalmente tale violazione. 

\begin{esempio}[Singolo qubit]
    Consideriamo nuovamente il caso di un qubit e immaginiamo di trovarci nello stato $\ket{0}$. Nella Sezione \ref{sec:osservabili} abbiamo visto che il più generale operatore hermitiano che agisce su un singolo qubit è dato da una combinazione lineare di matrici di Pauli (non consideriamo l'identità), ossia $\vec{\sigma} \cdot \vec{n}$ dove $\abs{\vec{n}} = 1$. Supponiamo che a seguito di una misurazione possiamo ottenere $\vec{\sigma} \cdot \vec{n} \ket{\vec{n}} = \ket{\vec{n}}$ e $\vec{\sigma} \cdot \vec{n} \ket{-\vec{n}} = - \ket{-\vec{n}}$, quindi il risultato è $\pm 1$. Perciò, ricordando la decomposizione $\ket{0} = c_1 \ket{\vec{n}} + c_2 \ket{-\vec{n}}$, la probabilità di misurare 1 è $P(\vec{\sigma} \cdot \vec{n} = 1) = \abs{c_1}^2 = \abs{\braket{0}{\vec{n}}}^2$. Al tempo stesso sappiamo anche che il generico qubit si scrive come in \eqref{generic qubit}, quindi $\vec{n}$ può essere specificato scegliendo gli angoli $\theta$ e $\phi$: dato che avevamo sottolineato che $\vec{\sigma} \cdot \vec{n} \ket{\psi} = \ket{\psi}$ allora la soluzione che cerchiamo è $\ket{\vec{n}} = \ket{\psi}$, quindi
    \begin{equation*}
        P(\vec{\sigma} \cdot \vec{n} = 1) = \abs{\braket{0}{\vec{n}}}^2 = \abs{\braket{0}{\psi}}^2 = \cos^2 \! \left( \frac{\theta}{2} \right) \, .
    \end{equation*}
    Vediamo se riusciamo a riprodurre la medesima distribuzione di probabilità utilizzando una teoria classica basata sulle variabili nascoste. Supponiamo che, oltre allo spin, la particella sia in realtà descritta da un'extra variabile $\lambda$: tutte le particelle sono specificate dalla coppia fissata $(a = \pm 1, \lambda)$, ossia hanno spin $a = \pm 1$ e un preciso valore di $\lambda$ persino prima di effettuare la misurazione. Per semplicità assumiamo $\lambda \in [0,1]$. Supponiamo di voler effettuare una misurazione dello spin in una particolare direzione $\vec{n}$: la misurazione, in questa teoria, corrisponde a particolari valori di spin e $\lambda$ con l'idea che una misura effettuata con angolo $\theta$ abbia risultato dipendente dal valore assunto da $\lambda$ nell'intervallo $[0,1]$. Più precisamente, consideriamo una teoria con variabili nascoste in cui  il risultato delle misure sia quello indicato nella formula seguente: assumendo di non poter rilevare il valore $\lambda$ e richiedendo che le particelle abbiano dei valori di tale variabile uniformemente distribuiti, un esperimento di questo tipo produce 
    \begin{equation*}
        \begin{cases}
            \text{spin }\ket{\uparrow} \, , & \text{per} \, \, \, 0 \leq \lambda \leq \cos^2 \theta/2 \\
            \text{spin }\ket{\downarrow} \, , &\text{per} \,\,  \cos^2 \theta/2 \leq \lambda \leq 1
        \end{cases} \, , \quad \Rightarrow \quad P(a = 1) = \cos^2 \! \left( \frac{\theta}{2} \right) \, .
    \end{equation*}
\end{esempio}

\noindent Nell'esempio precedente è stato analizzato il caso di un singolo qubit, che \`e riproducibile da una teoria con variabili nascoste. Prendiamo ora in esame il sistema di 2 qubit, dove sappiamo che l'entanglement gioca un ruolo centrale e dove ci sar\`a  differenza tra probabilità \textit{classica} e \textit{quantistica}. Esistono diversi modi per scrivere delle disuguaglianze che testino sperimentalmente questa profonda differenza. Uno dei più famosi è il seguente

\subsection{Disuguaglianza CHSH}
Si tratta di una semplice generalizzazione delle  disuguaglianze scritte originariamente da Bell e utilizzata per le verifiche sperimentali. Ancora una volta, consideriamo i due sperimentatori Alice e Bob situati in città differenti. Supponiamo che entrambi abbiano a disposizione un apparato identico su cui possano effettuare misure e che vengano riforniti (ad esempio da un terzo sperimentatore) di infinite copie costituite da coppie di particelle correlate sulle quali possono compiere dei test, e ciascuno possieda una particella della coppia. Alice misura le osservabili $a,a'$ per la sua particella, mentre Bob misura $b,b'$ per la sua, dove $a,a',b,b' = \pm 1$. Entrambi scelgono di fare misurazioni simultanee di una sola delle due osservabili, scelta ogni volta in maniera casuale.

\noindent In un'ipotetica teoria basata sulle variabili nascoste, dato questo sistema è impossibile stabilire immediatamente quale sia il risultato di una misura poiché ci sarà necessariamente una distribuzione di probabilità classica, che chiamiamo $P(a,a',b,b')$, associata alla nostra ignoranza. Consideriamo ora l'osservabile $ C = (a+a') b + (a-a') b'$; per costruzione sappiamo che
\begin{equation*}
    \begin{cases}
        a+a' = 0 \, , \; a-a' = \pm 2 \, , \quad &\text{se } a \neq a' \\
        a+a' = \pm 2 \, , \; a - a' = 0 \, , \quad &\text{se } a = a'
    \end{cases} \, , 
\end{equation*}
ma questo significa allora che per qualsiasi valore delle 4 osservabili in gioco si ha sempre $C = \pm 2$. Dalla teoria della probabilità classica sappiamo che $\abs{\expval{C}} \leqslant \expval{\abs{C}}$ dato che $\abs{\sum_c c p(c)} \leqslant \sum_c \abs{c} p(c)$. Siccome assumiamo che $a,a',b,b',C$ esistano indipendentemente dalla nostra misura, possiamo applicare questa disuguaglianza: in tutte le possibili configurazioni $\abs{C} = 2$ quindi $\expval{\abs{C}} = 2$ e allora
\begin{equation}\label{CHSH_inequality}
    \abs{\expval{C}} \leqslant 2 \, .
\end{equation}
La disuguaglianza precedente prende il nome di \textbf{disuguaglianza CHSH}\footnote{\textit{Clauser, J., Horne, M., Shimony, A., \& Holt, R. (1969). Proposed Experiment to Test Local Hidden-Variable Theories. Phys. Rev. Lett., 23, 880–884.}}. Notiamo che si tratta di un risultato classico derivante dalla teoria della probabilità. 

\noindent In QM è facile trovare un esempio nel quale questa disuguaglianza è violata. Supponiamo che Alice e Bob condividano lo stato entangled $\ket{\psi} = \frac{1}{\sqrt{2}} \left( \ket{01} - \ket{10} \right)$ e che entrambi decidano di misurare qualcosa che in QM abbia 2 possibili valori. In particolare misurano
\begin{align*}
    a &= \vec{\sigma} \cdot \hat{a} = \pm 1 \, , &a' &= \vec{\sigma} \cdot \hat{a}' = \pm 1 \, , \\
    b &= \vec{\sigma} \cdot \hat{b} = \pm 1 \, , &b' &= \vec{\sigma} \cdot \hat{b}' = \pm 1 \, ,
\end{align*}
dove il simbolo "$\hat{\,}$" indica un vettore di modulo unitario. È possibile dimostrare in QM che
\begin{equation}\label{formula_CHSH}
    \expval{(\vec{\sigma} \cdot \hat{c}) \otimes (\vec{\sigma} \cdot \hat{d})}{\psi} = - \hat{c} \cdot \hat{d} = - \cos \theta \, ,
\end{equation}
dove $\theta$ è l'angolo tra $\hat{c}$ e $\hat{d}$. Supponiamo che Alice e Bob decidano di misurare nelle direzioni indicate dagli angoli di Figura \ref{fig:CHSH}. Utilizzando la \eqref{formula_CHSH} possiamo calcolare il valore di aspettazione di $C$:
\begin{equation*}
    \expval{C} = \expval{ab + a'b + a b' - a'b'}{\psi} = - \left[ \frac{1}{\sqrt{2}} + \frac{1}{\sqrt{2}} + \frac{1}{\sqrt{2}} - \left( -\frac{1}{\sqrt{2}} \right) \right] = -2\sqrt{2} \, .
\end{equation*}
Abbiamo quindi ricavato che, secondo la QM, $\abs{\expval{C}} = 2\sqrt{2}$, in disaccordo\footnote{In realtà esiste un teorema che stabilisce che $2\sqrt{2}$ è il più grande valore che può essere ottenuto.} con il risultato classico in \eqref{CHSH_inequality}: la probabilità \textit{quantistica} è intrinsecamente differente dalla probabilità \textit{classica}! 

\begin{figure}[!t]
    \centering
    \includegraphics[scale=0.45]{images/CHSH}
    \caption{Direzioni spaziali delle 4 osservabili misurate da Alice e Bob. Si noti che l'apparato di uno è ruotato di 45° rispetto a quello dell'altro perciò $\theta_{a'b} = \theta_{ab} = \theta_{ab'} = 45$° e $\theta_{a'b'} = 135$°.}
    \label{fig:CHSH}
\end{figure}
\noindent Altri esperimenti degni di nota sono quelli condotti da Freedman e Clauser nel 1972, la serie di esperimenti condotti da Aspect negli anni 1981 e 1982, da Tittel e il gruppo Geneva nel 1988 e da Weihs sotto condizioni di località "strettamente einsteniane" nel 1998. La serie di esperimenti sulle disuguaglianze di Bell, di crescente sofisticazione, ha ridotto i critici, che mettono in discussione i risultati, a indicare falle in tale esperimenti, alcune delle quali distorcerebbero i risultati sperimentali in favore della meccanica quantistica. Nel 2015 è stato pubblicato il primo esperimento dichiarato totalmente privo di falle (loopholes-free), che ha confermato i risultati degli esperimenti precedenti.
    %%%%%%%%%%%%%
% LECTURE 5 %
%%%%%%%%%%%%%

\chapter{Algoritmi quantistici}

\vspace{1cm}

\lecture{5}{18/10/2021}

\vspace{0.5cm}

\noindent Prima di cominciare la vera discussione riguardante gli algoritmi più importanti e conosciuti della computazione quantistica, affrontiamo l'analisi della crittografia quantistica, la quale mostra ancora una volta la potenzialità dei metodi quantistici rispetto a quelli classici. 

\section{Crittografia quantistica}
Molti anni prima che fu introdotta la crittografia RSA che utilizziamo oggigiorno, molte persone pensavano che gli stati quantistici e il "bizzarro" comportamento della QM potessero essere utilizzati per scopi crittografici. Il nome del protocollo quantistico che fu pensato per trasmettere dati criptati è \textbf{protocollo BB84}. Consideriamo, come al solito, Alice e Bob in differenti città e supponiamo che vogliano comunicare tra loro tramite una linea criptata. Vediamo come viene affrontato questo problemi sia dal punto di vista classico che quantistico. 

\subsection{Esempio di crittorafia classica}
Classicamente entrambi possiedono una sequenza $S$ di bit casuali, chiamata \textbf{codepad}. Immaginiamo che Alice voglia inviare a Bob un messaggio $M$: un modo classico di inviare il messaggio criptato è quello di inviare la sequenza $M \oplus S$, dove il simbolo "$\oplus$" indica l'\textbf{addizione bit a bit modulo 2}. Ad esempio supponiamo che il codepad sia $S = 0110$ e il messaggio sia $M = 1111$: la sequenza $M \oplus S$ è data da 
\begin{table}[!ht]
	\centering
    \begin{tabular}{c|cccc|c}
        \toprule
        $S$ & 0 & 1 & 1 & 0 & $= 6$ \\
        $M$ & 1 & 1 & 1 & 1 & $=15$ \\
        \midrule
        $M \oplus S$ & 1 & 0 & 0 & 1 & $=9$ \\
        \bottomrule
    \end{tabular}
\end{table}

\noindent dove nell'ultima colonna abbiamo inserito il numero associato a quel messaggio (ad esempio, leggendo da destra verso sinistra, per $S$ si ha $0 \times 2^0 + 1 \times 2^1 + 1 \times 2^2 + 0 \times 2^3 = 6$). Il vantaggio di questo modo di crittografare messaggi risiede nel fatto che anche se si parte con una stringa $M$ sensata, l'operazione $M \oplus S$ la trasforma in una sequenza apparentemente casuale di 0 e 1 che può essere decifrata solamente se si possiede il codepad. Infatti, una volta che Bob riceve $M \oplus S$, si può facilmente ricostruire il messaggio originale calcolando $(S \oplus M) \oplus S = M \oplus (S \oplus S) = M$ dato che 
\begin{equation*}
    x \oplus x =
    \begin{cases}
        0 \oplus 0 = 0 \\
        1 \oplus 1 = 0 
    \end{cases} \, , \quad \forall \, x \, .
\end{equation*}
Il problema di un tale protocollo crittografico, oltre al fatto che il codepad non debba essere scoperto da nessun altro al di fuori di Alice e Bob, risiede nel fatto che non sia molto efficiente a seguito del cosiddetto "one-time codepad", dato che la stringa $S$ può essere utilizzata una volta sola. Per capirne il motivo supponiamo che Alice voglia inviare entrambi i messaggi $M_1$ e $M_2$: il messaggio ricevuto da Bob è $(M_1 \oplus S) \oplus (M_2 \oplus S) = M_1 \oplus M_2$, il quale è costituito da una sequenza di 0 e 1 abbastanza randomica. Se Bob è abbastanza abile e conosce almeno una parte del messaggio che Alice voleva inviare allora ci sono possibilità che riesca a decifrare $M_1$ e $M_2$ separatamente riconoscendo degli opportuni schemi in $M_1 \oplus M_2$; tuttavia non è detto che chi riceva il messaggio sia sempre così abile !


\subsection{Il protocollo BB84}
La versione quantistica viene chiamata \textbf{protocollo BB84} ed è abbastanza simile al caso classico ma molto più potente: lo scopo è quello di creare un codepad $S$ che non possa essere in alcun modo (o quasi\footnote{Si sfrutta la natura probabilistica della QM quindi se  i qubit inviati da Alice sono in gran numero, è solo una questione di tempo prima che una terza persona venga scoperta intercettare i messaggi. Si veda la discussione di seguito per chiarimenti più espliciti.}) intercettato da una terza persona, che chiameremo Eve, la quale vuole rovinare i piani di Alice e Bob. Il protocollo funziona come segue: Alice possiede una serie di qubit che vorrebbe inviare a Bob tramite un canale sicuro; invia allora casualmente dei qubit che sono preparati nella base computazionale $C = \{ \ket{0}, \ket{1} \}$ oppure nella base di Hadamard $H = \{ \ket{+}, \ket{-} \}$ (si vedano le \eqref{basi_di_sigma_12}). Quindi Alice, prima di inviare i qubit, effettua due scelte: sceglie la base e poi sceglie uno stato di quella base da inviare. Nel frattempo, Bob riceve i qubit inviati e tiene attentamente conto dell'ordine di ricezione di questi qubit, dopodiché effettua una misurazione scegliendo randomicamente la base $C$ oppure\footnote{Ad esempio, se Alice invia delle particelle dotate di spin, Bob può scegliere, mediante un apparato simile a quello dell'esperimento di Stern e Gerlach, di misurare lo spin lungo la direzione $z$ (base $C$) oppure lungo la direzione $x$ (base $H$).} o la base $H$. Dato che Bob sceglie una delle due basi, per ogni qubit che riceve ci sono due possibilità:
\begin{itemize}
    \item Sceglie la stessa base di Alice. Ad esempio se Alice avesse scelto $(C, \ket{0})$ allora necessariamente, dai postulati della QM, sappiamo che Bob misura obbligatoriamente $(C, \ket{0})$ con probabilità 1 (nel caso in cui nessuno abbia intercettato il messaggio).
    
    \item Sceglie una base differente da quella di Alice. Ad esempio se Alice invia $(C, \ket{0})$ e Bob sceglie la base $H$ allora sappiamo che ha il 50\% di possibilità di trovare $\ket{0}$ nella propria misurazione. 
\end{itemize}

\noindent Notiamo che i risultati ottenuti da Bob nelle proprie misurazioni non sono in alcun modo correlati con le informazioni che Alice vuole inviare. Riassumendo, gli step necessari sono i seguenti:
\begin{enumerate}
    \item Alice sceglie una base e invia casualmente i qubit.
    
    \item Bob riceve i qubit tenendo conto dell'ordine di arrivo e misura con il proprio apparato scegliendo casualmente una delle due basi. 
    
    \item Alice e Bob comparano \textbf{solamente le basi} di un numero arbitrario di qubit concordato a priori dai due (alcuni e non tutti perché in generale Alice potrebbe inviare un numero altissimo di qubit) tramite una linea non sicura (ossia che può essere intercettata da Eve). Questo significa che per ogni qubit che confontano, i due si scambiano la base in cui è stata effettuata la misura, non lo stato misurato.
    
    \item Infine Bob, usando parte dei qubit inviati da Alice, ossia solamente quelli che ha misurato nella sua stessa base, può costruire un codepad comune e stabilire se qualcuno ha intercettato i qubit inviati (si veda la discussione dopo l'esempio \ref{BB84_example}) confrontando direttamente i qubit delle misure con la stessa base. 
\end{enumerate}

\noindent Per capire al meglio il funzionamento di questo meccanismo illustriamolo con un esempio.

\begin{esempio}\label{BB84_example}
    Immaginiamo che le basi scelte e i risultati delle misurazioni effettuate da Alice e Bob siano quelli mostrati nella Tabella \ref{tab:BB84}. 
    
\begin{table}[!ht]
	\centering
    \begin{tabular}{c | c c c >{\columncolor[gray]{0.8}} c >{\columncolor[gray]{0.8}} c c >{\columncolor[gray]{0.8}} c >{\columncolor[gray]{0.8}} c c c >{\columncolor[gray]{0.8}} c}
        \toprule
        \text{Alice} & \text{base} & $\rightarrow$ & $C$ & $H$ & $H$ & $C$ & $C$ & $H$ & $C$ & $H$ & $C$ \\
        \text{} & \text{qubit} & $\rightarrow$ & 0 & 1 & 0 & 0 & 0 & 0 & 1 & 0 & 1 \\
        \midrule
        \text{Bob} & \text{base} & $\rightarrow$ & $H$ & $H$ & $H$ & $H$ & $C$ & $H$ & $H$ & $C$ & $C$ \\
        \text{} & \text{qubit} & $\rightarrow$ & 1 & 1 & 0 & 0 & 0 & 0 & 1 & 1 & 1 \\
        \bottomrule
    \end{tabular}\\
    \caption{Basi scelte e rispettive misurazioni effettuate da Alice e Bob. Si noti che nelle righe dei qubit misurati si è indicato solamente il bit di informazione inviato da Alice o ottenuto da Bob, ossia: $\ket{0}, \, \ket{+} \rightarrow 0$ e $\ket{1}, \ket{-} \rightarrow 1$. Nella tabella sono state colorate in grigio le colonne corrispondenti alle misurazioni effettuate nella medesima base.}
    \label{tab:BB84}
\end{table}

\noindent Una volta che Alice ha terminato\footnote{In generale esistono varie versioni di questa procedura perché dal punto di vista pratico è molto difficile accumulare una sequenza di qubit mantenendoli tutti inalterati. Una versione alternativa più conveniente e realistica prevede Alice che invia il suo qubit e subito dopo comunica immediatamente la base che ha scelto, in maniera tale che una volta che Bob abbia ricevuto il qubit possa scartare quelli misurati in basi differenti.} la sequenza di qubit che voleva inviare, comunica a Bob, mediante un canale classico, la sequenza di basi scelte, ossia la prima riga della tabella: dalle regole della QM sappiamo che ogniqualvolta che Bob sceglie (per coincidenza) la medesima base di Alice, il risultato della misurazione che ottiene è obbligatoriamente il medesimo qubit che Alice ha scelto di inviare (a tal proposito si vedano infatti le colonne colorate). Notiamo che per una coincidenza fortuita le misurazioni nelle colonne 4 e 7 sono le medesime sebbene la base scelta fosse differente: in questa situazione, ossia quando i due sperimentatori scelgono una base diversa, Bob ottiene casualmente 0 o 1 con probabilità $1/2$. Una volta effettuata la chiamata, i due decidono di tenere solamente i risultati in cui hanno scelto le stesse basi e formano con tali misure un codepad comune: nella situazione della Tabella \ref{tab:BB84}, solo le colonne colorate hanno la stessa base, quindi il codepad non è altro che $S = 10001$. Ovviamente questo codepad è comune perchè Alice e Bob si sono scambiati, oltre alle basi, anche i qubit delle misure effettuate nella stessa base.
\end{esempio}

\noindent Per quale ragione il codepad comune formato da Alice e Bob è più protetto di quello classico ? Come interviene Eve nella trasmissione delle informazioni per capire ciò che è stato inviato ? Eve può semplicemente intercettare il qubit durante il transito: dalla QM sappiamo che è obbligata ad effettuare una misurazione, la quale disturba inevitabilmente il sistema. Dato che Eve non conosce la base in cui il qubit è stato preparato è costretta a fare una scelta ! Nel caso in cui sia fortunata, scegliendo cioè la stessa base di Alice, Eve vede il qubit inviato senza modificare lo stato, tuttavia quando sceglie la base opposta ottiene un numero casuale 0 o 1 con probabilità $1/2$ e causa il collasso dello in uno dei due stati della base utilizzata. 

\noindent Capiamo meglio questo discorso con un esempio.

\begin{esempio}
    Supponiamo che Alice abbia scelto di inviare $(C, \ket{0})$ e che Eve scelga di misurare nella base $H$ ottenendo $\ket{+}$: lo stato è ora collassato in $\ket{+}$, quindi se Bob effettua una misurazione in $C$, egli può ottenere sia $\ket{0}$ sia $\ket{1}$ con probabilità $1/2$, nonostante Alice avesse inviato $(C, \ket{0})$. Se dovesse succedere che Bob misuri $(C,\ket{1})$, allora Alice e Bob concludono che hanno misurato due stati differenti, nonostante abbiano scelto la medesima base, ma questo è impossibile dalla QM se nessuno è intervenuto sullo stato !
\end{esempio}

\noindent A seguito del collasso dello stato in uno stato di una base differente da quella scelta da Alice, può accadere che nelle colonne colorate della Tabella \ref{tab:BB84} (misurazioni con stesse basi) i due sperimentatori ottengano uno stato differente: se nessuno sta intercettando gli stati in transito questo è impossibile per le leggi della QM ! In questo modo, una volta comunicati i qubit misurati nelle stesse basi, Bob capisce che qualcuno ha interferito con i qubit che Alice sta inviando. 

\noindent Statisticamente, quante volte Eve sta ascoltando sistematicamente il messaggio e vi è una possibilità che Alice e Bob non concordino su una misura effettuata nella stessa base ? Tipicamente per $1/4$ delle volte. Il motivo è dato dal fatto che Eve può essere fortunata e misurare nella stessa base di Alice (probabilità $1/2$ per questa scelta) e inoltre anche se Eve sceglie la base sbagliata, Bob deve effettuare una misurazione in cui ottiene lo stesso qubit di Alice il 50\% delle volte: quindi $\frac{1}{2} \times \frac{1}{2} = \frac{1}{4}$, dove il primo $1/2$ deriva dalla scelta di Eve e il secondo dalla misura di Bob.  

\subsection{Quantum non-demolition measures}
Chiaramente ci si potrebbe domandare se Eve possa fare di meglio. Esiste una possibilità in cui possa misurare senza recare alcun disturbo allo stato ? Delle volte queste misure vengono chiamate in letteratura \textbf{quantum non-demolition measures}: si trattano di particolari misure in cui Eve effettua la misurazione senza disturbare lo stato oppure disturba lo stato ma è in grado di resettarlo all'originale inviato da Alice. La risposta alla domanda precedente è no per un motivo simile alla dimostrazione del teorema di No-cloning.

\noindent Supponiamo che Alice stia inviando l'insieme di stati $\ket{\phi_\mu} = \{ \ket{0}, \ket{1}, \ket{+}, \ket{-} \}$, dove $\mu = 0, 1,2,3$, e inoltre assumiamo che Eve possieda un proprio computer quantistico sul quale può effettuare operazioni. Nell'intercettare il messaggio, Eve osserva lo stato $\ket{\phi_\mu} \otimes \ket{\phi}$, dove $\ket{\phi}$ si trova nel suo computer. Supponiamo inoltre che nel suo computer ci sia un altro insieme di stati $\ket{\psi_\mu}$, con $\mu = 0,1,2,3$, tale che possa essere distinto da una misura effettuata da Eve stessa. La domanda é: esiste qualche sorta di processo quantistico (gate unitario $U$) che agisce come
\begin{equation}\label{U_non_demolition_measure}
    U \left( \ket{\phi_\mu} \otimes \ket{\phi} \right) = \ket{\phi_\mu} \otimes \ket{\psi_\mu} \, ,
\end{equation}
ossia tale che quando Eve misura $\ket{\psi_\mu}$ e legge il valore $\mu$ allora con probabilità 1 legge anche lo stesso $\mu$ che Alice sta inviando, senza però disturbare $\ket{\phi_\mu}$? La risposta è no, similmente al teorema di No-cloning. Per dimostrare questo fatto calcoliamo il prodotto scalare di ambo i membri della \eqref{U_non_demolition_measure}, il quale, come sappiamo a seguito dell'unitarietà di $U$, deve rimanere preservato:
\begin{align*}
    \left( \bra{\phi_\mu} \otimes \bra{\phi} \right) \left( \ket{\phi_\nu} \otimes \ket{\phi} \right) &\overset{?}{=} \left( \bra{\phi_\mu} \otimes \bra{\psi_\mu} \right) \left( \ket{\phi_\nu} \otimes \ket{\psi_\mu} \right) \, , \; \text{ con } \mu \neq \nu \, \text{ in generale.} \\
    \Rightarrow \qquad \braket{\phi_\mu}{\phi_\nu} \underbrace{\braket{\phi}}_1 &\overset{?}{=} \braket{\phi_\mu}{\phi_\nu} \braket{\psi_\mu}{\psi_\nu} \, , \; \forall \text{ paia di indici } (\mu,\nu) \, .
\end{align*}
Analizziamo i casi in cui $\mu \neq \nu$. Quando $(\mu = 2, \nu = 3)$ e $(\mu = 0, \nu = 1)$ (o viceversa) si ha l'identità $0 = 0$, che non è interessante (ricordare sopra gli stati $\ket{\phi_\mu}$ di Alice). Nei casi invece $(\mu = 0, \nu = 2)$, $(\mu = 0, \nu = 3)$, $(\mu = 1, \nu = 2)$ oppure $(\mu = 1, \nu = 3)$ (o viceversa) i prodotti scalari $\braket{\phi_\mu}{\phi_\nu}$ non sono nulli e possono essere semplificati ad entrambi i membri. Per queste scelte otteniamo quindi che $\braket{\psi_\mu}{\psi_\nu} = 1$, ossia $\ket{\psi_\mu} = \ket{\psi_\nu}$ a meno di una fase. Ma questo significa allora che $\ket{\psi_0} = \ket{\psi_1} = \ket{\psi_2} = \ket{\psi_3}$ e quindi, non essendo stati differenti, Eve non può in alcun modo distinguere ciò che ha inviato Alice. 

\noindent La conclusione è che non esiste alcun modo di effettuare una misura con operazioni unitarie che distingua il qubit inviato da Alice senza necessariamente disturbare il sistema. 

\section{Proprietà dei gate}
Nella sezione \ref{sec:gate} abbiamo introdotto alcuni concetti preliminari riguardanti i gate, i circuiti e i computer quantistici. In molti casi nei computer si hanno degli algoritmi, ossia una ben precisa sequenza di istruzioni, che permettono di calcolare risultati desiderati. Approfondiamo le analogie e differenze dei gate classici e quantistici.

\subsection{Gate classici: il \texttt{TOFFOLI-gate}}
In CC si hanno i bit 0 e 1 e le funzioni classiche sono tali che $f: \; \{ 0,1 \}^{\otimes n} \rightarrow \{ 0,1 \}^{\otimes m}$, sono cioè mappe da $n$ a $m$ bit classici. Quindi in generale abbiamo
\begin{equation*}
    f_i(x_1, x_2, \ldots, x_n) = \{ 0, 1 \} \, , \; \text{ dove } i = 1, \ldots , n \, , \; \text{ e } x_i = 0, 1 \, .
\end{equation*}
Si vorrebbe che il computer sia in grado di calcolare tali funzioni e che inoltre gli strumenti a disposizione siano sufficientemente efficienti per farlo: quello che uno vorrebbe è poter calcolare funzioni generali con l'ausilio di solamente pochi gate. In CC si ha che con le seguenti operazioni è possibile calcolare quasi tutti i conti di algebra e aritmetica: \texttt{NOT}, $a \rightarrow -a$; \texttt{AND}, indicato con $a \land b$ e \texttt{OR}, indicato con $a \lor b$. Questo insieme di operazioni è detto \textbf{universale} perché utilizzando questi pochi gate è possibile calcolare tutte le operazioni di aritmetica di interesse. 

\noindent Sempre nella sezione \ref{sec:gate} abbiamo osservato che il CC \textbf{non} é \textbf{reversibile}\footnote{Si pensi ad esempio al fatto che le operazioni non reversibili dissipano calore all'interno della macchina. Si tratta di tutte quelle situazioni in cui si parte con molta informazione e si giunge alla fine ad un singolo risultato, creando nel frattempo numerosi risultati di scarto.} (in generale). Si pensi ad esempio all'\texttt{AND-gate}. Nel corso degli anni si è studiato numerosi metodi per implementare operazioni reversibili: questo è possibile mediante il cosiddetto \textbf{Toffoli gate} o \texttt{cont-cont-NOT}. Il circuito classico è
\begin{center}
    \mbox
    {
        \Qcircuit @C=2em @R=1.35em 
        {
            \lstick{x} & \ctrl{1} & \rstick{x} \qw \\
            \lstick{y} & \ctrl{1} & \rstick{y} \qw \\
            \lstick{z} & \targ & \rstick{z \oplus xy} \qw 
        }
    }
\end{center}
Quando uno dei due tra $x$ e $y$ é 0, allora $xy$ è 0 e niente succede all'output di $z$. L'output si modifica solamente quando sono $1$ perché controllano entrambi il risultato di $z$: quando $xy = 1$ allora $z \oplus 1$ inverte il valore iniziale di $z$. 

\begin{esempio}[Azione Toffoli gate]
    In pratica il \texttt{TOFFOLI-gate} agisce come mostrato in Tabella \ref{tab:Toffoli}:
    \begin{table}[!ht]
	    \centering
        \begin{tabular}{ccc|ccc}
            \toprule
            $\qquad$ & \text{Bit iniziali} & $\qquad \quad$ & $\qquad \quad$ & \text{Bit finali} & \text{} \\
            \midrule
            $x$ & $y$ & $z$ & $x$ & $y$ & $z \oplus xy$ \\
            \midrule
            0 & 0 & 0 & 0 & 0 & 0 \\
            0 & 0 & 1 & 0 & 0 & 1 \\
            0 & 1 & 0 & 0 & 1 & 0 \\
            0 & 1 & 1 & 0 & 1 & 1 \\
            1 & 0 & 0 & 1 & 0 & 0 \\
            1 & 0 & 1 & 1 & 0 & 1 \\
            1 & 1 & 0 & 1 & 1 & 1 \\
            1 & 1 & 1 & 1 & 1 & 0 \\
            \bottomrule
        \end{tabular}\\
        \caption{Azione del \texttt{TOFFOLI-gate} su tutti i possibili bit.}
        \label{tab:Toffoli}
    \end{table}
\end{esempio}

\noindent E' possibile dimostrare che mediante l'uso del \texttt{TOFFOLI-gate} si possono realizzare tutte le operazioni base, quindi è universale e reversibile. Notiamo che è reversibile poiché, come evidenziato nelle ultime due righe della Tabella \ref{tab:Toffoli}, esso agisce sulle stringhe facendo una \textbf{permutazione}, la quale è invertibile. 


\subsection{Gate quantistici: reversibili e continui}
Consideriamo ora il caso dei gate quantistici. Per definizione, essendo implementati da operatori unitari, sono sempre dei gate \textbf{reversibili}. Questo significa ad esempio che
\begin{center}
    \mbox
    {
        \Qcircuit @C=2em @R=1.35em 
        {
            \lstick{\ket{\psi}} & \gate{U} & \gate{U^\dag} & \rstick{\ket{\psi}} \qw
        }
    }
\end{center}
dato che $U U^\dag = \mathbb{I}$. E' possibile implementare il \texttt{TOFFOLI-gate} anche in un computer quantistico ? La risposta è sì: consideriamo una base di stati per 3 qubit 
\begin{equation*}
    \{ \ket{000}, \ket{001}, \ket{010}, \ket{100}, \ket{101}, \ket{011}, \ket{110},  \ket{111} \} \, .
\end{equation*}
La matrice unitaria $U_T$ $8 \times 8$ che agisce sul vettore contenenti gli stati della base è
\begin{equation*}
    \begin{pmatrix}
        1 & 0 & 0 & 0 & 0 & 0 & 0 & 0 \\
        0 & 1 & 0 & 0 & 0 & 0 & 0 & 0 \\
        0 & 0 & 1 & 0 & 0 & 0 & 0 & 0 \\
        0 & 0 & 0 & 1 & 0 & 0 & 0 & 0 \\
        0 & 0 & 0 & 0 & 1 & 0 & 0 & 0 \\
        0 & 0 & 0 & 0 & 0 & 1 & 0 & 0 \\
        0 & 0 & 0 & 0 & 0 & 0 & 0 & 1 \\
        0 & 0 & 0 & 0 & 0 & 0 & 1 & 0
    \end{pmatrix}
    \begin{pmatrix}
        \ket{000} \\ \ket{001} \\ \ket{010} \\ \ket{100} \\ \ket{101} \\ \ket{011} \\ \ket{110} \\ \ket{111}
    \end{pmatrix}
    =
    \begin{pmatrix}
        \ket{000} \\ \ket{001} \\ \ket{010} \\ \ket{100} \\ \ket{101} \\ \ket{011} \\ \ket{111} \\ \ket{110}
    \end{pmatrix} \, .
\end{equation*}
Notiamo infatti che $U_T$ è unitaria ($U_T U_T^\dag = \mathbb{I}$). Tramite $U_T$ possiamo realizzare su un computer quantistico le stesse operazioni che faremmo su un computer classico. Fino ad ora non abbiamo ancora detto se effettivamente queste operazioni possano essere eseguite in maniera più efficiente su un QC. 

\noindent Il secondo fatto importante dei gate quantistici è che sono \textbf{continui}: matrici unitarie possono dipendere da parametri reali continui. Ad esempio, per un sistema di 1 qubit abbiamo visto nella \eqref{general_2by2_matrix} come si scrive la più generale matrice $2 \times 2$ unitaria (gruppo $U(2)$) tramite l'implementazione delle rotazioni di angolo $\lambda$ sulla sfera di Bloch e lungo la direzione generica $\vec{n}$ (si veda la \eqref{rotation_n_lambda}). Abbiamo visto che la \eqref{general_2by2_matrix} dipende in generale da 4 parametri reali arbitrari, quindi persino per un singolo qubit si ha un insieme continuo di gate ! Il problema è che le cose si complicano notevolmente se si passa ad un sistema generico di $n$-qubit. In tale situazione lo spazio di Hilbert associato ha dimensione $2^n$, quindi le matrici unitarie che agiscono su tale spazio sono $2^n \times 2^n$, le quali formano il gruppo $U(2^n)$. Ognuna di queste matrici contiene in generale $2^{2n}$ parametri reali ! Il problema è quindi dato dal fatto che il numero di parametri cresce esponenzialmente con il numero di qubit: il numero di gate è estremamente grande e non vogliamo un QC in cui possiamo implementare qualsiasi trasformazione con $2^{2n}$ parametri reali. Una tale situazione è troppo difficile da realizzare, tuttavia preferiamo considerare un numero limitato di gate e costruire le operazioni desiderate tramite composizione. Si noti inoltre che un insieme continuo di operazioni é impossibile per la memoria limitata di un computer. 

\noindent Il meglio che possiamo fare è introdurre una nozione di \textbf{universalità} e approssimare abbastanza bene una generica trasformazione unitaria utilizzando solamente un insieme finito di gates. Qual è il significato di tale approssimazione ? Dobbiamo definire un'opportuna nozione di \textbf{distanza} tra matrici:

\begin{definizione}[\textbf{Distanza tra matrici}]
    Date due matrici $U$ e $V$, definiamo la seguente funzione \textbf{distanza}
    \begin{equation*}
        E(U,V) = \max\limits_{\ket{\psi}} \norm{(U-V) \ket{\psi}} \, ,
    \end{equation*}
    dove $\ket{\psi}$ è un vettore arbitrario.
\end{definizione}

\noindent E' possibile trovare un insieme discreto di matrici tale che tutte le possibili matrici unitarie possano essere realizzate a partire da tale insieme a meno di un errore $\varepsilon$ arbitrario ? La risposta è sì: un possibile insieme di gate che soddisfa la precedente nozione di "approssimazione universale" è dato dai seguenti 4
\begin{center}
    \mbox{
        \Qcircuit @C=2em @R=2em {
            & \gate{H} & \qw \\
        }
    } 
    , \ \ \ \ 
    \mbox{
        \Qcircuit @C=2em @R=2em {
            & \gate{S} & \qw \\
        }
    }
    , \ \ \ \ 
    \mbox{
        \Qcircuit @C=2em @R=2em {
            & \gate{T} & \qw \\
        }
    }
    , \ \ \ \
    \mbox{
        \Qcircuit @C=2em @R=1em {
            & \ctrl{1} & \qw \\
            & \targ & \qw
        }
    }
\end{center}
Si vedano esplicitamente le matrici \eqref{Hadamard_matrix} e \eqref{S_T_matrices}. Questi gate prendono il nome di \texttt{H-gate} $H$ (Hadamard gate), \texttt{Phase-gate} $S$, \texttt{$\pi/4$-gate} $T$ e il \texttt{CNOT-gate}. Si noti che il nome della matrice $T$ deriva dal fatto che
\begin{equation*}
    T = e^{i \frac{\pi}{8}} 
    \begin{pmatrix}
        e^{-i \frac{\pi}{8}} & 0 \\ 0 & e^{i \frac{\pi}{8}}
    \end{pmatrix} \, .
\end{equation*}
Non lo dimostriamo esplicitamente, ma l'insieme di questi 4 gate è universale. E' importante sottolineare che $H, S$ e $T$ sono gate agenti sui singoli qubit, mentre il \texttt{CNOT-gate} agisce sempre su almeno 2 qubit. Dato che avevamo visto che $T^2 = S$ potrebbe sorgere spontanea la domanda: perché è necessario considerare entrambi $T$ e $S$ ? Di solito si preferisce tenere anche $S$ per la cosiddetta \textbf{Fault tolerance computation}, che approfondiremo quando parleremo di propagazione degli errori nei circuiti quantistici. 

\noindent Alcune importanti proprietà che ci servirà sapere sui gate quantistici sono le seguenti:
\begin{enumerate}
    \item \textit{Tutti i gate agenti su $n$ qubit possono essere approssimati come un prodotto di un opportuno numero di gate agenti su $2$ qubit}. 
    
    \noindent Chiaramente il numero di gate agenti su 2 qubit deve essere sufficientemente grande per approssimare una matrice $2^n \times 2^n$ (matrice agente su $n$ qubit). Per capire il significato di questa affermazione si pensi al circuito seguente di 6 qubit:
    \vspace{-1.2cm}
    \begin{center}
        \mbox{
            \Qcircuit @C=2em @R=0.12em {
                & \multigate{5}{U} & \qw \\
                & \ghost{U} & \qw \\
                & \ghost{U} & \qw \\
                & \ghost{U} & \qw \\
                & \ghost{U} & \qw \\
                & \ghost{U} & \qw \\
            }
            $
            \quad
            \begin{matrix}
                \\
                \\
                \\
                \\
                \simeq \\
            \end{matrix}
            \quad
            $
            \Qcircuit @C=2em @R=0.12em {
                & \multigate{1}{U_1} & \qw \\
                & \ghost{U_1} & \qw \\
                & \multigate{1}{U_2} & \qw \\
                & \ghost{U_2} & \qw \\
                & \multigate{1}{U_3} & \qw \\
                & \ghost{U_3} & \qw \\
            }
        }
    \end{center}
    In questo esempio il gate originale $U$ è stato approssimato fattorizzandolo in 3 gate agenti ciascuno localmente solo su 2 qubit: il numero di operazioni per ricostruire la matrice $2^n \times 2^n$ del circuito a LHS è di ordine $\order{2^{2n}}$. Chiaramente si tratta solamente di algebra: questo non è un modo molto efficiente di approssimare un gate agente su $n$ qubit perché tipicamente si hanno comunque $2^{2n}$ fattori da tenere in considerazione. 
    
    \item \textit{I gate agenti su 2 qubit possono essere scritti in termini di un \texttt{CNOT-gate} e di un gate agente su un singolo qubit}.
    
    \noindent Notiamo che, a differenza della proprietà precedente, questa proprietà è esatta e può essere svolta senza alcuna approssimazione. In termini di circuiti stiamo dicendo che 
    \begin{center}
        \mbox{
            \Qcircuit @C=2em @R=0.12em {
                & \multigate{1}{U_4} & \qw \\
                & \ghost{U_4} & \qw \\
            }
            $
            \quad
            \begin{matrix}
                \\
                = \\
            \end{matrix}
            \quad
            $
            \Qcircuit @C=2em @R=0.5em {
                & \ctrl{1} & \qw \\
                & \targ & \qw \\
            }
            $
            \quad
            \begin{matrix}
                \\
                + \\
            \end{matrix}
            \quad
            $
            \Qcircuit @C=2em @R=0.12em {
                & \gate{U_2} & \qw \\
            }
        }
    \end{center}
    dove $U_4 \in U(4)$ e $U_2 \in U(2)$. In generale è possibile dimostrare che per costruire una generica matrice di $U(4)$ si possono utilizzare due opportune matrici $A, B \in U(2)$ tali che $\comm{A}{B} \neq 0$. 
    
    \item \textit{I gate agenti sui singoli qubit possono essere approssimati come prodotto di matrici $H$ e $T$ con un errore, il quale può essere arbitrariamente scelto più piccolo di $\varepsilon$}.
    
    Questo significa che data $V$ la generica matrice unitaria $2 \times 2$ da approssimare (ricordare che contiene $2^2 = 4$ parametri reali) possiamo scrivere un'opportuna sequenza di prodotti tra $H$ e $T$ tali che
    \begin{equation*}
        E(V, \ldots H H T H \ldots T \ldots H \ldots) < \varepsilon \, .
    \end{equation*}
    Chiaramente più lunga è la sequenza più piccolo sarà l'errore entro il quale si può approssimare $V$. In realtà $H$ e $T$ non sono matrici speciali: questo argomento funziona con qualsiasi $A, B \in U(2)$ tali che $\comm{A}{B} \neq 0$. Matematicamente questa proprietà è dovuta al fatto che il sottogruppo dato dai prodotti di $H$ e $T$ è \textbf{denso} in $U(2)$. 
    
    \noindent Ci si può chiedere se un'approssimazione mediante prodotti di $H$ e $T$ possa essere efficiente. Ancora una volta, fortunatamente la risposta è sì: esiste un teorema, chiamato \textbf{teorema di Solovay-Kitaev}, che stabilisce che il numero di prodotti tra $H$ e $T$ per approssimare una generica matrice di $U(2)$ è dell'ordine di $\order{\log_{10}^c(1/\varepsilon)}$ dove $c \sim 2$. 
    
\end{enumerate}

    %%%%%%%%%%%%%
% LECTURE 6 %
%%%%%%%%%%%%%
\vspace{1cm}

\noindent \lecture{6}{22/10/2021}

\section{Quantum Parallelism}

\begin{definizione}[\textbf{Quantum Parallelism}]
    Il \textbf{quantum parallelism} è una delle caratteristiche fondamentali di molti algoritmi quantistici. Consente ai computer quantistici di valutare una funzione $f(x)$ per molti valori diversi di $x$ contemporaneamente.
\end{definizione}
\noindent Supponiamo di considerare la più semplice funzione possibile $f(x):\{0,1\}^{\otimes n} \rightarrow \{0,1\}$ definita su un dominio (insieme di numeri costruiti con $n$ cifre di 0 e 1) e a elementi in un intervallo di bit. Assumiamo inoltre di saper calcolare efficientemente nel nostro computer tale funzione. Ciò che calcoliamo, dal punto di vista della computazione classica, lo possiamo valutare nella computazione quantistica, pertanto tutte le operazioni aritmetiche possono essere svolte dal calcolo quantistico. Un modo quindi di calcolare questa funzione su un computer quantistico è quello di considerare due differenti stati: immaginiamo un qubit $\ket{y}$ e uno stato che può essere un prodotto tensoriale di qubit, come ad esempio $\ket{0}^{\otimes n}$. Spesso considereremo lo stato $\ket{0}^{\otimes n}$ come stato iniziale in cui il computer quantistico viene preparato mediante una misurazione nella base computazionale perché è facilmente costruibile: ad esempio nel caso $n=3$ se, a seguito di una misurazione, lo stato nel QC collassa in $\ket{\psi} \rightarrow \ket{1} \otimes \ket{0} \otimes \ket{1}$, basterà applicare un \texttt{X-gate} al primo e al terzo qubit per costruire lo stato voluto $\ket{0}^{\otimes 3}$.

\noindent Chiamiamo lo stato iniziale totale $\ket{x,y}$, dove $x$ contiene l'informazione iniziale data in input e $y$ conterrà, dopo delle opportune operazioni, il risultato cercato. Con un'appropriata sequenza di gate è possibile effettuare la trasformazione
\begin{equation}\label{black_box_U_f}
    \ket{x,y} \overset{U_f}{\longrightarrow} \ket{x,y \oplus f(x)} \, ,
\end{equation}
dove $U_f$ è un opportuno gate unitario che implementa l'operazione desiderata. Il circuito che implementa la \eqref{black_box_U_f} è 
\begin{center}
    \mbox{
        \Qcircuit @C=1em @R=1em {
            \lstick{\ket{x}} & \multigate{1}{U_f} & \rstick{\ket{x}} \qw \\
            \lstick{\ket{y}} & \ghost{U_f} & \rstick{\ket{y\oplus f(x)}} \qw
        }
    }
\end{center}
dove $\ket{x}$ prende il nome di \textbf{data register} e $\ket{y}$ prende il nome di \textbf{target register}. Questa rappresentazione è utile perché quando $\ket{y} = \ket{0}$ l'output del target register è esattamente l'oggetto che si vuole calcolare
\begin{center}
    \mbox{
        \Qcircuit @C=1em @R=1em {
            \lstick{\ket{x}} & \multigate{1}{U_f} & \rstick{\ket{x}} \qw \\
            \lstick{\ket{0}} & \ghost{U_f} & \rstick{\ket{0\oplus f(x)}=\ket{f(x)}} \qw
        }
    }
\end{center}
Notiamo che la \eqref{black_box_U_f} è invertibile: se applichiamo $U_f$ due volte, otteniamo:
\begin{equation*}
    \ket{x,y} \rightarrow \ket{x, y \oplus f(x)} \rightarrow \ket{x,y \oplus f(x) \oplus f(x)} = \ket{x,y} \, ,
\end{equation*}
siccome $f(x) \oplus f(x) = 0$ indipendentemente dai valori di $f$. Fino ad ora avremmo potuto effettuare tutte queste operazioni in CC. L'importanza del QC risiede nel fatto che si possano considerare sovrapposizioni di stati appartenenti ad una base. Consideriamo il caso $n = 1$ (il data register è un qubit) e assumiamo il seguente stato iniziale
\begin{equation*}
    \ket{x,y} \equiv \underbrace{\frac{1}{\sqrt 2} (\ket{0}+\ket{1})}_{\ket{x}} \otimes \underbrace{\ket 0}_{\ket{y}} = \frac{1}{\sqrt 2} \left( \ket{00}+\ket{10} \right) \, ;
\end{equation*}
Se assumiamo che il computer sia preparato in $\ket{0} \otimes \ket{0}$ come possiamo rappresentare $\ket{x,y}$ in un circuito? Possiamo sfruttare l'\texttt{H-gate} in questo modo:
\begin{center}
    \mbox{
        \Qcircuit @C=1em @R=1em {
            \lstick{\ket{0}} & \gate{H} & \multigate{1}{U_f} & \qw \\
            \lstick{\ket{0}} & \qw & \ghost{U_f} & \qw
            %\gategroup{1}{4}{2}{4}{0.8em}{\}}
        }
    }
\end{center}
infatti, utilizzando la \eqref{black_box_U_f}, avremo
\begin{equation*}
    \ket{0,0} \overset{H}{\longrightarrow} \frac{1}{\sqrt{2}} \left( \ket{00}+\ket{10} \right) \overset{U_f}{\longrightarrow} \frac{1}{\sqrt{2}} \left( \ket{0, f(0)} + \ket{1, f(1)} \right) \, .
\end{equation*}
Questo circuito è particolarmente interessante perché l'output è una sovrapposizione di differenti stati contenenti informazioni riguardo la funzione: $f(0)$ e $f(1)$ appaiono simultaneamente nel medesimo stato. È come se avessimo valutato $f(x)$ per due valori di $x$ contemporaneamente, parallelamente! A differenza del classic parallelism, in cui più circuiti vengono costruiti per calcolare $f(x)$ ed eseguiti simultaneamente, qui viene impiegato un singolo circuito per valutare la funzione $f(x)$ per più valori di $x$ nello stesso momento: si sta sfruttando la capacità di un computer quantistico di essere in sovrapposizioni di stati diversi. Qui risiede il \textbf{quantum parallelism}.

\noindent Questo discorso può essere facilmente generalizzato al caso di $n$-qubit. Supponiamo che il data register si trovi in $\ket{0}^{\otimes n}$. Usiamo il fatto che l'azione dell'\texttt{H-gate} su $n$-qubit possa essere scritta nel seguente modo:
\begin{align}
    H^{\otimes n}\ket{0}^{\otimes n} &= \underbrace{H\otimes \cdots \otimes H}_{n\text{-volte}} \underbrace{\ket{0} \otimes \cdots \otimes \ket{0}}_{n\text{-volte}} = \frac{1}{\sqrt 2} (\ket 0 + \ket 1) \otimes \cdots \otimes \frac{1}{\sqrt 2} (\ket 0 + \ket 1) \notag \\
    &= \frac{1}{\sqrt{2^n}}(\ket{000 \ldots 0} + \ket{010 \ldots 0} + \ldots + \ket{111 \dots 1}) = \frac{1}{\sqrt{2^n}}\sum_{x=0}^{2^n-1}\ket{x} \, , \label{n_H_gates}
\end{align}
dove $x$ rappresenta tutte le possibili stringhe di $n$-volte $0$ e $1$. Se il target si trova in $\ket{y} = \ket{0}$ e applichiamo ora $U_f$, il risultato è:
\begin{equation*}
    \frac{1}{\sqrt{2^n}}\sum_{x=0}^{2^n-1}\ket{x} \otimes \ket{0} \overset{U_f}{\longrightarrow} \frac{1}{\sqrt{2^n}}\sum_{x=0}^{2^n-1}\ket{x,f(x)} \, ,
\end{equation*}
dove si è fatto uso della \eqref{black_box_U_f} con $\ket{y} = \ket{0}$. In termini di circuiti avremo
\begin{center}
    \mbox{
        \Qcircuit @C=1em @R=1em {
            \lstick{\ket{0}^{\otimes n}} & \gate{H^{\otimes n}} & \multigate{1}{U_f} & \qw & \qw \\
            \lstick{\ket{y} = \ket{0}} & \qw & \ghost{U_f}   & \qw      & \qw
        }
    }
\end{center}
In un certo senso, il quantum parallelism consente di valutare simultaneamente tutti i possibili valori della funzione $f(x)$, anche se apparentemente abbiamo valutato $f(x)$ in una singola volta. Precisiamo che la misura dello stato nel caso del qubit singolo ci darà solamente $\ket{0, f(0)}$ oppure $\ket{1, f(1)}$. In maniera analoga per il caso generale, la misura dello stato $\sum_x\ket{x,f(x)}$ ci darà un solo $f(x_0)$ per un singolo valore casuale $x_0$. Ovviamente un computer classico può farlo più facilmente! La computazione quantistica richiede qualcosa di più del semplice quantum parallelism per essere utile; richiede cioè la capacità di estrarre informazioni su più di un valore di $f(x)$ da stati di sovrapposizione, come $\sum_x\ket{x,f(x)}$. Come vedremo nella prossima sezione, il trucco di considerare una sovrapposizione lineare ci permetterà di estrarre alcune informazioni su $f$ in un modo più efficiente del CC.

\section{Algoritmo di Deutsch}
Una semplice modifica del circuito precedente dimostra come i circuiti quantistici possano essere più performanti rispetto a quelli classici. Nelle ultime righe del paragrafo precedente abbiamo detto che la computazione quantistica richiede qualcosa di più oltre al quantum parallelism per essere utilizzabile. L'\textbf{algoritmo di Deutsch} combina il meccanismo del \textbf{quantum parallelism} con la proprietà della meccanica quantistica dell'\textbf{interferenza}. 

\noindent Si tratta di un algoritmo un po' accademico (le funzioni sono banali), tuttavia utile per illustrare l'idea di algoritmo quantistico. Lasciamo che entrambi input e output register contengano ciascuno un solo qubit, quindi stiamo esplorando le funzioni $f(x)$ che convertono un singolo bit in un singolo bit: $f(x): \; \{ 0,1 \} \rightarrow \{ 0,1 \}$. Ci sono due modi piuttosto diversi di pensare a tali funzioni. Il primo modo è notare che ci sono solo quattro di queste funzioni, come mostrato nella Tabella \ref{tab:Deutsch_Fnct}.

\begin{table}[!ht]
	\centering
    \begin{tabular}{ccc}
        \toprule
        & $x = 0$ & $x=1$ \\
        \midrule
        $f_0$ & $0$ & $0$ \\
        $f_1$ & $0$ & $1$ \\
        $f_2$ & $1$ & $0$ \\
        $f_3$ & $1$ & $1$ \\
        \bottomrule
    \end{tabular} \\
    \caption{Possibili output delle sole quattro funzioni distinte $f_j(x)$ che convertono un bit in un bit. Esse sono tutte facilmente implementabili sia in un computer classico che quantistico.}
    \label{tab:Deutsch_Fnct}
\end{table}

\noindent Supponiamo che ci venga data una black-box (ossia un gate ignoto che indicheremo con \texttt{U-gate}) che calcola una di queste quattro funzioni eseguendo la seguente trasformazione unitaria:
\begin{equation*}
    U_{f_j} \ket{x,y} = \ket{x, y \oplus f_j(x)} \, .
\end{equation*}
In questo caso, se implementiamo in circuiti la Tabella \ref{tab:Deutsch_Fnct} avremo:

\begin{center}
    \mbox{
        $
        \begin{matrix}
             \\
             \\
            f_0: \\
        \end{matrix}
        $
        \Qcircuit @C=1em @R=1em {
            & \multigate{1}{U_{f_0}} & \qw \\
            & \ghost{U_{f_0}}& \qw \\
        }
        $
        \begin{matrix}
             \\
             \\
            \ = \\
        \end{matrix}
        $
        \Qcircuit @C=1em @R=1.9em {
            & \qw & \qw & \qw & \qw & \qw \\
            & \qw & \qw & \qw & \qw & \qw \\
        }
    }
    \qquad \qquad
    \mbox{
        $
        \begin{matrix}
             \\
             \\
            f_1: \\
        \end{matrix}
        $
        \Qcircuit @C=1em @R=1em {
            & \multigate{1}{U_{f_1}} & \qw \\
            & \ghost{U_{f_1}}& \qw \\
        }
        $
        \begin{matrix}
             \\
             \\
            \ = \\
        \end{matrix}
        $
        \Qcircuit @C=1em @R=1.35em {
            & \ctrl{1} & \qw & \qw \\
            & \targ & \qw & \qw  \\
        }
    }
\end{center}
\begin{center}
    \mbox{
        $
        \begin{matrix}
             \\
             \\
            f_2: \\
        \end{matrix}
        $
        \Qcircuit @C=1em @R=1em {
            & \multigate{1}{U_{f_2}} & \qw \\
            & \ghost{U_{f_2}}& \qw \\
        }
        $
        \begin{matrix}
             \\
             \\
            \ = \\
        \end{matrix}
        $
        \Qcircuit @C=1em @R=1.15em {
            & \qw & \ctrl{1} & \qw \\
            & \gate{X} & \targ & \qw \\
        }
    }
    \qquad \qquad
    \mbox{
        $
        \begin{matrix}
             \\
             \\
            f_3: \\
        \end{matrix}
        $
        \Qcircuit @C=1em @R=1em {
            & \multigate{1}{U_{f_3}} & \qw \\
            & \ghost{U_{f_3}}& \qw \\
        }
        $
        \begin{matrix}
             \\
             \\
            \ = \\
        \end{matrix}
        $
        \Qcircuit @C=1em @R=1.25em {
            & \qw & \qw \\
            & \gate{X} & \qw \\
        }
    }
\end{center}
Dato che la regola che vogliamo implementare è $\ket{x,0} \rightarrow \ket{x, f(x)}$ ($\ket{y}$ inizializzato a $\ket{0}$), in termini matematici questo significa scrivere:
\begin{align*}
    &f_0: &\ket{x,0} &\longrightarrow \ket{x,0} \, , \\
    &f_1: &\ket{x,0} &\overset{\texttt{CNOT}}{\longrightarrow} 
    \begin{cases}
        \ket{0,0} \, , &\text{per } x = 0 \\
        \ket{1,1} \, , &\text{per } x = 1
    \end{cases} \, , \\
    &f_2: &\ket{x,0} &\overset{X}{\longrightarrow} \ket{x,1} \overset{\texttt{CNOT}}{\longrightarrow}
    \begin{cases}
        \ket{0,1} \, , &\text{per } x = 0 \\
        \ket{1,0} \, , &\text{per } x = 1
    \end{cases} \, , \\
    &f_3: &\ket{x,0} &\overset{X}{\longrightarrow} \ket{x,1} \, , 
\end{align*}

\noindent Supponiamo che ci venga data una black-box che esegua $U_f$ per una delle quattro funzioni, ma non ci venga detto quale delle quattro operazioni. Ovviamente possiamo scoprirlo lasciando agire due volte la black-box, prima su $\ket0 \otimes \ket0$ e poi su $\ket 1 \otimes \ket 0$. Ma supponiamo di poter far agire la black-box solo una volta. Cosa possiamo conoscere di $f(x)$?

\noindent In un computer classico, dove siamo effettivamente limitati a lasciare che la black-box agisca sui qubit in uno dei quattro stati di base computazionale, possiamo conoscere il valore di:
\begin{itemize}
    \item $f(0)$, lasciando che $U_f$ agisca su uno dei due $\ket0 \otimes \ket0$ o $\ket0 \otimes \ket1$;
    \begin{itemize}
        \item In tal caso possiamo limitare la scelta a $f_0$ o $f_1$ (se $f(0) = 0$) oppure $f_2$ o $f_3$ (se $f(0) = 1$).
    \end{itemize}
    \item $f(1)$, lasciando che $U_f$ agisca su $\ket1 \otimes \ket0$ o $\ket1 \otimes \ket1$;
    \begin{itemize}
        \item In questa situazione abbiamo ristretto la funzione ad essere $f_0$ o $f_2$ (se $f(1) = 0$) oppure $f_1$ o $f_3$ (se $f(1) = 1$).
    \end{itemize}
\end{itemize}
In definitiva, un computer classico necessita di due esecuzioni per determinare se $f$ sia costante o meno. Sorprendentemente, risulta che con un computer quantistico questo non è necessario perché il problema può essere risolto con una singola esecuzione. Il punto interessante è che l'algoritmo non riguarda il calcolo preciso della funzione, ma piuttosto la comprensione di una o più sue proprietà: quando l'algoritmo viene lanciato non impariamo nulla sui valori individuali di $f(0)$ e $f(1)$, ma siamo comunque in grado di rispondere alla domanda sui loro valori relativi. Chiaramente otteniamo meno informazioni di quelle che otterremmo rispondendo alla domanda con un computer classico, ma, rinunciando alla possibilità di acquisire quella parte dell'informazione che è irrilevante per la domanda a cui vogliamo rispondere, possiamo ottenere la risposta con una sola applicazione di $U_f$.

\noindent Come sottolineato in precedenza l'algoritmo combina il quantum parallelism e l'interferenza: possiamo preparare il computer nello stato $\ket 0 \otimes \ket 1$ della base canonica e applicare l'\texttt{H-gate} a entrambi i qubit: 
\begin{equation}\label{eq:Deutsch_1}
    (H\otimes H) \ket{0} \otimes \ket{1} = \underbrace{\frac{\ket0 + \ket1}{\sqrt 2}}_{\substack{\text{quantum} \\ \text{parallelism}}} \otimes \underbrace{\frac{\ket0-\ket1}{\sqrt 2}}_{\text{interferenza}} \, ;
\end{equation}
in un circuito significa scrivere
\begin{center}
    \mbox{
        \Qcircuit @C=1em @R=1em {
            \lstick{\ket{0}} & \gate{H} & \multigate{1}{U_f} & \qw \\
            \lstick{\ket{1}} & \gate{H} & \ghost{U_f} & \qw
        }
    }
\end{center}
\vspace{0.2cm}
Chiamando per semplicità $\ket{x} \equiv \{ \ket{0} ,  \ket{1} \}$ e applicando $U_f$ alla \eqref{eq:Deutsch_1} tramite \eqref{black_box_U_f}, possiamo esplicitamente vedere che cosa implica il termine di interferenza:
\begin{align*}
    &\ket{x} \otimes \frac{1}{\sqrt 2} (\ket 0 - \ket 1) \overset{U_f}{\longrightarrow} \frac{1}{\sqrt 2} \left( \ket{x, 0 \oplus f(x)} - \ket{x, 1 \oplus f(x)} \right) \\
    &=
    \begin{cases}
        \frac{1}{\sqrt 2} \left( \ket{x, 0 \oplus 0} - \ket{x, 1 \oplus 0} \right) = \ket x \otimes \frac{1}{\sqrt 2} (\ket 0 - \ket 1) \, , \quad &\text{per } f(x) = 0 \\
        \frac{1}{\sqrt 2} \left( \ket{x, 0 \oplus 1} - \ket{x, 1 \oplus 1} \right) = - \ket x \otimes \frac{1}{\sqrt 2} (\ket 0 - \ket 1) \, , \quad &\text{per } f(x) = 1
    \end{cases} \, .
\end{align*}
Combinando i due casi in un'unica espressione compatta abbiamo ottenuto
\begin{equation}\label{black_box_action_U_f_on_x}
    \ket{x} \otimes \frac{1}{\sqrt 2} (\ket 0 - \ket 1) \overset{U_f}{\longrightarrow} (-1)^{f(x)} \ket x \otimes \frac{1}{\sqrt 2} (\ket 0 - \ket 1) \, ,
\end{equation}
Sostituendo $\ket{x}$ con lo stato iniziale che implementava il quantum parallelism avremo
\begin{equation*}
    \frac{\ket{0} + \ket{1}}{\sqrt{2}} \otimes \frac{\ket 0 - \ket 1}{\sqrt 2} \overset{U_f}{\longrightarrow} \frac{1}{\sqrt{2}} \left[ (-1)^{f(0)} \ket 0 + (-1)^{f(1)} \ket 1 \right] \otimes \frac{\ket 0 - \ket 1}{\sqrt 2} \, ;
\end{equation*}
dato che il segno relativo nella parentesi quadra dipende dal fatto che $f(0)$ e $f(1)$ siano uguali o meno, possiamo riscrivere quest'ultima espressione come
\begin{equation*}
    \begin{cases}
        (-1)^{f(0)}\frac{\ket 0 + \ket 1}{\sqrt 2}\otimes\frac{\ket 0-\ket 1}{\sqrt 2} \, , &\text{per }f(0) = f(1) \\
        (-1)^{f(0)}\frac{\ket 0 - \ket 1}{\sqrt 2}\otimes\frac{\ket 0-\ket 1}{\sqrt 2} \, , &\text{per }f(0) \neq f(1) 
    \end{cases} \, .
\end{equation*}
Come ultimo passaggio si applica l'\texttt{H-gate} al primo qubit in maniera tale che il circuito totale diventi:
\begin{center}
    \mbox{
        \Qcircuit @C=1em @R=1em {
            \lstick{\ket{0}} & \gate{H} & \multigate{1}{U_f} & \gate{H} & \qw \\
            \lstick{\ket{1}} & \gate{H} & \ghost{U_f} & \qw & \qw
        }
    }
\end{center}
Questa modifica trasforma il risultato precedente in 
\begin{equation*}
    \begin{cases}
        (-1)^{f(0)}\frac{\ket 0 + \ket 1}{\sqrt 2}\otimes\frac{\ket 0-\ket 1}{\sqrt 2} \overset{H}{\longrightarrow} (-1)^{f(0)}\ket 0\otimes\frac{\ket 0-\ket 1}{\sqrt 2} \, , &\text{per }f(0) = f(1) \\
        (-1)^{f(0)}\frac{\ket 0 - \ket 1}{\sqrt 2}\otimes\frac{\ket 0-\ket 1}{\sqrt 2} \overset{H}{\longrightarrow} (-1)^{f(0)}\ket 1\otimes\frac{\ket 0-\ket 1}{\sqrt 2} \, , &\text{per }f(0) \neq f(1) 
    \end{cases} \, .
\end{equation*}
Il risultato finale ci suggerisce che possiamo effettuare solamente una misurazione sul primo qubit: ottenendo $\ket{0}$ o $\ket{1}$ siamo in grado, con una singola misura, di stabilire se $f(0) = f(1)$ oppure $f(0) \neq f(1)$. Questo significa che siamo in grado di escludere 2 delle 4 funzioni con una singola esecuzione dell'algoritmo. 

\noindent Questo esempio permette di evidenziare quale sia la differenza tra il quantum parallelism e gli algoritmi randomizzati classici. Ingenuamente, si potrebbe pensare che lo stato finale corrisponda piuttosto a un calcolatore classico probabilistico che valuta $f(0)$ con probabilità $\frac 12$, o $f(1)$ con probabilità $\frac 12$. La differenza è che in un computer classico queste due alternative si escludono sempre mentre in un computer quantistico è possibile che le due alternative interferiscano l'una con l'altra per ottenere alcune proprietà globali della funzione $f(x)$. Utilizzando un opportuno gate (nel nostro caso l'\texttt{H-gate}) siamo in grado di ricombinare le diverse alternative.

\section{Algoritmo di Deutsch-Jozsa}
L'algoritmo di Deutsch è un semplice caso di un algoritmo quantistico più generale, noto come \textbf{algoritmo di Deutsch-Jozsa}, che evidenzia esplicitamente come il QC offra un grosso miglioramento rispetto ai metodi del CC. Supponiamo di avere una black-box che calcoli una funzione booleana $f(x): \; \{0,1\}^{\otimes n}\rightarrow \{0,1\}$ e supponiamo di sapere per certo che $f(x)$ sia solamente una delle seguenti alternative:
\begin{itemize}
    \item \textbf{Funzione costante} (\textit{constant}): l'output è sempre $0$ oppure $1$ indipendentemente dall'input.
    \item \textbf{Funzione bilanciata} (\textit{balanced}): l'output è costituito per metà dal valore $0$ e metà dal valore $1$.
\end{itemize}
Lo scopo dell'algoritmo è quello di capire quale delle due sia l'alternativa corretta con il minor numero di esecuzioni. Classicamente potremmo risolvere questo problema calcolando $2^{n-1}+1$ valori della funzione perché è necessario calcolare almeno una metà dei valori più un valore aggiuntivo. Chiaramente si tratta di un numero esponenzialmente grande. Quello che fa l'algoritmo di Deutsch-Jozsa è risolvere il problema perfettamente con una sola query quantistica. Cominciamo scrivendo il circuito che descrive tale algoritmo, il quale è molto simile a quello di Deutsch con la sola differenza che il data register non è un singolo qubit, ma piuttosto un prodotto tensoriale di $n$-qubit:
\begin{center}
    \mbox{
        \Qcircuit @C=1em @R=1em {
            \lstick{\ket{0}^{\otimes n}} & \gate{H^{\otimes n}} & \multigate{1}{U_f} & \gate{H^{\otimes n}} & \qw \\
            \lstick{\ket{1}} & \gate{H} & \ghost{U_f} & \qw & \qw
        }
    }
\end{center}
Vediamo nello specifico cosa succede all'interno del circuito:
\begin{enumerate}
    \item Viene inizializzato (preparato) lo stato in $\ket{0}^{\otimes n} \otimes \ket{1}$;
    \item Creiamo una sovrapposizione di stati usando l'\texttt{H-gate} su tutti gli $n+1$ qubit:
        \begin{equation*}
            \ket{0}^{\otimes n} \otimes \ket{1} \overset{H}{\longrightarrow} \frac{1}{\sqrt {2^n}}\sum_{x=0}^{2^n-1}\ket x \otimes \frac{\ket 0 - \ket 1}{\sqrt 2} \, ,
        \end{equation*}
        dove si è fatto uso della \eqref{n_H_gates}. Notiamo che ora nell'output register è presente lo stato che nella sezione precedente avevamo visto essere associato all'interferenza. 
    \item Valutiamo la funzione $f(x)$ usando la black-box di $U_f$
        \begin{equation}\label{dopo_U_f}
            \frac{1}{\sqrt {2^n}}\sum_{x=0}^{2^n-1}\ket x \otimes \frac{\ket 0 - \ket 1}{\sqrt 2} \overset{U_f}{\longrightarrow} \sum_{x=0}^{2^n-1}\frac{(-1)^{f(x)}}{\sqrt{2^n}}\ket x \otimes \frac{\ket 0 - \ket 1}{\sqrt 2} \, ,
        \end{equation}
        dove, essendo $\ket{x}$ arbitrario, abbiamo fatto uso della \eqref{black_box_action_U_f_on_x}. 
    \item Applichiamo nuovamente l'\texttt{H-gate} ai primi $n$ qubit. Per capire il risultato di $H^{\otimes n} \ket{x}$ consideriamo per semplicità il caso $n=1$: formalmente avremo 
    \begin{equation*}
        H \ket{x} = \sum_{z = 0}^1 \frac{(-1)^{xz}}{\sqrt{2}} \ket{z} \, , \; \text{ dove } x = 0 \text{ oppure } 1 \, .
    \end{equation*}
    Per $n$ generico possiamo generalizzare scrivendo
    \begin{align*}
        H^{\otimes n} \ket{x} &= (H \otimes \ldots \otimes H) \ket{x_0} \otimes \ket{x_1} \otimes \ldots \otimes \ket{x_{n-1}} \\
        &= \sum_{z_0=0}^1 \ldots \sum_{z_{n-1}=0}^1 \frac{(-1)^{x_0 z_0} (-1)^{x_1 z_1} \cdots (-1)^{x_{n-1} z_{n-1}}}{\sqrt{2^n}} \ket{z} \, ,
    \end{align*}
    dove $\ket{z} \equiv \ket{z_0, z_1, \ldots, z_{n-1}}$. In maniera più compatta possiamo scrivere quindi l'azione dell'\texttt{H-gate} sugli $n$ qubit (nonché risultato finale del circuito) come
        \begin{equation}\label{output_Deutsch_Jozsa}
            \sum_{z = 0}^{2^n-1} \sum_{x = 0}^{2^n-1} \frac{(-1)^{f(x) + x \cdot z}}{2^n}\ket z \otimes \frac{\ket 0 - \ket 1}{\sqrt 2} \, ,
        \end{equation}
        dove abbiamo indicato con $x\cdot z$ il \textbf{prodotto bit a bit modulo 2}:
        \begin{equation*}
            x\cdot z = (x_0z_0 + \ldots + x_{n-1}z_{n-1}) \mod{2} \, .
        \end{equation*}
    \item Infine misuriamo per ottenere lo stato finale $\ket{z}$. 
\end{enumerate}

\noindent Ricordiamo che il problema è quello di determinare se $f$ sia constant o balanced. Notiamo dal risultato in \eqref{output_Deutsch_Jozsa} che il data register ora contiene una sovrapposizione lineare di tutti i possibili stati che si scrivono come stringhe contenenti $n$ volte 0 e 1. In $\ket{z}$ è presente un caso particolare: consideriamo la situazione in cui $\ket z = \ket{00\ldots0} = \ket0^{\otimes n}$ e cerchiamo la probabilità di ottenere tale stato guardando il modulo quadro del coefficiente:
\begin{equation*}
    P \left( \ket{z} = \ket0^{\otimes n} \right) = \abs{\sum_{x=0}^{2^n-1}\frac{(-1)^{f(x)}}{2^n}}^2 = 
    \begin{cases}
    1 \, , &\text{se } f(x) \text{ è constant} \\
    0 \, , &\text{se } f(x) \text{ è balanced} 
    \end{cases} \, .
\end{equation*}
Notiamo che quando la probabilità è 1 a numeratore si hanno $2^n$ termini tutti uguali ($(-1)^1$ oppure $(-1)^0$) che si semplificano con il fattore $1/2^n$; quando invece la probabilità è nulla a numeratore si ha uno stesso numero di $(-1)^1$ e $(-1)^0$ che si cancellano esattamente. Come abbiamo detto $\ket z=\ket0^{\otimes n}$ è un caso particolare molto importante perché permette di risolvere il problema mediante la misura dello stato. Se misurando $z$ otteniamo $\ket{0}^{\otimes n}$ allora, con probabilità 1 (quindi sempre), lo stato è $\ket{0}^{\otimes n}$ e la funzione è constant; al contrario quando la misura di $z$ produce un qualsiasi stato differente da $\ket{0}^{\otimes n}$ allora, necessariamente $P \left( \ket{z} = \ket0^{\otimes n} \right) = 0$, lo stato $\ket{0}^{\otimes n}$ non è nemmeno presente in $z$ e possiamo stabilire con assoluta certezza che la funzione è balanced. Il fatto importante è che essendo queste misure mutuamente esclusive, possiamo determinare se $f$ sia constant o balanced con una singola misurazione. Quindi si tratta di effettuare una sola misurazione in QC contro $\mathcal{O}(2^n)$ misure in CC.

\noindent Osserviamo che il confronto tra algoritmi classici e quantistici è in qualche modo un confronto delicato, poiché il metodo per valutare la funzione è abbastanza diverso nei due casi. Se fosse consentito utilizzare un computer probabilistico classico, per valutare $f(x)$ per pochi $x$ scelti a caso, si può determinare molto rapidamente con alta probabilità se $f(x)$ è \textit{constant} o \textit{balanced}. Questo scenario probabilistico è forse più realistico dello scenario deterministico che abbiamo considerato.

\noindent Ribadiamo nuovamente che questo algoritmo è un esempio molto accademico in quanto non esistono problemi fisici o matematici reali che necessitano di sapere se una funzione sia constant o balanced. Nonostante ciò il fatto importante è che grazie a questo algoritmo quantistico non è più necessario aspettare un tempo esponenzialmente\footnote{Talvolta non si vuole sapere con precisione assoluta se $f$ sia constant o balanced, ma è sufficiente stabilirlo entro un errore dato $\varepsilon$. Un ipotetico algoritmo classico e probabilistico di questo tipo diventa di ordine polinomiale in $n$: passare da $\mathcal{O}(\text{polinomio in }n)$ a $\mathcal{O}(1)$ mediante la controparte quantistica non è più un miglioramento così estremo come passare da $\mathcal{O}(2^n)$ ad $\mathcal{O}(1)$!} crescente nel numero di bit per sapere il risultato. 

\section{Algoritmo di Bernstein-Vazirani}
Consideriamo un altro algoritmo di black-box per il quale gli algoritmi quantistici forniscono un vantaggio: l'\textbf{algoritmo di Bernstein-Vazirani}. Qui, a differenza dei due casi precedenti, abbiamo accesso alla funzione della black-box $f: \{0, 1\}^n \rightarrow \{0, 1\}$. Supponiamo che la funzioni sia data da\footnote{Come prima il simbolo "$\cdot$" indica il prodotto bit a bit modulo 2}:
\begin{equation*}
    f(x) = a\cdot x = (a_0 x_0 + \ldots + a_{n-1} x_{n-1})\mod{2} \, , \; \text{ dove } a \geq 0 \text{ e } x < 2^n \, .
\end{equation*}
Sappiamo che la funzione è lineare, tuttavia l'obiettivo di questo algoritmo è trovare il valore di $a$. Classicamente, questo problema potrebbe richiedere $n$ query poiché ogni query può fornire solo un nuovo bit di informazioni su $a$, ma $a$ possiede $n$ bit: dobbiamo valutare $f(1000\ldots) = a_0$, $f(0100\ldots) = a_1$ e così via con $n$ valutazioni fino a $f(111\ldots1) = a_{n-1}$. L'algoritmo di Bernstein-Vazirani, invece, risolve il problema quantisticamente utilizzando una sola query!

\noindent Consideriamo il medesimo circuito dell'algoritmo di Deutsch-Josza e il suo output in \eqref{output_Deutsch_Jozsa}: nel caso in cui $f(x) = a \cdot x$ esso diventa 
\begin{equation*}
    \sum_{z=0}^{2^n-1}\sum_{x=0}^{2^n-1}\frac{(-1)^{x\cdot (a+z)}}{2^n}\ket z \otimes \frac{\ket 0 - \ket 1}{\sqrt 2} \, .
\end{equation*}
Come nell'algoritmo precedente guardiamo il coefficiente di $\ket{z}$:
\begin{equation*}
        \frac{1}{2^n} \sum_{x=0}^{2^n-1}(-1)^{x\cdot (a+z)} = \frac{1}{2^n} \sum_{x=0}^{2^n-1}(-1)^{x_0(a_0+z_0) + \ldots + x_{n-1}(a_{n-1}+z_{n-1})} = \frac{1}{2^n} \prod_{j=0}^{n-1} \left( \sum_{x_j=0}^{1}(-1)^{x_j(a_j+z_j)} \right) \, ,
\end{equation*}
ma ogni termine nella parentesi tonda è la somma di termini che possono essere $\pm 1$ a seconda dell'esponente. Distinguiamo i due casi:
\begin{itemize}
    \item Se $(a_j+z_j=0)\mod2 $ allora il coefficiente è 
    \begin{equation*}
        \frac{1}{2^n} \prod_{j=0}^{n-1}(2) = 1 \, , \quad \Rightarrow \quad \text{Probabilità } 1 \, .
    \end{equation*}
    \item Al contrario quando $(a_j+z_j=1)\mod2$ allora il coefficiente diventa 
    \begin{equation*}
        \frac{1}{2^n} \prod_{j=0}^{n-1} \left[ (-1)^{0\cdot 1} + (-1)^{1 \cdot 1} \right] = 0 \, , \quad \Rightarrow \quad \text{Probabilità } 0 \, .
    \end{equation*}
\end{itemize}
Ancora una volta, i due casi della probabilità sono mutuamente esclusivi e quindi avremo
\begin{align*}
    &(a_j+z_j=0)\mod2 \, , \; \forall \, j \, ,  &\Rightarrow& \quad a = z \, , &\Rightarrow& \quad \text{Probabilità } 1 \, , \\
    &(a_j+z_j=1)\mod2 \, , \;  \text{per qualche } j \, , &\Rightarrow& \quad a \neq z \, , &\Rightarrow& \quad \text{Probabilità } 0 \, .
\end{align*}
Questo significa che il nostro stato, in realtà, non è una sovrapposizione lineare, ma contiene bensì solamente lo stato
\begin{equation*}
    \ket a \otimes \frac{\ket 0 - \ket 1}{\sqrt 2} \, ;
\end{equation*}
e quindi attraverso un'unica operazione di misura sui primi $n$-qubit, otteniamo $a$, la nostra incognita.
    %%%%%%%%%%%%%%%%%%%%%%%
%%%%%% Lezione 7 %%%%%%
%%%%%%%%%%%%%%%%%%%%%%%

\vspace{1.0cm}
\lecture{7}{25/10/2021}
\vspace{1.0cm}

\section{Screening}

Abbiamo osservato che l'interazione tra due cariche in un gas omogeneo di elettroni o in generale, in un qualque sistema di interazione tra elettroni, l'interazione è schermata dalle altre cariche presenti. Possiamo formalizzare questa idea con la \textbf{teoria della risposta lineare}

\subsection{Teoria della risposta lineare}
Questa teoria assume che la densità è lineare. Consideriamo un cambiamento nel $V_{\text{ext}}$ ed è tale per cui genera un cambiamento, a sua volta, nella densità in una relazione lineare:
\begin{equation*}
    \delta V_{\text{ext}}(\overline{x}') \longrightarrow \delta n(\overline x) = \int \dd[3]{\overline{x}'}\chi(\overline x, \overline{x}')\delta V_{\text{ext}}(\overline{x}')
\end{equation*}
$\chi(\overline x, \overline{x}')$ prende il nome di \textbf{suscettività} e in un gas omogeneo di elettroni dipende solo della distanza tra $\overline x$ e $\overline{x}'$:
\begin{equation*}
    \chi(\overline x, \overline{x}')=\chi(|\overline x - \overline{x}'|)
\end{equation*}
Questa espressione può essere introdotta usando le trasformazioni di Fourier, possiamo scegliere qualunque definizione, noi scegliamo quella con il fattore $(2\pi)^3$ a denominatore:
\begin{equation*}
    \delta n(\overline x) = \frac{1}{(2\pi)^3}\int \dd[3]{q} \delta n(\overline q) e^{i\overline q \cdot \overline x}
\end{equation*}
\begin{equation*}
    \delta n(\overline q) = \frac{1}{(2\pi)^3}\int \dd[3]{x} \delta n(\overline x) e^{-i\overline q \cdot \overline x}
\end{equation*}
Ora possiamo usare questa definizione per scrivere la risposta lineare:
\begin{equation*}
    \begin{aligned}
        \delta n(\overline x) &=\frac{1}{(2\pi)^3}\int \dd[3]q\delta n(\overline q)e^{i\overline q \cdot \overline x} \\
        & = \int \dd[3]{x'}\chi(|\overline x - \overline{x}'|)\frac{1}{(2\pi)^3}\int \dd[3]{q}\delta V_{\text{ext}}(\overline q)e^{i\overline q \cdot \overline x} \\
        & = \int \dd[3]{x'}\chi(|\overline x - \overline{x}'|)\frac{1}{(2\pi)^3}\int \dd[3]{q}\delta V_{\text{ext}}(\overline q)e^{i\overline q \cdot \overline x}e^{i\overline q \cdot \overline x}e^{-i\overline q \cdot \overline x} \\
        & = \frac{1}{(2\pi)^3}\int \dd[3]{q}e^{i\overline q \cdot \overline x}\int\dd[3]{x'}\chi(|\overline x - \overline{x}'|)e^{i\overline q \cdot (\overline{x}'-\overline{x})}\delta V_{\text{ext}}(\overline q)
    \end{aligned}
\end{equation*}
Confrontando la prima e l'ultima riga troviamo che:
\begin{equation*}
    \begin{aligned}
        \delta n(\overline q) &=\underbrace{\int \dd[3]{x'}\chi(|\overline x - \overline{x}'|)e^{i\overline q \cdot (\overline{x}' - \overline x)}}_{\text{Trasformata di Fourier di } \chi(-\overline{q})}\delta V_{\text{ext}}(\overline q) \\
        & = \chi(-\overline{q})\delta V_{\text{ext}}(\overline q)
    \end{aligned}
\end{equation*}
Siccome in un gas omogeneo di elettroni, la suscettività dipende dal modulo di $\overline q$, pertanto:
\begin{equation*}
    \delta n(\overline q) = \chi(\overline q)\delta V_{\text{ext}}(\overline q)
\end{equation*}
Questo è sempre vero, se consideriamo la convoluzione tra le due funzioni, abbiamo:
\begin{equation*}
    \delta n(\overline x)=\int \dd[3]{x'}\chi(\overline x, \overline{x}')\delta V_{\text{ext}}(\overline{x}')
\end{equation*}
Questa $\chi$ è la cosiddetta \textbf{suscettività longitudinale} perché $\dots$
Il cambiamento su $V_{\text{ext}}$ induce, come abbiamo detto, un cambiamento in $n(\overline x)$ che è responsabile del potenziale totale:
\begin{equation*}
    \begin{aligned}
        \delta V_{\text{tot}}(\overline x) &=\delta V_{\text H}+ \delta V_{\text{ext}} \\
        & = \int \dd[3]{x'}\frac{e_0^2}{4\pi\varepsilon_0|\overline{x} - \overline{x}'|}\delta n(\overline{x}') + \delta V_{\text{ext}}
    \end{aligned}
\end{equation*}
Scritto in termini del reticolo reciproco:
\begin{equation*}
    \delta V_{\text{tot}}(\overline q)=V_{\text C}(\overline q)\delta n (\overline q)+\delta V_{\text{ext}}(\overline q)
\end{equation*}
Infatti è possibile applicare la trasformata di Fourier al potenziale coulombiano:
\begin{equation*}
    \frac{e_0^2}{4\pi\varepsilon_0}\frac{1}{|\overline x|} \longrightarrow V_{\text C}(\overline q)
\end{equation*}
Questo potenziale può essere valutato nell'equazione di Poisson, ricordando il caso di una carica posta nell'origine:
\begin{equation*}
    \nabla^2 \frac{e_0^2}{4\pi\varepsilon_0|\overline{x}|}=-\frac{e_0^2}{\varepsilon_0}\delta(\overline x)
\end{equation*}
\begin{equation*}
    \nabla^2 \frac{1}{|\overline x|}= -4\pi\delta(\overline x)
\end{equation*}
Considerando la trasformata di Fourier e prendendone il laplaciano, abbiamo:
\begin{equation*}
    \begin{aligned}
        \nabla^2\frac{1}{(2\pi)^3}\int \dd[3]{q}V_{\text C}(q)e^{i\overline q \cdot \overline x} & =\frac{1}{(2\pi)^3}\int \dd[3]{q}(-q^2)V_{\text C}(q)e^{i\overline q \cdot \overline x} \\
        & = -\frac{e_0^2}{\varepsilon_0}\delta(\overline x) \text{ introduco la T.F. della } \delta\\
        & = -\frac{e_0^2}{\varepsilon_0}\frac{1}{(2\pi)^3}\int \dd[3]{q}\delta(\overline q)e^{i \overline q \cdot \overline x}
    \end{aligned}
\end{equation*}
Confrontando la seconda espressione con quest'ultima, troviamo:
\begin{equation*}
    -q^2V_{\text{C}}(\overline q)=-\frac{e_0^2}{\varepsilon_0}\delta(\overline q)
\end{equation*}
Esplicitando $V_{\text{C}}$
\begin{equation*}
    V_{\text{C}}(\overline q) = \frac{e_0^2}{\varepsilon_0 q^2}\delta(\overline q) = \frac{e_0^2}{\varepsilon_0 q^2}
\end{equation*}
Questo perché:
\begin{equation*}
    \delta q = \int \dd[3]{x} \delta(\overline x) e^{-i\overline q \cdot \overline x}=1
\end{equation*}
Possiamo inserire questo risultato nel potenziale totale:
\begin{equation*}
    \delta V_{\text{tot}}(\overline q)=\delta n(\overline q)\frac{e_0^2}{\varepsilon_0q^2}+\delta V_{\text{tot}}(\overline q)
\end{equation*}
Possiamo introdurre un ulteriore oggetto: la \textbf{funzione dielettrica}. Questo può essere introdotto attraverso lo studio dell'elettromagnetismo, ricordiamo che:
\begin{equation*}
    \overline D = \varepsilon \overline E
\end{equation*}
dove
\begin{equation*}
    \divergence{D}=\rho_{\text{ext}}(\overline x)
\end{equation*}
Se consideriamo un campo longitudinale, $\overline D$ è il campo generato dalle cariche esterne, mentre $\overline E$ rappresenta il campo totale.
\begin{equation*}
    \underbrace{\overline D}_{\mathllap{\text{esterno}}} = \varepsilon \underbrace{\overline E}_{\mathrlap{\text{totale}}}
\end{equation*}
da cui abbiamo:
\begin{equation*}
    \delta V_{\text{ext}}(\overline q) = \varepsilon(\overline q)\delta V_{\text{tot}}(\overline q)
\end{equation*}
cioè:
\begin{equation*}
    \delta V_{\text{tot}}(\overline q) = \frac{\delta V_{\text{ext}}(\overline q)}{\varepsilon(\overline q)}
\end{equation*}
Siccome $\varepsilon < 1$, il potenziale totale è maggiore del potenziale esterno. Torniamo alla definizione di potenziale totale:
\begin{equation*}
    \begin{aligned}
        \delta V_{\text{tot}}(\overline q) & =\delta n(\overline q)V_{\text C}(\overline q)+\delta V_{\text{ext}}(\overline q) \\
        & = \chi(\overline q)\delta V_{\text{ext}}(\overline q)V_{\text C}(\overline q)+\delta V_{\text{ext}}(\overline q) \\
        & = \delta V_{\text{ext}}(\overline q)(1+\chi(\overline q)V_{\text C}(\overline q))
    \end{aligned}
\end{equation*}
Mettendo insieme le due equazioni si ottiene:
\begin{equation*}
    \frac{1}{\varepsilon(\overline q)}=1+\chi(\overline q)V_{\text C}(\overline q)
\end{equation*}

\subsection{Suscettività dielettrica per un gas omogeneo di elettroni}
Possiamo pensare di calcolare la suscettività dielettrica per un gas omogeneo di elettroni, tuttavia non possiamo risolverla in maniera analitica, ma possiamo introdurre alcune approssimazioni. La più semplice delle approssimazione che si può fare è stivare $\varepsilon(\overline q)$ nel \textbf{modello di Thomas-Fermi}. Ricordiamo che l'energia è espressa come:
\begin{equation*}
    E\big[n(\overline x)\big]=\int \dd[3]{x} V_{\text{ext}}(\overline x)n(\overline x)+E_{\text H}\big[n(\overline x)\big]+\int \dd[3]{x}n(\overline x)\frac 35 \varepsilon_{\text F}(n(\overline x))
\end{equation*}
Attraverso la minimizzazione di questo funzionale di energia possiamo trovare la densità che lo minimizza:
\begin{equation*}
    \functionalderivative{E}{n(\overline x)}=\mu
\end{equation*}
Ricordiamo che $\mu$ è un moltiplicatore di Lagrange perché abbiamo questo vincolo $\int \dd[3]{x} n(\overline x)=N$. Pertanto, introducendo un cambiamento nel potenziale esterno, avremo:
\begin{equation*}
    V_{\text{ext}} \rightarrow V_{\text{ext}}+\delta V_{\text{ext}}
\end{equation*}
\begin{equation*}
    n(\overline{x}) \rightarrow n(\overline x)+\delta n(\overline x)
\end{equation*}
L'equazione che minimizza il funzionale energetico porta all'equazione di Thomas-Fermi:
\begin{equation*}
    \varepsilon_{\text F}(n(\overline x))+V_{\text H}(\overline x) + V_{\text{ext}}(\overline x) = \mu
\end{equation*}
Abbiamo però un cambiamento su $V_{\text{ext}}$:
\begin{equation*}
    \varepsilon_{\text{F}}(n(\overline x)+\delta n(\overline x))+V_{\text H}+\delta V_{\text{H}}+V_{\text{ext}}+\delta V_{\text{ext}}=\mu
\end{equation*}
Siamo nella teoria della risposta lineare, pertanto il cambiamento è piccolo e possiamo svilupparlo al primo ordine. Ricordando che:
\begin{equation*}
    \varepsilon_{\text{F}}=\frac{\hbar^2}{2m}(3\pi^2n)^{\frac 13}
\end{equation*}
Possiamo espandere per piccoli cambiamenti:
\begin{equation*}
    \varepsilon_{\text{F}}(n(\overline x))+\frac{\hbar^2}{2m}(3\pi^2)^{\frac 23}\frac 23 \frac{n^{\frac 23}}{n}\delta n
\end{equation*}
Inserendo questo risultato troviamo che tutti i termini imperturbati si semplificano e rimangono solamente:
\begin{equation*}
    \frac 23 \frac{\varepsilon_{\text F}}{n}\delta n(\overline x)+\delta V_{\text H}(\overline x)+\delta V_{\text{ext}}(\overline x)=0
\end{equation*}
Ora vogliamo studiare il cambiamento del potenziale in un gas omogeneo di elettroni. Quando abbiamo discusso del gas omogeneo di elettroni, abbiamo introdotto la densità di stati $D(\varepsilon) \propto \sqrt{\varepsilon}$. In particolar modo abbiamo visto che:
\begin{equation*}
    D(\varepsilon_{\text{F}})=\frac 32 \frac n{\varepsilon_{\text F}}
\end{equation*}
In questo modo possiamo riscrivere l'equazione precedente come:
\begin{equation*}
    \frac{\delta n(\overline x)}{D(\varepsilon_{\text F})}+\delta V_{\text H}+\delta V_{\text{ext}}=0
\end{equation*}
Possiamo scrivere la stessa equazione nel reticolo reciproco utilizzando la trasformata di Fourier:
\begin{equation*}
    \frac{\delta n(\overline q)}{D(\varepsilon_{\text F})}+\delta n(\overline q)V_{\text C}(\overline q)+\delta V_{\text{ext}}(\overline q)=0
\end{equation*}
\begin{equation*}
    \delta n(\overline q)\Bigg[\frac{1}{D(\varepsilon_{\text{F}})}+V_{\text{C}}(\overline q)\Bigg]=-\delta V_{\text{ext}}(\overline q)
\end{equation*}
\begin{equation*}
    \delta n(\overline q)= -\frac{D(\varepsilon_{\text F})}{(1+D(\varepsilon_{\text{F}})V_{\text{C}}(\overline q))}\delta V_{\text{ext}}(\overline q)
\end{equation*}
Da cui ricaviamo:
\begin{equation*}
    \chi(\overline q)=-\frac{D(\varepsilon_{\text F})}{(1+D(\varepsilon_{\text{F}})V_{\text{C}}(\overline q))}
\end{equation*}
Se conosciamo $\chi(\overline q)$ possiamo ottenere:
\begin{equation*}
    \begin{aligned}
        \frac{1}{\varepsilon(\overline q)} &=1+\chi(\overline q)V_{\text{C}}(\overline q) \\
        & = 1-\frac{D(\varepsilon_{\text F})}{1+D(\varepsilon_{\text F})V_{\text C}(\overline q)}V_{\text C}(\overline q) \\
        & = \frac{1+D(\varepsilon_{\text F})V_{\text C}(\overline q)- D(\varepsilon_{\text F})V_{\text{C}}(\overline q)}{1+D(\varepsilon_{\text F})V_{\text{C}}(\overline q)} \\
        & = \frac{1}{1+D(\varepsilon_{\text F})V_{\text{C}}(\overline q)}
    \end{aligned}
\end{equation*}
Invertendo la relazione troviamo:
\begin{equation*}
    \begin{aligned}
        \varepsilon(\overline q) &=1+D(\varepsilon_{\text F})V_{\text{C}}(\overline q) \\
        & = 1+\frac{D(\varepsilon_{\text F})e_0^2}{\varepsilon_0 q^2} \\
        & = 1+ \frac{q_{\text{TF}}^2}{q^2}
    \end{aligned}
\end{equation*}
Dove abbiamo introdotto:
\begin{equation*}
    q_{\text{TF}}^2=\frac{D(\varepsilon_{\text{F}})e_0^2}{\varepsilon_0}=\frac{1}{\lambda_{\text{TF}}}
\end{equation*}
In particolar modo siamo interessati al suo inverso $\lambda_{\text{TF}}$ che prende il nome di \textbf{lunghezza di screening di Thomas-Fermi}. \\
Viene introdotto questo concetto perché possiamo calcolare
\begin{equation*}
    \delta V_{\text{tot}}(\overline q)=\frac{\delta V_{\text{ext}}(\overline q)}{\varepsilon(\overline q)}
\end{equation*}
Se consideriamo come potenziale esterno il potenziale generato da una carica puntiforme: $\delta V_{\text{ext}}(\overline q)=\frac{Ze_0^2}{\varepsilon_0 q^2}$, otteniamo:
\begin{equation*}
    \delta V_{\text{tot}}(\overline q)=\frac{Ze_0^2}{\varepsilon_0q^2\bigg(1+\frac{q_{\text{TF}}^2}{q^2}\bigg)}=\frac{Ze_0^2}{\varepsilon_0(q^2+q_{\text{TF}}^2)}
\end{equation*}
Applicando la trasformata di Fourier e il teorema dei residui (in coordinate sferiche), si trova:
\begin{equation*}
    \delta V_\text{tot}(\overline x)=\frac{1}{(2\pi)^3}\int\dd[3]{q}\frac{Ze_0^2}{\varepsilon_0(q^2+q_{\text{TF}}^2)}e^{i \overline q \cdot \overline x} = \frac{Ze_0^2}{4\pi\varepsilon_0r}e^{-\frac r {\lambda_{\text{TF}}}}
\end{equation*}
Ricordando $2 < r_{\text S} < 6$ e $\varepsilon_\text{F}$ inserendoli in $q_{\text{TF}}^2$ troviamo che:
\begin{equation*}
    \lambda_{\text{TF}}=r_{\text S}^{\frac 12}a_0\frac{2\pi}{\big(\frac{12}{\pi}\big)^\frac 13}\approx 4.01 r_{\text S}^{\frac 12}a_0
\end{equation*}
    \vspace{0.5cm}

\noindent  \lecture{8}{2/11/2021}

\section{Misurazioni indirette}\label{sec:mis_ind}

Per ora abbiamo analizzato il problema di \textit{cosa} avviene al sistema successivamente a una misurazione. Tuttavia non abbiamo ancora cercato di spiegare \textit{come} avviene una misurazione.
Generalmente, fra l'osservatore e il sistema quantistico, vi è un cosiddetto dispositivo classico di misura (\textit{classical measuring device})\footnote{Esempi di dispositivi di questo tipo sono le foto-emulsioni, la camera a nebbia di Wilson etc. etc.}. Le misurazioni in cui un dispositivo di misura classico interagisce propriamente con l'oggetto quantistico studiato vengono chiamate \textbf{misure dirette}. 
Chiaramente, in una misura di questo genere, stiamo utilizzando un sistema a molti gradi di libertà che vanno inevitabilmente a perturbare fortemente il sistema (abbiamo perturbazioni ben più grandi di quelle previste dai limiti di Heisenberg). Inoltre è raro che il sistema modifichi unicamente il grado di libertà studiato: con ogni probabilità porterà a una grossa perturbazione su una vasta gamma di osservabili.
Per questo si tende a preferire quelle che vengono chiamate \textbf{misure indirette}. Tali misurazioni sfruttano una sonda quantistica (\textit{quantum probe}) per mediare l'interazione sistema classico - sistema quantistico.
Abbiamo un processo a due fasi:
\begin{enumerate}
    \item la sonda quantistica interagisce con l'oggetto da misurare e fra di essi si stabilisce una correlazione;
    \item la sonda viene rilevata e misurata direttamente da un dispositivo classico.
\end{enumerate}
\noindent La sonda viene preparata in modo adeguato quando è ancora del tutto sconnessa al sistema e ha, in particolare, un osservabile detto \textbf{\textit{pointer observable}} che ha un legame con l'osservabile d'interesse nel sistema quantistico: ovvero c'è un rapporto (idealmente 1:1) fra i valori dei due osservabili.
Durante l'interazione sonda - sistema, d'altro canto, dobbiamo aspettarci che i due sistemi diventino \textit{entangled} e questo limiterà parzialmente la nostra misura finale: benché possiamo pensare di avere alte perturbazioni nella seconda fase del processo (perché non ci interessa conservare la sonda) dovremo limitarci per non perturbare troppo il sistema originale.
Per raggiungere un'alta precisione nel processo di misura cercheremo di rispettare sempre due semplici condizioni:
\begin{itemize}
    \item il secondo step della misurazione deve avvenire solo quando il primo risulta completato;
    \item il secondo step non deve contribuire significativamente all'errore totale della misura.
\end{itemize}
Se seguiamo queste regole, dunque, le uniche fonti di errore nella misura e le sole perturbazioni del sistema saranno quelle relative a incertezze intrinseche al processo di preparazione della sonda\footnote{Non stiamo considerando interazioni coi gradi di libertà ambientali.}.
\vspace{1cm}
\noindent Analizziamo, dunque, il processo di misura.
Abbiamo due differenti sistemi quantistici (la sonda e l'oggetto studiato) descritti dalle rispettive matrici densità $\hat \rho _P$ e $\hat \rho _{INIT}$. L'evoluzione di entrambe può essere descritta introducendo $\hat U (t) = e^{-\frac{i}{\hbar}\hat H t}$ dove $\hat H$ è l'hamiltoniana totale del sistema ($\hat H = \hat H_{probe} + \hat H _{obj} + \hat H _{interaction}$):
\begin{equation*}
    (\hat \rho_P \hat \rho_{init})'=\hat U (t) \hat \rho_P \hat \rho_{init} \hat U ^\dagger (t)
\end{equation*}
\noindent Abbiamo detto che il processo di misurazione non è unitario, ma a questo punto la non-unitarietà è tutta contenuta nella fase di interazione fra sonda e strumentazione di misura.
Lo stato finale della sonda, dopo l'interazione, è dato dalla seguente formula:
\begin{equation*}
    \hat \rho_{probe}' = \Tr_{obj} \left( \hat U \hat \rho_{probe} \hat \rho_{int} \hat U ^\dagger \right)
\end{equation*}
\noindent Solo a questo punto utilizziamo una misura proiettiva sulla sonda. 
Misuriamo il \textit{pointer observable} $P\tilde a$ nello spazio della sonda che è collegato con l'osservabile $A$ nello spazio dell'oggetto studiato. Avremo l'autostato dell'operatore puntatore: $\ket{\tilde a}$ che corrisponderà all'autovalore (e rispettivo autostato) $\tilde a$ per l'oggetto.
La probabilità di avere un certo autovalore sarà:
\begin{equation*}
    P(\tilde a) = \Tr \left( \ket{\tilde a}\bra{\tilde a} \hat \rho_{probe}'\right)
\end{equation*}
E la traccia che scriviamo qui è una traccia totale che opera nello spazio della sonda.
Possiamo riscrivere questa equazione:
\begin{equation*}
    P(\tilde a) = \Tr \left( \hat \Pi (\tilde a) \rho_{init} \right)
\end{equation*}
Dove, a questo punto, la traccia opera solo sullo spazio dell'oggetto e abbiamo introdotto l'operatore $\hat \Pi$ (anch'esso opera sullo spazio dell'oggetto)  che definiamo come:
\begin{equation*}
    \hat \Pi (\tilde a) = \Tr_{probe}\left( \hat U ^\dagger \ket{\tilde a}\bra{\tilde a}\hat U \hat \rho_{probe}  \right)
\end{equation*}
A questo punto l'operatore $\hat \Pi$ contiene lo stato in cui la sonda è preparata ($\hat \rho_{probe}$), lo stato finale della sonda ($\ket{\tilde a}\bra{\tilde a}$) e l'evoluzione temporale (che è l'unico operatore che operava sia sullo spazio della sonda, dipendenza eliminata dalla traccia parziale, che sullo spazio dell'oggetto).
Abbiamo visto la misura dal punto di vista della sonda, ma cosa sta succedendo al nostro oggetto?
Avevamo scritto l'equazione \ref{eq:prob_omega} rimandandone la dimostrazione che, però, vediamo ora.
Possiamo scrivere la matrice finale dell'oggetto (post interazione):
\begin{equation*}
    \hat \rho_{fin} (\tilde a) = \frac{1}{P(\tilde a)}\bra{\tilde a}\hat U \hat \rho_{probe}\hat \rho_{init} \hat U^\dagger \ket{\tilde a}
\end{equation*}
Se possiamo scrivere lo stato iniziale della sonda come: $\hat \rho_{probe} = \sum_i w_i \ket{\psi_i}\bra{\psi_i}$ arriviamo facilmente, per sostituzione, a:
\begin{equation*}
    \hat \rho_{fin} (\tilde a) = \frac{1}{P(\tilde a)} \sum_i \bra{\tilde a}\hat U \ket{\psi_i}\hat \rho_{init} \bra{\psi_i}\hat U ^\dagger \ket{\tilde a}
\end{equation*}
E ricordiamo che $\ket{\tilde a}$ e $\ket{\psi_i}$ sono stati della sonda e $\hat U$ agisce su entrambi gli stati. Dunque gli operatori $\bra{}\hat U \ket{}$ sono operatori dell'oggetto.
Se la sonda si trova, inizialmente, in uno stato puro $\hat \rho_{probe} = \ket{\psi}\bra{\psi}$, allora abbiamo:
\begin{equation*}
    \hat \rho_{fin} (\tilde a) = \frac{1}{P(\tilde a)}\bra{\tilde a}\hat U \ket{\psi}\hat \rho_{init} \bra{\psi}\hat U ^\dagger \ket{\tilde a}
\end{equation*}
E possiamo identificare:
\begin{equation*}
    \bra{\tilde a} \hat U \ket{\psi} = \hat \Omega = \hat U_{obj}\sqrt{\hat M}
\end{equation*}
Con $\hat U_{obj}$ (diverso dall'operatore $\hat U$ precedente) che contiene le informazioni sulla perturbazione dell'oggetto causata dalla sonda.
\vspace{0.5cm}
Abbiamo già visto che abbiamo una misurazione senza demolizione nel caso in cui $[\hat A, \hat \Omega]=0$ che ora possiamo riscrivere:
\begin{equation*}
    \bra{\tilde a}[\hat A, \hat \Omega] \ket{\psi}=0  
\end{equation*}
Tale equazione deve essere vera per ogni autostato $\ket{\tilde a}$, perciò:
\begin{equation*}
    (\hat A \hat U - \hat U \hat A)\ket{\psi}= 0
\end{equation*}
Se moltiplichiamo per $\hat U^\dagger$ otteniamo:
\begin{equation*}
    \hat U ^\dagger (\hat A \hat U - \hat U \hat A)\ket{\psi}= (\hat U^\dagger \hat A \hat U - \hat A ) \ket{\psi}=0
\end{equation*}
Otteniamo, dunque, questa equazione che è la condizione per avere una QND (\textit{Quantum NonDemolition measurement}). Si noti che l'operatore fra parentesi corrisponde alla variazione dell'osservabile $A$ nella rappresentazione di Heisenberg.
Perché questa condizione sia rispettata vi sono due possibilità:
\begin{itemize}
    \item che valga sempre $\hat U ^\dagger \hat A \hat U-\hat A = 0$;
    \item che lo stato iniziale della sonda sia un autostato della differenza $\hat U^\dagger \hat A \hat U - \hat A$.
\end{itemize}
Il secondo caso non è stato particolarmente analizzato né dal punto di vista sperimentale né da quello teorico, mentre ci si è concentrati sulla prima condizione.
Quest'ultima equivale a richiedere $[\hat A, \hat U] = 0$. Tale condizione è necessaria e sufficiente per avere una misura senza demolizione, ma è generalmente complesso verificare che sia verificata poiché è necessario conoscere l'evoluzione di oggetto e sonda.
Per questo si preferisce richiedere una condizione più forte (sufficiente, ma non necessaria): $[\hat A, \hat H]=0$.
Tipicamente avremo l'hamiltoniana scrivibile come:
\begin{equation*}
    \hat H = \hat H_{obj} + \hat H_{probe}+\hat H_{int}
\end{equation*}
Chiaramente abbiamo immediatamente la commutatività con la parte relativa alla sonda: $[\hat A, \hat H_{probe}]$=0 (poiché $\hat A$ è un osservabile dell'oggetto); mentre dovremo esplicitamente richiedere:
\begin{align}
    [\hat A , \hat H_{obj} ] &= 0\\
    [\hat A , \hat H_{int} ] &= 0
\end{align}
    %%%%%%%%%%%%%%%%%%%%%%%
%%%%%% Lezione 9 %%%%%%
%%%%%%%%%%%%%%%%%%%%%%%

\vspace{1.0cm}
\lecture{7}{5/11/2021}
\vspace{1.0cm}
L'ultima volta abbiamo parlato dello screening in un gas omogeneo di elettroni nell'approssimazione di Lindhard ottenendo
\begin{equation*}
    \varepsilon(q)=1-V_C\chi^0(q)=1+\frac{q_{\text{TF}^2}}{q^2}F\left(\frac{q}{q_{\text{TF}}}\right)
\end{equation*}
dove $q_{\text{TF}}$ è il vettore d'onda Thomas-Fermi e $F(x)=\frac 12 + \frac{1-x^2}{4x}\log\frac{\abs{1+x}}{\abs{1-x}}$. Questa funzione $F(x)$ è la stessa che avevamo incontrato quando abbiamo introdotto l'energia di scambio in un gas omogeneo di elettroni e possiede un punto di flesso a tangente verticale. Analizziamo da un punto di vista fisico $\varepsilon(q)$:
\begin{itemize}
    \item Per valori piccoli di $q$, cioè per grandi lunghezze d'onda, otteniamo l'approssimazione di Thomas Fermi, $V_{\text{ext}}$ varia lentamente;
    \item Per valori grandi di $q$, cioè per piccole lunghezze d'onda, abbiamo che lo screening è meno forte e gli elettroni non possono riorganizzarsi per schermare la perturbazione.
\end{itemize}
Possiamo spiegare questa rapida decrescita dello screening per $q>k_{\text F}$ tornando all'espressione
\begin{equation*}
    \chi^0(q)=\frac{2}{(2\pi)^3}\int_{\mathbb{R}^3}\dd[3]{k}\frac{f_{\overline k}-f_{\overline k+q}}{\varepsilon_{\overline k}-\varepsilon_{\overline k+q}}
\end{equation*}
notiamo che c'è un grande contributo a $\chi^0(q)$ quando il denominatore è nullo, quando andiamo verso il punto di singolarità e l'integrale non diverge. Rappresentiamo la \textbf{sfera di Fermi} e facciamo alcune considerazioni:
\begin{itemize}
    \item Se $\overline k$ è interno alla sfera di Fermi, perché dobbiamo iniziare dagli stati occupati, possiamo considerare questa sorta di transizione virtuale verso gli stati vuoti, fuori dalla sfera di Fermi. Questo è il processo che contribuisce all'integrale.
    \item La differenza in energia tra lo stato iniziale e finale può tendere a zero, quando il punto iniziale è esattamente sulla superficie della sfera così da poter avere una transizione al di fuori della sfera di Fermi che sia poco distante dal punto iniziale.
    \item Possiamo raggiungere gli stati vuoti con un'energia che è la stessa del punto iniziale, per qualunque valore di $q$ fino al diametro di questa sfera, perché se $q>2k_{\text F}$, dobbiamo andare molto lontani dalla sfera di Fermi e l'energia non è più uguale all'energia di Fermi.
\end{itemize}
Per valori di $q \leq 2k_{\text F}$, possiamo avere questa eccitazione virtuale che coinvolge gli stati con la stessa energia, cioè stati che contribuiscono maggiormente al valore di $\chi^0(q)$.
Per valori di $q > 2k_{\text F}$, non ci sono stati con la stessa energia per cui $\chi^0(q)$ inizia a decrescere.\\
Conoscendo $\varepsilon(q)$, possiamo calcolare la variazione $\delta n (q)$, questo perché
\begin{equation*}
    \begin{aligned}
        \delta n(q) &= \chi(q)\delta V_{\text{ext}}(q) \qquad \text{Consideriamo }\delta V_{\text{ext}}(\overline x)=\frac{Ze_0^2}{4\pi\varepsilon_0\abs{\overline x}}\\
                     &= \left[\frac{1}{\varepsilon(q)}-1\right]\frac{1}{\frac{e_0^2}{\varepsilon_0 q^2}}\frac{Ze_0^2}{\varepsilon_0q^2} \\
                     &= \left[\frac{1}{\varepsilon(q)-1}\right]Z
    \end{aligned}
\end{equation*}
ricordando che
\begin{equation*}
    \varepsilon(q)=1+\frac{q_{\text {TF}^2}}{q^2}F\left(\frac{q}{2k_{\text F}}\right)
\end{equation*}
e inserendola in $\delta n(x)$:
\begin{equation*}
    \delta n(x)=\frac{1}{(2\pi)^3}\int \dd[3]q \delta n(q)e^{iq\cdot x}
\end{equation*}
Nel limite di grandi valori di $x$, il che significa grandi rispetto alle lunghezze d'onda tipiche $x\gg \frac{1}{k_{\text F}}$ si può dimostrare che
\begin{equation*}
    \delta n(x)\rightarrow \frac{Z}{\abs{x}^3}\cos (2k_{\text F}r)
\end{equation*}
Queste oscillazioni vengono chiamate \textbf{oscillazioni di Friedel} e sono presenti se stiamo considerando l'approssimazione di Lindhard.
[GRAFICI]
Queste oscillazioni di Friedel possono essere osservate sperimentalmente attraverso l'utilizzo dell'\textbf{STM}: \textbf{Scanning Tunneling Microscope} inventato da Binnig e Rohrer nel 1985 all'IBM di Zurigo. Si tratta di un potente strumento per lo studio delle superfici a livello atomico che sfrutta l'effetto tunnel. Quando una punta conduttrice è portata molto vicino alla superficie da esaminare, una differenza di potenziale applicata tra i due può permettere agli elettroni di attraversare il vuoto tra di loro per effetto tunnel. La corrente di tunnelling che ne risulta dipende dalla posizione della punta, della tensione applicata e della densità locale degli stati del campione. \\
Simili effetti di screening possono essere visti tra due elettroni in interazione. Se guardiamo a un elettrone in una determinata posizione, ci aspettiamo di avere uno svuotamento di elettroni attorno a questo elettrone per via delle repulsioni elettrostatiche (descritto dall'energia di correlazione). Questo svuotamento di elettroni agisce come uno screening, infatti se consideriamo l'interazione tra due elettroni non è dato solamente dal potenziale di Coulomb, ma è anche dato anche dallo screening degli altri elettroni. \\
Tutto ciò che abbiamo imparato può essere utilizzato per descrivere un gas omogeneo di elettroni a livello del metodo di Hartree-Fock
\begin{equation*}
    \varepsilon_{\text{HF}}(k)=\frac{\hbar^2k^2}{2m}-\frac{e_0^2}{4\pi\varepsilon_0}\frac 1V \sum_{k'<k_{\text F}} \underbrace{\int \dd[3]{x}\frac{e^{i(\overline k - \overline k')\cdot \overline x}}{\abs{\overline x}}}_{\mathclap{\text{Trasformazione di Fourier di }V_C}}
\end{equation*}
L'idea è quella di schermare questo $V_C$ con ciò che abbiamo imparato. La trasformata di Fourier di $V_C$ è
\begin{equation*}
    \int \dd[3]{x}\frac{4\pi}{q^2 \varepsilon(q)}\frac{e^{i(\overline k - \overline k')\cdot \overline x}}{\abs{\overline x}} \qquad q=(k-k')
\end{equation*}
In questo modo, abbiamo una modifica all'energia di Hartee-Fock che risolve la divergenza di $v_k$ e questo è stato reso possibile dall'approssimazione di Lindhard. In un certo senso è come se avessimo introdotto in maniera fenomenologica la correlazione, ma questo non è ancora sufficiente per una buona descrizione del gas omogeneo di elettroni. Questo screening è una descrizione statica dello spazio, dovremmo considerare uno screening dinamico.
Supponiamo di considerare $\delta V_{\text{ext}}(x', t')$ che comporta ad una variazione nella densità di elettroni
\begin{equation*}
    \delta n(x, t)=\int \dd[3]{x'}dt'\chi(x,t, x', t')\delta V_{\text{ext}}(x', t') \qquad t'<t
\end{equation*}
Nel caso di un gas omogeneo di elettroni $\chi(\abs{x-x'}, t-t')$. Possiamo considerare ora di realizzare una trasformazione di Fourier di questa quantità
\begin{equation*}
    \delta n(\overline x,t)= \frac{1}{(2\pi)^3}\int \dd[3]{q} \int \frac{\dd{\omega}}{2\pi}\delta n(q, \omega) e^{iqx}e^{-i\omega t}
\end{equation*}
dove $\omega$ è il coniugato del tempo e il segno meno è convenzionale.\\
Ancora una volta, il cambiamento nella densità è una convoluzione tra due funzioni
\begin{equation*}
    \delta n(q, \omega)=\chi(q, \omega)\delta V_{\text{ext}}(q, \omega)
\end{equation*}
Considerando l'approssimazione di Lindhard ($\delta V_{\text{tot}}=\delta V_{\text{ext}}+\delta V_{\text H}$)
\begin{equation*}
    \delta n(q, \omega)=\chi^0(q, \omega)\delta V_{\text{tot}}(q, \omega)
\end{equation*}
dove $\chi^0(q, \omega)$ è valutato nell'approssimazione RPA (random phase approximation) come la risposta in tempo e spazio di un indipendente gas omogeneo di elettroni il cui calcolo è effettuato in regime perturbativo (time-dependent). Se utilizzassimo il metodo di Hartree-Fock introducendo l'approssimazione RPA time-dependent otterremmo una buona descrizione per l'energia di correlazione. Ma questo non può essere fatto perché abbiamo trattato hamiltoniane indipendenti dal tempo. Per introdurre uno screening dinamico, dovremmo introdurre le funzioni di Green. Se utilizzassimo uno schema più sofisticato rispetto a quello di Hartree-Fock che sfrutta le funzioni di Green e l'approssimazione di Lindhard, otterremmo l'energia di correlazione di Gell-Mann e Brueckner.\\
Tornando allo screening time-independent abbiamo visto che $\chi(q)$, nel caso di un gas omogeneo di elettroni dipende dalla distanza $\chi(\abs{\overline x'-\overline x})$. Per non un gas omogeneo di elettroni, ad esempio nel particolare caso di un cristallo $\chi(\overline x, \overline x')=\chi(\overline x + \overline n, \overline x' + \overline n)$, dove $\overline n$ è il vettore di un reticolo. Questa periodicità spaziale, si riflette nella trasformazione di Fourier, abbiamo un'invarianza, infatti
\begin{equation*}
    \chi(\overline q, \overline q')=\chi(\overline q+ \overline G, \overline q+ \overline G')
\end{equation*}
dove $\overline q$ è la zona di Brilloun, mentre $\overline G$ e $\overline G'$ sono il reciproco del vettore reticolo.\\
In questo caso possiamo valutare $\varepsilon_{\text{RPA}}$ in un cristallo
\begin{equation*}
    \varepsilon_{\text RPA} \rightarrow \varepsilon_{\text RPA}(\overline q+ \overline G, \overline q+ \overline G')
\end{equation*}
\section{Teoria del funzionale densità}
Nelle sezioni precedenti, il problema dei molti elettroni è stato affrontato approssimando il più possibile l'esatta funzione d'onda a molti elettroni dello stato fondamentale. Nella teoria del funzionale densità l'enfasi si sposta dalla funzione d'onda dello stato fondamentale alla densità elettronica $n(\vec x)$ di un corpo di uno stato fondamentale, molto più gestibile. La teoria del funzionale  densità mostra che l'energia dello stato fondamentale di un sistema a molte particelle può essere espressa come un funzionale della densità a un corpo; la minimizzazione di questo funzionale consente in linea di principio la determinazione dell'effettiva densità dello stato fondamentale. Il successo della teoria è anche quello di fornire una ragionevole approssimazione del funzionale da minimizzare. La particolarità dell'approccio del funzionale densità alla teoria dei molti corpi, è quella di ottenere rigorosamente un'equazione di Schrodinger a un elettrone con un potenziale effettivo locale nello studio delle proprietà dello stato fondamentale dei sistemi a molti elettroni.\\
Questa teoria è basata su un teorema introdotto da Hohenberg e Kohn nel 1964\footnote{Hohenberg, P., \& Kohn, W. (1964). Inhomogeneous Electron Gas. Phys. Rev., 136, B864–B871.}
\begin{theorem}[\textbf{Teorema di Hohenberg e Kohn}]
    Consideriamo un sistema di $N$ elettroni interagenti un un campo elettrico esterno, descritto dall'hamiltoniana standard a molti elettroni
    \begin{equation*}
        \hat H = \hat T + \hat U_{\text{ee}} + \sum_{i=1}^N \hat V_{\text{ext}}(\vec{x}_i).
    \end{equation*}
    Siamo interessati a conoscere l'energia di stato fondamentale, in particolare l'energia è un funzionale della densità di elettroni nello stato fondamentale
    \begin{equation*}
        E\left[n(\vec x)\right] \qquad n(\vec x)= \text{densità di elettroni}.
    \end{equation*}
    Se conosciamo l'hamiltoniana siamo in grado di risolvere l'equazione agli autovalori
    \begin{equation*}
        \hat H \psi_{\text{GS}}=E_{\text{GS}}\psi_{\text{GS}},
    \end{equation*}
    e dalla $\psi_{\text{GS}}$ siamo in grado di ottenere $n(\vec x)$. Il \textbf{teorema di Hohenberg e Hohn} afferma che esiste una corrispondenza biunivoca tra la densità dello stato fondamentale di un sistema di $N$ elettroni e il potenziale esterno che agisce su di esso
    \begin{equation*}
        V_{\text{ext}}(\vec x)\longleftrightarrow n_{\text{GS}}(\vec x)
    \end{equation*}
    in questo senso, la densità elettronica allo stato fondamentale diventa la variabile di interesse, poiché da $n_{\text{GS}}(\vec x)$ ricaviamo $\hat V_{\text{ext}}(\vec x)$, da cui otteniamo $E_{\text{GS}}$. Avremo quindi che se $V_{\text{ext}}'\neq V_{\text{ext}}$, allora $n_{\text{GS}}(\vec x)'\neq n_{\text{GS}}(\vec x)$
\end{theorem}

\noindent Quali sono le conseguenze di questo teorema?
\begin{equation*}
    \begin{aligned}
        E_{\text{GS}}\left[n(\vec{x})\right] &=\int \dd[3]x \hat V_{\text{ext}}(\vec x)n(\vec x)+\underbrace{\mel{\psi_{\text{GS}}\left[n(\vec{x})\right]}{\hat T + \hat U_{\text{ee}}}{\psi_{\text{GS}}\left[n(\vec{x})\right]}}_{\text{È un funzionale di } n(\vec x)} \\
        &= \int \dd[3]{x}\hat V_{\text{ext}}(\vec x)n(\vec x)+F\left[n(\vec x)\right]
    \end{aligned}
\end{equation*}
dove $F\left[n(\vec x)\right]$ è un funzionale universale che non dipende da $\hat V_{\text{ext}}(\vec x)$. \\
Se scegliessimo $n(\vec x)=n_{\text{GS}}(\vec x)$ otterremmo l'energia dello stato fondamentale. In realtà possiamo considerare $E_{\text{GS}}\left[n(\vec{x})\right]$ perché se $n(\vec x)\neq n_{\text{GS}}(\vec x)$, allora possiamo vederlo come risultato di un cambiamento nel potenziale esterno $\hat V_{\text{ext}}(\vec x)'\neq \hat V_{\text{ext}}(\vec x)$. Supponiamo di calcolare questo come
\begin{equation*}
    E\left[n(\vec x)\right])= \mel{\psi_{\text{GS}}}{\hat T + \hat U_{\text{ee}+V_{\text{ext}}}}{\psi_{\text{GS}}}
\end{equation*}
possiamo vedere gli altri stati come associati ad altri stati fondamentali e quindi, per il principio variazionale
\begin{equation*}
    E\left[n(\vec x)\right]) \geq E_{\text{GS}}\left[n_\text{GS}(\vec{x})\right].
\end{equation*}
Possiamo minimizzare il funzionale con il vincolo sul numero totale di elettroni $\int \dd[3]{x}n(\vec x)=N$:
\begin{equation*}
    \functionalderivative{E\left[n(\vec x)\right]-\mu\left[\int \dd[3]{x'}n(\vec x')- N\right]}{n(\vec x)}=0
\end{equation*}
il cui risultato è
\begin{equation*}
    \functionalderivative{E}{n(\vec x)}=\hat V_{\text{ext}}(\vec x)+\functionalderivative{F}{n(\vec x)}=\mu
\end{equation*}
cioè al moltiplicatore di Lagrange.\\
Trovare l'energia di stato fondamentale esatta necessita conoscere $F$, ma il teorema non ci dice nulla sulla sua forma, si limita ad affermare la sua esistenza. Nel \textbf{modello di Thomas Fermi} abbiamo visto una forma di $F$, esso risulta essere espresso come
\begin{equation*}
    F\left[n(\vec x)\right]=E_{\text H}\left[n(\vec x)\right]+\int \dd[3]{x} n(\vec x) \frac 35 \varepsilon_{\text F}\left(n(\overline x)\right)+\underbrace{\int \dd[3]x n(\vec x)\varepsilon_X(n(\vec x))}_{\text{Termine di Dirac}}
\end{equation*}
Tuttavia, l'energia cinetica quantistica risulta essere troppo approssimata, vedremo come costruire un funzionale più affidabile.\\
Limitiamoci per ora a dimostrare il \textbf{teorema di Hohenberg e Kohn}:
\begin{proof}
    Il teorema afferma che esiste una relazione biunivoca tra $n_{\text{GS}}(\vec x)$ e $\hat V_{\text{ext}}(\vec x)$. Questo significa che se $V_{\text{ext}}'\neq V_{\text{ext}}$, allora $n_{\text{GS}}(\vec x)'\neq n_{\text{GS}}(\vec x)$. Proseguiamo per assurdo e richiediamo che se $V_{\text{ext}}'\neq V_{\text{ext}}$, allora $n_{\text{GS}}(\vec x)' = n_{\text{GS}}(\vec x)$, quindi $\psi_{\text{GS}}'\neq \psi_{\text{GS}}$. Mostriamo che è una contraddizione.\\
    Andiamo a valutare i valori di aspettazione
    \begin{equation*}
        \begin{array}{l}
            \mel{\psi_{\text{GS}}'}{\hat T+ \hat U_{\text{ee}}+\hat V_{\text{ext}}}{\psi_{\text{GS}}'} \\
            \mel{\psi_{\text{GS}}}{\hat T+ \hat U_{\text{ee}}+\hat V_{\text{ext}}}{\psi_{\text{GS}}},
        \end{array}
    \end{equation*}
    chiamiamo
    \begin{equation*}
        \begin{array}{l}
            E'=\mel{\psi_{GS}'}{\hat T + \hat U_{\text{ee}}}{\psi_{GS}'} \\
            E=\mel{\psi_{GS}}{\hat T + \hat U_{\text{ee}}}{\psi_{GS}},
        \end{array}
    \end{equation*}
    avremo quindi che
    \begin{equation*}
        E'+\mel{\psi_{GS}'}{\hat V_{\text{ext}}}{\psi_{GS}'} > E+\mel{\psi_{GS}}{\hat V_{\text{ext}}}{\psi_{GS}}
    \end{equation*}
    se $n_{\text{GS}}(\vec x)' = n_{\text{GS}}(\vec x)$ allora
    \begin{equation*}
        E' > E
    \end{equation*}
    Ora consideriamo
    \begin{equation*}
        \begin{array}{l}
            \mel{\psi_{\text{GS}}}{\hat T+ \hat U_{\text{ee}}+\hat V_{\text{ext}}'}{\psi_{\text{GS}}} \\
            \mel{\psi_{\text{GS}}'}{\hat T+ \hat U_{\text{ee}}+\hat V_{\text{ext}}'}{\psi_{\text{GS}}'},
        \end{array}
    \end{equation*}
    chiamiamo
    \begin{equation*}
        \begin{array}{l}
            E'=\mel{\psi_{GS}'}{\hat T + \hat U_{\text{ee}}}{\psi_{GS}'} \\
            E=\mel{\psi_{GS}}{\hat T + \hat U_{\text{ee}}}{\psi_{GS}}
        \end{array}
    \end{equation*}
    avremo quindi che
    \begin{equation*}
        E+\mel{\psi_{GS}}{\hat V_{\text{ext}}'}{\psi_{GS}} > E'+\mel{\psi_{GS}'}{\hat V_{\text{ext}}'}{\psi_{GS}'}
    \end{equation*}
    se $n_{\text{GS}}(\vec x)' = n_{\text{GS}}(\vec x)$  allora
    \begin{equation*}
        E > E'
    \end{equation*}
    Ma questo è il risultato opposto rispetto al precedente, abbiamo quindi trovato un assurdo, per cui se $V_{\text{ext}}'\neq V_{\text{ext}}$, allora $n_{\text{GS}}(\vec x)'\neq n_{\text{GS}}(\vec x)$
\end{proof}
    %%%%%%%%%%%%%%
% LECTURE 10 %
%%%%%%%%%%%%%%
\vspace{1cm}
\noindent\lecture{10}{08/11/2021}
\vspace{0.5cm}

\section{Interazione con l'ambiente}
Focalizziamo la nostra attenzione sul discorso riguardante i qubit e sul modo con cui interagiscono con l'ambiente in cui sono immersi, ossia come sono influenzati dal rumore, dalla temperatura, ecc. Lo spazio di Hilbert generale è il prodotto tensoriale tra quello del qubit e quello dell'ambiente: $\mathcal{H} = \mathcal{H}_q \otimes \mathcal{H}_E$ (pedice $E$ per "environment"). Come visto dai postulati della QM, possiamo assumere che l'evoluzione temporale in $\mathcal{H}$ sia unitaria: se assumiamo uno stato puro iniziale $\ket{\psi} \in \mathcal{H}$, allora esso evolverà in $\ket{\psi}' = U \ket{\psi}$ dove $U$ è un operatore unitario. 

\noindent Come evidenziato nella sezione precedente, non vogliamo studiare il qubit mantenendo tutti i gradi di libertà dell'ambiente, quindi possiamo calcolare una traccia parziale su $\mathcal{H}_E$: si ricordi che anche se $\ket{\psi}$ è uno stato puro, alla fine otteniamo una matrice densità $\rho$ che descrive il sottosistema $\mathcal{H}_q$. Ad esempio, se il qubit è preparato nella sovrapposizione $\ket{\psi} = a \ket{0} + b \ket{1}$, dal punto di vista della fisica del qubit esso si trova in uno stato puro; quando si considera tuttavia l'intero sistema costituito anche dall'ambiente, la traccia parziale su $\mathcal{H}_E$ produce una matrice densità del qubit tale che possa corrispondere ad una miscela di stati nonostante si partisse da uno stato puro!

\noindent Un punto importante da tenere sempre in considerazione è che l'evoluzione temporale del sottosistema del qubit non è necessariamente unitaria: supponiamo ad esempio che l'evoluzione di $\mathcal{H}$ sia descritta dal seguente circuito
\begin{center}
    \mbox{
        \Qcircuit @C=2em @R=1em {
            \lstick{\text{Qubit}} & \multigate{1}{U} & \gate{} & \rstick{\ket{\psi}} \qw \\
            \lstick{\text{Ambiente}} & \ghost{U} & \qw & \qw \gategroup{1}{3}{1}{3}{.7em}{--}
        }
    }
\end{center}
dove l'evoluzione totale $U$ è unitaria. Il problema è che il risultato $\ket{\psi} \in \mathcal{H}_q$ dell'evoluzione del qubit potrebbe derivare da un operatore non unitario di $\mathcal{H}_q$! (si veda il gate ignoto nel riquadro tratteggiato).
 
\noindent Uno dei modi per realizzare un qubit è quello di considerare un atomo che presenta due livelli energetici vicini tra loro, ma facilmente isolabili rispetto ai livelli restanti. Chiamiamo $\ket{1}$ e $\ket{0}$ lo stato eccitato e il ground state rispettivamente. Un fenomeno che accade spontaneamente in natura è l'\textbf{emissione spontanea} di fotoni
\begin{center}
    \mbox{
        \Qcircuit @C=2em @R=2em {
            & \qw & \rstick{\ket{1}} \qw \\
            & \raisebox{.3em}{\begin{huge}$\downarrow$\end{huge}} & \raisebox{.05em}{\begin{huge} $\rightsquigarrow$ \end{huge}}\\
            & \qw & \rstick{\ket{0}} \qw
        }
    }
\end{center}
\vspace{0.2cm}
(la freccia verticale indica la transizione $\ket{1} \rightarrow \ket{0}$, mentre quella orizzontale indica l'emissione del fotone). In generale questo accade sempre se si aspetta un tempo sufficientemente lungo. Chiaramente il fotone emesso viene perso nell'ambiente, quindi dal punto di vista del qubit, $\mathcal{H}_q$ è lo spazio di Hilbert dei livelli energetici, mentre la radiazione ambientale di background, che è parte del campo elettromagnetico quantizzato, costituisce l'ambiente in cui il qubit è immerso. Come detto sopra, dal punto di vista del sistema totale l'evoluzione è unitaria, tuttavia in $\mathcal{H}_q$ l'emissione spontanea è descritta dal seguente operatore $\tilde U$
\begin{equation*}
    \tilde U \, : \, 
    \begin{cases}
        \ket{1} \rightarrow \ket{0} \\
        \ket{0} \rightarrow \ket{0}
    \end{cases} \, , \quad \Rightarrow \quad 
    \tilde U = 
    \begin{pmatrix}
        0 & 1 \\ 0 & 0 
    \end{pmatrix} \, ,
\end{equation*}
e chiaramente la matrice $\tilde U$ non è unitaria!

\noindent Diamo uno sguardo più dettagliato a questi processi generali. Supponiamo che $\mathcal{H}$ sia un sistema \textbf{chiuso} descritto da un evoluzione unitaria $U$ tale che
\begin{align*}
    \ket{\psi} &\rightarrow U \ket{\psi} \, , \\
    \bra{\psi} &\rightarrow \bra{\psi} U^\dag \, ;
\end{align*}
usando il formalismo della matrice densità possiamo scrivere $\rho = \ketbra{\psi}$: come evolve $\rho$ a seguito di evoluzioni unitarie degli stati? Semplicemente
\begin{equation*}
    \rho = \ketbra{\psi} \rightarrow U \ketbra{\psi} U^\dag = U \rho \, U^\dag \, ,
\end{equation*}
quindi la matrice densità evolve per \textbf{coniugazione}. Questo vale anche nel caso di miscele, infatti
\begin{equation*}
    \rho = \sum_i \rho_i \ketbra{\psi_i} \rightarrow \sum_i \rho_i \, U \ketbra{\psi_i} U^\dag = U \rho \, U^\dag \, .
\end{equation*}
Ritorniamo al sistema totale descritto da $\tilde{\mathcal{H}} = \mathcal{H}_q \otimes \mathcal{H}_E$: chiamiamo $\tilde \rho$ la matrice densità totale di $\tilde{\mathcal{H}}$, la quale può descrivere sia stati puri sia miscele (in generale a seguito della presenza di una temperatura finita si parte sempre con una miscela di stati). Scriviamo $\tilde \rho = \rho \otimes \rho_E$, dove chiaramente $\rho$ è la matrice densità in $\mathcal{H}_q$. Cosa succede se si aspetta un tempo abbastanza lungo? Analizziamo la seguente situazione
\begin{center}
    \mbox{
        \Qcircuit @C=2em @R=1em {
            \lstick{\text{Qubit: } \; \rho} & \multigate{1}{U} & \gate{} & \rstick{\mathcal{E}(\rho)} \qw \\
            \lstick{\text{Ambiente: } \; \rho_E} & \ghost{U} & \qw & \qw \gategroup{1}{3}{1}{3}{.7em}{--}
        }
    }
\end{center}
dove $\mathcal{E}(\rho)$ è l'evoluto di $\rho$ in $\mathcal{H}_q$. Sappiamo che se il sistema totale qubit-ambiente è chiuso allora $\tilde \rho$ evolverà per coniugazione come $\tilde \rho \rightarrow U (\rho \otimes \rho_E) U^\dag$. Come al solito, per ignorare i gradi di libertà dell'ambiente prendiamo una traccia parziale su di esso: definiamo la matrice densità
\begin{equation}\label{evolution_rho_qubit}
    \mathcal{E}(\rho) = \Tr_E \left[ U (\rho \otimes \rho_E) U^\dag \right] \, ,
\end{equation}
che dà una descrizione efficace del qubit e ne descrive la fisica dal suo punto di vista. La mappa $\rho \rightarrow \mathcal{E}(\rho)$ non è unitaria in generale! Come possiamo caratterizzare l'evoluzione dal punto di vista del qubit? Abbiamo detto che
\begin{center}
    \mbox{
        $
        \begin{matrix}
             \\
             \\
            \Large\substack{\text{Sistema} \\ \text{completo}}: \\
        \end{matrix}
        $
        $
        \begin{matrix}
             \\
             \\
            \qquad \\
        \end{matrix}
        $
        $
        \begin{matrix}
             \\
             \\
            \qquad \\
        \end{matrix}
        $
        $
        \begin{matrix}
             \\
             \\
            \qquad \\
        \end{matrix}
        $
        \raisebox{-.95em}{
            \Qcircuit @C=1em @R=1em {
                \lstick{\tilde \rho} & \gate{U} & \rstick{U \tilde \rho \, U^\dag \; ,} \qw
            }
        }
    }
    \qquad \qquad \qquad
    \mbox{
        $
        \begin{matrix}
             \\
             \\
            \Large\substack{\text{Sistema} \\ \text{ridotto}}: \\
        \end{matrix}
        $
        $
        \begin{matrix}
             \\
             \\
            \qquad \\
        \end{matrix}
        $
        $
        \begin{matrix}
             \\
             \\
            \qquad \\
        \end{matrix}
        $
        $
        \begin{matrix}
             \\
             \\
            \qquad \\
        \end{matrix}
        $
        $
        \begin{matrix}
             \\
             \\
            \qquad \\
        \end{matrix}
        $
        $
        \begin{matrix}
             \\
             \\
            \qquad \\
        \end{matrix}
        $
        \Qcircuit @C=1em @R=1em {
            \lstick{\rho} & \multigate{1}{U} & \rstick{\mathcal{E}(\rho)} \qw \\
            \lstick{\rho_E} & \ghost{U} & \qw
        }
        \raisebox{-1.2em}{\qquad \quad ,}
    }
\end{center}
Supponiamo che $\mathcal{H}_E$ possieda la base $\{ \ket{e_k} \}$, allora la \eqref{evolution_rho_qubit} diventa 
\begin{equation*}
    \mathcal{E}(\rho) = \sum_k \expval{U (\rho \otimes \rho_E) U^\dag}{e_k} \, ,
\end{equation*}
dove sottolineiamo che la notazione non deve trarre in inganno perché $U$ e $U^\dag$ agiscono sia sul qubit sia sull'ambiente (il RHS rimane un operatore, non una somma di elementi di matrice). Supponiamo che l'ambiente sia in uno stato puro iniziale, che chiamiamo $\ket{e_0}$: possiamo sempre assumere questa condizione senza alcuna perdita di generalità grazie al teorema di purificazione; quindi $\rho_E = \ketbra{e_0}$. In questo modo la precedente diventa
\begin{equation*}
    \mathcal{E}(\rho) = \sum_k \mel{e_k}{U}{e_0} \rho \mel{e_0}{U^\dag}{e_k} \, ,
\end{equation*}
dove come prima notiamo che $\mel{e_k}{U}{e_0}$ e $\mel{e_0}{U^\dag}{e_k}$ non sono elementi di matrice ma rimangono operatori agenti sul qubit. Chiamiamo 
\begin{equation}\label{E_k}
    E_k \equiv \mel{e_k}{U}{e_0} \, ,
\end{equation}
questi operatori ottenuti come elementi di matrice parziali con $e_k$ e $e_0$: spesso vengono chiamati \textbf{operation elements}. In questo modo $\mathcal{E}(\rho)$ può essere scritto come
\begin{equation}\label{operator_sum_repr}
    \mathcal{E}(\rho) = \sum_k E_k \rho E_k^\dag \, ,
\end{equation}
il quale prende il nome di \textbf{operator-sum representation}; la mappa $\mathcal{E}(\rho)$, invece, prende il nome di \textbf{quantum operation}. Dato che $\mathcal{E}(\rho)$ è una matrice densità deve soddisfare la condizione $\Tr \mathcal{E}(\rho) = 1$:
\begin{equation*}
    \Tr \mathcal{E}(\rho) = \Tr \left( \sum_k E_k \rho E_k^\dag \right) = \sum_k \Tr \left( E_k \rho E_k^\dag \right) = \sum_k \Tr \left( \rho E_k^\dag E_k \right) \overset{!}{=} 1 \, ;
\end{equation*}
dove nell'ultimo passaggio abbiamo usato la proprietà ciclica della traccia. Dato che $\Tr \rho = 1$, allora gli operation elements devono soddisfare il vincolo seguente
\begin{equation}\label{constraint_E_k}
    \sum_k E_k^\dag E_k = \mathbb{I} \, .
\end{equation}
Notiamo che nella \eqref{constraint_E_k} si ha in generale $E^\dag_k E_k \neq \mathbb{I}$, quindi $E_k$ non sono unitari! 

\noindent Qual è l'interpretazione fisica della \eqref{operator_sum_repr}? Si assume che esistano numerosi processi, etichettati da $k$, ognuno dei quali è visto come un sottosistema in cui è eseguita una particolare evoluzione temporale, data da $\rho \rightarrow E_k \rho E_k^\dag$. Ancora una volta, non si tratta di una coniugazione perché gli $E_k$ non sono in generale unitari. Ciascun operatore $E_k$ descritto dalla \eqref{E_k} è visto come un salto degli stati dell'ambiente da $\ket{e_0}$ a $\ket{e_k}$. 

\noindent Cominciamo a capire meglio come l'ambiente interagisce con i qubit guardando degli esempi espliciti. Supponiamo un qubit nella sovrapposizione $a\ket{0} + b\ket{1}$ che interagisce con l'ambiente: gli stati possono essere invertiti, possono essere aggiunte delle fasi (cambiano i segni dei coefficienti) e più in generale i coefficienti stessi possono cambiare. Vediamo esplicitamente che cosa può succedere. 

\begin{esempio}[\textbf{Bit flip channel}]
    Il primo esempio che analizziamo è il caso in cui gli stati vengano invertiti: $\ket{0} \rightarrow \ket{1}$ e $\ket{1} \rightarrow \ket{0}$. Che cosa ci aspettiamo per gli operatori $E_k$? Chiaramente uno degli operatori può essere $X$, dato che agisce sui singoli qubit; tuttavia $X$ da solo non è sufficiente per soddisfare il vincolo \eqref{constraint_E_k}, quindi aggiungiamo anche l'identità: scriviamo
    \begin{equation*}
        E_0 = \sqrt{p} \, \mathbb{I} = \sqrt{p}
        \begin{pmatrix}
            1 & 0 \\ 0 & 1
        \end{pmatrix} \, , \qquad
        E_1 = \sqrt{1-p} X = \sqrt{1-p}
        \begin{pmatrix}
            0 & 1 \\ 1 & 0
        \end{pmatrix} \, ;
    \end{equation*}
    se imponiamo che $0 \leq p \leq 1$ allora possiamo interpretarla come una probabilità: $p$ è la probabilità che non accada nulla, mentre $1-p$ è la probabilità che lo stato venga invertito. Ricordando che $X^2 = \mathbb{I}$, ora il vincolo \eqref{constraint_E_k} è soddisfatto:
    \begin{equation*}
        E_0^\dag E_0 + E_1^\dag E_1 = p \mathbb{I} + (1-p) X^2 = \mathbb{I} \, .
    \end{equation*}
    Vediamo cosa accade esplicitamente alla matrice densità del qubit: usiamo la \eqref{operator_sum_repr} sulla \eqref{density_matrix_Pauli}
    \begin{align*}
        \rho \rightarrow \sum_k E_k \rho E_k^\dag &= E_0 \rho E_0^\dag + E_1 \rho E_1^\dag = p \frac{\mathbb{I} + \vec{r} \cdot \vec{\sigma}}{2} + (1-p) X \frac{\mathbb{I} + \vec{r} \cdot \vec{\sigma}}{2} X \\
        &= p \frac{\mathbb{I} + r_1 \sigma_1 + r_2 \sigma_2 + r_3 \sigma_3}{2} + (1-p) \frac{\mathbb{I} + r_1 \sigma_1 - r_2 \sigma_2 - r_3 \sigma_3}{2} \\
        &= \frac{\mathbb{I} + r_1 \sigma_1 + (2p-1) r_2 \sigma_2 + (2p-1) r_3 \sigma_3}{2} \equiv \mathcal{E}(\rho) \, ,
    \end{align*}
    dove nella seconda linea abbiamo utilizzato $\sigma_i \sigma_j = i \varepsilon_{ijk} \sigma_k$ per $i \neq j$. Esplicitamente, in funzione del punto $\vec{r}$ della sfera di Bloch avremo
    \begin{equation*}
        (r_1,r_2,r_3) \rightarrow \left[ r_1, (2p-1)r_2, (2p-1)r_3 \right] \, .
    \end{equation*}
    Come evidenziano i plot di Figura \ref{subfig:bit_flip_Bloch_1}, quando $p$ diminuisce, allora anche $2p-1$ diminuisce: più $p$ è prossima a $\frac{1}{2}$, più la sfera di Bloch risulta schiacciata lungo $y$ e $z$ (lungo $x$ rimane costante). In corrispondenza del valore $p = \frac{1}{2}$, come evidenziato nel plot \ref{subfig:bit_flip_Bloch_2}, la sfera collassa ad un segmento lungo $x$: dato che i punti sulla superficie della sfera erano gli stati $\ket{+}$ e $\ket{-}$, allora ogni generico punto del segmento consiste in una sovrapposizione classica (miscela) di $\ket{+}$ e $\ket{-}$ con pesi dati dalla distanza dai due estremi del segmento.    
    
    \begin{figure}[!ht]
	\centering	
	\subfloat[][\text{La sfera di Bloch è un ellissoide per }$\frac{1}{2} \leq p \leq 1$.\label{subfig:bit_flip_Bloch_1} ]{\includegraphics[scale=.37,keepaspectratio]{images/bit_flip_Bloch_1}} \quad
	\subfloat[][\text{Caso }$p=\frac{1}{2}$.\label{subfig:bit_flip_Bloch_2} ]{\includegraphics[scale=.37,keepaspectratio]{images/bit_flip_Bloch_2}}
	\caption{Cambiamento della sfera di Bloch nel bit flip channel al variare di $p$.}
    \label{fig:bit_flip_Bloch}
    \end{figure}
    \noindent Notiamo che, viceversa, al diminuire di $p$ con $0 \leq p \leq \frac{1}{2}$ la sfera di Bloch ritorna ad essere un ellissoide: passa dall'essere il segmento di Figura \ref{subfig:bit_flip_Bloch_2} fino alla sfera unitaria originale per $p= 0$. 
\end{esempio}


\begin{esempio}[\textbf{Phase flip channel}]\label{es:phase_flip}
    Si tratta del caso analogo all'esempio precedente in cui $X$ è sostituita da $Z$. In questa situazione sappiamo che $Z$ agisce come $a \ket{0} + b \ket{1} \rightarrow a \ket{0} - b \ket{1}$ (si ricordi che $-1 = e^{i \pi}$), quindi cambia la fase relativa tra gli stati (importante per fenomeni di interferenza). In questo caso gli operatori $E_k$ sono
    \begin{equation*}
        E_0 = \sqrt{p} \, \mathbb{I} = \sqrt{p}
        \begin{pmatrix}
            1 & 0 \\ 0 & 1
        \end{pmatrix} \, , \qquad
        E_1 = \sqrt{1-p} Z = \sqrt{1-p}
        \begin{pmatrix}
            1 & 0 \\ 0 & -1
        \end{pmatrix} \, ;
    \end{equation*}
    dove analogamente $1-p$ ha l'interpretazione della probabilità che lo stato possa subire un cambio di fase. Come nell'esempio precedente la trasformazione di $\rho$ sarà
    \begin{align*}
        \rho \rightarrow \sum_k E_k \rho E_k^\dag &= E_0 \rho E_0^\dag + E_1 \rho E_1^\dag = p \frac{\mathbb{I} + \vec{r} \cdot \vec{\sigma}}{2} + (1-p) Z \frac{\mathbb{I} + \vec{r} \cdot \vec{\sigma}}{2} Z \\
        &= p \frac{\mathbb{I} + r_1 \sigma_1 + r_2 \sigma_2 + r_3 \sigma_3}{2} + (1-p) \frac{\mathbb{I} - r_1 \sigma_1 - r_2 \sigma_2 + r_3 \sigma_3}{2} \\
        &= \frac{\mathbb{I} + r_3 \sigma_3 + (2p-1) r_1 \sigma_1 + (2p-1) r_2 \sigma_2}{2} \equiv \mathcal{E}(\rho) \, ;
    \end{align*}
    perciò in termini della sfera di Bloch avremo
    \begin{equation*}
        (r_1,r_2,r_3) \rightarrow \left[ (2p-1) r_1, (2p-1)r_2, r_3 \right] \, .
    \end{equation*}
    Le situazioni sono mostrate in Figura \ref{fig:phase_flip_Bloch}. Il caso è simile al precedente esempio, tuttavia l'ellissoide risulta questa volta schiacciato lungo $y$ e $x$ mantenendo $z$ costante. Quando $p = \frac{1}{2}$ la sfera collassa ad un segmento lungo $z$: a seconda del valore di $r_3$ lo stato corrisponde ad una miscela classica di $\ket{0}$ e $\ket{1}$. 
    \begin{figure}[!ht]
	\centering	
	\subfloat[][\text{La sfera di Bloch è un ellissoide per }$\frac{1}{2} \leq p \leq 1$.\label{subfig:phase_flip_Bloch_1} ]{\includegraphics[scale=.37,keepaspectratio]{images/phase_flip_Bloch_1}} \quad
	\subfloat[][\text{Caso }$p=\frac{1}{2}$.\label{subfig:phase_flip_Bloch_2} ]{\includegraphics[scale=.37,keepaspectratio]{images/phase_flip_Bloch_2}}
	\caption{Cambiamento della sfera di Bloch nel phase flip channel al variare di $p$.}
    \label{fig:phase_flip_Bloch}
    \end{figure}
    
    \noindent Esplicitamente per $p = \frac{1}{2}$ avremo
    \begin{equation*}
        \mathcal{E}(\rho) = \frac{\mathbb{I}+ \sigma_3 r_3}{2} = 
        \begin{pmatrix}
            \frac{1+r_3}{2} & 0 \\ 0 & \frac{1-r_3}{2}
        \end{pmatrix}
        = \frac{1+r_3}{2} \ketbra{0} + \frac{1-r_3}{2} \ketbra{1} \, .
    \end{equation*}
    Notiamo che questo fenomeno accade anche quando si parte da uno stato puro: se si partisse dallo stato $\ket{+}$ (intersezione della sfera di Bloch con l'asse positivo delle $x$), allora esso sarebbe spinto, al diminuire di $p$, verso l'origine fino a quando $\rho = \frac{\mathbb{I}}{2}$.
\end{esempio}

\noindent Il cambiamento della fase relativa a seguito dell'interazione con l'ambiente è correlato al fenomeno (lo analizzeremo in dettaglio più avanti) che va sotto il nome di \textbf{decoerenza}. Per capire di che cosa si tratta supponiamo di partire con lo stato puro $\ket{\psi}$, dato dalla sovrapposizione generica $\ket{\psi} = a \ket{0} + b \ket{1}$. In termini di $\rho$ abbiamo visto che
\begin{equation*}
    \rho = \ketbra{\psi} =
    \begin{pmatrix}
        a \\ b
    \end{pmatrix}
    \otimes \begin{pmatrix} a^\ast & b^\ast \end{pmatrix} =
    \begin{pmatrix}
        \abs{a}^2 & a b^\ast \\ a^\ast b& \abs{b}^2
    \end{pmatrix} \, ;
\end{equation*}
i termini non diagonali misurano la sovrapposizione degli stati $\ket{0}$ e $\ket{1}$, quindi hanno a che fare con la fase relativa dello stato. Se si aspetta un tempo sufficientemente lungo, il qubit interagirà con l'ambiente tramite i phase flip channel producendo la seguente matrice diagonale:
\begin{equation*}
    \rho = 
    \begin{pmatrix}
        \abs{a}^2 & a b^\ast \\ a^\ast b& \abs{b}^2
    \end{pmatrix}
    \rightarrow
    \mathcal{E}(\rho) =
    \begin{pmatrix}
        (\ldots) & 0 \\ 0 & (\ldots)
    \end{pmatrix} \, ;
\end{equation*}
si ottiene quindi una miscela di stati (non è più uno stato puro) in cui tutte le informazioni legate alle fasi relative (termini non-diagonali) vengono perse! Questo fenomeno è la \textbf{decoerenza}: l'interazione con l'ambiente tramite phase flip channel tende a sopprimere tutti i termini non-diagonali di $\rho$. 

\begin{esempio}[\textbf{Bit-phase flip channel}]
    Ricordando che $XZ = -i Y$, se combiniamo il bit flip channel con il phase flip channel possiamo scrivere l'interazione con l'ambiente dovuta ai seguenti operatori
    \begin{equation*}
        E_0 = \sqrt{p} \, \mathbb{I} = \sqrt{p}
        \begin{pmatrix}
            1 & 0 \\ 0 & 1
        \end{pmatrix} \, , \qquad
        E_1 = \sqrt{1-p} Y = \sqrt{1-p}
        \begin{pmatrix}
            0 & -i \\ i & 0
        \end{pmatrix} \, ;
    \end{equation*}
    la trasformazione di $\rho$ non è altro che
    \begin{align*}
        \rho \rightarrow \sum_k E_k \rho E_k^\dag &= E_0 \rho E_0^\dag + E_1 \rho E_1^\dag = p \frac{\mathbb{I} + \vec{r} \cdot \vec{\sigma}}{2} + (1-p) Y \frac{\mathbb{I} + \vec{r} \cdot \vec{\sigma}}{2} Y \\
        &= p \frac{\mathbb{I} + r_1 \sigma_1 + r_2 \sigma_2 + r_3 \sigma_3}{2} + (1-p) \frac{\mathbb{I} - r_1 \sigma_1 + r_2 \sigma_2 - r_3 \sigma_3}{2} \\
        &= \frac{\mathbb{I} + r_2 \sigma_2 + (2p-1) r_1 \sigma_1 + (2p-1) r_3 \sigma_3}{2} \equiv \mathcal{E}(\rho) \, ;
    \end{align*}
    e quindi la sfera di Bloch si modifica come in Figura \ref{fig:bit_phase_flip_Bloch}. 
    \begin{figure}[!ht]
	\centering	
	\subfloat[][\text{La sfera di Bloch è un ellissoide per }$\frac{1}{2} \leq p \leq 1$.\label{subfig:bit_phase_flip_Bloch_1} ]{\includegraphics[scale=.37,keepaspectratio]{images/bit_phase_flip_Bloch_1}} \quad
	\subfloat[][\text{Caso }$p=\frac{1}{2}$.\label{subfig:bit_phase_flip_Bloch_2} ]{\includegraphics[scale=.37,keepaspectratio]{images/bit_phase_flip_Bloch_2}}
	\caption{Cambiamento della sfera di Bloch nel bit phase flip channel al variare di $p$.}
    \label{fig:bit_phase_flip_Bloch}
    \end{figure}
\end{esempio}

\begin{esempio}[\textbf{Depolarizing channel}]
    Un altro possibile modo di interagire con l'ambiente porta al cosiddetto \textbf{canale di depolarizzazione} per il quale la matrice densità del qubit diventa
    \begin{equation}\label{rho_depolarizing_channel}
        \mathcal{E}(\rho) = (1-p) \rho + \frac{p}{3} \left( X \rho X + Y \rho Y + Z \rho Z \right) \, ,
    \end{equation}
    dove $\frac{p}{3}$ non è altro che la probabilità che l'ambiente interagisca tramite ognuno dei termini nella parentesi tonda. Se utilizziamo la formula 
    \begin{equation*}
        \frac{\mathbb{I}}{2} = \frac{\rho + X \rho X + Y \rho Y + Z \rho Z}{4} \, , \quad \forall \; \rho = \frac{\mathbb{I}+ \vec{r} \cdot \vec{\sigma}}{2} \, ,
    \end{equation*}
    la quale è facilmente dimostrabile con la proprietà $\sigma_i \sigma_j = i \varepsilon_{ijk} \sigma_k$ per $i \neq j$, allora la \eqref{rho_depolarizing_channel} può essere scritta come
    \begin{equation}
        \mathcal{E}(\rho) = \tilde p \frac{\mathbb{I}}{2} + (1 - \tilde p) \rho \, , \; \text{ con } \; \tilde p = \frac{4}{3} p \, .
    \end{equation}
    Quest'ultima formula asserisce che nel canale di depolarizzazione vi è una probabilità di $1-\tilde p$ che nulla accada a $\rho$ e una probabilità di $\tilde p$ che ci sia un improvviso salto da $\rho$ a $\frac{\mathbb{I}}{2}$, ossia alla situazione in cui lo stato è il più indeterminato o depolarizzato possibile. In termini di effetti sulla sfera di Bloch, la situazione è mostrata nella Figura \ref{fig:depolarizing_channel}: quando $\tilde p \to 1$ la sfera di Bloch viene "schiacciata" in tutte le 3 possibili direzioni fino a quando, per $\tilde p = 1$, collassa ad un punto nell'origine. 
    \begin{figure}[!ht]
	\centering	
	\subfloat[][\text{La sfera di Bloch si restringe per }$\tilde p \to 1$.\label{subfig:depolarizing_channel_1} ]{\includegraphics[scale=.37,keepaspectratio]{images/depolarizing_channel_1}} \quad
	\subfloat[][\text{Caso }$\tilde p = 1$.\label{subfig:depolarizing_channel_2} ]{\includegraphics[scale=.37,keepaspectratio]{images/depolarizing_channel_2}}
	\caption{Cambiamento della sfera di Bloch nel canale di depolarizzazione al variare di $\tilde p$.}
    \label{fig:depolarizing_channel}
    \end{figure}
\end{esempio}

\noindent La caratterizzazione più completa dell'interazione dei qubit con l'ambiente è offerta dai fenomeni dell'\textbf{amplitude damping} ("smorzamento dell'ampiezza") e \textbf{phase damping} ("smorzamento della fase"): il primo è associato alla perdita di energia, tipicamente a seguito dell'emissione spontanea di fotoni, mentre il secondo è dovuto alla perdita di fase, a causa degli scattering (in realtà sono una sorta di diffusioni) con le particelle dell'ambiente (gli stati rimangono tali, $\ket{0} \to \ket{0}$ e $\ket{1} \to \ket{1}$). Cominciamo con l'analisi del primo dei due.

%\subsection{Amplitude damping}
%In questa situazione si ha una perdita di energia del qubit nell'ambiente causata dall'emissione spontanea di fotoni, a seguito della quale il valore assoluto dei coefficienti dello stato %diminuisce. La situazione schematica è la seguente: 
%\begin{center}
%    $
%        \begin{matrix}
%             \\
%             \\
%            \text{Emissione spontanea}: \\
%        \end{matrix}
%        $
%    \raisebox{1.2em}{
%        \mbox{
%            \Qcircuit @C=2em @R=2em {
%                & \qw & \rstick{\ket{1}} \qw \\
%                & \raisebox{.3em}{\begin{huge}$\downarrow$\end{huge}} & \raisebox{.05em}{\begin{huge} $\rightsquigarrow$ \end{huge}}\\
%                & \qw & \rstick{\ket{0}} \qw
%            }
%        }
%    }
%\end{center}
%\vspace{0.2cm}
%quindi se aspettiamo un tempo sufficientemente lungo il generico stato $a \ket{0} + b \ket{1}$ collassa in $\ket{0}$ a seguito di $\abs{b}^2 \to 0$. Possiamo descrivere la fisica %dell'emissione spontanea del fotone dal punto di vista dell'ambiente nel modo seguente: assumiamo che esso presenti due stati tali che
%\begin{align*}
%    \ket{0}_E &= \text{Nessuna emissione di fotoni}. \\
%    \ket{1}_E &= \text{1 fotone emesso}. 
%\end{align*}
%In questo modo il sistema totale $\mathcal{H}_q \otimes \mathcal{H}_E$ evolve con l'evoluzione unitaria $U$ tale che
%\begin{equation}\label{variazione_stati_emissione_spontanea}
%    U \, : \; 
%    \begin{cases}
%        \ket{0} \otimes \ket{0}_E &\rightarrow  \ket{0} \otimes \ket{0}_E \\
%        \ket{1} \otimes \ket{0}_E &\rightarrow  \sqrt{1-p} \ket{1} \otimes \ket{0}_E + \sqrt{p} \ket{0} \otimes \ket{1}_E
%    \end{cases} \, ,
%\end{equation}
%dove risulta evidente che $p$ è la probabilità in cui si abbia emissione spontanea, mentre $1-p$ è la probabilità che non accade nulla. Notiamo che le trasformazioni %\eqref{variazione_stati_emissione_spontanea} sono unitarie perché è immediato verificare che il prodotto scalare è conservato. Cosa succede quando calcoliamo la traccia parziale su %$\mathcal{H}_E$? Dobbiamo calcolare gli operation elements \eqref{E_k} dove $\ket{e_0} \equiv \ket{0}_E$ e $\ket{e_1} \equiv \ket{1}_E$. Dalla forma delle trasformazioni %\eqref{variazione_stati_emissione_spontanea} è facile vedere che hanno la seguente espressione:
%\begin{equation}\label{E_k_spontaneous_emission}
%    E_0 = 
%    \begin{pmatrix}
%        1 & 0 \\ 0 & \sqrt{1-p}
%    \end{pmatrix} \, , \qquad
%    E_1 =
%    \begin{pmatrix}
%        0 & \sqrt{p} \\ 0 & 0
%    \end{pmatrix} \, .
%\end{equation}
    \vspace{1cm}
\newline
\lecture{11}{11/11/2021}
\noindent Come possiamo arrivare a una descrizione quantistica di qubit superconduttivi? Le giunzioni Josephson sono degli oggetti che hanno uno strano comportamento e abbiamo trattato questo effetto utilizzando un approccio classico poiché $\delta$ è trattabile solamente in modo classico. Per vedere effetti quantistici necessitiamo di sapere quando gli SQUIDS possono essere utilizzati come qubit. Per vedere effetti quantistici, dove la fase non è ben definita, trattiamo un basso numero di coppie di Cooper. L'energia associata alla capacità è data da
\begin{equation*}
    E_c=\frac{e^2}{2C} \, ,
\end{equation*}
per avere un comportamento quantistico $E_c\gg k_BT$. Procediamo con un esempio numerico
\begin{esempio}
    Supponiamo di essere a una temperatura di $18 \text{ mK}$, questo fornisce un'energia termica pari a $k_BT=10^{-6} \text{ eV}$. Affinché si abbia un comportamento quantistico, la capacità del condensatore deve essere $C \ll 0.1 \text{ pF}$. Cioè l'area della giunzione Josephson dovrà essere $10^{-6} \text{cm}^3$.
\end{esempio}
\noindent In principio noi vorremmo lavorare con temperature più alte, non solo, dobbiamo considerare la scala temporale del \textbf{principio di indeterminazione di Heisenberg}
\begin{equation*}
    \Delta E \Delta t = \frac \hbar 2 \, ,
\end{equation*}
quindi
\begin{equation*}
    E_c > \Delta E =\frac{\hbar}{2\Delta t} \, ,
\end{equation*}
siccome $\Delta t \sim RC$, $R$ deve essere superiore di $6 \text{ k}\Omega$, cioè $R=\frac{2\hbar}{(2e)^2}$. Questo è una possibile considerazione, altrimenti uno può valutare ad esempio l'energia della giunzione $E_J > k_BT$, dove $E_J = \frac{\hbar}{2e}I_0$, dove $I_0$ dipende dall'area. Il comportamento della giunzione utilizzata nei qubit dipende da $E_J$, $E_c$ e $k_BT$. L'idea di base per lavorare con i qubit è quindi avere una giunzione di piccole dimensioni e lavorare a basse temperature.

\section{Quantizzazione di un circuito LC}
Dal momento che vogliamo quantizzare un qubit necessitiamo di una lagrangiana da cui costruire la corrispondente hamiltoniana e, applicando la quantizzazione canonica, otteniamo il nostro sistema quantizzato.
Supponiamo di considerare un semplice circuito LC come mostrato in Figura \ref{fig:lc-circuit}.

\begin{figure}[!ht]
    \centering
    \begin{circuitikz}
        \draw
        (0,0)   to[C=$C$] ++ (0, 2) -- ++ ( 2,0) 
                to[L=$L$] ++ (0,-2) -- ++ (-2,0);
    \end{circuitikz}
    \caption{Circuito LC.}
    \label{fig:lc-circuit}
\end{figure}
\noindent In generale, in ciascun ramo abbiamo una tensione $V_b$ e una corrente $I_b$, ciò che vogliamo andare a calcolare sono $\Phi_b$ e $Q_b$, dove
\begin{equation*}
    \begin{aligned}
        \Phi_b(t)=\int_{-\infty}^t \dd{t'} V_b(t') \\
        Q_b(t)=\int_{-\infty}^t \dd{t'} I_b(t') \, .
    \end{aligned}
\end{equation*}
Ciascun elemento è caratterizzato da una relazione costitutiva che collega le variazioni di corrente e tensione. Dobbiamo distinguere tra elementi capacitivi per cui la relazione è della forma
\begin{equation*}
    V_b = f(Q_b) \, ,
\end{equation*}
ed elementi induttivi per cui la relazione è della forma
\begin{equation*}
    I_b = g(\Phi_b) \, .
\end{equation*}
Capacità e induttanze lineari usuali sono dei casi speciali in cui
\begin{equation*}
    \begin{aligned}
        f(Q_b) = \frac{Q_b}{C} \\
        g(\Phi_b) = \frac{\Phi_b}{L} \, .
    \end{aligned}
\end{equation*}
Un controesempio è dato dalla giunzione tunnel Josephson che è un elemento induttivo e la funzione $g(\Phi_b)$ è un seno. Tuttavia lavorare con i rami non è una scelta ottimale. Le leggi di Kirchhoff da risolvere sono sulla corrente e tensione, ma invece di utilizzare $I_b$ e $V_b$ usiamo la carica $Q_b$ e il flusso $\Phi_b$. Questo perché il flusso e la carica in un nodo sono uguali al flusso e alla carica in un ramo, anche se in generale non sono sempre lo stesso. Siccome
\begin{equation*}
        \delta = \frac{2e}{\hbar}\Phi \mod{2\pi} = 2\pi \frac{\Phi}{\Phi_0}
\end{equation*}
\begin{equation*}
        I_L = \frac{\Phi}{L} = \ddot{\Phi}C = I_C \, ,
\end{equation*}
avremo che la nostra lagrangiana sarà
\begin{equation*}
    L(\dot \Phi, \Phi) = C\frac{\dot{\Phi}^2}{2} - \frac{\Phi^2}{2L} \, .
\end{equation*}
Applicando la \textbf{trasformata di Legendre} con cui calcoliamo il momento coniugato
\begin{equation*}
    Q = \partialderivative{L}{\dot \Phi} = C\dot \Phi \, ,
\end{equation*}
possiamo scrivere ora l'hamiltoniana del nostro sistema:
\begin{equation*}
    H = Q\dot \Phi - L = \frac{Q^2}{2C} + \frac{\Phi^2}{2L} \, .
\end{equation*}
Avendo ora a disposizione l'hamiltoniana possiamo procedere ad una quantizzazione canonica in cui
\begin{align*}
    &\Phi \longrightarrow \hat \Phi \\
    &Q \longrightarrow \hat Q = -i\hbar \partialderivative{\Phi} \\
    &H \longrightarrow \hat H = \frac{\hat Q^2}{2C} + \frac{\hat \Phi^2}{2L}
\end{align*}
Come possiamo notare l'hamiltoniana quantizzata appena scritta assomiglia all'hamiltoniana di un oscillatore armonico, infatti possiamo notare un'analogia con
\begin{equation*}
    \hat H = \frac{\hat p^2}{2m} + \frac 12 m\omega^2\hat x^2 \, ,
\end{equation*}
\begin{align*}
    Q &\longleftrightarrow p \\
    \Phi &\longleftrightarrow x \\
    C &\longleftrightarrow m \\
    L &\longleftrightarrow \frac 1 k
\end{align*}
con pulsazione
\begin{equation*}
    \omega = \sqrt{\frac{k}{m}} = \frac{1}{\sqrt{LC}} \, .
\end{equation*}
In questo contesto, ricordando che
\begin{equation*}
    E_c = \frac{e^2}{2C} \qquad \text{e} \qquad Q = 2en \, ,
\end{equation*}
possiamo scrivere
\begin{equation*}
    \frac{Q^2}{2C} = \frac{4e^2n^2}{2C} = 4E_cn^2
\end{equation*}
Allo stesso modo
\begin{equation*}
    E_L = \left(\frac{\Phi_0}{2\pi}\right)^2\frac 1L \, ,
\end{equation*}
quindi 
\begin{equation*}
    \frac{\Phi^2}{2L} = \frac{\delta^2\Phi_0^2}{(2\pi)^2}\frac{1}{2L} = \frac{1}{2}\delta^2E_L \,
\end{equation*}
Un altro modo dunque di scrivere l'hamiltoniana è il seguente
\begin{equation*}
    \hat H = 4 E_C \hat n^2 + \frac 12 E_L \hat \delta^2 \, .
\end{equation*}
se prima $\comm{\hat \Phi}{\hat Q} = i\hbar$, ora abbiamo che $\comm{\hat \delta}{\hat n} = i$.
Conoscendo l'oscillatore armonico abbiamo che le soluzioni degli autovettori $\ket k$ e degli autovalori $E{k+1} - E_k = \hbar \omega$ sono anche soluzioni del nostro circuito LC. Usando gli operatori di creazione $\hat a^\dagger$ e distruzione $\hat a$ possiamo riscrivere l'hamiltoniana come
\begin{equation*}
    \hat H = \hbar \omega \left(\hat a^\dagger \hat a + \frac 12 \right)
\end{equation*}
con 
\begin{equation*}
    \hat a = \frac{1}{\sqrt{2\hbar z}}\left(\hat \Phi + iz \hat Q\right) \qquad \text{e} \qquad \hat a^\dagger = \frac{1}{\sqrt{2\hbar z}}\left(\hat \Phi - iz \hat Q\right) \qquad \qquad z = \sqrt{\frac{L}{C}}
\end{equation*}
e le relazioni inverse
\begin{equation*}
    \hat Q = i\sqrt{\frac{\hbar}{2z}}\left(\hat a^\dagger - \hat a\right) \qquad \qquad \hat \Phi = \sqrt{\frac{\hbar z}{2}}\left(\hat a^\dagger + \hat a\right)
\end{equation*}
Possiamo inoltre valutare i valori medi di $\hat Q^2$ e $\hat \Phi^2$
\begin{align*}
    \expval{\hat Q^2} = \expval{\hat Q^2}{0} = \frac{\hbar}{2z} \\
    \expval{\hat \Phi^2} = \expval{\hat \Phi^2}{0} = \frac{\hbar z}{2}
\end{align*}
e prendono il nome di \textbf{zero-point fluctuations} perché $\Delta Q \Delta \Phi = \frac{\hbar}{2}$.
Come possiamo notare, il circuito LC non può essere utilizzato come qubit, questo perché il gap energetico tra i vari livelli è lo stesso, inoltre non possiamo vedere un comportamento quantistico perché $\hbar \omega \ll k_BT$. Se prendessimo $T=10 \text{ mK}$, allora $\hbar\omega > 10^{-5}\text{ eV}$, cioè $\omega \gg 10 \text{GHz}$.
Questo fatto è interessante, perché per avere un sistema quantistico, a questa temperatura, necessitiamo solamente di una frequenza dell'ordine di $10 \text{GHz}$. Questo significa che il sistema in questione deve avere una dimensione $l < \frac{\omega}{2\pi c}$, ma questo non è un problema. Il problema risiede nel fatto che dobbiamo accoppiarlo con l'ambiente.
Un altro fatto di cui dobbiamo tenere conto è quello riguardante la possibilità di avere tunneling, necessitiamo quindi di lavorare con livelli meno energetici del gap che c'è tra i vari livelli in un circuito LC. La probabilità di tunneling aumenta più si eccita il sistema. Martinis, Devoret e altri fisici che studiarono questo sistema rimpiazzarono l'induttore $L$ con una giunzione Josephson. La giunzione, comportandosi da elemento non lineare, va ad alterare la distribuzione dei livelli energetici.
    %%%%%%%%%%%%%%
% LECTURE 12 %
%%%%%%%%%%%%%%
\chapter{Quantum error correction}

\lecture{12}{15/11/21}
\vspace{0.5cm}

\noindent Entrambi il CC e il QC sono soggetti ad errori: si pensi ad esempio all'imperfezione dei fili nei circuiti elettrici, ai gate difettosi che non restituiscono il corretto risultato oppure anche al caso dell'interazione con l'ambiente che, come visto nel capitolo precedente, porta sempre ad una perdita di coerenza e di ampiezza nella sovrapposizione quantistica del qubit. A tal proposito, tipicamente il tempo di decoerenza è molto più piccolo rispetto al tempo di decadimento dell'ampiezza, quindi la perdita di interferenza nello stato avviene quasi immediatamente. 

\noindent Nonostante la gestione degli errori di un qubit sia molto più complessa rispetto al caso classico, a causa della continua influenza dell'ambiente, cominciamo la nostra discussione analizzando il caso del CC.

\section{Correzione classica degli errori}\label{sec:classical_correction}
In CC esistono alcuni protocolli standard per affrontare i problemi sopraelencati: il più famoso prevede una \textbf{codifica} del messaggio contenuto nella stringa di bit in esame in una ripetizione di questi ultimi. Questo significa rimpiazzare, ad esempio, $0 \to 000$ e $1 \to 111$: chiaramente la sequenza di 3 bit finali, chiamati \textbf{bit logici}, ha il medesimo significato del bit iniziale (3 bit uguali $=$ 1 bit), tuttavia la differenza è che si hanno molti più bit. 

\noindent Quale vantaggio porta questa sostituzione? Assumendo che gli errori tipici siano equivalenti ad un \texttt{OR} (bit flip: $0 \to 1$ e $1 \to 0$) e che siano statisticamente indipendenti tra loro, allora la probabilità che due o più errori avvengano simultaneamente nella sequenza finale è molto più piccola rispetto ad averne solamente uno: possiamo quindi correggere un qualsiasi errore usando la \textbf{majority rule}. Se la sequenza di bit ricevuti è costituita da $xxx$, allora il ricevente legge "bit inviato $= 0$" se la maggior parte dei bit sono 0, viceversa, quando la maggior parte sono degli 1 allora legge "bit inviato $= 1$". 

\noindent Più formalmente, chiamiamo $p$ la probabilità di avere un singolo bit flip. Quando si codifica il messaggio in un bit singolo si ha probabilità $p$ di avere una qualche sorta di errore, tuttavia se la codifica avviene in 3 bit allora la sicurezza è maggiore. Per vederlo più esplicitamente calcoliamo la probabilità di avere 2 o 3 bit flip contemporaneamente: nel caso di due bit flip avremo $3 p p (1-p) = 3 p^2(1-p)$ (il 3 indica che il primo errore può avvenire in uno dei 3 bit qualsiasi), mentre la probabilità di avere 3 errori è ovviamente $p^3$, dunque la probabilità totale di avere 2 o più errori è data dalla somma $3p^2-2p^3$. Ma si ha
\begin{equation}\label{prob_2_simult}
    3p^2-2p^3 < p \quad \text{per} \quad p < \frac{1}{2} \, ;
\end{equation}
bastano quindi probabilità minori del 50\%, di avere un singolo bit flip, per fare in modo che la probabilità di avere 2 o più errori simultanei sia trascurabile. Codificare il messaggio in 3 bit è più sicuro che lasciare il singolo bit. In generale questo discorso funziona ragionevolmente bene in CC. 

\noindent Analizziamo ora l'analogo quantistico.

\section{Introduzione alla correzione quantistica degli errori}\label{sec:intro_error_corr}
Proviamo a svolgere il medesimo procedimento in QC: rimpiazziamo i singoli qubit con $\ket{0} \to \ket{000}$ e $\ket{1} \to \ket{111}$. Nonostante il problema della duplicazione dei qubit (non è affatto banale raddoppiarli o triplicarli), esistono altri problemi concettuali da tenere necessariamente in considerazione:
\begin{itemize}
    \item[A.] \textbf{Teorema di no-cloning}: non possiamo clonare stati generici. 
    
    \item[B.] \textbf{Errori continui}: in CC tutti gli errori che avvengono sono discreti, ma in QC, dato che i gate dipendono da parametri continui, gli errori possono essere continui (si pensi banalmente all'amplitude e phase damping).
    
    \item[C.] \textbf{Regola di Born}: sappiamo dalla QM che la misurazione porta al collasso della funzione d'onda; per applicare la majority rule sui qubit è quindi necessario interagire con essi e causare irreversibilmente il collasso dello stato.
\end{itemize}

\noindent Nelle prossime sezioni vedremo che tutte queste problematiche possono essere affrontate nella teoria generale della correzione degli errori. 

\noindent Cominciamo ad introdurre il procedimento che si attua per rimpiazzare i qubit iniziali con dei \textbf{qubit logici}. Come prima cosa si vuole rimpiazzare
\begin{align*}
    \ket{0} & \rightarrow \ket{000} \equiv \ket{\overline{0}} \equiv \ket{0}_L \, , \\
    \ket{1} & \rightarrow \ket{111} \equiv \ket{\overline{1}} \equiv \ket{1}_L \, ,
\end{align*}
dove il pedice $L$ sta per "logico". Ricordiamo che un qubit si trova quasi sempre in una sovrapposizione quantistica di stati, quindi si vuole scrivere
\begin{equation}\label{logical_qubit}
    a \ket{0} + b \ket{1} \to a \ket{000} + b \ket{111} \, .
\end{equation}
Possiamo effettuare questa operazione senza violare il punto A.? La risposta è sì ed è quasi banale perché l'operazione precedente non è la stessa che clonare lo stato. Ciò che non è permesso dal \textit{teorema di no-cloning} è \begin{align}\label{forbidden_no_cloning}
    a \ket{0} + b \ket{1} \equiv \ket{\psi} & \rightarrow \ket{\psi} \otimes \ket{\psi} \otimes \ket{\psi} \notag \\
    &= (a \ket{0} + b \ket{1}) \otimes (a \ket{0} + b \ket{1}) \otimes (a \ket{0} + b \ket{1}) \, ;
\end{align} 
chiaramente \eqref{logical_qubit} $\neq$ \eqref{forbidden_no_cloning} perché ciò che vogliamo fare è molto più semplice. 

\noindent Possiamo disegnare un circuito che sia in grado di operare la codifica \eqref{logical_qubit} ricordando che il \texttt{CNOT-gate} inverte lo stato solamente quando il control qubit è in $\ket{1}$:
\begin{center}
    \mbox{
        \Qcircuit @C=2em @R=1em {
            \lstick{a \ket{0} + b \ket{1}} & \ctrl{1} & \ctrl{2} & \qw \\
            \lstick{\ket{0}} & \targ & \qw & \qw \\
            \lstick{\ket{0}} & \qw & \targ & \qw
        }
    }
\end{center}
è chiaro vedere che quando il primo qubit è in $\ket{0}$ allora si ottiene $\ket{000}$ e viceversa quando è in $\ket{1}$ i due \texttt{CNOT-gate} agiscono e producono $\ket{111}$. Questo circuito non contraddice il teorema di no-cloning: ad esempio se il primo qubit si trova in $\ket{1}$ allora il circuito avrà come output
\begin{equation*}
    b \ket{1} \otimes \ket{0} \otimes \ket{0} \to b \ket{1} \otimes \ket{1} \otimes \ket{1} \, ,
\end{equation*}
il quale è effettivamente il risultato di una clonazione perché se chiamiamo $\ket{1} \equiv \ket{\psi}$ allora l'operazione è
\begin{equation*}
    \ket{\psi} \otimes \ket{0} \otimes \ket{0} \to  \ket{\psi} \otimes \ket{\psi} \otimes \ket{\psi} \, ;
\end{equation*}
nonostante ciò, la dimostrazione del teorema di no-cloning falliva per stati ortogonali, il che significa che non c'è contraddizione fisica nel clonare stati appartenenti ad una base. Il teorema non è violato perché lo stato clonato non è uno stato generico. 

\noindent Vogliamo cercare di capire come rilevare possibili errori. Supponiamo che si possano verificare errori di bit flip: gli errori causano $\ket{0} \to \ket{1}$ e $\ket{1} \to \ket{0}$ e questa operazione può essere implementata in un circuito con un \texttt{X-gate}. Immaginiamo che il circuito precedente sia soggetto ad un errore di questo tipo: 
\begin{center}
    \mbox{
        \Qcircuit @C=2em @R=1em {
            \lstick{} & \ctrl{1} & \ctrl{2} & \gate{X} & \qw \\
            \lstick{} & \targ & \qw & \qw & \qw \\
            \lstick{} & \qw & \targ & \qw & \qw
            \gategroup{1}{4}{1}{4}{.7em}{--}
        }
    }
\end{center}
(il gate è stato inserito nel primo qubit, ma il discorso è analogo anche se fosse stato nel secondo o terzo). Il nostro scopo è quello di correggere l'errore tratteggiato senza disturbare in maniera eccessiva il sistema. Chiaramente essendo l'output il seguente
\begin{equation*}
    a \ket{000} + b \ket{111} \overset{X}{\longrightarrow} a \ket{100} + b \ket{011} \, ,
\end{equation*}
dobbiamo cercare di capire come rilevare questo errore senza causare il collasso dello stato e come intervenire per ripristinare l'informazione desiderata. 

\noindent Vediamo questo errore in generale. Se solamente singoli\footnote{Stiamo assumendo che la probabilità che avvengano 2 o più errori simultaneamente è soppressa rispetto alla probabilità che avvenga un singolo bit flip. Si veda l'equazione \eqref{prob_2_simult} per ulteriori dettagli.} \texttt{X-gate} su singoli qubit possono intervenire allora, a seconda della posizione di questo gate in uno dei 3 qubit, possono verificarsi 4 casi (compresa la situazione in cui non avviene alcun errore):
\begin{equation}\label{distinguish_cases}
    a \ket{000} + b \ket{111} \rightarrow
    \begin{cases}
    a \ket{000} + b \ket{111} \, , &\text{Caso I} \\
    a \ket{100} + b \ket{011} \, , &\text{Caso II} \\
    a \ket{010} + b \ket{101} \, , &\text{Caso III} \\
    a \ket{001} + b \ket{110} \, , &\text{Caso IV}
    \end{cases} \, .
\end{equation}
Come possiamo distinguere in quale dei 4 casi ci troviamo? Possiamo pensare di costruire un circuito che distingua le situazioni senza disturbare i 3 qubit in gioco. Per farlo aggiungiamo, dopo i due \texttt{CNOT-gate} che servono per produrre \eqref{logical_qubit}, due qubit extra, chiamati \textbf{ancilla qubit}\footnote{Il nome "ancilla" deriva dal latino \textit{ancilla}, che significa "ancella", "serva", "schiava". Sì, sono qubit schiavi.}, e costruiamo il seguente circuito:
\begin{center}
    \mbox{
        \Qcircuit @C=2em @R=1em {
            \lstick{} & \targ & \targ & \qw & \qw & \meter & \rstick{\ket{x} = \ket{0}, \ket{1}} \qw \\
            \lstick{\raisebox{2.8em}{$\large{\substack{\text{Ancilla qubit} \\ \text{preparati in } \ket{0}}}$ \ }} & \qw & \qw & \targ & \targ & \meter & \rstick{\ket{y} = \ket{0}, \ket{1}} \qw \gategroup{1}{1}{2}{1}{1.2em}{\{} \\
            \lstick{} & \ctrl{-2} & \qw & \qw & \qw & \qw & \qw \\
            \lstick{} & \qw & \ctrl{-3} & \ctrl{-2} & \qw & \qw & \qw \\
            \lstick{\raisebox{2.8em}{$\large{\substack{\text{Qubit dei casi} \\ \text{I, II, III e IV}}}$}\ \ } & \qw & \qw & \qw & \ctrl{-3} & \qw & \qw \gategroup{3}{1}{5}{1}{1em}{\{}
        }
    }
\end{center}
In questo circuito i 3 qubit in esame agiscono sempre da control qubit, quindi rimangono tali non essendo soggetti ad alcun flip dovuto ai diversi \texttt{CNOT-gate} (questo significa che il punto C. è rispettato). Come indicato, si svolge una misurazione sugli ancilla qubit: il risultato, che non causa alcun collasso dei 3 qubit in esame, è un insieme di due numeri $(x,y)$ che possono essere utilizzati per distinguere quale dei 4 errori è avvenuto. Tenendo presente gli stati indicati in \eqref{distinguish_cases} nei differenti casi, avremo le seguenti situazioni:
\begin{enumerate}
    \item \textbf{Caso I}: $(x,y) = (0,0)$.
    
    \noindent La parte che riguarda $\ket{000}$ non produce alcuna modifica perché i 4 \texttt{CNOT-gate} non agiscono; la parte di $\ket{111}$, invece, attiva tutti e 4 i \texttt{CNOT-gate} ma non produce alcuna variazione finale degli stati $\ket{x} = \ket{0}$ e $\ket{y} = \ket{0}$ perché si hanno due flip consecutivi. 
    
    \item \textbf{Caso II}: $(x,y) = (1,0)$.
    
    \noindent Lo stato $\ket{100}$ attiva solamente il primo \texttt{CNOT-gate} sul primo ancilla; lo stato $\ket{011}$ attiva gli altri 3, di cui uno sul primo ancilla (come in precedenza) e due consecutivi sul secondo ancilla. Il risultato è quindi $\ket{x} = \ket{1}$ e $\ket{y} = \ket{0}$ per entrambi. 
    
    \item \textbf{Caso III}: $(x,y) = (1,1)$. 
    
    \noindent Lo stato $\ket{010}$ attiva i due \texttt{CNOT-gate} centrali che agiscono sui due diversi ancilla; lo stato $\ket{101}$ attiva il primo e il quarto \texttt{CNOT-gate}, che anch'essi agiscono su due diversi ancilla. In entrambi i casi si ha un flip su ogni stato e quindi avremo $\ket{x} = \ket{1}$ e $\ket{y} = \ket{1}$. 
    
    \item \textbf{Caso IV}: $(x,y) = (0,1)$.
    
    \noindent Lo stato $\ket{001}$ attiva solamente l'ultimo \texttt{CNOT-gate}, che modifica il secondo ancilla; lo stato $\ket{110}$ attiva i primi 3 \texttt{CNOT-gate}: i primi due non producono alcuna differenza sul primo ancilla, mentre il terzo inverte il secondo ancilla. Il risultato per entrambe le situazioni è quindi $\ket{x} = \ket{0}$ e $\ket{y} = \ket{1}$. 
\end{enumerate}

\noindent Una volta effettuata la misurazione sugli ancilla e distinto il caso in esame, si può facilmente intervenire sul qubit corrotto inserendo un \texttt{X-gate} che lo riporti alla situazione iniziale. Si noti, ancora una volta, che non si è misurato il qubit difettoso, ma si sa della sua presenza grazie al risultato degli ancilla. In generale, al posto che effettuare le misure sugli ancilla, è possibile implementare un circuito che svolga questo lavoro in automatico (compreso l'inserimento dell'\texttt{X-gate} per correggere il qubit). 

\section{Stabilizers}\label{sec:stabilizers}
Qual è il significato fisico in termini di osservabili della misurazione $(x,y)$ che viene effettuata sugli ancilla qubit? 

\noindent Prima di rispondere a questa domanda fissiamo le notazioni: i 3 qubit soggetti ad errori vengono generalmente indicati con un numero, come ad esempio
\begin{center}
    \mbox{
        \Qcircuit @C=2em @R=1em {
            \lstick{0} & \gate{X} & \qw \\
            \lstick{1} & \qw & \qw \\
            \lstick{2} & \qw & \qw 
        }
    }
\end{center}
\vspace{0.2cm}
L'\texttt{X-gate} mostrato, invece, corrisponde, in termini di operatori agenti sui 3 qubit, al prodotto tensoriale $X_0 \otimes \mathbb{I}_1 \otimes \mathbb{I}_2$. Per semplicità scriveremo sempre $X_0 \otimes \mathbb{I}_1 \otimes \mathbb{I}_2 \equiv X_0$, $\mathbb{I}_0 \otimes X_1 \otimes \mathbb{I}_2 \equiv X_1$ e così via: i pedici sugli operatori indicano il qubit su cui esso sta agendo. 

\noindent Lo spazio di Hilbert dei 3 qubit, che indicheremo con $\mathcal{H}$, ha dimensione $\dim \mathcal{H} = 2^3 = 8$ perché i qubit logici che stiamo considerando sono triplette dei qubit singoli iniziali: $\ket{\overline{0}} = \ket{000}$ e $\ket{\overline{1}} = \ket{111}$. Chiamiamo \textbf{codewords} gli stati del seguente sottospazio
\begin{equation}\label{codewords}
    C = \left\lbrace \ket{\psi} \in \mathcal{H} \; : \; \ket{\psi} = a \ket{000} + b \ket{111} \right\rbrace \subset \mathcal{H} \, .
\end{equation}
Gli operatori che agiscono su $\mathcal{H}$ che ci interessano particolarmente  sono della forma $M = M_0 \otimes M_1 \otimes M_2$ dove $M_i = \{ \mathbb{I}, X, Y, Z \}$. Questi operatori $M$ sono molto speciali perché sono fattorizzati come prodotto di matrici: non stiamo agendo su $\ket{\overline{0}}$ e $\ket{\overline{1}}$ con generiche matrici $8 \times 8$; in aggiunta, se consideriamo le singole matrici $M_i$ $2 \times 2$ che agiscono sui singoli qubit esse non sono matrici generali di $U(2)$. Alcuni semplici esempi di operatori $M$ sono: $X \otimes \mathbb{I} \otimes \mathbb{I}$, $X \otimes Z \otimes \mathbb{I}$ e $Y \otimes Z \otimes X$.

\noindent Perché questi operatori sono così importanti? In primo luogo perché grazie alla proprietà $\sigma_i^2 = \mathbb{I}$ allora $M^2 = \mathbb{I}$. Ad esempio se $M = X \otimes Z \otimes \mathbb{I}$ avremo
\begin{equation*}
    M^2 = (X \otimes Z \otimes \mathbb{I}) (X \otimes Z \otimes \mathbb{I}) = X^2 \otimes Z^2 \otimes \mathbb{I}^2 = \mathbb{I} \otimes \mathbb{I} \otimes \mathbb{I} \equiv \mathbb{I} \, .
\end{equation*}
In secondo luogo, le matrici di Pauli sono hermitiane, il che vuol dire che possono essere diagonalizzate. I possibili autovalori ($\lambda = \pm 1$) e autovettori delle matrici di Pauli sono mostrati nella Tabella \ref{tab:Pauli_eig} della Sezione \ref{sec:osservabili}. 

\noindent Grazie alla due ragioni precedenti vale il seguente risultato: se $M$ è non banale (almeno una $M_i \neq \mathbb{I}$) allora lo spazio di Hilbert dei qubit può essere decomposto come somma diretta di due sottospazi della medesima dimensione corrispondenti agli autovalori $\lambda = \pm 1$ dell'operatore $M$ (autospazi di $M$). In termini matematici possiamo scrivere 
\begin{equation}\label{decomposition_direct_sum}
    \mathcal{H} = \mathcal{H}_1 \oplus \mathcal{H}_2 \, , \quad \text{con} \quad \dim \mathcal{H}_1 = \dim \mathcal{H}_2 \, ;
\end{equation}
questo significa quindi che $\mathcal{H}$ è "tagliato" da $M$ in due sottospazi della stessa dimensione. Cerchiamo di capirlo meglio con un esempio:

\begin{esempio}\label{es:M}
    Supponiamo che l'operatore sia $M = X \otimes \mathbb{I} \otimes \mathbb{I}$. Dato che abbiamo la matrice $\sigma_x$ è meglio utilizzare la base composta dai suoi autostati: si ha $X \ket{\pm} = \pm \ket{\pm}$, quindi la tripletta di qubit è costituita dagli stati della forma $\{ \ket{\pm \pm \pm} \}$. Lo spazio di Hilbert totale è tagliato negli autospazi:
    \begin{equation}\label{H_1_H_2}
        \mathcal{H}_1 = \{ \ket{+ \pm \pm} \} \, , \qquad \mathcal{H}_2 = \{ \ket{- \pm \pm} \} \, ,
    \end{equation}
    dove entrambi gli spazi hanno dimensione 4 poiché sono costituiti da 4 stati differenti. Tutti gli stati di $\mathcal{H}_1$ sono autostati di $M$ con autovalore 1 e similmente tutti gli stati di $\mathcal{H}_2$ sono autostati di $M$ con autovalore $-1$.  
\end{esempio}

\noindent Il fatto che valga la \eqref{decomposition_direct_sum} non è una coincidenza perché per ogni operatore $M$ non banale esiste un operatore invertibile $S$ tale che $S \; : \; \mathcal{H}_1 \to \mathcal{H}_2$ (i due sottospazi $\mathcal{H}_1$ e $\mathcal{H}_2$ sono infatti isomorfi perché possiamo sempre tornare indietro). 

\begin{esempio}
    In riferimento all'Esempio \ref{es:M}, possiamo scegliere una matrice che anticommuta con $X$, come $Z$ ad esempio, e scrivere $S = Z \otimes \mathbb{I} \otimes \mathbb{I}$. Come agisce $Z$ su $\ket{\pm}$? Ricordiamo che
    \begin{equation*}
        Z \ket{\pm} = Z \frac{\ket{0} \pm \ket{1}}{\sqrt{2}} = \frac{\ket{0} \mp \ket{1}}{\sqrt{2}} = \ket{\mp} \, ;
    \end{equation*}
    in riferimento ai sottospazi in \eqref{H_1_H_2}, è evidente che se partiamo da $\ket{\psi_1} \in \mathcal{H}_1$ allora $S \ket{\psi_1} = \ket{- \pm \pm} \equiv \ket{\psi_2} \in \mathcal{H}_2$. Chiaramente è invertibile perché basta applicare nuovamente $S$ a $\ket{\psi_2}$ per ottenere uno stato di $\mathcal{H}_1$. 
\end{esempio}

\noindent L'esempio precedente funzionava perché l'operatore scelto anticommutava con $M$: se possiamo trovare un operatore $S$ tale che $\acomm{S}{M} = 0$ allora tale operatore agisce come $S \; : \; \mathcal{H}_1 \to \mathcal{H}_2$ per una ragione algebrica molto semplice. Consideriamo uno stato $\ket{\psi_1} \in \mathcal{H}_1$, quindi $M \ket{\psi_1} = \ket{\psi_1}$ perché $\mathcal{H}_1$ è autospazio di $M$ con autovalore associato $\lambda = +1$. Allora
\begin{equation*}
    M \left( S \ket{\psi_1} \right) = - S \left( M \ket{\psi_1} \right) = - S \ket{\psi_1} \, ,
\end{equation*}
quindi $S \ket{\psi_1}$ è autovettore di $M$ con autovalore associato $\lambda = -1$: questo significa necessariamente che $S \ket{\psi_1} \in \mathcal{H}_2$.

\noindent Dopo questa digressione matematica ritorniamo al nostro problema originale nel capire qual è il significato fisico delle misure effettuate sugli ancilla qubit. Cosa c'è di speciale negli stati che si scrivono come in \eqref{codewords}? Possiamo identificarli in qualche modo? Definiamo gli operatori
\begin{align*}
    Z_0 Z_1 &= Z \otimes Z \otimes \mathbb{I} \, , \\
    Z_1 Z_2 &= \mathbb{I} \otimes Z \otimes Z \, ,
\end{align*}
dove i pedici indicano su quali qubit le matrici $Z$ stanno agendo. Tutti gli stati di $C$ in \eqref{codewords} sono autostati di questi due operatori con autovalori $+1$ per entrambi: questo è ovvio perché $Z \ket{0} = \ket{0}$ e $Z \ket{1} = - \ket{1}$, infatti
\begin{align*}
    Z_0 Z_1 \ket{000} &= \ket{000} \, , & Z_1 Z_2 \ket{000} &= \ket{000} \, , \\
    Z_0 Z_1 \ket{111} &= \ket{111} \, , & Z_1 Z_2 \ket{111} &= \ket{111} \, .
\end{align*}

\noindent Affermiamo inoltre che gli stati di $C$ sono il più generale autovettore comune di $Z_0 Z_1$ e $Z_1 Z_2$. La ragione è la seguente: entrambi gli operatori sono della forma di $M$ e ognuno dei due, come autospazio, taglia lo spazio di Hilbert totale $\mathcal{H}$ in 2 sottospazi, le cui dimensioni sono la metà della dimensione dello spazio di partenza (ossia 4). Entrambi gli operatori $Z_0 Z_1$ e $Z_1 Z_2$ possono avere autovalori $\pm 1$: focalizzandoci sull'autovalore in comune, ossia $+1$, abbiamo che la dimensionalità viene tagliata in $8/2/2 = 2$, dove il primo taglio è operato da $Z_0 Z_1$ e il secondo da $Z_1 Z_2$. La dimensionalità rimanente, ossia 2, indica che l'autospazio comune agli operatori ha dimensione 2, e infatti $C$ ha proprio dimensione 2 (tutti gli stati della \eqref{codewords} dipendono da due coefficienti generici, quindi $\dim C = 2$). 

\noindent Alla luce di questo discorso affermiamo che le misure sugli ancilla qubit sono legate agli autovalori $x$ di $Z_0 Z_1$ e $y$ di $Z_1 Z_2$. Gli autovalori dell'azione di questi operatori sugli stati della \eqref{distinguish_cases} sono mostrati nella Tabella \ref{tab:eigs_ZZ}. Si noti che ciascun caso è un sottospazio di $\mathcal{H}$ di dimensione 2 (gli stati dipendono da due coefficienti generici) quindi riempiono tutto lo spazio di Hilbert totale ($8 = 2 + 2 + 2 +2$). 
Gli autovalori di questi operatori permettono di distinguere i 4 differenti autospazi, ossia le 4 differenti situazioni (qubit intatto o qubit corrotto da uno dei 3 errori).

\begin{table}[!ht]
	\centering
    \begin{tabular}{llcc}
        \toprule
        Caso & Autospazio & Autovalore di $Z_0 Z_1$ & Autovalore di $Z_1 Z_2$ \\
        \midrule
        I & $C = \{ a \ket{000} + b \ket{111} \}$ & $+1$ ($x=0$) & $+1$ ($y=0$) \\
        II & $E_{II} = \{ a' \ket{100} + b' \ket{011} \}$ & $-1$ ($x=1$) & $+1$ ($y=0$) \\
        III & $E_{III} = \{ a'' \ket{010} + b'' \ket{101} \}$ & $-1$ ($x=1$) & $-1$ ($y=1$) \\
        IV & $E_{IV} = \{ a''' \ket{001} + b''' \ket{110} \}$ & $+1$ ($x=0$) & $-1$ ($y=1$) \\
        \bottomrule
    \end{tabular}\\
    \caption{Autovalori degli operatori $Z_0 Z_1$ e $Z_1 Z_2$ sugli autospazi delle differenti situazioni in \eqref{distinguish_cases}. Se associamo agli autovalori le misure $(\lambda = +1) \to (x \text{ oppure } y = 0)$ e $(\lambda = -1) \to (x \text{ oppure } y = 1)$ allora questi autovalori permettono di distinguere i 4 differenti autospazi perché i valori corrispondenti di $(x,y)$ sono esattamente le misurazioni effettuate sugli ancilla qubit.}
    \label{tab:eigs_ZZ}
\end{table}

\noindent Abbiamo detto che un errore non è altro che un \texttt{X-gate} agente su un qubit: dato che $\acomm{X}{Z} = 0$ allora, come evidenzia anche la Tabella \ref{tab:acomm_ZZ}, le misure sugli ancilla qubit non sono altro che i segni rimanenti nelle anticommutazioni di $X_i$ con gli operatori $Z_0 Z_1$ e $Z_1 Z_2$. Esplicitamente avremo:
\begin{align*}
    (Z_0 Z_1) \mathbb{I} &= \mathbb{I} (Z_0 Z_1) \, , &(Z_1 Z_2) \mathbb{I} &= \mathbb{I} (Z_1 Z_2) \, , \\
    (Z_0 Z_1) X_0 &= - X_0 (Z_0 Z_1) \, , &(Z_1 Z_2) X_0 &= X_0 (Z_1 Z_2) \, , \\
    (Z_0 Z_1) X_1 &= - X_1 (Z_0 Z_1) \, , &(Z_1 Z_2) X_1 &= - X_1 (Z_1 Z_2) \, , \\
    (Z_0 Z_1) X_2 &= X_2 (Z_0 Z_1) \, , &(Z_1 Z_2) X_2 &= - X_2 (Z_1 Z_2) \, .
\end{align*}
Si noti che un determinato $X_i$ commuta sempre con le matrici appartenenti ad un differente sottospazio (differente qubit).

\begin{table}[!hb]
	\centering
    \begin{tabular}{c|cccc}
        \toprule
        Operatore & Segno $\mathbb{I}$ & Segno $X_0$ & Segno $X_1$ & Segno $X_2$ \\
        \midrule
        $Z_0 Z_1$ & $+1$ ($x = 0$) & $-1$ ($x = 1$) & $-1$ ($x = 1$) & $+1$ ($x = 0$) \\
        $Z_1 Z_2$ & $+1$ ($y = 0$) & $+1$ ($y = 0$) & $-1$ ($y = 1$) & $-1$ ($y = 1$) \\
        \bottomrule
    \end{tabular}\\
    \caption{Il segno rimanente dell'anticommutazione degli operatori $X_i$ (gli \texttt{X-gate} che implementano l'errore) con gli operatori $Z_0 Z_1$ e $Z_1 Z_2$ è esattamente la misura (mostrata tra parentesi) che viene effettuata sui due ancilla qubit.}
    \label{tab:acomm_ZZ}
\end{table}

\noindent Questo formalismo che abbiamo utilizzato per rilevare gli errori è spesso chiamato \textbf{syndrome error correction}. In generale il prodotto delle matrici di Pauli in questi operatori è molto utile per scrivere degli algoritmi per la rilevazione di errori: codici basati su questa logica vengono detti \textbf{stabilizer codes}; gli operatori $Z_0 Z_1$ e $Z_1 Z_2$ sono detti \textbf{stabilizers}. Nelle sezioni successive vedremo alcuni di questi codici famosi.

\noindent Abbiamo detto che lo spazio di Hilbert totale dei 3 qubit può essere visto come somma dei sottospazi $C$, $E_{II}$, $E_{III}$ e $E_{IV}$ perché ciascuno ha dimensione 2, quindi $2^3 = 8$ è scrivibile come $2 + 2 + 2 + 2$. Gli spazi degli errori sono 3 poiché stiamo studiando solamente i bit flip errors, i quali possono accadere in 3 posizioni differenti dei qubit logici. La domanda è: avremmo potuto fare di meglio se avessimo scelto di codificare il messaggio in un numero maggiore di qubit? Se si vuole correggere gli errori di bit flip codificando 1 qubit in $n$ qubit, qual è il minimo valore di $n$ da utilizzare? Per rispondere a queste domande ripartiamo dalla \eqref{codewords}: sappiamo che $C$ ha dimensione 2 e immaginiamo che codifichi i singoli qubit di partenza in $n$ qubit logici, che chiamiamo $\ket{\overline{0}}$ e $\ket{\overline{1}}$; per ognuno di essi è possibile il verificarsi di un errore e in tal caso si necessita uno "spazio dell'errore" ortogonale\footnote{Se lo spazio è ortogonale a tutti gli altri allora la misurazione non influenza in alcun modo gli altri spazi.} a tutti gli altri. Il numero degli spazi necessari sarà quindi $ 2 + 2n$: il primo termine è la dimensione del codewords mentre il secondo è la dimensione del numero di spazi necessari per correggere $n$ qubit\footnote{Il $2$ deriva dal fatto che si debbano correggere entrambi i qubit logici $\ket{\overline{0}}$ e $\ket{\overline{1}}$, mentre il fattore $n$ indica che l'errore si può trovare in uno degli $n$ qubit di quello logico. Ad esempio per $n=4$ si avrà $\ket{\overline{0}} = \ket{0000}$ e $\ket{\overline{1}} = \ket{1111}$: vanno corretti entrambi i qubit logici e in più l'errore si può trovare in ciascuno dei quattro "posti" nel ket di stato.}. Dato che la dimensione dello spazio di Hilbert di $n$ qubit è $2^n$, allora per avere un codewords e degli spazi degli errori mutuamente ortogonali dovremo avere
\begin{equation*}
    2^n \geq 2 + 2 n \, , \quad \Rightarrow \quad 2^{n-1} \geq 1 + n \, .
\end{equation*}
Il minimo numero che soddisfa la disuguaglianza precedente è proprio $n = 3$.

\noindent Il \textbf{bit flip error} appena analizzato non è l'unico errore che si può verificare: talvolta può accadere ad esempio un \textbf{phase flip error}, il quale è facilmente implementabile nei circuiti con un semplice \texttt{Z-gate} ricordando che $Z \ket{0} = \ket{0}$ e $Z \ket{1} = -\ket{1}$. Per analizzare una situazione di questo tipo bisogna ricordarsi che per cambiare base da quella di $Z$, $\{ \ket{0}, \ket{1} \}$, a quella di $X$, $\{ \ket{+}, \ket{-} \}$, basta applicare un \texttt{H-gate} (si ricordi la matrice \eqref{Hadamard_matrix}): $H \ket{0} = \ket{+}$ e $H \ket{1} = \ket{-}$. 

\noindent Per correggere gli errori di phase flip è quindi necessario codificare gli stati in maniera differente: $\ket{0} \to \ket{+++}$ e $\ket{1} \to \ket{---}$. Se si parte con uno stato generico $a \ket{+++} + b \ket{---}$ allora l'azione dello \texttt{Z-gate} è semplicemente quella di scambiare $+ \leftrightarrow -$:
\begin{align*}
    a \ket{+++} + b \ket{---} &\overset{Z_0}{\longrightarrow} a \ket{-++} + b \ket{+--} \, , \\
    a \ket{+++} + b \ket{---} &\overset{Z_1}{\longrightarrow} a \ket{+-+} + b \ket{-+-} \, , \\
    a \ket{+++} + b \ket{---} &\overset{Z_2}{\longrightarrow} a \ket{++-} + b \ket{--+} \, .
\end{align*}
A questo punto la logica è esattamente la stessa di quella che abbiamo adoperato in precedenza! In termini di circuiti, per costruire una tale codifica è necessario aggiungere 3 \texttt{H-gate} al termine del circuito originale:
\begin{center}
    \mbox{
        \Qcircuit @C=2em @R=1em {
            \lstick{a \ket{0} + b \ket{1}} & \ctrl{1} & \ctrl{2} & \gate{H} & \qw \\
            \lstick{\ket{0}} & \targ & \qw & \gate{H} & \qw \\
            \lstick{\ket{0}} & \qw & \targ & \gate{H} & \qw
        }
    }
\end{center}
Infatti dopo i due \texttt{CNOT-gate} lo stato è come in \eqref{logical_qubit}, mentre dopo i 3 \texttt{H-gate} si avrà 
\begin{equation}\label{eig_X_1}
    a \ket{000} + b \ket{111} \overset{H^{\otimes 3}}{\longrightarrow} a \ket{+++} + b \ket{---} \, .
\end{equation}
Da qui in poi per correggere gli errori si agisce esattamente come prima: questa volta lo stato \eqref{eig_X_1} è autostato degli operatori $X_0 X_1$ e $X_1 X_2$ con autovalore $+1$ e ogni qualvolta avverrà un phase flip error si potrà rilevare l'errore per mezzo del cambiamento di tale autovalore. La discussione è identica alla precedente sostituendo $Z \leftrightarrow X$, $\ket{0} \leftrightarrow \ket{+}$ e $\ket{1} \leftrightarrow \ket{-}$: si tratta solamente di un cambio di base!

\noindent Chiaramente può sorgere spontanea la domanda: come faccio se voglio correggere entrambi i bit-phase flip errors? Nelle prossime sezioni vedremo che il codice che si utilizza, dovuto nuovamente a Shor, utilizza solamente 9 qubit per correggere qualsiasi tipologia di errore (assicurando in questo modo che anche il problema B. di inizio Sezione \ref{sec:intro_error_corr} sia risolto). Vedremo che correggere tutti gli errori riguardanti i gate $X,Y,Z$ è equivalente a correggere qualsiasi tipologia di errore continuo!


    %%%%%%%%%%%%%%
% LECTURE 13 %
%%%%%%%%%%%%%%
\vspace{1cm}
\noindent \lecture{13}{19/11/21}

\section{Codice di correzione di Shor a 9 qubit}
Nel 1995 Peter Shor propose il primo codice di correzione degli errori quantistici in grado di correggere errori arbitrari a singolo qubit. La sua proposta, in breve, consisteva in una concatenazione dei circuiti di bit flip error e phase flip error che abbiamo visto nelle sezioni precedenti. Per il primo livello di codifica, si garantisce la protezione dagli errori legati al phase flip codificando i qubit logici $\ket{\overline 0}$ e $\ket{\overline 1}$ utilizzando il codice di phase flip a 3 qubit:
\begin{equation}\label{qe-shor1}
    \ket{\overline 0} \rightarrow \ket{+++} \, , \qquad \qquad \ket{\overline 1} \rightarrow \ket{---} \, .
\end{equation}
Come abbiamo notato sopra, tuttavia, questa codifica è suscettibile a errori di bit flip. Per proteggerci da questo tipo di errori prendiamo ciascuno degli stati $\ket +$ e $\ket -$ e, ricordando le \eqref{basi_di_sigma_12}, codifichiamo ciascun $\ket{0}$ e $\ket{1}$ utilizzando il codice di bit flip a 3 qubit. In questo modo la codifica finale è costituita da 9 qubit in totale:
\begin{align}
    &\ket{+++} \rightarrow \frac{\ket{000}+\ket{111}}{\sqrt 2}\frac{\ket{000}+\ket{111}}{\sqrt 2}\frac{\ket{000}+\ket{111}}{\sqrt 2} \equiv \ket{\overline{0}} \, , \label{qe-shor2} \\
    &\ket{---} \rightarrow \frac{\ket{000}-\ket{111}}{\sqrt 2}\frac{\ket{000}-\ket{111}}{\sqrt 2}\frac{\ket{000}-\ket{111}}{\sqrt 2} \equiv \ket{\overline{1}} \, . \label{qe-shor3}
\end{align}
Mettendo insieme i due passaggi che coinvolgono prima la correzione di eventuali phase flip in \eqref{qe-shor1} e dopo la correzione di eventuali bit flip in \eqref{qe-shor2} e \eqref{qe-shor3} avremo quindi la codifica
\begin{equation*}
    \ket{\overline 0} = \left(\frac{\ket{000}+\ket{111}}{\sqrt 2}\right)^{\otimes 3} \, , \qquad \qquad \ket{\overline 1} = \left(\frac{\ket{000}-\ket{111}}{\sqrt 2}\right)^{\otimes 3} \, .
\end{equation*}
Questi passaggi possono essere implementati nel circuito seguente
\begin{center}
    \mbox{
        \Qcircuit @C=1em @R=0.1 em {
            & \qw & \ctrl{3} & \ctrl{6} & \gate{H} & \qw & \ctrl{1} & \ctrl{2} & \qw \gategroup{1}{6}{3}{9}{.7em}{--} \\
            &     &          &          &          &     & \targ    & \qw      & \qw \\
            &     &          &          &          &     & \qw      & \targ    & \qw \\
            & \qw & \targ    & \qw      & \gate{H} & \qw & \ctrl{1} & \ctrl{2} & \qw 
            \gategroup{4}{6}{6}{9}{.7em}{--} \\
            &     &          &          &          &     & \targ    & \qw      & \qw \\
            &     &          &          &          &     & \qw      & \targ    & \qw \\
            & \qw & \qw      & \targ    & \gate{H} & \qw & \ctrl{1} & \ctrl{2} & \qw \gategroup{7}{6}{9}{9}{.7em}{--} \\
            &     &          &          &          &     & \targ    & \qw      & \qw \\
            &     &          &          &          &     & \qw      & \targ    & \qw
        }
    }
\end{center}
Come sopra descritto, la prima parte del circuito (fino a dopo gli \texttt{H-gate}) codifica il qubit utilizzando il codice relativo al phase flip a tre qubit. La seconda parte del circuito codifica ciascuno di questi tre qubit mediante il codice relativo al bit flip; in particolare fa uso di tre copie del circuito di codifica bit flip (si vedano le 3 subroutine tratteggiate). Questo metodo di codifica che utilizza una gerarchia di livelli è noto come \textit{concatenazione}. 

\noindent Per capire se il codice di Shor sia effettivamente in grado di proteggere da errori di phase flip e bit flip su qualsiasi qubit dobbiamo trovare un insieme di operatori tali che $\ket{\overline{0}}$ e $\ket{\overline{1}}$ siano autostati col medesimo autovalore. Per capire la logica generale di funzionamento facciamo un semplice esempio. 

\begin{esempio}
    Supponiamo che si verifichi un bit flip sul primo qubit (blocco I). Per quanto riguarda il codice relativo al bit flip, possiamo eseguire una misurazione di $Z_0Z_1$ per confrontare i primi due qubit, scoprendo che sono diversi ($x=1$). In questo modo stabiliamo che si è verificato un errore di bit flip sul primo o sul secondo qubit. Successivamente possiamo confrontare il secondo e il terzo qubit eseguendo una misurazione di $Z_1Z_2$: in tal caso scopriremmo che sono uguali ($y=0$), quindi non potrebbe essere stato il secondo qubit a capovolgersi. Concludiamo che il primo qubit deve essere stato capovolto e risolviamo l'errore invertendo nuovamente il primo qubit, riportandolo allo stato originale attraverso il gate $X_0$.
\end{esempio}

\noindent In modo del tutto analogo all'esempio precedente possiamo rilevare e correggere gli effetti degli errori legati al bit flip su uno qualsiasi dei nove qubit nel codice. Riassumiamo gli stabilizers coinvolti nella rilevazione nella Tabella \ref{tab:shor-bit-flip-cases}:
\begin{table}[!ht]
	\centering
    \begin{tabular}{lc}
        \toprule
        Operatori & Blocco \\
        \midrule
        $Z_0Z_1$, $Z_1Z_2$ & I \\
        $Z_3Z_4$, $Z_4Z_5$ & II \\
        $Z_6Z_7$, $Z_7Z_8$ & III \\
        \bottomrule
    \end{tabular}\\
    \caption{Stabilizers coinvolti nei diversi blocchi per identificare eventuali errori legati al bit flip. Il blocco I coinvolge i primi 3 qubit, il blocco II i qubit 3, 4 e 5 e infine il blocco III contiene i qubit 6, 7 e 8.}
    \label{tab:shor-bit-flip-cases}
\end{table}

\noindent Vediamo ora come affrontare errori di phase flip sui 9 qubit. Supponiamo che si verifichi un phase flip sul primo qubit, quindi
\begin{equation*}
    Z_0\left(\ket{000}+\ket{111}\right)=\left(\ket{000}-\ket{111}\right) \, ;
\end{equation*}
in questa situazione notiamo che c'è qualcosa di strano, perché se si verificasse un phase flip sul secondo o terzo qubit avremmo 
\begin{align*}
    Z_1\left(\ket{000}+\ket{111}\right) &= \ket{000}-\ket{111} \, , \\
    Z_2\left(\ket{000}+\ket{111}\right) &= \ket{000}-\ket{111} \, ,
\end{align*}
dunque questi due errori producono entrambi lo stesso effetto di $Z_0$: non possiamo stabilire con precisione in quale qubit si trovi l'errore, ma solamente il blocco di appartenenza. Per tale ragione il codice di Shor si dice \textbf{degenere} in quanto $Z_0$, $Z_1$ e $Z_2$ producono lo stesso effetto, o meglio, in generale qualunque errore di phase flip che comporta un cambiamento di segno sui 3 qubit all'interno dello stesso blocco (I, II o III) è lo stesso (degenerazione $= 3$). 

\noindent Per trovare un errore legato al phase flip dobbiamo quindi utilizzare un \texttt{X-gate}, ma deve essere eseguito come tripletta, cioè
\begin{equation*}
    X_0X_1X_2\left(\ket{000}\pm\ket{111}\right)= \pm \left( \ket{000} \pm \ket{111} \right) \, ,
\end{equation*}
quindi lo stato di partenza è autostato di questa tripletta di operatori con autovalore $\pm 1$. Per tale ragione, gli stabilizers, i cui autostati sono \eqref{qe-shor2} e \eqref{qe-shor3}, da considerare nel codice di Shor per individuare in quale blocco avvengono errori di phase flip sono
\begin{equation}\label{X_stabilizers}
    X_0X_1X_2X_3X_4X_5 \, , \qquad \qquad X_3X_4X_5X_6X_7X_8 \, .
\end{equation}
 

\noindent Un fatto importante da evidenziare è che nonostante gli operatori \eqref{X_stabilizers} permettano di stabilire solamente il blocco in cui è avvenuto l'errore, ciò non limita la sua risoluzione. Più precisamente, in riferimento ai casi scritti sopra, supponiamo che si verifichi un phase flip error nel blocco I: indipendentemente dall'operatore $Z_i$ che ha causato l'errore ($i = 0, 1, 2$), possiamo sempre applicare nuovamente al primo blocco uno qualsiasi di questi 3 operatori per correggere e riportare lo stato alla situazione originale. Il discorso è analogo per gli altri due blocchi. 

\noindent Riassumendo\footnote{Per esercizio si piò dimostrare che effettivamente gli 8 operatori del codice di Shor (6 in Tabella \ref{tab:shor-bit-flip-cases} e 2 in \eqref{X_stabilizers}) possono rilevare qualsiasi errore di bit flip, phase flip e bit-phase flip.}:
\begin{itemize}
    \item I sei operatori contenenti $Z$ della Tabella \ref{tab:shor-bit-flip-cases} identificano la posizione di un eventuale bit flip sui 9 qubit;
    \item I due operatori $X$ in \eqref{X_stabilizers} identificano la posizione di un eventuale phase flip nei 3 diversi blocchi.
\end{itemize}
Questa tipologia di misurazioni vengono definite \textbf{syndrome measurements}.

\noindent Per chiarire al meglio il funzionamento del codice di Shor consideriamo il seguente esempio che coinvolge tutte le casistiche possibili: bit flip, phase flip e bit-phase-flip.

\begin{esempio}[\textbf{Errori sul primo qubit}]
Consideriamo la Tabella \ref{tab:Shor_first_qubit_errors}: supponiamo di considerare separatamente tutti i possibili tre tipi di errori discreti che possono avvenire sul primo qubit
    \begin{table}[!hb]
	\centering
        \begin{tabular}{lcccc}
            \toprule
            Stabilizers & Codewords & bit flip ($X_0$) & phase flip ($Z_0$) & Bit-phase flip ($Y_0$) \\
            \midrule
            $Z_0Z_1$ & $+1$ & $-1$ & $+1$ & $-1$ \\
            $Z_1Z_2$ & $+1$ & $+1$ & $+1$ & $+1$ \\
            $Z_3Z_4$ & $+1$ & $+1$ & $+1$ & $+1$ \\
            $Z_4Z_5$ & $+1$ & $+1$ & $+1$ & $+1$ \\
            $Z_6Z_7$ & $+1$ & $+1$ & $+1$ & $+1$ \\
            $Z_7Z_8$ & $+1$ & $+1$ & $+1$ & $+1$ \\
            $X_0X_1X_2X_3X_4X_5$ & $+1$ & $+1$ & $-1$ & $-1$ \\
            $X_3X_4X_5X_6X_7X_8$ & $+1$ & $+1$ & $+1$ & $+1$ \\
            \bottomrule
        \end{tabular}\\
        \caption{Autovalori degli stabilizers del codice di Shor per tutti i possibili errori discreti che avvengono sul primo qubit (il codewords è dato dalla generica combinazione lineare degli stati \eqref{qe-shor2} e \eqref{qe-shor3}). Essendo tutti i casi distinguibili,  tutti i tre tipi di errori discreti possono essere identificati e corretti. }\label{tab:Shor_first_qubit_errors}
    \end{table}
    
    \noindent Si noti che, come nella Tabella \ref{tab:acomm_ZZ}, gli autovalori mostrati possono essere verificati ulteriormente andando a vedere se gli operatori che implementano i vari errori anticommutano con i rispettivi stabilizers. Ad esempio $\acomm{Z_0Z_1}{X_0}=0$ (prima riga e seconda colonna). È sempre possibile distinguere il tipo di errore (syndrome), dove avviene (in che blocco o qubit) e automatizzare questo processo.
\end{esempio}

\noindent Alla luce di questo discorso ci possiamo chiedere: che cosa ne è stata della problematica B. di inizio Sezione \ref{sec:intro_error_corr}? Il codice di Shor è sufficiente a correggere tutte le tipologie di errori, comprese quelli continui? 

\noindent In effetti, il codice Shor protegge da molto più di semplici errori di bit e phase flip su un singolo qubit: ora mostriamo che protegge da errori completamente arbitrari, a condizione che influiscano solo su un singolo qubit! La cosa interessante è che non è necessario eseguire alcun lavoro aggiuntivo per proteggersi da errori arbitrari: la procedura già descritta funziona perfettamente. Questo è un esempio del fatto straordinario che l'apparente continuum di errori che può verificarsi su un singolo qubit può essere corretto correggendo solo un sottoinsieme discreto di quegli errori; tutti gli altri possibili errori vengono corretti automaticamente da questa procedura! La procedura di discretizzazione degli errori è fondamentale per il motivo per cui la correzione degli errori quantistica funziona e dovrebbe essere considerata in contrasto con la correzione degli errori classica per i sistemi analogici, dove tale discretizzazione degli errori non è possibile. 

\noindent Che cosa produce un generico errore? Immaginiamo di avere un sistema quantistico descritto da una matrice densità $\rho=\op{\psi}{\psi}$ e di considerare solamente il nostro qubit tracciando $\rho$ sull'ambiente descritto dallo spazio di Hilbert $\mathcal{H}_E$:
\begin{equation*}
    \rho =\op{\psi}{\psi} \rightarrow \mathcal{E}(\rho) = \sum_k E_k \rho E_k^\dagger \, .
\end{equation*}
Supponendo che lo stato del qubit codificato sia $\ket \psi = \alpha\ket{\overline 0} + \beta\ket{\overline 1}$ prima che il rumore agisca, allora successivamente all'interazione con l'ambiente lo stato è descritto dalla matrice densità $\mathcal{E}(\op{\psi}{\psi}) = \sum_k E_k \op{\psi}{\psi} E_k^\dagger$. Per analizzare gli effetti della correzione dell'errore è più facile concentrarsi sull'effetto che essa ha su un singolo termine in questa somma, diciamo $E_k \op{\psi}{\psi} E_k^\dagger$. Come operatore sul solo qubit, $E_k$ può essere espanso come una generica (si veda la \eqref{generical_matrix_C2}) combinazione lineare dell'identità, del bit flip $X$, del phase flip $Z$ e del bit-phase flip $Y$:
\begin{equation*}
    E_k\ket \psi = \left(\alpha \mathbb{I} +\beta_x X + \beta_y Y + \beta_z Z\right)\ket \psi \, ,
\end{equation*}
dove $\alpha$, $\beta_x$, $\beta_y$ e $\beta_z$ sono coefficienti arbitrari reali: il comportamento continuo è chiaramente in questi coefficienti!
La syndrome measurement dell'errore fa collassare\footnote{Il coefficiente è irrilevante perché possiamo sempre normalizzare lo stato.} questa sovrapposizione in uno dei quattro stati $\ket \psi$, $X\ket\psi$, $Y\ket\psi$ o $Z\ket\psi$ da cui poi si può recuperare lo stato iniziale $\ket{\psi}$ applicando l'opportuna operazione di inversione. Lo stesso vale per tutti gli altri operation elements $E_k$: la misura causa il collasso dello stato in $S\ket{\psi}$, dove $S = \{ \mathbb{I}, X, Y, Z \}$, e tutti gli stati $S\ket{\psi}$ appartengono a differenti sottospazi degli errori mutualmente ortogonali tra loro.

\noindent Pertanto, la correzione degli errori comporta il ripristino dello stato originale, nonostante il fatto che l'errore sul qubit fosse arbitrario. Questo è un fatto fondamentale e profondo sulla correzione degli errori quantistica: correggendo solo un insieme discreto di errori (il bit flip, il phase flip e il bit-phase flip) un codice di correzione quantistica degli errori è in grado di correggere automaticamente una classe di errori apparentemente molto più ampia (continua!). Ricordiamo però che tutto questo discorso è basato sull'assunzione che il rumore può coinvolgere  solo un singolo qubit.

\section{Codice di correzione di Steane a 7 qubit}
Nel 1996 il fisico inglese Andrew Steane propose un codice di correzione degli errori basato sull'utilizzo di soli 7 qubit. Il codice di Steane utilizza i seguenti 6 operatori per la diagnostica degli errori:
\begin{align*}
    M_0 &= X_0X_4X_5X_6, &N_0 &= Z_0Z_4Z_5Z_6, \\
    M_1 &= X_1X_3X_5X_6, &N_1 &= Z_1Z_3Z_5Z_6, \\
    M_2 &= X_2X_3X_4X_6, &N_2 &= Z_2Z_3Z_4Z_6.
\end{align*}
Notiamo che soddisfano le proprietà seguenti:
\begin{enumerate}
    \item $M_i^2 = N_i^2=\mathbb{I}$ per $i = 0, 1, 2$;
    \item Sono operatori commutanti, quindi $\comm{M_i}{M_j}=\comm{N_i}{N_j}=\comm{M_i}{N_j}=0$.
    
    I primi due commutatori sono ovvi. L'ultimo commutatore, invece, è meno immediato. È utile osservare che i termini che agiscono sullo stesso qubit anticommutano (da $\acomm{X_i}{Z_i} = 0$) e producono un segno meno. Nel caso in cui $i=j$, abbiamo quattro meno moltiplicati tra loro mentre per $i\neq j$ solo due meno moltiplicati tra loro: in ogni caso i segni meno scompaiono e la commutazione è dimostrata. 
\end{enumerate}

\noindent Trattandosi di operatori commutanti possiamo simultaneamente diagonalizzarli: l'idea è quella di utilizzare questo autospazio comune per codificare i qubit logici del codewords. Lavorando con 7 qubit, lo spazio totale di cui necessitiamo deve essere di dimensione $\text{dim} \mathcal{H}=2^7$: se ci restringiamo agli autovettori di $M_i$ e $N_i$ che hanno autovalore $+1$ allora effettivamente la dimensione del codewords sarà $2^7/2/2/2/2/2/2 = 2$, ossia la dimensione di uno spazio di un singolo qubit logico.

\noindent Il problema è quindi come costruire questo autospazio comune per $\ket{\overline 0}$ e $\ket{\overline 1}$. Il punto importante è che non è necessario conoscere la forma degli stati $\ket{\overline 0}$ e $\ket{\overline 1}$. Consideriamo uno stato generico $\ket{\psi}$; notiamo che
\begin{equation}\label{eigs_psi_M}
    M_i\left( (\mathbb{I}+M_i) \ket \psi \right) = (M_i+\mathbb{I}) \ket{\psi} \, ,
\end{equation}
quindi $(\mathbb{I}+M_i) \ket \psi$ è autostato di $M_i$ con autovalore $+1$ indipendentemente dalla forma di $\ket{\psi}$. L'idea è quindi quella di iniziare con gli stati $\ket{0000000}$ e $\ket{1111111}$ e di applicare degli operatori come in \eqref{eigs_psi_M}:
\begin{align*}
    \ket{\overline 0} &= \frac{\mathbb{I}+M_2}{\sqrt 2}\frac{\mathbb{I}+M_1}{\sqrt 2}\frac{\mathbb{I}+M_0}{\sqrt 2}\ket{0000000} \, , \\
    \ket{\overline 1} &= \frac{\mathbb{I}+M_2}{\sqrt 2}\frac{\mathbb{I}+M_1}{\sqrt 2}\frac{\mathbb{I}+M_0}{\sqrt 2}\ket{1111111} \, .
\end{align*}
Questo rappresenta il modo corretto per codificare i qubit logici $\ket{ \overline{0}}$ e $\ket{\overline{1}}$ perché sono entrambi autostati di tutti gli operatori $M_i$ e $N_i$. 

\noindent Dimostriamolo esplicitamente. Il fatto che siano autostati di $M_i$ con autovalore $+1$ è evidente dalla \eqref{eigs_psi_M}, perciò la domanda è: che cosa succede agli $N_i$? Ricordiamo che $\comm{N_i}{M_j} = 0$ per qualsiasi $i,j$ e notiamo inoltre che ciascun $N_i$ è dato da un prodotto di 4 operatori $Z$, i quali hanno autostati $\ket{0}$ e $\ket{1}$ con autovalori $+1$ e $-1$ rispettivamente: il fatto che $N_i \ket{\overline{0}} = \ket{\overline{0}}$ è quindi ovvio, mentre $N_i \ket{\overline{1}} = (-1)^4 \ket{\overline{1}} = \ket{\overline{1}}$ perché si hanno sempre 4 operatori $Z$. 

\noindent È possibile verificare\footnote{Esercizio! Si utilizzi $(\mathbb{I} + M_i)^2 = 2(\mathbb{I} + M_i)$. Per verificare la normalizzazione si noti che nei 3 prodotti di questi operatori solamente il prodotto delle 3 identità sopravvive: il motivo deriva dal fatto che l'azione di ciascun $M_i$ su $\ket{\overline{0}}$ o $\ket{\overline{1}}$ inverte $0 \leftrightarrow 1$, quindi il braket rimanente coinvolgerà sempre almeno un prodotto scalare $\braket{0}{1} = \braket{1}{0} = 0$.} che i qubit $\ket{\overline 0}$ e $\ket{\overline 1}$, definiti a partire da $M_0$, $M_1$ ed $M_2$, possono essere utilizzati come qubit logici perché sono un sistema ortonormale:
\begin{align*}
    \ip{\overline 0}{\overline 0}=\ip{\overline 1}{\overline 1}=1 \, ,\\
    \ip{\overline 0}{\overline 1}=\ip{\overline 1}{\overline 0}=0 \, .
\end{align*}

\noindent Consideriamo ora il generico stato del codewords nella base logica, ossia $\ket \psi=\alpha \ket{\overline 0} + \beta \ket{\overline 1}$, e supponiamo che avvenga un'interazione esterna, ad esempio con l'ambiente. Lo stato dopo l'interazione, la quale è implementata da un opportuno operatore che agisce come errore, sarà descritto da $E_k\ket \psi$ con l'assunzione che l'errore possa avvenire su un singolo qubit. I possibili errori che possono avvenire non sono altro che generiche matrici $2 \times 2$ che possono essere parametrizzate da una combinazione lineare di matrici di Pauli. Questo significa che gli $E_k$ non sono altro che una collezione di 
\begin{equation*}
    \{X_i, Y_i, Z_i \}_{i=0,\dots,6} \, ;
\end{equation*}
in totale ci sono quindi $3 \times 7 = 21$ possibili errori da distinguere: il $3$ è dovuto al fatto che abbiamo come sorgente di errore $X$, $Y$ e $Z$ (3 errori indipendenti su ciascun qubit) mentre il $7$ perché stiamo lavorando con un codice a $7$ qubit. Necessitiamo quindi di $21$ "spazi degli errori" mutualmente ortogonali per correggere tutti i possibili errori. Consideriamo ora la Tabella \ref{tab:steane-bit-flip-cases}, la quale mostra se un determinato operatore $M_i$ o $N_i$ contiene o meno l'operatore corrispondente.  

\begin{table}[!ht]
	\centering
    \begin{tabular}{lccccccc}
        \toprule
        bit flip & $X_0$     & $X_1$     & $X_2$     & $X_3$     & $X_4$     & $X_5$     & $X_6$    \\
        \midrule
        $M_0$    & $\bullet$ &           &           &           & $\bullet$ & $\bullet$ & $\bullet$ \\
        $M_1$    &           & $\bullet$ &           & $\bullet$ &           & $\bullet$ & $\bullet$ \\
        $M_2$    &           &           & $\bullet$ & $\bullet$ & $\bullet$ &           & $\bullet$ \\
        \toprule
        phase flip & $Z_0$     & $Z_1$     & $Z_2$     & $Z_3$     & $Z_4$     & $Z_5$     & $Z_6$    \\
        \midrule
        $N_0$      & $\bullet$ &           &           &           & $\bullet$ & $\bullet$ & $\bullet$ \\
        $N_1$      &           & $\bullet$ &           & $\bullet$ &           & $\bullet$ & $\bullet$ \\
        $N_2$      &           &           & $\bullet$ & $\bullet$ & $\bullet$ &           &           \\
        \bottomrule
    \end{tabular}\\
    \caption{I 6 error-syndrome operators $M_i$ e $N_i$ ($i = 0, 1, 2$) per il codice di Steane a 7 qubit. Un punto ($\bullet$) indica se un dato operatore $X_i$ appare in $M_j$ e se un dato operatore $Z_i$ appare in $N_j$.}
    \label{tab:steane-bit-flip-cases}
\end{table}

\noindent Abbiamo già discusso la logica generica del codice di correzione degli errori nella Sezione \ref{sec:stabilizers}: il codice deve essere realizzato in maniera tale che se ho un errore tale per cui $\comm{E_k}{M}=0$ ($M$ è uno degli error-syndrome) e sono in un autostato $\ket{\psi}$ di $M$ ($M\ket \psi = \ket \psi$), allora $M\left(E\ket \psi\right)=EM\ket \psi=E\ket\psi$ e rimaniamo quindi nel medesimo autospazio; viceversa se $\acomm{E_k}{M}=0$, allora $M\left(E\ket \psi\right)=-EM\ket \psi=-E\ket\psi$ e quindi finiamo in un autospazio con un valore differente dell'osservabile. Per capire meglio questo discorso e il significato della Tabella \ref{tab:steane-bit-flip-cases} si veda il seguente esempio. 

\begin{esempio}
    Supponiamo di avere un bit flip error $X_2$: sappiamo che $\comm{X_2}{M_i}=0$ per ogni $i$, ma $\comm{X_2}{N_2}\neq 0$ perché è l'unico operatore che contiene $Z_2$. Se eseguissimo una misura degli error-syndrome avremmo tutti autovalori $+1$, tranne che per la riga corrispondente a $N_2$, nella quale si ha un autovalore pari a $-1$ per la presenza di $Z_2$. Questo non solo ci dice se c'è un bit flip error, ma ci dice anche dove è localizzato, così da poterlo correggere. I simboli "$\bullet$" nella Tabella \ref{tab:steane-bit-flip-cases} indicano dove il segno sarà invertito, ossia dove l'autovalore della misura dell'error-syndrome sarà $-1$! Lo stesso discorso ovviamente lo si può fare per un phase flip error $Z_i$: si otterrà un segno $-1$ nella corrispondente riga di $M_j$ che conterrà l'operatore $X_i$ dello stesso autospazio dell'errore. Questa procedura può essere utilizzata anche per errori di bit-phase flip $Y_i$: in questo caso i segni $-1$ appaiono in entrambi gli operatori $M_j$ e $N_j$. 
\end{esempio}

\noindent Ricapitolando: 
\begin{itemize}
    \item I bit flip error $X$ sono rilevati da dei "$\bullet$" nella parte bassa della tabella.
    
    \item I phase flip error $Z$ sono rilevati da dei "$\bullet$" nella parte alta della tabella.
    
    \item I bit-phase flip error sono rilevati da dei "$\bullet$" sia nella parte bassa sia in quella alta della tabella.
\end{itemize}
In questo modo siamo in grado di distinguere separatamente senza alcuna degenerazione tutti i possibili $21$ errori perché tutti gli errori a singolo qubit sono localizzati in \textbf{singole colonne}. 

\noindent Supponiamo di considerare errori multipli: saremmo in grado di rilevarli utilizzando la medesima trattazione? La risposta è affermativa perché lo spazio di Hilbert $\mathcal{H}$ dei 7 qubit ha dimensione $2^7= 128$, quindi è grande abbastanza da poter trattare anche questo tipo di errori. Per capire questo fatto notiamo che fino ad ora abbiamo lavorato con spazi del tipo $C\oplus E_a C$, ossia spazi che contenevano la somma diretta tra il codewords e gli spazi degli errori che servivano per correggere $C$. In questo caso avremo $\dim (C\oplus E_a C) = 2+2\times21=44$: se valutiamo la differenza di dimensione tra lo spazio di Hilbert totale $\mathcal{H}$ e lo spazio $C\oplus E_a C$ che serve per correggere tutti i possibili 21 errori a singolo qubit, ci rimane un sottospazio di dimensione $128-44 = 84$. Questi spazi rimanenti non sono nient'altro che gli spazi degli errori multipli! Ad esempio,  se si verifica  un errore simultaneo sui qubit $i$ e $j$ con $i \neq j$, possiamo implementarlo come $X_i Z_j \ket{\overline 0}$ e $X_i Z_j \ket{\overline 1}$: in totale avremo quindi $7 \times 6 + 7 \times 6 = 84$ errori simultanei, i quali sono rilevabili per mezzo della Tabella \ref{tab:steane-bit-flip-cases} dalla presenza di "$\bullet$" in \textbf{2 colonne simultaneamente}. In principio siamo quindi in grado di correggere tutti i tipi di errori su singoli qubit o su coppie di qubit.

\noindent Chiaramente, dopo aver analizzato il funzionamento del codice di Steane potrebbe sorgere spontanea la domanda: se esistono, quali sono i vantaggi del codice di Steane rispetto al codice di Shor?
\begin{itemize}
    \item Innanzitutto nel codice di Steane è relativamente più semplice il meccanismo di localizzazione e correzione degli errori, dal momento che il numero di qubit coinvolti è minore (7 rispetto a 9). In generale, una volta rilevati gli errori, è necessario intervenire con degli opportuni gate per risolverli: lavorare con un sistema a 7 qubit è più semplice rispetto a 9 qubit in quanto le matrici (gate) con cui si è costretti a lavorare sono $2^7\times 2^7$, molto più piccole rispetto ai gate di risoluzione degli errori nel codice di Shor. 
    Inoltre gli operatori  logici \texttt{X-gate}, \texttt{Z-gate} e \texttt{H-gate} che implementano le operazioni elementari sulle codewords hanno una forma particolarmente semplice nel codice di Steane:
    \begin{align*}
        \overline X &= X_0X_1X_2X_3X_4X_5X_6 \, , \\
        \overline Z &= Z_0Z_1Z_2Z_3Z_4Z_5Z_6 \, , \\
        \overline H &= H_0H_1H_2H_3H_4H_5H_6 \, .
    \end{align*}
    
    \item Un altro vantaggio del codice di Steane riguarda il concetto della \textbf{fault tolerance}, che approfondiremo nelle prossime sezioni. Il punto  fondamentale verte sul fatto che sia necessario correggere gli errori in maniera sufficientemente veloce per poter effettuare correttamente il calcolo desiderato.
\end{itemize}

\noindent Se volessimo codificare un qubit logico usando $n$ qubit necessiteremmo uno spazio di Hilbert di dimensione $\dim \mathcal{H}=2^n$, il quale dovrebbe necessariamente contenere $C \oplus E_aC$: la condizione per $n$ che deve essere soddisfatta prende il nome di \textbf{Quantum Hamming Bound} ed è data da
\begin{equation*}
    \dim \mathcal{H} \geq \dim (C \oplus E_a C) \, , \quad \Rightarrow \quad 2^n \geq 2 + 2 \times 3 \times n \, ,
\end{equation*}
($2 \times 3$ perché è necessario correggere entrambi i qubit logici di $C$ per tutti e 3 gli errori $X, Y, Z$). Semplificando un 2, la relazione non è altro che 
\begin{equation}\label{quantum_hamming_bound}
    2^{n-1}\geq 1+3n \, ,
\end{equation}
la quale ci dice qual è la dimensione minima che possiamo usare. Questo risultato è applicabile unicamente a codici \textbf{non-degeneri}, ossia codici in cui possiamo distinguere e localizzare precisamente l'errore. Per contro, il codice di Shor è degenere quindi ad esso non si applica la disuguaglianza \eqref{quantum_hamming_bound}. 
Il Quantum Hamming Bound è saturato per $n=5$, ci chiediamo quindi se esista un codice che, sfruttando soli 5 qubit, corregga efficientemente tutti i possibili errori. La risposta è affermativa ed è possibile verificare che gli error-syndrome operators sono i seguenti
\begin{align*}
    M_0 &= Z_1X_2X_3Z_4 \, , &M_1 &= Z_0 X_3 X_4 Z_0 \, , \\
    M_2 &= Z_3X_4X_0Z_1 \, , &M_3 &= Z_4X_0X_1Z_2 \, .
\end{align*}
Non ci addentriamo nel suo studio, ma ci limitiamo a dire che lavora in maniera simile al codice di Steane a 7 qubit perché è possibile dimostrare che esistono dei qubit logici con $n=5$ che sono autovettori degli operatori precedenti con autovalore $+1$. Lo spazio di Hilbert è quindi diviso da questi operatori in $2^5/2/2/2/2 = 2$, ossia proprio la dimensione del codewords. 

\noindent Infine ci chiediamo: che cosa succederebbe se abbandonassimo l'ipotesi di codice non-degenere? Esiste un codice di correzione con $n < 5$? Intuitivamente possiamo pensare che, dato che lo spazio degli errori $E_a C$ diventa più piccolo, allora si necessiterebbe di dimensioni inferiori per $\mathcal{H}$ poiché serve meno spazio per correggere gli errori. Nonostante ciò, la risposta, che è complicata da dimostrare, è no: $n = 5$ è il numero minimo di qubit per un un generico codice di correzione degli errori anche per codici degeneri. 

    %%%%%%%%%%%%%%
% LECTURE 14 %
%%%%%%%%%%%%%%
\newpage 

\noindent \lecture{14}{22/11/2021}
\vspace{0.5cm}

\section{Fault tolerance}
Prima di discutere in dettaglio il successivo codice di correzione degli errori spendiamo alcune parole riguardanti le diverse tipologie di codici. Ne esistono di diversi esempi:
\begin{itemize}
    \item I \textbf{CSS codes} (da Calderbank-Shor-Steane), i quali generalizzano gli analoghi classici della correzione degli errori al contesto del QC;
    \item Gli \textbf{stabilizer codes}, dei quali abbiamo visto alcuni esempi nelle sezioni precedenti;
    \item Vi è il cosiddetto \textbf{toric/surface code}, il quale fornisce un approccio topologico\footnote{Si tratta di codici particolarmente ben protetti rispetto all'interazione dei qubit con l'ambiente.} al QC;
    \item E molti altri\dots
\end{itemize}
In generale esiste un'intera teoria generale riguardante questi codici, la quale evidenzia tutte le loro analogie e differenze. Si pensi ad esempio al modo con cui correggono gli errori, al numero di qubit utilizzati, ecc.

\noindent Quanto velocemente questi codici correggono gli errori? Questo è il soggetto della cosiddetta \textbf{fault tolerance} in QC (esiste un analogo classico), perché quando si ha un errore bisogna essere certi di non averne troppi (nel senso che sono talmente tanti da non poter essere corretti in un tempo ragionevole) e inoltre bisogna assicurarsi che i circuiti non possiedano gate che propaghino questi errori. Anche in questo caso esiste ua teoria generale riguardante il come costruire circuiti quantistici che siano ottimizzati per la correzione degli errori. 

\noindent La logica della fault tolerance è la seguente: si fissa un probabilità $p$ che un qualche elemento del circuito non svolga correttamente il proprio lavoro (si pensi ad esempio alla probabilità di fallimento di un filo o un gate) e, data $p$, si vuole conoscere quanti qubit aggiuntivi è necessario introdurre per correggere tutti questi errori. In generale si fissa una soglia, la quale non è altro che la probabilità che il circuito fornisca l'output desiderato: lo scopo è quello di bilanciare opportunamente le componenti del circuito affinché esso lavori al di sotto di tale soglia. Più precisamente: il fine ultimo è quello di conoscere la probabilità che un singolo componente del circuito fallisca; in questo modo, con un numero polinomiale di qubit extra, si possono correggere gli errori avendo la certezza che il risultato sia corretto a meno di una soglia fissata. In generale, i codici fault tolerant sono quelli in cui si introducono un ragionevole ammontare di componenti extra senza modificare l'efficienza e la velocità di esecuzione dei codici. 

\noindent Non entreremo nel dettaglio, tuttavia è bene ricordare ciò che sottolineammo nella Sottosezione \ref{subsec:quantum_gates}: l'insieme di gate $\{ H, T, S, \texttt{CNOT} \}$ è universale sebbene $S$ e $T$ non siano indipendenti perché $S= \sqrt{Z}$ e $T = \sqrt{S}$; nonostante $S$ sembri ridondante nella descrizione, affinché si abbia un codice fault tolerant è necessario tenere in considerazione anche questo gate. 

\section{Toric code}
L'ultimo esempio che analizziamo di codice di correzione degli errori è il cosiddetto \textbf{toric code} (conosciuto con questo nome nella letteratura della fisica della materia condensata), detto anche più genericamente \textbf{surface code}. È un codice peculiare per diverse ragioni: innanzitutto, dal punto di vista della correzione degli errori, è un codice fault tolerant perché è molto "robusto" contro gli errori; in secondo luogo è molto interessante perché è legato ad altre branche della fisica oltre al QC: si tratta di un esempio di una situazione in cui appare una fase topologica non banale della materia e, per tale motivo, è stato in passato uno dei modelli che ha condotto all'idea della cosiddetta \textbf{topological quantum computing}. Dal nostro punto di vista è interessante per la correzione degli errori e per l'approccio topologico al QC. 

\noindent Come mostra la Figura \ref{subfig:Toric_lattice_1}, consideriamo un reticolo $L \times L$ di qubit in cui questi ultimi "vivono" sui link (collegamenti) del reticolo (si vedano i puntini rossi sui lati dei quadrati). Dal punto di vista pratico si costruisce un array periodico di qubit su un reticolo. Dato che sono presenti 2 qubit indipendenti per ogni faccia (assumiamo il qubit a sinistra e in basso nei diversi quadrati) allora la dimensione dello spazio di Hilbert totale non è altro che
\begin{equation*}
    \dim \mathcal{H} = 2^n = 2^{2 L^2} \, ,
\end{equation*}
dove $L^2$ è il numero di link/facce (si hanno $L$ righe e $L$ colonne). Quindi in totale avremo $n = 2 L^2$ qubit indipendenti. In generale questo codice funziona molto bene quando si ha un grande numero di qubit.

\begin{figure}[!h]
	\centering	
	\subfloat[][Reticolo di qubit.\label{subfig:Toric_lattice_1} ]{\includegraphics[scale=.43,keepaspectratio]{images/Toric_lattice_1}} \quad
	\subfloat[][Operatore $A_v$.\label{subfig:Toric_lattice_2} ]{\includegraphics[scale=.43,keepaspectratio]{images/Toric_lattice_2}} \quad 
	\subfloat[][Operatore $B_p$.\label{subfig:Toric_lattice_3} ]{\includegraphics[scale=.43,keepaspectratio]{images/Toric_lattice_3}}
	\caption{\eqref{subfig:Toric_lattice_1} Reticolo $L \times L$ costituito da $2L^2$ qubit indipendenti. Per ogni plaquette ci sono 2 qubit indipendenti, quello in basso e quello a sinistra. \eqref{subfig:Toric_lattice_2} L'operatore rappresentato è esplicitamente $A_v = X_1 X_2 X_3 X_4$. \eqref{subfig:Toric_lattice_3} L'operatore rappresentato è esplicitamente $B_p = Z_1 Z_2 Z_3 Z_4$.}
    \label{fig:Toric_lattice}
\end{figure}

\noindent Il codice fu proposto per la prima volta dal fisico Alexei Kitaev ed è realizzato su un reticolo con condizioni al bordo periodiche (PBC: Periodic Boundary Conditions): dal punto di vista topologico considerare PBC sul quadrato di Figura \ref{subfig:Toric_lattice_1} significa porre il reticolo di qubit su un \textbf{toro} (il reticolo è bidimensionale, ma il volume è tridimensionale). Per realizzazioni concrete il toro risulta tuttavia poco pratico: in realtà la fisica del codice può essere ben rappresentata anche mediante strutture planari con opportune condizioni al bordo (per tale motivo il codice è anche detto \textbf{surface code}); anche l'idea di realizzazione di questa procedura su strutture planari fu proposta da Kitaev.

\noindent Ancora una volta la logica del codice è la stessa di quelle viste nelle precedenti sezioni perché è un \textbf{stabilizer code}. Vi sono due tipologie di stabilizers: per ogni vertice $v$ del reticolo si costruisce un operatore $A_v$ dato dal prodotto degli \texttt{X-gate} di ogni link appartenente al vertice e, similmente, per ogni faccia $p$ del reticolo (detta \textbf{plaquette}) si definisce l'operatore $B_p$ come prodotto dei 4 \texttt{Z-gate} sui link della plaquette. Si faccia riferimento alle Figure \ref{subfig:Toric_lattice_2} e \ref{subfig:Toric_lattice_3} per una rappresentazione grafica di questi operatori (chiaramente, così come i qubit, anche i gate si trovano sui link del reticolo). Dal punto di vista degli operatori avremo 
\begin{equation}\label{A_B}
    A_v = \prod_{j \in v} X_j \, , \qquad B_p = \prod_{j \in p} Z_j \, .
\end{equation}
In totale si hanno $L^2$ differenti operatori $A_v$ (uno per ognuno degli $L^2$ vertici) e $L^2$ differenti operatori $B_p$ (uno per ognuna delle $L^2$ plaquette): abbiamo quindi $2L^2$ differenti stabilizers. L'idea è, come al solito negli stabilizer codes, quello di codificare i qubit logici nel sottospazio (dato un certo autovalore) di questo insieme di operatori: più precisamente sappiamo che possiamo codificare il codewords nell'autospazio comune di questi operatori se commutano tra loro e il loro quadrato è l'identità. È evidente che per ogni $v$ e $p$, in quanto prodotto di matrici di Pauli, avremo $A^2_v = B^2_p = \mathbb{I}$. Inoltre si ha
\begin{equation}\label{commutatori_A_B}
    \comm{A_v}{B_p} = 0 \, , \qquad \comm{A_v}{A_{v'}} = 0 \, , \qquad \comm{B_p}{B_{p'}} = 0 \, \, \quad \forall \, v,p,v',p' \, ;
\end{equation}
il secondo e il terzo commutatore sono banali, tuttavia il primo è meno ovvio. Chiaramente questo commutatore è nullo quando $A_v$ e $B_p$ agiscono su qubit diversi (vertice e plaquette disgiunti), tuttavia potrebbe non essere nullo nel caso in cui $A_v$ sia localizzato in uno dei quattro vertici di una plaquette $p$ (si pensi al vertice \ref{subfig:Toric_lattice_2} posto su uno dei quattro vertici della plaquette \ref{subfig:Toric_lattice_3}): nonostante questa situazione, le matrici $X$ e $Z$ che agiscono sui qubit di un medesimo sottospazio sono sempre 2. Questo significa che i due anticommutatori $\acomm{Z_i}{X_i} = 0$ producono $(-1)^2 = 1$, quindi anche in questo caso il primo commutatore è dimostrato. 

\noindent Possiamo definire come codewords il sottospazio di $\mathcal{H}$ corrispondente all'autospazio comune agli operatori $A_v$ e $B_p$ con autovalore $+1$, ossia
\begin{equation*}
    C = \{ \text{Sottospazio comune agli operatori con autovalore } A_v = B_p = +1 \} \, .
\end{equation*}
Qual è la dimensione di $C$? Ricordando dalla \eqref{A_B} che gli stabilizers sono prodotti di matrici di Pauli, essi "tagliano" sempre $\mathcal{H}$ in due sottospazi della medesima dimensione: dato che abbiamo $L^2$ operatori $A_v$ e $L^2$ operatori $B_p$ allora si ha $\dim C = 2^{2L^2}/2^{L^2}/2^{L^2} = 1$, quindi sembrerebbe che non possiamo codificare i qubit in $C$. Il problema è che ci siamo dimenticati che non tutti questi stabilizers sono indipendenti! Essi soddisfano infatti
\begin{equation}\label{constraint_A_B}
    \prod_v A_v = \mathbb{I} \, , \qquad \prod_p B_p = \mathbb{I} \, ;
\end{equation}
queste proprietà derivano dal fatto che nei prodotti di tutti i possibili vertici e plaquette ci sono sempre almeno 2 matrici di Pauli in comune tali che $X_i^2 = \mathbb{I}$ e $Z_i^2 = \mathbb{I}$. Si pensi ad esempio all'operatore $A_{v_1}$ della Figura \ref{subfig:Toric_lattice_2} adiacente ad un altro $A_{v_2}$, i quali hanno un link in comune e quindi gli \texttt{X-gate} di quel link daranno $X^2 = \mathbb{I}$; discorso simile per due plaquette adiacenti $B_{p_1}$ e $B_{p_2}$ della Figura \ref{subfig:Toric_lattice_3}, le quali hanno un link comune che darà $Z^2 = \mathbb{I}$. I vincoli \eqref{constraint_A_B} fanno sì che si abbiano in totale  $(L^2-1) + (L^2-1)$ operatori $A_v$ e $B_p$ indipendenti: dunque il codewords ha dimensione 
\begin{equation*}
    \dim C = 2^{2 L^2} / 2^{L^2-1} / 2^{L^2-1} = 4 \, .
\end{equation*}
Questo significa che nel toro  possiamo codificare fino a 4 stati logici, ovvero 2 qubit logici. In realtà nel caso planare  si ha $\dim C = 2$, quindi possiamo codificare un singolo qubit logico (come nei codici precedenti). 

\noindent Come costruiamo gli stati logici del codewords $C$? Possiamo procedere in maniera esattamente analoga al caso di Steane: partiamo dallo stato $\ket{000 \ldots 0}$ (prodotto tensoriale dei $2L^2$ qubit indipendenti del reticolo) e calcoliamo 
\begin{equation}\label{logical_00}
    \ket{\overline{00}} = \prod_v \frac{(\mathbb{I}+A_v)}{\sqrt{2}} \ket{000 \ldots 0} \, ;
\end{equation}
sappiamo che in questo modo otteniamo automaticamente un autostato di qualsiasi $A_v$ con autovalore $+1$ perché $A_v (\mathbb{I} + A_v) \ket{\psi} = (A_v + \mathbb{I}) \ket{\psi}$. Analogamente avremo che $\ket{\overline{00}}$ è anche autostato di ogni $B_p$ con autovalore $+1$ perché valgono i commutatori \eqref{commutatori_A_B} e perché $B_p$ è un prodotto di \texttt{Z-gate} ($Z \ket{0} = \ket{0}$):
\begin{equation*}
    B_p (\mathbb{I} + A_v) \ket{000 \ldots 0} = (\mathbb{I} + A_v) B_p \ket{000 \ldots 0} = (\mathbb{I} + A_v) \ket{000 \ldots 0} \, .
\end{equation*}
In aggiunta allo stato $\ket{\overline{00}}$, dato che $\dim C = 4$, ci sono altri 3 stati logici che chiamiamo $\ket{\overline{01}}$, $\ket{\overline{10}}$ e $\ket{\overline{11}}$. Al posto che costruirli scrivendo formule analoghe alla \eqref{logical_00} possiamo identificare degli opportuni operatori logici $\overline{X}_1$, $\overline{X}_2$, $\overline{Z}_1$ e $\overline{Z}_2$ tali che permettano di costruire questi ultimi a partire da $\ket{\overline{00}}$: $\overline{X}_1 \ket{\overline{00}} = \ket{\overline{10}}$, ecc. Come riferimento grafico per la discussione che segue si vedano i due reticoli di Figura \ref{fig:logical_X_Z}.   

\begin{figure}[!h]
	\centering	
	\subfloat[][Operatori logici $\overline{Z}_1$ e $\overline{Z}_2$.\label{subfig:logical_X_Z_1} ]{\includegraphics[scale=.43,keepaspectratio]{images/logical_X_Z_1}} \quad
	\subfloat[][Operatori logici $\overline{X}_1$ e $\overline{X}_2$.\label{subfig:logical_X_Z_2} ]{\includegraphics[scale=.43,keepaspectratio]{images/logical_X_Z_2}} 
	\caption{Le linee che rappresentano gli operatori logici $\overline{Z}_1$, $\overline{Z}_2$, $\overline{X}_1$ e $\overline{X}_2$ non sono altro che linee chiuse grazie alle PBC, quindi sono loop attorno al reticolo toroidale. I pallini rappresentati in corrispondenza dei link indicano i singoli gate che costituiscono il prodotto di quell'operatore logico.}
    \label{fig:logical_X_Z}
\end{figure}

\noindent Come mostrato nella Figura \ref{subfig:logical_X_Z_1}, definiamo $\overline{Z}_1$ come prodotto di tutti gli \texttt{Z-gate} individuati dall'intersezione della linea verticale passante per i link del reticolo; similmente $\overline{Z}_2$ è definito come prodotto degli \texttt{Z-gate} individuati dall'intersezione della linea orizzontale passante per i link. Esplicitamente avremo
\begin{equation}\label{logical_Z_1_2}
    \overline{Z}_1 = \prod_{i \in \text{vline}} Z_i \, , \qquad \overline{Z}_2 = \prod_{i \in \text{hline}} Z_i \, ,
\end{equation}
dove le diciture "vline" e "hline" indicano rispettivamente la linea verticale e la linea orizzontale passante per i link del reticolo. 

\noindent Similmente agli operatori \eqref{logical_Z_1_2} consideriamo ora la Figura \ref{subfig:logical_X_Z_2}. Se consideriamo questa volta il \textbf{reticolo duale}, ossia l'analogo reticolo che si costruisce passando per i punti medi dei link del reticolo di partenza, possiamo definire $\overline{X}_2$ come prodotto degli \texttt{X-gate} intercettati dalla linea verticale passante per il reticolo duale; infine definiamo $\overline{X}_1$ come prodotto degli \texttt{X-gate} intercettati dalla linea orizzontale passante per il reticolo duale. In termini di operatori scriviamo
\begin{equation}\label{logical_X_1_2}
    \overline{X}_1 = \prod_{i \in \text{hline}} X_i \, , \qquad \overline{X}_2 = \prod_{i \in \text{vline}} X_i \, ,
\end{equation}
dove questa volta le diciture "hline" e "vline" indicano rispettivamente la linea orizzontale e la linea verticale passante per i link del reticolo duale. È importante notare che grazie alle PBC le linee degli operatori $\overline{Z}_1$, $\overline{Z}_2$, $\overline{X}_1$ e $\overline{X}_2$ rappresentate nella Figura \ref{fig:logical_X_Z} non sono altro che \textbf{loop} (linee chiuse) passanti attorno al reticolo toroidale. D'ora in avanti ci riferiremo a queste linee chiamandole equivalentemente con il termine "loop".  

\noindent Come possiamo essere certi che gli operatori \eqref{logical_Z_1_2} e \eqref{logical_X_1_2} siano i corretti operatori logici? Sappiamo che gli operatori logici agiscono sul sottospazio $C$ e producono un nuovo stato $\ket{\psi} \in C$; questo significa che se gli operatori appena definiti commutano con tutti gli $A_v$ e $B_p$ allora la loro azione su stati del codewords produce altri stati di $C$, altrimenti, se anticommutano con $A_v$ e $B_p$, la loro azione su $C$ produce nuovi stati non più facenti parte del codewords, ossia si tratta di operatori corrispondenti ad errori. In altre parole dobbiamo quindi verificare che per ogni $i = 1 ,2$ e per qualsiasi $v$ e $p$ avremo
\begin{align}
    \comm{ \,\overline{X}_i}{ A_v} &= 0 \, , &\comm{ \,\overline{Z}_i}{ B_p} &= 0 \label{comm_log_banali} \\
    \comm{ \,\overline{X}_i}{ B_p} &= 0 \, , &\comm{ \,\overline{Z}_i}{ A_v} &= 0 \, . \label{comm_log_meno_banali}
\end{align}
Chiaramente, ricordando le definizioni \eqref{A_B}, le \eqref{comm_log_banali} sono banali. Per quanto riguarda invece le \eqref{comm_log_meno_banali} il discorso è più sottile: se si considerano operatori $A_v$ (vertici di Figura \ref{subfig:Toric_lattice_2}) e $B_p$ (plaquette di Figura \ref{subfig:Toric_lattice_3}) disgiunti rispetto alle linee individuate rispettivamente da $\overline{Z}_i$ e $\overline{X}_i$ allora i commutatori sono ancora una volta banali. Se si considera tuttavia un vertice $A_v$ sulla linea $\overline{Z}_i$ allora esso presenterà alcuni operatori agenti sul medesimo sottospazio di quelli del loop: gli \texttt{X-gate} di $A_v$ sul loop $\overline{Z}_i$ saranno sempre due (sopra e sotto per $\overline{Z}_1$ e destra e sinistra per $\overline{Z}_2$), quindi come per \eqref{commutatori_A_B}, le due anticommutazioni producono $(-1)^2 = +1$ e il commutatore è dimostrato. Vale un discorso analogo per gli operatori $B_p$: quando la plaquette di $B_p$ si interseca con una delle due linee $\overline{X}_i$ allora saranno sempre e solamente 2 gli \texttt{Z-gate} di $B_p$ agenti sul medesimo sottospazio degli \texttt{X-gate} del loop $\overline{X}_i$ (sopra e sotto per $\overline{X}_2$ e destra e sinistra per $\overline{X}_1$); perciò le due anticommutazioni producono come prima $(-1)^2 = +1$ e il commutatore è verificato\footnote{Per capire ancora meglio questo discorso si provi a sovrapporre gli operatori $A_v$ e $B_p$ delle Figure \ref{subfig:Toric_lattice_2} e \ref{subfig:Toric_lattice_3} con le linee delle Figure \ref{subfig:logical_X_Z_1} e \ref{subfig:logical_X_Z_2} rispettivamente: apparirà evidente come sono sempre due gli operatori anticommutanti agenti sul medesimo sottospazio.}.

\noindent Affinché $\overline{Z}_1$, $\overline{Z}_2$, $\overline{X}_1$ e $\overline{X}_2$ siano i corretti operatori logici non solo devono essere verificati i commutatori sopra, ma inoltre deve valere
\begin{equation*}
    \acomm{\overline{X}_1}{\overline{Z}_1} = 0 \, , \qquad \acomm{\overline{X}_2}{\overline{Z}_2} = 0 \, ;
\end{equation*}
queste relazioni sono ovvie se si pensano ai loop di Figura \ref{fig:logical_X_Z}: nell'intersezione di $\overline{X}_1$ con $\overline{Z}_1$ e di $\overline{X}_2$ con $\overline{Z}_2$ vi è precisamente un solo operatore ($X$ per $\overline{X}_i$ e $Z$ per $\overline{Z}_i$) che agisce sul medesimo sottospazio comune perché l'intersezione tra le due linee avviene in un solo punto. Questo fatto fa in modo che grazie all'anticommutatore $\acomm{X}{Z} = 0$ gli anticommutatori logici precedenti siano anch'essi verificati.  

\noindent Dati quindi gli operatori logici in \eqref{logical_Z_1_2} e \eqref{logical_X_1_2} possiamo calcolare i 3 stati logici rimanenti ($\ket{\overline{01}}$, $\ket{\overline{10}}$ e $\ket{\overline{11}}$) a partire dallo stato \eqref{logical_00}. Nonostante ciò qui si evidenzia la natura topologica del codice: avremmo potuto proporre come operatori logici moltissimi altri loop oltre alle linee di Figura \ref{fig:logical_X_Z}, ma ciò non avrebbe fatto alcuna differenza perché si tratta sempre di loop omotopi a quelle linee!

\noindent Il motivo profondo dell'affermazione precedente è mostrato nell'esempio di Figura \ref{fig:topological_loops}. 

\begin{figure}[!ht]
    \centering
    \includegraphics[scale=0.33]{images/topological_loops}
    \caption{Una possibile scelta di operatore logico $\overline{Z}_1$ è dato dal loop spezzato a sinistra. Questo loop è in realtà equivalente al loop dritto della Figura \ref{subfig:logical_X_Z_1} per il loop chiuso a destra, il quale è omotopo a zero, ossia all'identità quando agisce sul codewords.}
    \label{fig:topological_loops}
\end{figure}

\noindent Nella parte sinistra della figura è presentato un esempio di scelta differente di $\overline{Z}_1$, il quale non è altro che un loop (linea) lungo il toro che non è più dritto. Si può infatti dimostrare con argomenti analoghi ai precedenti che è un operatore logico $\overline{Z}_1$ perché soddisfa i commutatori \eqref{comm_log_banali} e \eqref{comm_log_meno_banali}.  Sembrerebbe da questa scelta che possiamo avere un'infinità di loop analoghi, tuttavia ora dimostriamo che in realtà il loop a sinistra è equivalente al prodotto del loop dritto al centro della Figura \ref{fig:topological_loops} (ossia quello della Figura \ref{subfig:logical_X_Z_1}) per il loop chiuso a destra, dove quest'ultimo è dato dal prodotto di tutti gli \texttt{Z-gate} lungo tale loop. Per capire questo fatto consideriamo il loop chiuso a destra: con la stessa logica possiamo pensare a questo loop, che circonda due plaquette, come al prodotto dei due loop più piccoli che circondano ciascuno una singola plaquette. In tale situazione il link orizzontale intermedio comune ai due loop conta 2 operatori $Z$, ciascuno da uno dei due loop: grazie alla proprietà $Z^2 = \mathbb{I}$ allora effettivamente questo loop attorno alle due plaquette è equivalente al prodotto dei due loop singoli. 

\noindent Quale sarà il loop più piccolo possibile? Chiaramente questo non è altro che il prodotto di 4 \texttt{Z-gate} attorno ad una plaquette, ossia, dalla definizione \eqref{A_B}, proprio lo stabilizer $B_p$ di Figura \ref{subfig:Toric_lattice_3}; ma ricordiamo che ciascun $B_p$ sul codewords ha autovalore $+1$! Questo significa che ciascun loop non dritto del reticolo è equivalente ad un loop dritto (Figura \ref{subfig:logical_X_Z_1}) grazie al fatto che ogni loop chiuso abbia un contributo banale su $C$: 2 loop omotopi sono equivalenti quando agiscono sul codewords! È questo il motivo fondamentale per cui il codice è chiamato \textbf{topologico}: è possibile deformare i contorni di $\overline{Z}_i$ e $\overline{X}_i$ senza cambiare l'azione di questi operatori logici sugli stati del codewords!

\noindent Notiamo che con la stessa logica precedente un qualsiasi prodotto di \texttt{Z-gate} lungo un loop chiuso contraibile è equivalente all'identità quando agisce su $C$:
\begin{equation}\label{product_Z_closed_loop}
    \prod_{i \in \substack{\text{loop} \\ \text{chiuso}\\ \text{contraibile}}} Z_i \equiv \mathbb{I} \, \text{ agendo su } C \, .
\end{equation}
Perciò il contributo del prodotto precedente è banale su $C$ perché un qualsiasi loop chiuso può essere decomposto come prodotto di plaquette singole, ossia come prodotto di $B_p$, le quali hanno autovalore $+1$ sul codewords. Questo fatto può essere espresso in altre parole dicendo che qualsiasi loop chiuso contraibile sia omotopo a zero, ossia è equivalente all'azione dell'identità sul codewords. 

\noindent Qual è quindi la ragione per cui abbiamo esattamente 4 operatori logici non banali? Il motivo è che qualsiasi operatore non banale sul toro deve essere periodico: i due loop della Figura \ref{fig:torus} (ricordare che sono prodotti di operatori), a differenza di qualsiasi altro loop, non sono omotopi a zero e ciascuno dei due può essere costituito dal prodotto di \texttt{Z-gate} oppure \texttt{X-gate}. Abbiamo quindi in totale 4 operatori non banali (2 loop non-contraibili con prodotti di $Z$ più 2 loop con prodotti di $X$). 

\begin{figure}[!h]
    \centering
    \includegraphics[scale=0.33]{images/torus}
    \caption{Abbiamo in totale 4 operatori logici non banali: 2 loop rossi e 2 loop blu . Ogni tipologia di loop (rosso o blu) può essere originata da prodotti di \texttt{Z-gate} o \texttt{X-gate}. Qualsiasi altro loop non rappresentato in figura è banale, ossia è omotopo a zero e agisce come un'identità sul codewords. Notare che questi 4 operatori, se pensati in una rappresentazione planare con PBC, non sono altro che le 4 linee (loop) delle Figure \ref{subfig:logical_X_Z_1} e \ref{subfig:logical_X_Z_2}.}
    \label{fig:torus}
\end{figure}

\subsection{Correzione degli errori}
Discutiamo ora come avviene la correzione degli errori nel toric code. Ricordiamo ancora una volta che negli stabilizer codes gli operatori commutanti con gli stabilizers sono operatori logici, mentre coloro che non commutano, ma anticommutano, sono errori.

\begin{figure}[!ht]
	\centering	
	\subfloat[][Bit flip error.\label{subfig:bit_phase_toric_1} ]{\includegraphics[scale=.32,keepaspectratio]{images/bit_phase_toric_1}} \quad
	\subfloat[][Phase flip error.\label{subfig:bit_phase_toric_2} ]{\includegraphics[scale=.32,keepaspectratio]{images/bit_phase_toric_2}} 
	\caption{(\ref{subfig:bit_phase_toric_1}) Esempio di rilevazione di un bit flip error. Solamente le due plaquette (riquadri arancioni) adiacenti all'errore contano perché viene invertito il loro autovalore. (\ref{subfig:bit_phase_toric_2}) Esempio di rilevazione di un phase flip error. Solamente i due vertici (croci arancioni) adiacenti all'errore contano perché viene invertito il loro autovalore.}
    \label{fig:bit_phase_toric}
\end{figure}

\noindent Cominciamo col considerare un \textbf{bit flip error}. Immaginiamo, come in Figura \ref{subfig:bit_phase_toric_1}, di avere un qubit su un link qualsiasi con un errore dato da un \texttt{X-gate}.  In tale situazione solamente le due plaquette adiacenti al link contano: in generale $\comm{X}{A_v} = 0$, ma dato che $B_p = \prod_{j \in p} Z_j$ allora solamente per le due plaquette mostrate avremo $\comm{X}{B_p} \neq 0$. Dato che in particolare si ha $\acomm{X}{B_p} = 0$ per la presenza di uno \texttt{Z-gate} agente sullo stesso sottospazio di $X$ in ciascuna delle due plaquette, allora l'autovalore di questi due $B_p$ su $C$ è diventato $-1$. In questo modo possiamo rilevare un bit flip error misurando l'autovalore di queste due plaquette. Questo significa, in generale, che quando si ha una situazione in cui tutti gli operatori $B_p$ hanno autovalore $+1$ eccetto per due plaquette adiacenti allora sia ha la certezza che è presente un bit flip error nel qubit tra le due. 

\noindent Consideriamo ora un \textbf{phase flip error}. Come evidenziato nella Figura \ref{subfig:bit_phase_toric_2} la situazione è simile alla precedente, ma questa volta l'errore è rappresentato da uno \texttt{Z-gate} su un link: per ogni plaquette avremo $\comm{Z}{B_p} = 0$, ma essendo $A_v = \prod_{i \in v} X_i$ allora per i due vertici adiacenti mostrati si ha $\comm{Z}{A_v} \neq 0$. Come in precedenza, l'errore agisce sul medesimo sottospazio degli operatori $X$ nei due $A_v$, quindi a causa dell'anticommutatore $\acomm{Z}{A_v} = 0$ l'autovalore di questi due vertici sarà $-1$. Esattamente come il caso precedente, quando tutti gli operatori $A_v$ hanno autovalore $+1$ tranne due vertici adiacenti allora si ha la certezza che è presente un phase flip error nel qubit tra i due. 

\noindent Ricapitolando, abbiamo esplicitamente mostrato che gli operatori $A_v$ e $B_p$ agiscono come stabilizers: misurando gli autovalori degli \eqref{A_B} è possibile rilevare direttamente bit flip error e phase flip error. Gli altri errori, ossia la combinazione bit-phase flip, sono semplicemente dati da una combinazione dei casi precedenti. 

\noindent Una delle ragioni principali per cui questo codice di correzione degli errori è così popolare nella letteratura del QC è data dal fatto che sia possibile rilevare e correggere gli errori aggiungendo qubit extra (che chiamiamo anche qui \textbf{ancilla qubits}) nei vertici e nelle facce del reticolo. Tutte le tipologie di qubit presenti nel reticolo (compresi gli ancilla) sono mostrate in Figura \ref{subfig:ancilla_toric_1}. I qubit sui link (pallini rossi pieni) sono i qubit fisici, ossia coloro che codificano l'informazione. Viceversa, i qubit aggiunti sui vertici e al centro delle facce del reticolo (pallini arancio vuoti) sono gli ancilla qubit che hanno lo scopo di effettuare la misurazione e correggere eventuali errori. Chiamiamo \textbf{qubit of type Z} gli ancilla nelle facce perché effettuano la misurazione di $B_p$ sui 4 qubit dei link della plaquette, mentre definiamo \textbf{qubit of type X} gli ancilla presenti nei vertici, i quali similmente effettuano una misurazione di $A_v$ sui 4 qubit dei link entranti in quel vertice.

\noindent Più precisamente, consideriamo la singola plaquette di qubit di Figura \ref{subfig:ancilla_toric_2}. L'ancilla qubit effettua una misurazione dell'operatore $B_p = Z_1 Z_2 Z_3 Z_4$: questa misurazione può essere portata a termine per mezzo del seguente circuito
\begin{center}
    \mbox{
        \Qcircuit @C=2em @R=1em {
            \lstick{\text{Ancilla: }\ket{0}} & \targ & \targ & \targ & \targ & \meter & \rstick{\ket{0}, \ket{1}} \qw \\
            \lstick{1} & \ctrl{-1} & \qw & \qw & \qw & \qw & \qw \\
            \lstick{2} & \qw & \ctrl{-2} & \qw & \qw & \qw & \qw \\
            \lstick{3} & \qw & \qw & \ctrl{-3} & \qw & \qw & \qw \\
            \lstick{4} & \qw & \qw & \qw & \ctrl{-4} & \qw & \qw
        }
    }
\end{center}
L'autovalore del prodotto dei 4 \texttt{Z-gate} di $B_p$ sarà $+1$ o $-1$ a seconda del numero di flip che vengono operati dai \texttt{CNOT-gate}: un numero dispari di stati $\ket{1}$ nei 4 qubit produrrà $B_p = -1$ perché l'ancilla sarà in $\ket{1}$, viceversa un numero pari di $\ket{1}$ vorrà dire $B_p = +1$ perché la misurazione sull'ancilla ha output $\ket{0}$. Misurando quindi lo stato dell'ancilla siamo in grado di stabilire l'autovalore di $B_p$. 

\begin{figure}[!ht]
	\centering	
	\subfloat[][Reticolo di qubit.\label{subfig:ancilla_toric_1} ]{\includegraphics[scale=.45,keepaspectratio]{images/ancilla_toric_1}} \quad
	\subfloat[][Misura di $B_p$.\label{subfig:ancilla_toric_2} ]{\includegraphics[scale=.45,keepaspectratio]{images/ancilla_toric_2}} \quad
	\subfloat[][Misura di $A_v$.\label{subfig:ancilla_toric_3} ]{\includegraphics[scale=.45,keepaspectratio]{images/ancilla_toric_3}}
	\caption{Il reticolo è cosparso di qubit fisici (pallini rossi pieni sui link) e di ancilla qubits (pallini arancio vuoti nelle facce e sui vertici). Gli ancilla delle plaquette permettono di effettuare una misurazione dell'autovalore di $B_p$, mentre gli ancilla dei vertici effettuano una misura dell'autovalore di $A_v$.}
    \label{fig:ancilla_toric}
\end{figure}

\noindent In maniera del tutto analoga consideriamo ora il singolo vertice di Figura \ref{subfig:ancilla_toric_3}. È possibile dimostrare che l'ancilla sul vertice effettua una misurazione dell'operatore $A_v = X_1 X_2 X_3 X_4 $ utilizzando un circuito analogo al precedente (vengono solamente inseriti alcuni \texttt{H-gate}): anche in questo caso, una misurazione sullo stato finale dell'ancilla permette di stabilire se l'autovalore è $A_v = +1$ oppure $A_v = -1$. 

\noindent Analizziamo ora gli errori mostrati in Figura \ref{subfig:strange_errors_toric_1}. Come evidenziato anche nella Figura \ref{subfig:bit_phase_toric_1}, la misura nel reticolo di due autovalori $B_p = -1$ corrisponde ad un bit flip error sul qubit tra le due plaquette adiacenti (situazione azzurra in alto nel reticolo). Nonostante la situazione precedente possono avvenire altre tipologie di errori: supponiamo di misurare due autovalori $B_p = -1$ in corrispondenza di due plaquette non adiacenti (si vedano i $-1$ in rosso nelle due facce). Da che cosa sono prodotti errori come i precedenti? Questi non sono altro che il risultato di una serie di bit flip lungo una linea che connette le due plaquette (si veda le "X" in arancio): dato che tutti i link tra le plaquette intermedie hanno cambiato segno due volte a seguito del bit flip, allora per tali facce la misura produce $B_p = +1$, mentre per le plaquette iniziali e finali l'autovalore risulta invertito! Siamo quindi in grado di dare una corretta interpretazione anche di questa tipologia di errore. 

\noindent Perché la natura topologica del codice è così importante? Come possiamo essere certi che l'errore appena spiegato sia dovuto alla linea arancio di bit flip e non, ad esempio, alla linea verde? In principio ogni possibile cammino di \texttt{X-gate} che connette le due plaquette invertite potrebbe dare lo stesso identico errore. Se non sappiamo distinguere quale percorso di \texttt{X-gate} ha causato l'errore, come possiamo correggerlo? Il punto fondamentale è che non importa quale sia il giusto percorso di errori: possiamo correggere gli errori applicando un cammino arbitrario di \texttt{X-gate} lungo un qualsiasi cammino aperto che connette le due plaquette invertite: si tratta quindi di scegliere un cammino di \texttt{X-gate} che connette le due plaquette invertite! Il cammino originale che causa l'errore (non lo conosciamo) più il cammino di \texttt{X-gate} scelto è un operatore banale perché uno compensa l'altro: il percorso finale risultante non è altro che un cammino chiuso nel reticolo duale, quindi la combinazione di errori risulta in un prodotto di \texttt{X-gate} lungo un loop chiuso (Figura \ref{subfig:strange_errors_toric_2})! Dato che, come illustra la Figura \ref{subfig:strange_errors_toric_2}, un qualsiasi prodotto di \texttt{X-gate} su un loop chiuso contraibile può essere scritto come prodotto di $A_p$, allora in analogia alla \eqref{product_Z_closed_loop} avremo
\begin{equation}\label{product_X_closed_loop}
    \prod_{i \in \substack{\text{loop} \\ \text{chiuso}\\ \text{contraibile}}} X_i \equiv \mathbb{I} \, \text{ agendo su } C \, .
\end{equation}

\begin{figure}[!ht]
	\centering	
	\subfloat[][\label{subfig:strange_errors_toric_1} ]{\includegraphics[scale=.38,keepaspectratio]{images/strange_errors_toric_1}} \qquad
	\subfloat[][\label{subfig:strange_errors_toric_2} ]{\includegraphics[scale=.38,keepaspectratio]{images/strange_errors_toric_2}}
	\caption{(\ref{subfig:strange_errors_toric_1}) La misura di due autovalori $B_p = -1$ in corrispondenza di due plaquette non adiacenti (facce in rosso) può essere causata da una concatenazione di bit flip errors lungo un cammino che connette le due plaquette. Per correggere tale errore basta applicare un percorso di \texttt{X-gate} che connette le due plaquette. (\ref{subfig:strange_errors_toric_2}) Un prodotto di \texttt{X-gate} su un loop chiuso nel reticolo duale è equivalente ad un prodotto di operatori $A_v$ (vertici azzurri), i quali hanno contributo banale su $C$.}
    \label{fig:strange_errors_toric}
\end{figure}

\noindent Ricapitolando: un cammino chiuso di \texttt{X-gate} nel reticolo duale può sempre essere visto come prodotto di tutti i vertici (operatori $A_v$) contenuti nel loop; ogni $A_p$ inserisce un \texttt{X-gate} sui link esterni mentre, grazie a $X^2 = \mathbb{I}$, nulla accade nei link interni comuni ai vertici. Dato che su $C$ tutti gli $A_p$ hanno autovalore $+1$, allora i loop chiusi di \texttt{X-gate} nel reticolo duale sono omotopi a zero, ossia sono l'operatore banale (identità). 

\subsection{Interpretazione in meccanica statistica}
Esiste una curiosa interpretazione del toric code alla luce della meccanica statistica considerando un reale modello di spin quantistici. Supponiamo un reticolo in cui, su ogni link, è possibile avere uno spin quantistico (up o down). In un tale sistema l'hamiltoniana è data da 
\begin{equation}\label{hamilt_toric}
    H = - J \sum_v A_v - J \sum_p B_p \, , \qquad J > 0 \, .
\end{equation}
Come mostra la Figura \ref{subfig:stat_mec_toric_1}, l'interazione degli spin avviene in due modi: interagiscono i 4 spin in un vertice (prima somma in $H$) oppure i 4 spin di una plaquette (seconda somma in $H$). Qual è lo stato (o gli stati) di minima energia? I ground state corrispondono a tutte quelle situazioni in cui gli autovalori sono $A_v = B_p = +1$ per ogni vertice e plaquette del reticolo. In un'interpretazione di questo tipo i 4 qubit logici $\ket{\overline{00}}$, $\ket{\overline{01}}$, $\ket{\overline{10}}$ e $\ket{\overline{11}}$ (con autovalori $A_v = B_p = +1$) del QC sono mappati nei ground state dell'hamiltoniana \eqref{hamilt_toric}; viceversa gli errori nel reticolo corrispondono a delle eccitazioni, ossia a delle configurazioni in cui almeno un autovalore di $A_v$ e/o $B_p$ è uguale a $-1$. 

\begin{figure}[!ht]
	\centering	
	\subfloat[][\label{subfig:stat_mec_toric_1} ]{\includegraphics[scale=.35,keepaspectratio]{images/stat_mec_toric_1}} \quad
	\subfloat[][\label{subfig:stat_mec_toric_2} ]{\includegraphics[scale=.35,keepaspectratio]{images/stat_mec_toric_2}}
	\caption{(\ref{subfig:stat_mec_toric_1}) L'analogo sistema in meccanica statistica del toric code è costituito da un reticolo di spin in cui questi ultimi possono interagire in un vertice oppure in una plaquette. (\ref{subfig:stat_mec_toric_2}) Se si espande la produttoria in \eqref{logical_00} si hanno diverse interazioni di spin a causa delle combinazioni degli operatori $A_v$.}
    \label{fig:stat_mec_toric}
\end{figure}

\noindent Dal punto di vista dei singoli spin lo stato \eqref{logical_00} è estremamente complicato perché è uno stato molto entangled. Se si espande la produttoria in $\ket{\overline{00}}$ si ottengono dei termini del tipo $\mathbb{I} + A + AA + AAA + \ldots$, quindi avvengono diverse interazioni: come illustra la Figura \ref{subfig:stat_mec_toric_2}, il singolo operatore $A$ agisce sul vertice e inverte i 4 spin (in blu in alto a sinistra dove gli spin $\uparrow$ sono diventati $\downarrow$); gli operatori della forma $AAA$, invece, creano un loop di spin capovolti nel reticolo originale (si vedano i tre vertici in basso in verde). In totale la produttoria in \eqref{logical_00} produce una sovrapposizione lineare di tutti i possibili loop costituiti da spin invertiti: c'è entanglement tra tutti gli spin del reticolo, persino tra quelli più distanti. 

\noindent Che cosa sono gli errori in QC? Non sono altro che operatori che agiscono sui link del reticolo: ad esempio se si pensa ad un singolo bit flip error (\texttt{X-gate} su un link) allora esso cambia autovalore agli operatori $B_p$ delle placchette adiacenti (si veda la Figura \ref{subfig:bit_phase_toric_1}); similmente quando si ha un phase flip error (Figura \ref{subfig:bit_phase_toric_2}), allora lo \texttt{Z-gate} sul link corrotto cambierà autovalore ai due operatori $A_v$ adiacenti. 

\noindent Data l'hamiltoniana \eqref{hamilt_toric}, agli errori nel QC corrispondono delle eccitazioni nel sistema di spin: avremo una variazione di energia $\Delta E_z = 2 J$ per il singolo bit flip e una variazione di $\Delta E_x = 2 J$ per il singolo phase flip. In teoria della materia condensata questi stati eccitati corrispondono alle cosiddette \textbf{quasiparticles}. In questo contesto ne esistono di due tipologie: le quasiparticle \textbf{elettriche}, a seguito dell'errore causato da $Z$ e le quasiparticle \textbf{magnetiche}, generate dall'errore di $X$; entrambe hanno la medesima energia e  vedremo nelle prossime sezioni che presentano alcune proprietà strane e insolite. Dato che per ogni errore (bit flip o phase flip) si hanno due autovalori $-1$ ($A_v = -1$ per $Z$ e $B_p = -1$ per $X$) allora è come se potessimo associare due paia di quasiparticle: 2 elettriche e 2 magnetiche. 

\noindent Una delle proprietà più insolite (vedremo) è la seguente: se si considera una quasiparticle elettrica e la si muove attorno ad una quasiparticle magnetica ritornando poi al punto di origine allora, a seguito della QM, si origina una fase: questo fenomeno può essere interpretato come un doppio scambio di particelle; per tale ragione questa particolare tipologia di particelle costituiscono un perfetto "toy model" per i cosiddetti \textbf{anyons} (diffusi in letteratura nella teoria della materia condensata). Queste particolari quasiparticle, presenti unicamente in sistemi bidimensionali, sono simili a particelle che presentano una statistica frazionaria, ossia non sono né bosoni né fermioni!
    %%%%%%%%%%%%%%
% LECTURE 15 %
%%%%%%%%%%%%%%
\vspace{1cm}
\newline
\lecture{15}{26/11/2021}
\subsection{Topological quantum computing}
Nella sezione precedente abbiamo associato le eccitazioni nel sistema di spin (errori nel toric code) a due diverse tipologie di quasiparticle:  elettriche e  magnetiche. Queste quasiparticle si originano in coppia quando si verifica un bit flip o un phase flip in un qualunque punto del reticolo. Notiamo che queste quasiparticle possono essere separate all'interno del reticolo senza alcun costo in termini energetici, perché possiamo costruire un percorso generico che le colleghi attraverso l'applicazione ripetuta di X-gate (quasiparticle magnetiche) o Z-gate (quasiparticle elettriche). In Figura \ref{fig:x-equiv-lattice} è evidenziata come la posizione delle quasiparticle (magnetiche ed elettriche) all'interno dei due reticoli abbia la stessa differenza in energia $\Delta E=2J$ indipendentemente dalla loro vicinanza.

\begin{figure}[!ht]
	\centering
	\subfloat[][\label{subfig:x-nopath-lattice}]{\includegraphics[scale=.5,keepaspectratio]{images/x-nopath-lattice.jpg}} \qquad \qquad
	\subfloat[][\label{subfig:x-path-lattice}]{\includegraphics[scale=.5,keepaspectratio]{images/x-path-lattice.jpg}}
	\caption{(\ref{subfig:x-nopath-lattice}) Coppia di quasiparticle elettriche (asterischi arancio per $A_v = -1$) e magnetiche (asterischi viola per $B_p = -1$) create da un singolo errore. (\ref{subfig:x-path-lattice}) Indipendentemente dal cammino di errori che origina queste particelle, il costo di energia $E - E_0 = 2 J$ è sempre lo stesso.}
    \label{fig:x-equiv-lattice}
\end{figure}

\noindent In aggiunta queste quasiparticle presentano delle ulteriori proprietà interessanti. Supponiamo di considerare due coppie, una di quasiparticle elettriche e l'altra di quasiparticle magnetiche, come mostrato in Figura \ref{fig:z-path-around-x}. Immaginiamo di considerare solamente la coppia elettrica: come visto nella scorsa sottosezione, se muoviamo una delle due particelle applicando una stringa di \texttt{Z-gate} lungo un loop chiuso, allora l'effetto è banale perché il percorso può essere fattorizzato in un prodotto di operatori $B_p$, i quali hanno tutti autovalori $+1$ sul codewords. 

\noindent Il comportamento risulta tuttavia differente se consideriamo in aggiunta una coppia di particelle magnetiche. Come in figura, realizziamo un circuito chiuso (quindi tornerà nel suo punto iniziale) a partire da una particella elettrica che passi tra le due quasiparticle magnetiche. Quando il percorso va ad avvolgere una quasiparticle magnetica, siccome $X$ e $Z$ anticommutano tra loro, lo stato ottiene un termine di fase $ \ket{\psi} \to - \ket{\psi} =  e^{i\pi} \ket{\psi}$ dovuto al fatto che questa volta il prodotto di plaquette abbia autovalore $-1$. Questo risultato assomiglia un po' all'effetto Aharonov-Bohm\footnote{Dopo aver percorso un circuito chiuso torniamo allo stato iniziale con una fase extra.}: per questo motivo si dice che queste quasiparticle abbiano una statistica reciproca non banale.

\begin{figure}[!ht]
    \centering
    \includegraphics[scale=0.45]{images/z-path-around-x.jpg}
    \caption{Il loop chiuso di \texttt{Z-gate} (loop rosso) della quasiparticle elettrica attorno a quella magnetica produce un termine di fase a seguito di $\acomm{X}{Z} = 0$.}
    \label{fig:z-path-around-x}
\end{figure}

\noindent Riassumiamo ciò che abbiamo imparato sull'interpretazione in meccanica statistica del toric code. Questo codice di correzione degli errori presenta un approccio topologico: mentre gli errori agenti sui singoli qubit possono essere corretti similmente ai codici visti in precedenza, la situazione è differente quando si considerano errori agenti su più di un qubit simultaneamente. Immaginiamo che in un codice di protezione dagli errori come quello di Shor o di Steane avvengano 3 errori simultanei che causino $\ket{000} \to \ket{111}$. In generale, dato che un qubit logico si è trasformato in un altro qubit logico, una tale situazione è molto difficile da distinguere e correggere per quel tipo di codici. Al contrario, nel toric code, errori tali che trasformino qubit logici in altri qubit logici sono molto difficili da avere perché sono necessari degli interi loop di errori che attraversano tutto il reticolo (si pensi a $\overline{Z}_i$ e $\overline{X}_i$ della Figura \ref{fig:torus}). Se il reticolo è grande abbastanza questi errori sono molto improbabili: un reticolo sufficientemente grande assicura una protezione da questa tipologia di errori!

\noindent Dal punto di vista della sua interpretazione in meccanica statistica, questa robustezza del toric code contro gli errori si traduce in una robustezza della degenerazione dei vuoti, ossia degli stati di ground, contro le perturbazioni. Possiamo formalizzare questa proprietà nel seguente modo. Per tutti gli operatori \textit{locali}\footnote{Operatori costruiti a partire da una collezione di $X$, $Z$ e $Y$ che agiscono solo localmente in una regione finita del piano del reticolo; non sono gli operatori $\overline{Z}_i$ e $\overline{X}_i$, ossia tali che attraversino l'intero reticolo.} $\hat{O}$, chiamando $\ket{\overline{x}\overline{y}}$ e $\ket{\overline{x}'\overline{y}'}$ due differenti stati di ground, vale
\begin{equation*}
    \mel{\overline{x}\overline{y}}{\hat{O}}{\overline{x}'\overline{y}'} = \delta_{\overline{x} \overline{x}'} \delta_{\overline{y} \overline{y}'} \, , \quad \text{per} \quad \overline{x}, \overline{y}, \overline{x}', \overline{y}' = 0,1 \, .
\end{equation*}
Questo significa che operatori locali non possono tramutare stati di ground in altri stati di ground. Il motivo della precedente proprietà è il seguente: se $\hat{O}$ è locale allora è un loop di $X$ o $Z$, che agisce banalmente come identità, oppure è un singolo errore $X$ o $Z$, il quale sappiamo che genera uno stato non più facente parte del codewords. L'unico modo per ottenere qualcosa di non banale è quello di utilizzare un operatore non locale: qualsiasi perturbazione può agire solo localmente!


\noindent Il toric code è un esempio di \textbf{fase della materia con ordine topologico}. Queste sono caratterizzate da un entanglement a lungo raggio (LRE = long range entanglement, soprattutto per gli stati fondamentali) e presentano alcune proprietà comuni:
\begin{itemize}
    \item Degenerazione robusta dello stato fondamentale su varietà compatte (ad esempio un toro). Questo fatto è associato alla topologia della varietà: è molto difficile eliminare la degenerazione del ground state utilizzando perturbazioni locali;
    \item Ci sono eccitazioni  (quasiparticles elettriche e magnetiche) che presentano proprietà non locali, come ad esempio statistiche non banali. Si pensi all'effetto Aharonov-Bohm tipico di fasi topologiche della materia;
    \item La low-energy theory è in qualche modo topologica, ossia dipende unicamente dalla topologia della varietà che si sta considerando. Si tratta di una descrizione in termini di una cosiddetta \textbf{topological quantum field theory} (TQFT)\footnote{Introdotta da Edward Witten nel 1988.}. Per esempio ampiezze di processi in QM dipendono solo dalla topologia del cammino della particella e non dalla sua forma o velocità (Figura \ref{fig:qtft-equiv}).
    \begin{figure}[H]
        \centering
        \includegraphics[scale=0.4]{images/qtft-equiv.jpg}
        \caption{Equivalenza delle ampiezze dei processi in una TQFT, i quali non dipendono dalla forma o dalla velocità dei cammini.}
        \label{fig:qtft-equiv}
    \end{figure}
\end{itemize}

\noindent Queste caratteristiche erano dal punto di vista della teoria della materia condensata. Ritornando al QC, il toric code è solitamente discusso come uno dei primi esempi di \textbf{topological quantum computing}, poiché alcune generalizzazioni di ciò che abbiamo visto nel corso di questa sezione possono essere utilizzate per il QC. Senza alcuna pretesa di essere formali, cerchiamo di comprendere quale sia l'idea che ne sta alla base. 

\noindent Dal punto di vista dell'interpretazione in meccanica statistica (sistema fisico di spin su reticolo), consideriamo l'ampiezza in QM generata dal processo in Figura \ref{subfig:circling-quasi-particles1}: se la quasiparticle elettrica circonda quella magnetica, allora, come abbiamo visto, otteniamo un termine di fase $(-1)=e^{i\pi}$ tale per cui la funzione d'onda passa da uno stato $\ket \psi$ (stato iniziale delle 4 quasiparticle) a uno stato $-\ket \psi$. Questa situazione, come mostra la Figura \ref{subfig:circling-quasi-particles2}, è assimilabile ad uno scambio di particelle: a metà del processo (linea tratteggiata in rosso in \ref{subfig:circling-quasi-particles1}) si ha una situazione in cui la particella elettrica è stata mossa nel passato di quella magnetica, quindi è come se si avesse uno scambio delle due. 

\begin{figure}[!ht]
	\centering	
	\subfloat[][\label{subfig:circling-quasi-particles1}]{\includegraphics[scale=.4,keepaspectratio]{images/circling-quasi-particles1}} \qquad \qquad
	\subfloat[][\label{subfig:circling-quasi-particles2}]{\includegraphics[scale=.4,keepaspectratio]{images/circling-quasi-particles2}}
	\caption{(\ref{subfig:circling-quasi-particles1}) Ampiezza del processo in cui una quasiparticle elettrica gira attorno ad una magnetica. (\ref{subfig:circling-quasi-particles2}) L'ampiezza qui a sinistra è analoga ad una situazione di scambio di queste particelle.}
    \label{fig:circling-quasi-particles}
\end{figure}

\noindent In una situazione di scambio di particelle dobbiamo stare attenti in QM quando si tratta di particelle identiche (si pensi alla Figura \ref{subfig:circling-quasi-particles2} con 4 particelle identiche come stato iniziale e finale): dobbiamo prestare attenzione al principio di esclusione di Pauli! 

\noindent Supponiamo di avere un sistema (in teoria della materia condensata) con un insieme di quattro\footnote{Ne consideriamo quattro perché in questi modelli le quasiparticle vengono prodotte in coppia.} quasiparticle tali per cui lo stato iniziale sia descritto dalla sovrapposizione $\ket \psi = \alpha \ket{\psi_1} + \beta \ket{\psi_2}$, la quale è degenere: ci sono due stati quantistici $\ket{\psi_1}$ e $\ket{\psi_2}$ che corrispondono alle quattro particelle identiche con i medesimi numeri quantici. Quando scambiamo due di loro come in Figura \ref{subfig:circling-quasi-particles2} avremo in generale
\begin{equation*}
    \ket \psi = \alpha \ket{\psi_1} + \beta \ket{\psi_2} \longrightarrow \ket \psi' = \alpha' \ket{\psi_1} + \beta' \ket{\psi_2} \, .
\end{equation*}
La relazione che lega questo scambio può essere vista come una rotazione non banale operata dalla matrice unitaria $\hat{U}$:
\begin{equation*}
    \begin{pmatrix}
        \alpha \\
        \beta
    \end{pmatrix}
    \longrightarrow
    \begin{pmatrix}
        \alpha' \\
        \beta'
    \end{pmatrix}
    =\hat U
    \begin{pmatrix}
        \alpha \\
        \beta
    \end{pmatrix} \, ;
\end{equation*}
in particolare ritorniamo alla condizione iniziale, ma con lo stato che può essere trasformato in un altro stato dello stesso spazio di Hilbert degenere. L'unico vincolo quantomeccanico è il fatto che $\hat U$ sia unitario. 

\noindent Questi stati presentano quella che si definisce \textbf{statistica non-abeliana} (il toric code ha solo uno stato per cui la statistica era abeliana, ossia data da una semplice fase del gruppo $U(1)$). Quando le particelle sono identiche e presentano statistica frazionaria, allora, non essendo né fermioni né bosoni, sono note come \textbf{anyons} e possono esistere solamente in una situazione bidimensionale\footnote{Non nello spazio tridimensionale, perché in 3D un doppio scambio deve essere banale per via della topologia, cosicché un singolo scambio possa dare solo $\pm1$ (bosoni o fermioni).}. 

\noindent La cosa curiosa è che, in teoria, gli anyons non-abeliani possono essere usati per il QC! Costruendo in laboratorio un array di quasiparticle e intrecciandole tra loro possiamo dare luogo a trasformazioni unitarie non-abeliane proprio come l'esempio di Figura \ref{fig:tqc}. Ogni volta che si scambia una quasiparticle si ha una trasformazione unitaria sul sistema: l'intera sequenza di scambi può essere pensata come un singolo operatore (gate) unitario $U$.

\begin{figure}[!t]
    \centering
    \includegraphics[scale=0.4]{images/tqc.jpeg}
    \caption{Implementazione in QC di un circuito utilizzando un array di quasiparticle. Si suppone che i fili del circuito possano essere sostituiti dalle particelle, le quali possono essere spostate tra loro. Il rumore può intervenire unicamente su un singolo intreccio: per cambiare l'intero $U$ dovrebbe annullare ogni singolo intreccio!}
    \label{fig:tqc}
\end{figure}

\noindent Se fossimo in grado di realizzare abbastanza trasformazioni unitarie con scambi multipli, potremmo eseguire del QC grazie ad essi. Perché sarebbe molto interessante poter costruire computer quantistici basati su questo funzionamento? Questi dispositivi sarebbero molto robusti e fault-tolerant perché il rumore o altri disturbi locali andrebbero ad agire solo localmente sulla linea delle quasiparticle. Affinché il disturbo causato dall'ambiente possa essere in grado di cambiare l'intero $U$, esso dovrebbe poter annullare tutte le singole trasformazioni.

\noindent I tipi più comuni di anyons (non-abeliani) sono:
\begin{itemize}
    \item \textbf{Ising Anyons}: non sono universali nel senso del QC, ma appaiono teoreticamente in molti sistemi della materia condensata:
        \begin{itemize}
            \item Effetto Hall quantistico frazionario ($\nu=5/2$; SU(2)$_2$ nel contesto della TQFT);
            \item Majorana zero-mode in superconduttori topologici bidimensionali (generalizzano il più semplice modello monodimensionale chiamato catena di Kitaev).
        \end{itemize}
    \item \textbf{Fibonacci Anyons}: sono universali, sono stati sviluppati nella teoria dei modelli che generalizzano il toric code e si suppone che appaiano nell'effetto Hall quantistico frazionario ($\nu=12/5$; SU(2)$_3$ nel contesto della TQFT).
\end{itemize}
Due figure di spicco che proposero e diedero inizio al campo del topological quantum computing sono:
\begin{itemize}
    \item Alexei Kitaev, il quale propose nel 1997 un topological quantum computing basato sugli anyons;
    \item Michael Freedman, vincitore nel 1986 della medaglia Fields per aver risolto la congettura di Poincaré in dimensione 4. Era interessato a calcolare degli invarianti della teoria dei nodi noti come polinomi di Jones,  un problema di difficoltà esponenziale dal punto di vista del CC, quando scoprì che in QC in realtà può essere risolto in un tempo polinomiale. Attualmente dirige il gruppo di Microsoft Station Q a Santa Barbara per lo sviluppo del topological quantum computing. Nel 2018 il gruppo annunciò di aver realizzato sperimentalmente i Majorana zero-mode, ma nel 2021 l'articolo fu ritirato.
\end{itemize}






\chapter{Realizzazione fisica dei qubit}

\section{Introduzione}
Nella parte introduttiva del primo capitolo abbiamo visto che esistono moltissimi modi per poter realizzare fisicamente un qubit, dato che, dal punto di vista della QM, si tratta di un qualsiasi sistema quantistico a due livelli. Lasciando stare il discorso sulla realizzazione dei qubit nell'ambito della topological quantum computing, i più semplici da realizzare sono basati su:
\begin{itemize}
    \item Sistemi costituiti da particelle con \textbf{Spin} $1/2$; le particelle possono essere semplicemente controllate attraverso un campo magnetico $\vec B$, in quanto l'hamiltoniana che descrive questo tipo di interazione con lo spin è
    \begin{equation*}
        \hat H = - \mu \hat{\vec S} \cdot \vec B \, ,
    \end{equation*}
    dove $\mu$ è il momento magnetico.
    
    \item Sistemi che sfruttano la \textbf{polarizzazione dei fotoni}; sono in generale più difficili da gestire rispetto ai precedenti, tuttavia possono essere opportunamente controllati attraverso filtri polarizzatori (ad esempio un prisma per controllare i cambi di fase), beam splitters, ecc.
\end{itemize}

\noindent Altre realizzazioni fisiche, alcune delle quali approfondiremo nelle prossime sezioni, riguardano i qubit superconduttivi\footnote{Utilizzati da IBM e Google.}, i qubit basati sulla risonanza magnetica nucleare\footnote{Nota meglio come NMRQC: Nuclear Magnetic Resonance Quantum Computing.}, i qubit a trappola ionica\footnote{Utilizzata da IonQ.}, ecc. Spendiamo alcune parole sui qubit realizzati mediante sistemi basati sulla trappola ionica e sulla superconduttività. 

\noindent Nei primi dispositivi si vuole andare a creare un campo elettromagnetico che confini un insieme di ioni in una catena lineare lungo una direzione privilegiata (per convenzione è l'asse $z$). Si veda la Figura \ref{fig:ion-trap1} per una raffigurazione schematica. 

\begin{figure}[!t]
    \centering
    \includegraphics[scale=0.5]{images/ion-trap1}
    \caption{Rappresentazione schematica del sistema della trappola ionica.}
    \label{fig:ion-trap1}
\end{figure}

\noindent L'idea è quella di raffreddare questo sistema fino allo stato fondamentale in maniera tale che l'unico grado di libertà rimanente a questi ioni riguardi piccole vibrazioni lungo l'asse z. In questo modo si può incorporare un qubit in ognuno di essi combinando due differenti tipi di sistemi a due livelli:
\begin{itemize}
    \item Ogni ione presenta una struttura iperfine nei livelli energetici, si avrà quindi uno stato fondamentale $\ket 0$ e uno stato eccitato $\ket 1$, il quale è metastabile. Essi sono separati da un'energia pari a $\hbar\omega_1$.
    
    \item Ogni ione ha però al tempo stesso un modo vibrazionale, il quale è assimilabile a un oscillatore armonico. In termini di fisica della materia condensata, ogni \textbf{fonone} ha quindi due stati energetici separati da un'energia pari a $\hbar\omega_2$.
\end{itemize}

\noindent Perciò è possibile costruire due qubit identificando lo stato con la notazione $\ket{nm}$, dove $n$ indica il primo sistema a due livelli (livelli iperfini dello ione) ed $m$ il secondo sistema a due livelli (modi del fonone), proprio come mostrato in Figura \ref{fig:ion-trap2}. Utilizzando infine degli impulsi generati da laser, si può andare a interagire e a manipolare questo sistema (si pensi alla Figura \ref{fig:ion-trap1}).

\begin{figure}[!ht]
    \centering
    \includegraphics[scale=0.45]{images/ion-trap2.jpg}
    \caption{Sistema doppio a due livelli che coinvolge gli stati energetici degli ioni e i modi vibrazionali dei fononi. Notare che la differenza energetica tra gli stati fondamentali e gli stati eccitati è maggiore rispetto alla differenza energetica dovuta ai modi vibrazionali.}
    \label{fig:ion-trap2}
\end{figure}

\noindent Per quanto riguarda invece i sistemi superconduttivi l'idea che sta alla base è quella di realizzare un semplice circuito LC.  Risolvendo le equazioni di Kirchhoff è possibile mostrare che il comportamento di tale circuito sia quello di un oscillatore armonico, il quale può essere facilmente quantizzato attraverso la quantizzazione canonica. Si parla di superconduttività poiché si cerca di lavorare con elementi che non presentano alcuna resistenza a bassa temperatura. 

\noindent Il problema, che ribadiremo più volte, è che questo circuito LC per come è fatto non è molto adatto a descrivere un sistema a due livelli, dunque bisogna fare in modo che il comportamento dell'oscillatore risulti anarmonico: solitamente lo si fa introducendo la cosiddetta \textbf{giunzione Josephson}, per mezzo della quale si passa da un sistema a livelli equispaziati a un sistema con livelli non più equidistanti e che presentano dell'anarmonicità. Per interagire con questi sistemi è poi quindi necessario introdurre un generatore di impulsi a microonde mentre per ottenere informazioni sui qubit si utilizzano delle cavità elettromagnetiche. La differenza sostanziale tra i due dispositivi sta nel campo elettromagnetico, in quanto nel primo è classico mentre nel secondo è quantizzato. Si veda la Figura \ref{fig:lc-circuit-cavity} per una rappresentazione schematica.

\begin{figure}[!ht]
    \centering
    \begin{circuitikz}
        \draw
        (0,0)   to[C=$C$] ++ (0, 2) -- ++ ( 2,0) 
                to[L=$L$] ++ (0,-2) -- ++ (-2,0)
        %
        (0.5,2) |- ++ (-1.5,0.5) node[left, draw] {V(t)}
        (1.5,2) |- ++ ( 1.5,0.5) node[right,draw, align=center] {E.M.\\ Cavity};
    \end{circuitikz}
    \caption{Circuito LC con generatore di microonde e cavità elettromagnetica.}
    \label{fig:lc-circuit-cavity}
\end{figure}


\section{Oscillatore armonico quantistico}
Prima di poter descrivere accuratamente un modello fisico completo per un computer quantistico realizzabile, richiamiamo alcune nozioni su un sistema fisico molto elementare: l'oscillatore armonico quantistico\footnote{Non ci dilunghiamo molto su questo argomento perché è stato già trattato in maniera sufficientemente approfondita nel corso di MQ.}.
\noindent Supponiamo di avere un sistema che oscilla con una frequenza pari a $\omega$. Il suo moto sarà descritto dall'equazione del moto 
\begin{equation}\label{EOM_HO}
    \dv[2]{x}{t} + \omega^2 x = 0 \, ,
\end{equation}
le cui soluzioni sono del tipo $x(t) = e^{\pm i \omega t}$. L'hamiltoniana associata a questo sistema risulta quindi in
\begin{equation}\label{eq:ham-oa}
    H = \frac{p^2}{2m}+ \frac 12 m \omega^2 x^2 \, .
\end{equation}
Per passare ad una descrizione quantomeccanica si procede  con la quantizzazione canonica identificando le osservabili posizione e momento lineare con i rispettivi operatori hermitiani $\hat x$ e $\hat p$; inoltre introduciamo la regola di commutazione $\comm{\hat x}{\hat p}=i\hbar$. In questo modo l'hamiltoniana \eqref{eq:ham-oa} può essere scritta immediatamente come
\begin{equation*}
    \hat H = \frac{\hat p ^2}{2m} + \frac 12 m\omega^2 \hat x^2 \, ,
\end{equation*}
oppure
\begin{equation}\label{eq:ham-oa2}
    \hat H = \hbar \omega\left(\hat a^\dagger \hat a+\frac 12\right)=\hbar \omega\left(\hat n +\frac 12\right) \, ,
\end{equation}
dove in quest'ultima equazione abbiamo introdotto tre nuovi operatori
\begin{align*}
    &\text{Operatore di creazione:} &\hat a^\dagger &= \frac{1}{\sqrt{2m\hbar \omega}}(m\omega \hat x - i\hat p) \, , \\
    &\text{Operatore di distruzione:} &\hat a &= \frac{1}{\sqrt{2m\hbar \omega}}(m\omega \hat x + i\hat p) \, , \\
    &\text{Operatore numero:} &\hat n &= \hat a^\dagger \hat a \, .
\end{align*}
Dalla relazione di commutazione precedente avremo $\comm{a}{a^\dagger}=1$. Con queste definizioni, possiamo anche riscrivere $\hat x$ e $\hat p$ in funzione dei primi due nuovi operatori
\begin{align}
    \hat x &= \sqrt{\frac{\hbar}{2m\omega}}\left(\hat a + \hat a^\dagger\right) \, , \label{x_a_adag} \\
    \hat p &= -i\sqrt{\frac{m\omega\hbar}{2}}\left(\hat a - \hat a^\dagger\right) \, .
\end{align}
Con queste nuove definizioni avremo che gli autostati dell'hamiltoniana \eqref{eq:ham-oa2} sono gli stessi dell'operatore numero $\hat n$, quindi possiamo considerare una base comune di autostati $\ket n$ tale per cui $\hat n \ket n = n \ket n$. Esplicitamente avremo
\begin{equation*}
    \hat H \ket n = \hbar \omega \left(n + \frac 12 \right)\ket n \, , \quad \text{dove} \quad n \in \mathbb{N} \, .
\end{equation*}
La cosa importante da sottolineare è che gli operatori $\hat a^\dagger$ e $\hat a$ non a caso sono chiamati operatori di creazione e distruzione, perché a partire dallo stato fondamentale $\ket 0$, che ha energia $\frac{\hbar\omega}{2}$, possiamo costruire tutti gli altri livelli. L'azione su un autostato $\ket n$ non è altro che
\begin{equation}\label{a_adag_action_states}
\begin{aligned}
    &\hat a^\dagger \ket n = \sqrt{n+1}\ket{n+1}\, , \\
    &\hat a \ket n = \sqrt n \ket{n-1} \, .
\end{aligned}
\end{equation}
Pertanto, per costruire il livello $n$-esimo, sarà sufficiente applicare la definizione sul ground state $\ket{0}$:
\begin{equation*}
    \ket n = \frac{(a^\dagger)^n}{\sqrt{n!}}\ket 0 \, .
\end{equation*}
Se ora andassimo a rappresentare graficamente il potenziale armonico
\begin{equation*}
    V(x) = \frac 12 m \omega^2 x^2 \, ,
\end{equation*}
con le varie funzioni d'onda (\textit{polinomi di Hermite}), potremmo osservare l'andamento rappresentato in Figura \ref{fig:qha}. Dal momento che tutti i livelli sono equispaziati, ciascun livello differisce dal precedente e dal successivo per un termine $\hbar\omega$: l'idea, per realizzare un sistema a due livelli, è quindi quella di considerare i livelli $n=0$ e $n=1$, che identifichiamo con $\ket 0$ e $\ket 1$, come stati del nostro qubit. Allo stesso tempo però dobbiamo essere in grado di controllare il passaggio dallo stato fondamentale $\ket 0$ allo stato eccitato $\ket 1$ e questo può essere facilmente realizzato attraverso un impulso laser. Tuttavia è possibile che il nostro sistema si trovi già in uno stato eccitato e mediante un altro impulso laser passi a un livello eccitato che si trova al di fuori del nostro sistema a due livelli, ad esempio si può verificare
\begin{align*}
    \ket{1} &\longrightarrow \ket{2} \, , \\
    \ket{2} &\longrightarrow \ket{3} \, , \\
    \vdots \; &\longrightarrow \; \vdots
\end{align*}
Per cui la descrizione dell'oscillatore armonico quantistico è sì utile per la realizzazione di sistemi a due livelli, però presenta dei difetti correlati al fatto che i livelli energetici siano equidistanti. Vedremo nelle sezioni successive come si può intervenire per ovviare a questo inconveniente.

\noindent Sebbene quest'ultimo aspetto sia spesso visto come un problema per la realizzazione di un qubit, l'oscillatore armonico quantistico può essere utilizzato per realizzare alcuni gate non banali. Vediamo il seguente esempio accademico.

\begin{figure}[!t]
    \centering
    \includegraphics[scale=0.6]{images/qho.png}
    \caption{Potenziale armonico con le funzioni d'onda associate. I livelli energetici sono chiaramente equidistanti dal punto di vista energetico, quindi è molto difficile interagire con dei livelli a piacere mediante l'utilizzo della radiazione.}
    \label{fig:qha}
\end{figure}

\begin{esempio}[\textbf{Codifica del CNOT gate.}]
    Supponiamo di voler eseguire un calcolo quantistico che faccia uso di un \texttt{CNOT-gate} costruito con l'oscillatore armonico quantistico descritto sopra. Che cosa possiamo fare? La scelta più naturale per la rappresentazione dei qubit sono gli autostati energetici $\ket n$. Questa scelta ci permette di eseguire un \texttt{CNOT-gate} nel seguente modo: codifichiamo i seguenti 4 qubit logici utilizzando l'identificazione
    \begin{equation}\label{eq:cnot-qho}
        \begin{aligned}
            \ket{00}_L &= \ket{0} \, , &\ket{01}_L &= \ket{2} \, ,\\
            \ket{10}_L &= \frac{\ket{4}+\ket{1}}{\sqrt 2} \, , &\ket{11}_L &= \frac{\ket{4}-\ket{1}}{\sqrt 2} \, ,
        \end{aligned}
    \end{equation}
    \noindent dove il pedice $L$ è usato per distinguere chiaramente gli stati logici in contrasto con gli autostati dell'hamiltoniana dell'oscillatore armonico. Dal punto di vista concettuale, il fatto che stiamo utilizzando stati energetici come $\ket 2$ e $\ket 4$ sarà presto chiaro.
    
    \noindent La manipolazione dei qubit, come abbiamo visto all'inizio, può essere effettuata, ad esempio, tramite l'applicazione di un campo magnetico nel caso di un sistema con degli spin: in generale può essere necessario sottoporre il sistema a delle perturbazioni esterne. In questo caso possiamo semplicemente sfruttare l'evoluzione temporale degli stati: assumiamo di aver preparato il sistema in uno stato $\ket{n}$ e decidiamo di lasciarlo evolvere nel tempo, quindi
    \begin{equation*}
        \ket n \rightarrow \hat{U}(t) \ket{n} =  e^{-\frac{i}{\hbar}\hat{H} t} \ket{n} = e^{-\frac{i}{\hbar} E_n t} \ket{n} \, ;
    \end{equation*}
    lo stato finale rimane nello stato di partenza acquisendo tuttavia una fase: il punto importante che ci permette di costruire un \texttt{CNOT-gate} è che l'evoluzione temporale non crea una sovrapposizione di stati, ma mantiene bensì un singolo stato stazionario. Ricordando che i livelli energetici sono dati da
    \begin{equation*}
        E_n = \hbar \omega \left(n+\frac 12\right) \, ,
    \end{equation*}
    allora, trascurando il fattore $1/2$ perché fase comune a tutti i livelli, possiamo procedere nell'applicare l'operatore di evoluzione temporale agli stati logici in \eqref{eq:cnot-qho}
    \begin{align*}
        \ket{00}_L &= \ket{0} \, , &\ket{01}_L &= e^{-2i\omega t}\ket{2} \, , \\
        \ket{10}_L &= \frac{e^{-4i\omega t}\ket{4}+e^{-i\omega t}\ket{1}}{\sqrt 2} \, , &\ket{11}_L &= \frac{e^{-4i\omega t}\ket{4}-e^{-i\omega t}\ket{1}}{\sqrt 2} \, .
    \end{align*}
    Se scegliamo un valore di tempo particolare, come ad esempio $t=\pi/\omega$, l'evoluzione temporale porterà allora a
    \begin{align*}
        \ket{00}_L &\rightarrow \ket{0} \equiv \ket{00}_L \, , &\ket{01}_L &\rightarrow \ket{2} \equiv \ket{01}_L \, , \\
        \ket{10}_L &\rightarrow \frac{\ket{4}-\ket{1}}{\sqrt 2} \equiv \ket{11}_L \, , &\ket{11}_L &\rightarrow \frac{\ket{4}+\ket{1}}{\sqrt 2} \equiv \ket{10}_L \, \, .
    \end{align*}
    Abbiamo ottenuto esattamente l'azione di un \texttt{CNOT-gate} utilizzando semplicemente l'evoluzione temporale del sistema!
\end{esempio}

\noindent Perché non è possibile utilizzare fisicamente un \texttt{CNOT-gate} così costruito? Perché ogni conto in QC è basato sul calcolo quantistico, non sul calcolo analogico: per usare i qubit ottenuti è necessario intervenire mediante l'utilizzo della radiazione, ma così ritorniamo al problema originale dei livelli energetici equispaziati! È semplice costruire un \texttt{CNOT-gate}, ma purtroppo è molto difficile controllare un tale sistema.  


    \vspace{0.5cm}
\noindent \lecture{16}{2/12/2021}
\vspace{0.5cm}
Analizziamo ora come possiamo riscrivere l'hamiltoniana nel nuovo sistema di riferimento.
Il primo termine di $H_{rf}$ diventa:
\begin{equation*}
    i\hbar \dot U U^\dagger = i \hbar \frac{i H_0}{\hbar} e^{\frac{i H_0 t}{\hbar}}e^{-\frac{i H_0 t}{\hbar}}=-H_0
\end{equation*}
E il secondo:
\begin{equation*}
    UHU^\dagger = U(H_0 + H(t)) U^\dagger = UH_0 U^\dagger + U H(t) U^\dagger
\end{equation*}
E, dunque, unendo tutto:
\begin{equation*}
    H_{rf} = U H(t)U^\dagger
\end{equation*}
Studiamo ora lo specifico caso di un qubit che non interagisce con nulla, dove conosciamo la forma dei termini dell'hamiltoniana:
\begin{equation*}
    \hat H_0 = -\frac{\hbar \omega_q}{2}\hat \sigma_z
\end{equation*}
Avremo ora $U(t)=e^{-\frac{i\omega_q}{\hbar}\sigma_z t}$. 
Ricordiamo alcune proprietà degli operatori:
\begin{align*}
    A &= \sum a_n \ket{a_n} \bra{a_n} \\
    e^A &= \sum e^{a_n}\ket{a_n}\bra{a_n}
\end{align*}
E nostro caso:
\begin{equation*}
    \sigma_z = \ket 0 \bra 0 - \ket 1 \bra 1 
\end{equation*}
Perciò l'operatore $U$ diventa:
\begin{equation*}
    U(t) = e^{-i\frac{\omega_q}{2}t}\ket 0 \bra 0 + e^{i\frac{\omega_q}{2}t}\ket 1 \bra 1
\end{equation*}
A questo punto aggiungiamo un \textit{drive} al qubit (ovvero un termine di accoppiamento volto al controllo di esso): $\hat H_d = -A(t) \hat \sigma_x$ (senza perdita di generalità non consideriamo qui una dipendenza, pur possibile, da $\hat \sigma_y$).
Supponiamo che la forzante esterna sia data da: $A(t) = A\cos \left( \omega_d t \right)$.
L'hamiltoniana risulta:
\begin{equation*}
    \hat H = - \frac{\hbar \omega_q}{2}\hat \sigma_z -A (t) \hat \sigma_x = H_0 + H_d
\end{equation*}
Vediamo ora come si comporta il nostro sistema nel caso in cui usiamo un sistema di riferimento in rotazione con $\omega_d$ (ma sempre intorno all'asse z). Usiamo l'operatore $\hat U= e^{-i \omega_d \hat \sigma_z t}$.
Il primo termine sarà:
\begin{equation*}
    i \hbar \dot U U^\dagger = - i \hbar \cdot i \frac{\omega_d}{2}\sigma_z = \frac{\hbar }{2}\omega_d \sigma_z
\end{equation*}
Il secondo termine, invece, risulta:
\begin{equation*}
    U H U^\dagger = U H_0 U^\dagger + U H_d U^\dagger =  -\frac{\hbar}{2}\omega_q \hat \sigma_z -UA(t)\sigma_x U^\dagger
\end{equation*}
Dal primo termine arriviamo a scrivere nell'hamiltoniana finale un fattore (considerando anche quanto abbiamo calcolato alla precedente equazione): $-\frac{\hbar}{2}(\omega_q - \omega_d)\sigma_z$.
Dobbiamo d'altra parte studiare più a fondo il secondo termine. Dobbiamo utilizzare alcune relazioni (che qui diamo per note in partenza):
\begin{align*}
    e^{\pm \frac{i}{2} \omega \sigma_z t } &= \cos \left( \frac{\omega}{2}t \right) \pm i \sigma_z\sin \left( \frac{\omega}{2}t \right) \\
    \sigma_z \sigma_{x,y} \sigma_z &= - \sigma_{x,y}
\end{align*}
Scrivendo il coseno come $C$ e il seno come $S$ per semplicità, otteniamo da $UA(t)U^\dagger$ il termine:
\begin{equation*}
    \left(C-i \sigma_z S\right)\left( C + i \sigma_z S\right) = \sigma_x \left( C ^2 - S^2 \right) + i \left( \sigma_x \sigma_z - \sigma_z \sigma_x \right) SC = \sigma_x \cos \left(\omega_d t \right) +  \sigma_y \sin \left( \omega_d t \right)
\end{equation*}
E, dunque, il termine nell'hamiltoniana risulta:
\begin{equation*}
    UA(t)\sigma_x U^\dagger = A \left[ \sigma_x \cos^2 \left(\omega_d t \right) +  \sigma_y \sin \left( \omega_d t \right) \cos \left( \omega_d t \right) \right]
\end{equation*}
Tramite formule trigonometriche, possiamo riscrivere l'equazione:
\begin{equation*}
    UA(t)\sigma_x U^\dagger = A \left[ \frac{\sigma_x}{2}\left( 1 - \cos (2\omega_d t ) \right) + \frac{\sigma_y}{2}\sin (2 \omega_d t ) \right]
\end{equation*}
L'approssimazione RWA ci dice che, nel caso in cui $2 \omega_d \gg \abs{\omega_q - \omega_d}$ (e per un qubit siamo proprio in tale situazione) i termini oscillanti con $2\omega_d$ sono trascurabili. Dunque:
\begin{equation*}
    UA(t) \sigma_x U^\dagger \approx A\frac{\sigma_x}{2}
\end{equation*}
E l'hamiltoniana complessiva risulta essere indipendente dal tempo e scrivibile come:
\begin{equation*}
    \hat H = -\frac{\hbar}{2} (\omega_q - \omega_d ) \hat \sigma_z - \frac{A}{2}\hat \sigma_x
\end{equation*}
Si nota, inoltre, che tutti i termini di questa hamiltoniana sono in generale noti poiché la forzante è interamente controllata da noi (dunque sappiamo $A$ e $\omega_d$) e il qubit ha una frequenza misurabile tramite uno scan di $\omega_d$ e graficando $\Delta$ in funzione di $t$ (quindi sappiamo anche $\omega_q$).
Definendo il \textit{detuning} $\Delta = \omega_q - \omega_d$, la nuova hamiltoniana è scrivibile in forma matriciale come:
\begin{equation*}
    \hat H = \frac{\hbar}{2} \begin{pmatrix} - \Delta & -A \\ -A & +\Delta \end{pmatrix}
\end{equation*}
Se siamo in risonanza ($\Delta = 0$), l'hamiltoniana è allineata lungo lasse $x$ e abbiamo dunque una precessione di $\vec \rho$ intorno a tale asse. Siccome conosciamo la velocità angolare ($A/2$) data dalle caratteristiche dell'impulso RF, possiamo calcolare il tempo necessario per portare, ad esempio, uno stato $\ket 0$ in $\ket 1$ (in tale caso si parlerebbe di impulso $\pi$).
In generale non abbiamo le frequenze perfettamente uguali, perciò il primo passo consiste nel diagonalizzare l'hamiltoniana per trovare gli autovalori:
\begin{equation*}
    E_{1,2} = \pm \frac{\hbar}{2}\sqrt{A^2 + \Delta^2}
\end{equation*}
Perciò la differenza di energia fra i due livelli sarà $\hbar \sqrt{A^2 + \Delta^2}$. I nuovi autostati, scritti in funzione degli autostati dell'hamiltoniana funzione della sola $\sigma_z$ risultano:
\begin{align*}
    \ket{E_1} & = \cos \theta \ket 1 - \sin \theta \ket 0 \\
    \ket{E_2} & = \sin \theta \ket 1 + \cos \theta \ket 0 
\end{align*}
Dove abbiamo definito $\theta = \arctan\left( \frac{A}{\sqrt{A^2 + \Delta^2}-\Delta}\right)$.
Avremo di nuovo un'oscillazione di Rabi attorno a un asse ottenuto ruotando l'asse z iniziale di un angolo $2\theta$. Nel caso in cui $\Delta = 0$ ritroviamo il risultato precedente (con $\theta=\pi/4$ e l'asse di rotazione coincidente con l'asse $x$).
Se lo stato iniziale è scritto come $\ket{\psi} = C_1 \ket{E_1}+ C_2 \ket{E_2}$ la sua evoluzione temporale è scrivibile come:
\begin{equation*}
    \ket{\psi (t)} = C_1 e^{-\frac{i}{\hbar} E_1 t}\ket{E_1} + C_2 e^{-\frac{i}{\hbar} E_2 t}\ket{E_2}
\end{equation*}
Esprimendola nella base iniziale:
\begin{equation*}
    \ket{\psi (t)}= \left( C_1 e^{-\frac{i}{\hbar} E_1 t} \sin \theta + C_2 e^{-\frac{i}{\hbar} E_2 t}\cos \theta \right) \ket 1 + \left( C_1 e^{-\frac{i}{\hbar} E_1 t} \cos \theta - C_2 e^{-\frac{i}{\hbar} E_2 t}\sin \theta \right) \ket 0
\end{equation*}
Supponiamo ora che lo stato iniziale è $\ket{\psi_0} = \ket 0$ (abbiamo $C_1=\cos \theta$ e $C_2 =-\sin \theta$), l'evoluzione sarà (usiamo l'uguaglianza $E_1 = -E_2$):
\begin{equation*}
    \ket{\psi(t)}= -i \sin (E_1 t) \sin (2\theta) \ket 1 + \cos (E_1 t) \sin (2 \theta) \ket 0
\end{equation*}
Possiamo scrivere, a questo punto, la probabilità di ottenere tramite una misura $\ket 0$ o $\ket 1$ in funzione del tempo. Nel caso della probabilità di $\ket 1$ abbiamo:
\begin{equation*}
    P(\ket 1) = \sin ^ 2 (2 \theta) \sin ^2 \left( \frac{E_1 t}{\hbar}\right) = \frac{A^2}{A^2 + \Delta^2}\sin ^2 \left( \frac{\sqrt{A^2 + \Delta^2}}{2}t\right)
\end{equation*}
L'oscillazione massima, corrispondente con l'ottenimento di uno stato $\ket 1$ con probabilità piena, è ottenibile solo senza \textit{detuning} e si ha (prendendo il t minore) per $t=\frac{\pi}{A}$.
\begin{figure}[H]
    \centering
    \includegraphics[width=\textwidth]{images/rabi_RWA.png}
    \caption{La linea rossa (blu) rappresenta l'evoluzione di un qubit guidato nel sistema del laboratorio (sistema rotante del drive). \textbf{a} con $\Delta=0$ e \textbf{b} per $\Delta \neq 0$. \url{https://arxiv.org/pdf/1904.09291.pdf}}
    \label{fig:my_label}
\end{figure}

\subsection{Controllo XY}
Consideriamo ora l'hamiltoniana per un qubit TRANSMON accoppiato tramite una capacità a una sorgente che porta un termine dipendente da $\sigma_y$ tale che:
\begin{equation*}
    \hat H= - \hbar \frac{\omega_q}{2}\hat \sigma_z + A(t) \hat \sigma_y
\end{equation*}
Passo al sistema di riferimento in rotazione con $\hat U = e^{-\frac{i}{\hbar}\omega_q t}$.
In questo sistema potrò applicare rotazioni sia rispetto all'asse x che rispetto all'asse y.
L'hamiltoniana si potrà scrivere come:
\begin{equation*}
    \hat H = U H_d U^\dagger = A(t) \left[ \cos \left( \omega_q t \right)\sigma_y -  \sin \left( \omega_q t \right)\sigma_x \right]
\end{equation*}
In questo caso scegliamo $A(t) = A v(t)$ ponendo attenzione non solo sull'ampiezza, ma anche sulla fase:
\begin{equation*}
    v(t) = s(t) \sin \left( \omega_d t + \phi \right)
\end{equation*}
Dove $s(t)$ è una funzione adimensionale che funge da \textit{envelope} e mi porta a poter scrivere l'ampiezza come $As(t)$, mentre la fase $\phi$ è scelta arbitrariamente.
Possiamo riscrivere $v(t)$ come:
\begin{equation*}
    v(t) = s(t) \left( \cos (\phi) \sin (\omega_d t) + \sin (\phi) \cos (\omega_d t) \right)
\end{equation*}
La notazione più in voga per questi termini (che riconosciamo essere le due quadrature del segnale RF) ci dice:
\begin{align*}
    I &= \cos \phi \qquad \text{chiamata per ragioni storiche: componente in fase} \\
    Q &= \sin \phi \qquad \text{chiamata per ragioni storiche: componente fuori fase} 
\end{align*}
E abbiamo le proprietà:
\begin{equation*}
    Q^2 + I^2 = 1 \qquad , \qquad \phi = \arctan\frac{Q}{I}
\end{equation*}
E riscriviamo la nostra hamiltoniana come:
\begin{equation*}
    \hat H = As(t) \left[  I \sin (\omega_d t) - Q \cos (\omega_d t)  \right]\times \left[ \cos (\omega_q t) \sigma_y - \sin (\omega_q t) \sigma_x \right]
\end{equation*}
Svolgendo il prodotto, usando un po' di trigonometria e applicando la RWA (rimuovo i termini dipendenti da $\omega_q + \omega_d$) arrivo a:
\begin{equation*}
    \hat H = \frac{1}{2}As(t) \left[ \left(-I \cos ( \Delta t) + Q \sin ( \Delta t) \right) \sigma_x + \left(I\sin ( \Delta t) - Q \cos ( \Delta t)  \right) \sigma_y \right] 
\end{equation*}
Che, espressa in forma matriciale, diventa:
\begin{equation*}
    \hat H = -\frac{A}{2}s(t) \begin{pmatrix} 0 & e^{i (\Delta t + \phi)} \\ e^{-i (\Delta t + \phi)} & 0 \end{pmatrix}
\end{equation*}
    \noindent \lecture{17}{9/12/2021}
\vspace{0.5cm}
\noindent Per semplicità ci poniamo adesso, per fare un esempio, in risonanza ($\Delta=0$).
L'hamiltoniana risulta:
\begin{equation*}
    \hat H = -\frac{A}{2}s(t) \left(I\sigma_x + Q \sigma_y \right)
\end{equation*}
Se utilizzo un segnale 'in fase' ($I=1$, $Q=0$) posso scrivere l'operatore di evoluzione temporale del nostro qubit come:
\begin{equation*}
    \hat U (t) = \exp\left[ \left(\frac{iA}{2} \int_0^t dt'\, s(t') \right) \sigma_x \right]
\end{equation*}
Se, ad esempio, scegliamo:
\begin{equation*}
    s(t') = \begin{cases} 0 \qquad \text{se $t' < 0$ o $t'> t$} \\ 1 \qquad \text{altrimenti} \end{cases}
\end{equation*}
Allora abbiamo $\hat U (t) = e^{\frac{i}{2}A t \sigma_x}$, operatore che descrive una rotazione lungo l'asse x. Dunque se stimoliamo il qubit con un segnale di questo tipo possiamo ruotare lo stato di un angolo a piacere intorno a $x$. Se procediamo nello stesso modo, ma con un segnale completamente fuori fase, otterremo in modo analogo una rotazione attorno all'asse $y$.

\subsubsection{Mixer IQ e realizzazione hardware}
Un mixer semplice (non IQ), è un componente elettronico fondamentale per la modulazione di frequenze nell'ordine del GHz. Ha 3 porte: RF, LO (\textit{local}), IF (\textit{intermediate frequency)}. LO è sempre un input, mentre le altre due porte possono essere usate alternativamente come input o output.
Supponiamo di usare RF come input e IF come output. Possiamo in questo modo fare una \textit{down conversion}: ovvero ci portiamo da alte frequenze a basse frequenze. L'output da IF sarà proporzionale al prodotto del segnale RF e del segnale LO perciò, scegliendo in modo adeguato i segnali, potremo avere una situazione del tipo:
\begin{align*}
    V_{LO} &= V_0^{LO} \cos (\omega_{LO} t) \\
    V_{RF} &= V_0^{RF} \cos (\omega_{RF} t ) \\
    V_{IF} &\propto \cos ((\omega_{LO} + \omega_{RF}) t)- \cos ((\omega_{LO} - \omega_{RF}) t)
\end{align*}
Perciò il segnale in uscita sarà diviso in due frequenze: una alta (data dalla somma delle $\omega$ che saranno quasi uguali) e una bassa (data dalla differenza fra le frequenze). Con un filtro adeguato potremo poi scegliere solo una di queste frequenze.
Per una \textit{up conversion} useremo, invece, la porta IF come input e la porta RF come output ottenendo, in modo analogo due frequenze: una data da $\omega_{LO} - \omega_{IF}$ e una data da $\omega_{LO} + \omega_{IF}$ .
Un mixer IQ è, invece, una combinazione di due diversi mixer con un totale di 4 'porte libere'. 
\begin{figure}[H]
    \centering
    \includegraphics[width=0.6\textwidth]{images/iq_mixer.png}
\end{figure}
\noindent Supponiamo ora che I, Q e LO siano input cosinusoidali. 
A RF, dalla parte di I, arriverà un segnale $\propto \cos(\omega_{LO} + \omega_I)+ \cos (\omega_{LO} - \omega_I) $ e dalla parte di Q arriverà $\propto  \sin(\omega_{LO} + \omega_Q)+ \sin (\omega_{LO} - \omega_Q)$.
Cambiando la fase relativa fra i segnali I e Q, possiamo modulare a piacimento il segnale in uscita.

\vspace{0.5cm}

\noindent Consideriamo ora un qubit TRANSMON collegato tramite una capacità alla porta RF di un mixer IQ. Un oscillatore locale è connesso alla porta LO. Un generatore di onde a due uscite, connesso alle porte I e Q.
Con questa configurazione possiamo controllare il qubit con relativa facilità.
Assumiamo, in un primo semplice caso, che l'oscillatore sia in perfetta risonanza col qubit ($\omega_q = \omega_{LO}$). Allora, se l'output del AWG (generatore di funzioni arbitrarie) è semplicemente un segnale on/off, quando I e Q sono nulli il segnale su RF sarà anch'esso nullo (questo in realtà è vero solo in linea teorica, in realtà servirebbero minimi correttivi). Se solo I è diverso da zero otterremo un segnale totalmente fuori fase rispetto a LO, mentre se solo Q è diverso da zero il segnale sarà totalmente in fase.
Se, invece, il segnale di AWG ha frequenza $\omega_{AWG}$ e usiamo un oscillatore con frequenza $\omega_{LO}=\omega_q - \omega_{AWG}$, possiamo usare I e Q in modo da avere a piacimento un segnale in fase o fuori fase. Così facendo otterremo sempre due frequenze, di cui solo una utile al controllo. Ciò che viene fatto usualmente è di sfasare il segnale in ingresso di I e Q di 90°, in modo che il segnale in uscita RF abbia un'unica frequenza caratteristica (si parla di \textit{Single Sideband modulation - SSB}).
\begin{figure}[H]
    \centering
    \includegraphics[width= \textwidth]{images/single_sideband_modulation.png}
    \caption{Single Sideband Modulation (SSB): By up-converting the LO signal by $\Omega_{SSB}$ the mixer
outputs at qubit frequency. The relative phase of SSB pulses determines the phase of the output signal thus
the direction of the rotation for the qubit. \url{https://arxiv.org/pdf/1904.09291.pdf}}
\end{figure}
\noindent In questo modo possiamo a piacimento ruotare lo stato del qubit intorno agli assi x-y coi meccanismi di controllo XY che abbiamo già illustrato.
Nel caso avessimo più qubit tutti collegati al medesima linea (connessa a RF) possiamo variare la frequenza $\omega_{AWG}$ in modo da stimolare in risonanza solo un qubit alla volta. Questo chiaramente a patto che i qubit abbiano frequenze caratteristiche diverse (proprietà facilmente ottenibile con un qubit TRANSMON simmetrico e una corretta modulazione del flusso interno al circuito).

\section{Misurazioni e circuit QED}

Per la misurazione di un qubit, considereremo un TRANSMON. Utilizzeremo la teoria QED per circuiti (\textit{circuit QED}) che è un'estensione della \textit{cavity QED}. Quest'ultima teoria considera una cavità RF dove, essendoci delle proprietà di risonanza, possono stabilirsi delle onde stazionarie (dei modi di oscillazione che corrispondono a fotoni a frequenze diverse). 
La cosa interessante è che posso partire da una cavità "vuota" (in un qualche \textit{ground state} $\ket 0$) ed eccitare il sistema a $\ket 1$ o $\ket 2$ stimolando la cavità con l'appropriata onda.
Se posso limitarmi ai primi due stati ho, in principio, un qubit.

\subsection{Cavity QED}
\subsubsection{Misurazione di una cavità con un atomo di Rydberg}
Per misurare il suo stato posso far attraversare la cavità da un atomo con frequenza di transizione (una cavità è un oscillatore armonico quantistico) uguale a quella caratteristica della cavità/qubit (che posso modulare). L'atomo entra in interazione col qubit tramite oscillazioni del suo stato simili a quelle di Rabi perciò, scegliendo con cautela velocità dell'atomo e dimensioni della cavità, posso far sì che l'atomo subisca una rotazione di $\pi$ mentre la cavità non ruoti (dunque abbiamo una QND).
Negli esperimenti che portarono all'assegnazione del premio Nobel per la fisica del 2012 \footnote{\url{https://www.nobelprize.org/uploads/2018/06/advanced-physicsprize2012.pdf}}, venivano usati atomi di rubidio (Rb) (utili perché dotati di grande dipolo elettrico) sfruttando la cosiddetta interferometria Ramsey. La configurazione hardware dell'esperimento comprendeva, in particolare, due segnali RF che l'atomo attraversava prima (R1) e dopo (R2) attraverso il qubit atti alla misurazione e al controllo dello stato atomico (tramite tecnologie più classiche).
Dell'atomo vengono usati, in particolare, 3 stati: $\ket i$, $\ket g$ e $\ket e$ (dati da livelli di Rydberg in approssimazione di atomo sferico). Dove la differenza di energia fra $\ket g$ e $\ket e$ corrisponde alla frequenza di risonanza del qubit.
Assumiamo che l'atomo sia inizialmente nello stato $\ket g$ e che vogliamo misurare se la cavità si trova nello stato $\ket 0$ o $\ket 1$. Lo inviamo attraverso la cavità R1 (che ha frequenza caratteristica data dalla differenza di energia fra $\ket g$ e $\ket i$) e gli diamo un impulso di $\pi/2$. L'atomo, dopo R1, si troverà quindi in uno stato: $\frac{1}{\sqrt 2} \left( \ket g + \ket i \right)$. 
A questo punto l'atomo attraversa la cavità principale interagendo col campo elettrostatico quantizzato tramite il proprio dipolo elettrico. A seconda dello stato della cavità, l'accoppiamento fra campo elettrico e dipolo opererà in modo differente. L'hamiltoniana di interazione fra atomo e cavità viene detta hamiltoniana di Jaynes-Cumming e avrà chiaramente degli autostati che dipendono dagli stati del campo e dell'atomo. Essendoci una sovrapposizione di stati, avremo anche una precessione data dalla differenza di energia degli autostati ovvero avremo oscillazioni di Rabi. Progettando in modo adeguato l'esperimento, posso far sì che la cavità mi dia un impulso $\pi$. L'evoluzione dello stato è data da:
\begin{equation*}
    \cos \left( \frac{\Omega t}{2}\right) \ket{g, 1}+ \sin \left( \frac{\Omega t}{2}\right) \ket{e, 0}
\end{equation*}
L'atomo, dunque, varierà stato solo se la cavità è in $\ket 1$ e arriverà a $\frac{1}{\sqrt{2}}\left( -\ket g + \ket i \right)$ (ho $\Omega t/2=2\pi$).
Attraversando poi R2 (anch'essa $\pi/2$ come R1) lo stato andrà in $\ket g$ o $\ket e$ (a seconda del segno relativo fra $\ket i$ e $\ket g$).
    %%%%%%%%%%%%%%
% LECTURE 18 %
%%%%%%%%%%%%%%

\vspace{1cm}
\noindent\lecture{18}{10/12/2021}
\vspace{0.5cm}

\noindent Abbiamo visto come la CQED\footnote{Abbreviazione per Cavity Quantum Electrodynamics.} modellizzi l'interazione tra un qubit e il campo elettromagnetico quantizzato generato all'interno di una cavità o un risonatore. L'hamiltoniana di Jaynes-Cummings \eqref{eq:ham-jaynes-cummings} fu studiata per la prima volta nel contesto dell'ottica quantistica: non appare solo in questo caso particolare, ma in tutti quei casi in cui un qubit interagisce con uno dei modi quantizzati del campo elettromagnetico, ossia descrive tutte quelle interazioni della forma $\vec d \cdot \vec E$, $\vec \mu \cdot \vec B$, ecc.

\noindent Giunti a questo punto vogliamo discutere il significato fisico che si trova dietro a questa hamiltoniana, in particolare vedremo qualche semplice esempio di codifica di un qubit in un modello CQED. Innanzitutto notiamo che l'hamiltoniana originale non approssimata della relazione \eqref{H_da_riscrivere_4} non poteva essere risolta esattamente, tuttavia grazie alla RWA può essere invece diagonalizzata! 

\noindent Dal punto di vista della QM, l'hamiltoniana \eqref{eq:ham-jaynes-cummings} costituisce un semplice problema di accoppiamento tra spin e oscillatore armonico (le eccitazioni di questo oscillatore sono fotoni). Lo spazio di Hilbert totale è infinito dimensionale in quanto è frutto del prodotto $\mathcal{H}_c \otimes \mathcal{H}_q$, dove $\dim \mathcal{H}_c = \infty$ (infiniti oscillatori). Nonostante ciò, l'hamiltoniana sopra è diagonalizzabile perché è una matrice diagonale a blocchi; per tale ragione suddividiamo gli stati utilizzando la seguente notazione:
\begin{align*}
    &\ket{0} \equiv \ket{g} &\Rightarrow& &&\text{Stato fondamentale del qubit.} \\
    &\ket{1} \equiv \ket{e} &\Rightarrow& &&\text{Stato eccitato del qubit.} \\
    &\ket{n} = \ket{0}, \ket{1}, \ket{2}, \ldots &\Rightarrow& &&\text{Numero di fotoni campo elettromagnetico.}
\end{align*}
La struttura a blocchi è evidente notando che $\ket{e,n} \leftrightarrow \ket{g,n+1}$, ossia sono trasformati l'uno nell'altro dai termini dell'interazione: infatti
\begin{align*}
    \hat a^\dagger \hat \sigma_+ \ket{e,n} &= \sqrt{n+1} \ket{g,n+1} \, , \\
    \hat a \hat \sigma_- \ket{g, n+1} &= \sqrt{n+1} \ket{e,n} \, ,
\end{align*}
perché nel primo caso diseccitiamo lo stato del qubit e creiamo un fotone di diseccitazione (lo stato finale è lo stato fondamentale con un fotone in più), mentre nel secondo caso distruggiamo un fotone della cavità, il quale viene assorbito dallo stato fondamentale, che sarà poi eccitato. Alla luce di questa osservazione possiamo suddividere lo spazio di Hilbert totale $\mathcal{H}$ in
\begin{equation}\label{CQED_states}
    \left\{\ket{g,0}\right\} \, , \quad \left\{ \ket{e,n}, \, \ket{g,n+1}\right\} \, ;
\end{equation}
decomponendo $\mathcal{H}$ in questo modo, ogni qualvolta che si agisce con i termini di interazione in \eqref{eq:ham-jaynes-cummings} si rimane sempre nello stesso sottospazio. Tenendo presente che $\hat{H}_0$ è diagonale, mentre $\hat{H}_I$ è off-diagonal, allora in forma matriciale l'hamiltoniana diventa
\begin{equation*}
    \hat H = \frac 12 \omega_c \mathbb{I} + 
    \begin{pmatrix}
        -\frac{\omega_q}2 & & & \\
        & \begin{pmatrix}
            \omega_c - \frac{\omega_q}2 & g \\
            g & \omega_c + \frac{\omega_q}2
          \end{pmatrix} & & \\
        & & \ddots & \\
        & & & \begin{pmatrix}
            \omega_c(n+1) - \frac{\omega_q}2 & g\sqrt{n+1} \\
            g\sqrt{n+1} & \omega_c n + \frac{\omega_q}2
        \end{pmatrix} \\
    \end{pmatrix}
    \begin{matrix}
        \ket{g,0}\\
        \ket{g,1}\\
        \ket{e,0}\\
        \\
        \ket{g,n+1}\\
        \ket{e,n}
    \end{matrix}\, ,
\end{equation*}
dove il termine iniziale rappresenta l'energia di punto zero, detta \textbf{ZPE} ("Zero Point Energy"). (Gli stati a destra sono per ricordare ciò a cui fanno riferimento i blocchi di questa matrice). Ricordando che il \textbf{detuning} è definito come $\Delta=\omega_q - \omega_c$ e tenendo conto della ZPE, possiamo allora riscrivere il blocco generico (ultimo elemento) della matrice precedente come
\begin{equation*}
    \begin{pmatrix}
        (n+1)\omega_c - \frac{\Delta}2 & \sqrt{n+1}g \\
        \sqrt{n+1}g & (n+1)\omega_c + \frac{\Delta}2
    \end{pmatrix} \, ;
\end{equation*}
diagonalizzando questo blocco, lo spettro dell'hamiltoniana è dato dai seguenti autovalori e autostati
\begin{align*}
    E_+ &= (n+1)\omega_c + \frac 12 \sqrt{\Delta^2+4g^2(n+1)} \, , &\ket{n_+} &= \sin\theta_n \ket{g,n+1} + \cos\theta_n\ket{e,n} \, , \\
    E_- &= (n+1)\omega_c - \frac 12 \sqrt{\Delta^2+4g^2(n+1)} \, , &\ket{n_-} &= \cos\theta_n\ket{g,n+1}-\sin\theta_n\ket{e,n} \, ;
\end{align*} 
ad essi si aggiunge il singolo stato $\ket{g,0}$ con energia $E_0 = - \frac{\Delta}{2}$. Gli stati $\ket{n_-}$ e $\ket{n_+}$ prendono il nome di \textbf{dressed states} e l'angolo $\theta_n$ risulta essere definito come
\begin{equation*}
    \tan {2 \theta_n} = \frac{2g\sqrt{n+1}}{\Delta} \, .
\end{equation*}

\noindent Che cosa succede ad un sistema come questo? L'evoluzione temporale, che ricordiamo essere data da $e^{-i\hat H t}$ (hamiltoniana indipendente dal tempo), realizza delle \textbf{oscillazioni di Rabi} sulla coppia dei \textbf{dressed states}: ogni stato si comporta come un sistema a due livelli che oscilla coerentemente in cicli di assorbimento ed emissione di fotoni in maniera tale che gli stati in \eqref{CQED_states} si trasformino continuamente l'uno nell'altro, ossia $\ket{g,n+1} \leftrightarrow \ket{e,n}$. Confrontando con il caso dei campi esterni, qui abbiamo due accoppiamenti: l'interazione è quantificata da $g$, la forza dell'accoppiamento tra i due sistemi, alla quale si aggiunge però $n$, ovvero il numero dei fotoni. La frequenza di Rabi delle oscillazioni risulta quindi essere data da $\sqrt{\Delta^2+4g^2(n+1)}$: maggiore è il numero di fotoni, più grande sarà la frequenza di oscillazione del qubit.

\noindent In ottica quantistica si è spesso interessati a guardare a situazioni in cui si ha una sovrapposizione di differenti stati con numero di fotoni fissato. In questi contesti l'oscillazione generale è più complicata: ogni insieme di fotoni si accoppia in blocchi di 2 cosicché ogni blocco oscilli a coppie. Il punto fondamentale è che la sovrapposizione di oscillazioni crea una situazione in cui non ci sono oscillazioni! Talvolta è possibile aspettare del tempo a sufficienza fino a quando si ritorna ad osservare un pattern di oscillazioni: tipicamente, quando si sovrappongono molti sistemi oscillatori, si ha interferenza, quindi se si aspetta un tempo sufficiente si possono di nuovo osservare delle oscillazioni (rilevate sperimentalmente).   

\begin{figure}[!ht]
    \centering
    \includegraphics[scale=1]{images/graph.pdf}
    \caption{Differenza in energia dei dressed states con lo stato fondamentale in funzione del detuning. Si noti che la minima differenza di energia tra $E_+$ e $E_-$ si trova in corrispondenza di $\Delta = 0$, mentre la maggior differenza è data quando $\Delta$ diverge. L'asintoto orizzontale si trova in corrispondenza dei limiti: $\lim_{\Delta \to -\infty} (E_+-E_0) = \lim_{\Delta \to +\infty} (E_--E_0) = (n+1) \omega_c$. In questo caso si sono impostati i seguenti valori $\omega_c = 25$, $g=10$, $n=1$.}
    \label{fig:plot-dressed-states-detuning}
\end{figure}

\noindent Nel caso invece del qubit è possibile "giocare" con $g$, $n$ e $\Delta$ per osservare questo pattern di oscillazioni. Nella scorsa sezione abbiamo visto che in una situazione di risonanza esatta ($\Delta = 0$) vi era certezza che ad un certo punto il qubit avesse subito una transizione $\ket{0} \to \ket{1}$. Come vedremo tra poco, in altre situazioni può essere utile considerare il cosiddetto \textbf{regime dispersivo}, ossia $\Delta \neq 0$. Consideriamo l'energia dei \textbf{dressed states} come funzione del \textbf{detuning}: la differenza in energia con lo stato fondamentale non è altro che
\begin{equation*}
    E_\pm - E_0 = (n+1)\omega_c \pm \frac 12\sqrt{\Delta^2+4g^2(n+1)}+\frac \Delta 2 \, ;
\end{equation*}
se disegniamo un plot in funzione di $\frac{\Delta}{g}$ otteniamo la Figura \ref{fig:plot-dressed-states-detuning}. I due regimi particolarmente interessanti sono:
\begin{itemize}
    \item \textbf{Regime di risonanza}, quindi $\Delta = 0$ ($\omega_c=\omega_q$): in questo caso $\theta_n=\frac \pi 4$, per cui $\cos\theta_n=\sin\theta_n=\frac1{\sqrt2}$. Si dice che vi è un'\textbf{ibridizzazione massima} degli stati poiché
    \begin{equation*}
        \ket{n_\pm}= \frac{\ket{g,n+1}\pm\ket{e,n}}{\sqrt 2} \, ;
    \end{equation*}
    essi prendono il nome di \textbf{polarons}. La differenza in energia qui è la minima possibile e dipende da $n$:
    \begin{equation*}
        E_\pm = (n+1)\omega_c \pm g \sqrt{n+1} \, .
    \end{equation*}
    In generale è una situazione abbastanza simile alle oscillazioni di Rabi della Figura \ref{fig:Rabi}.
    
    \item \textbf{Regime dispersivo}, tale che $\Delta \gg g$: qui $\theta_n\ll 1$ e gli stati originali rimangono pressoché invariati a meno di piccole correzioni
    \begin{align*}
        \ket{n_-} &= \ket{g,n+1} + \dots \, , \\
        \ket{n_+} &= \ket{e,n} + \dots \, ;
    \end{align*}
    notiamo che questo comportamento è previsto dato che $\frac{\Delta}{g} \to \infty$ significa equivalentemente che $\Delta = \text{ cost}$ e $g \simeq 0$: siamo vicini al caso libero in cui il qubit e la radiazione e.m. sono quasi disaccoppiati. 

    \noindent Questo regime è interessante per diverse ragioni: ad esempio può essere utile tenere le piccole correzioni negli stati $\ket{n_+}$ e $\ket{n_-}$. Sviluppando la radice in $\frac{g^2}{\Delta^2}$ si ha
    \begin{align*}
        E_\pm &= (n+1)\omega_c \pm \frac 12 \sqrt{\Delta^2+4g^2(n+1)} \\
              &= (n+1)\omega_c \pm \frac{\Delta}{2} \sqrt{1+ \frac{4g^2}{\Delta^2}(n+1)} \\
              &= (n+1)\omega_c \pm \frac{\Delta}{2}\left(1+\frac{2g^2}{\Delta^2}(n+1)+\dots\right) \\
              &= (n+1)\omega_c \pm \left(\frac\Delta 2 +\frac{g^2}{\Delta}(n+1)+\dots\right) \, ;
    \end{align*}
    Osserviamo che possiamo ricavare le medesime energie dando una descrizione efficace del sistema con la seguente hamiltoniana
    \begin{equation}\label{eq:effective-hamiltonian}
        \hat H^{(2)}=\omega_c\left(\hat a^\dagger \hat a + \frac 12\right)-\frac{\omega_q}2\hat \sigma_3 - \frac{g^2}{\Delta}\left(\hat a^\dagger \hat a + \frac 12\right) \hat{\sigma}_3 + \frac{g^2}{2\Delta}\mathbb{I}_{2\times2} \, ,
    \end{equation}
    infatti
    \begin{align*}
        \hat H^{(2)}\ket{g, n+1} &= \omega_c\left(n+1+\frac 12\right)-\frac{\omega_q}{2}-\frac{g^2}{\Delta}(n+1) \\ 
        &= \omega_c(n+1)-\frac{\Delta}2 -\frac{g^2}\Delta(n+1) \equiv E_- \, , \\
        \hat H^{(2)}\ket{e,n} &= \omega_c\left(n+\frac 12\right) + \frac{\omega_q}2 + \frac{g^2}{\Delta}\left(n+ \frac 12\right) + \frac{g^2}{2\Delta} \\
        &= \omega_c(n+1)+\frac{\Delta}2 + \frac{g^2}\Delta(n+1) \equiv E_+ \, .
    \end{align*}
    Dunque possiamo dire che l'hamiltoniana efficace $\hat{H}^{(2)}$ descrive la fisica del sistema nel regime dispersivo. Sui libri si trovano spesso hamiltoniane più complicate: questa hamiltoniana è un esempio della cosiddetta \textbf{trasformazione di Schrieffer-Wolff}. Si tratta di scegliere l'operatore $\hat{U}$ della \eqref{S_eq_rotated_state} indipendente dal tempo e della forma 
    \begin{equation*}
        \hat U = e^{\hat{S}} \, , \quad \text{dove} \quad S^\dag = - S \, .
    \end{equation*}
    L'operatore $\hat{S}$ è scelto in maniera tale che si possa effettuare un'espansione in teoria delle perturbazioni su un opportuno parametro. Se l'hamiltoniana si scrive come $\hat{H} = \hat{H}_0 + \hat{H}_I$, allora si sceglie un $\hat{S}$ tale che $\comm{\hat{S}}{\hat{H}_0} = -\hat{H}_I$. Con un po' di algebra si dimostra che
    \begin{equation*}
        \hat{U} \hat{H} \hat{U}^\dag = \hat{H}_0 + \frac{1}{2} \comm{\hat{S}}{\hat{H}_I} + \order{S^3} \, .
    \end{equation*}
    Nel nostro caso, l'espressione di $\hat H^{(2)}$ si ottiene dall'espansione fino a $\order{g^2/\Delta^2}$ nel sistema ruotato dall'operatore
    \begin{equation*}
        \hat U = e^{-\frac{g}{\Delta}(\hat \sigma_+\hat a^\dagger - \hat \sigma_-\hat a)} \, .
    \end{equation*}
    
    Riscriviamo l'hamiltoniana \eqref{eq:effective-hamiltonian} senza i termini costanti:
    \begin{equation*}
        \hat H^{(2)} = \omega_c  \hat a^\dagger \hat a - \frac{\omega_q}2\hat \sigma_3 - \frac{g^2}{\Delta}\left(\hat a^\dagger \hat a + \frac 12\right) \hat{\sigma}_3 \, .
    \end{equation*}
    A seconda di ciò che si sta facendo è utile raggruppare gli operatori interni a questa espressione in due modi:
    
    \begin{itemize}
        \item Possiamo raggruppare scrivendo 
    \begin{equation*}
        \hat H^{(2)}= \left(\omega_c-\chi\hat \sigma_3\right)\hat a^\dagger \hat a-\frac{\tilde{\omega}_q}{2}\hat \sigma_3 \, ,
    \end{equation*}
    dove $\chi = g^2/\Delta$ e $\tilde{\omega}_q=\omega_q+g^2/\Delta$. È evidente che l'energia della cavità dipenderà da una frequenza che risulta shiftata di una costante $\chi$ dipendente dallo stato del qubit; similmente $\omega_q$ è ridefinito a seguito dell'interazione con il campo elettromagnetico (si parla infatti di \textbf{Lamb shift}, in onore dell'analogo in fisica atomica). Per tale ragione questa situazione può essere utilizzata per realizzare una \textbf{quantum nondemolition measurement}: possiamo stabilire lo stato in cui si trova il qubit misurando la frequenza della cavità! Si noti che questo non viola le leggi della QM.
    
    \item Un modo equivalente è quello di raggruppare tutte le matrici $\hat{\sigma_3}$ scrivendo
    \begin{equation*}
        \hat H^{(2)}= \omega_c\hat a^\dagger \hat a - \frac{\hat \sigma_3}{2}\left(\omega_q + \frac{g^2}{\Delta}+\frac{2g^2}{\Delta}\hat a^\dagger \hat a \right) \, ;
    \end{equation*}
    in questa visione la frequenza del qubit è ridefinita a seguito di due fattori: il Lamb shift (secondo termine della parentesi) e il cosiddetto \textbf{AC - Stark Effect} (ultimo termine), il quale indica il numero di fotoni che creano rumore nella frequenza dei qubit; quindi in questo caso anche il numero di fotoni influenza la frequenza del qubit. 
    \end{itemize}
\end{itemize}

\subsection{Operazioni su 2 qubit}
Il regime dispersivo nel sistema qubit-cavità può essere utilizzato per codificare delle operazioni (gate) che coinvolgono due qubit. Ricordiamo che nella sezione precedente abbiamo visto che le operazioni sui \textbf{singoli} qubit sono realizzate abbastanza semplicemente per mezzo delle oscillazioni di Rabi. Qui il trucco è quello introdurre uno stato addizionale, che chiamiamo $\ket{\gamma}$, al fuori della cavità risonante (in cui sono presenti gli stati $\ket{g}$ e $\ket{e}$ che interagiscono con i fotoni):
\begin{center}
    \mbox{
        \Qcircuit @C=1em @R=2em {
            \lstick{\ket{e}} & \qw & \qw & \qw \\
            \lstick{\ket{g}} & \qw & \qw & \qw
        }
    }
    \raisebox{-1em}{\mbox{
        \Qcircuit @C=1em @R=2em {
            & \qw & \qw & \qw & \rstick{\ket{\gamma}} \\
        }
    }}
\end{center}
Precisiamo che $\ket{e}$ e $\ket{g}$ sono accoppiati con la cavità, mentre $\ket \gamma$ è disaccoppiato dal sistema qubit-cavità. L'idea è quella di codificare un qubit con gli stati $( \ket g, \ket \gamma)$ e uno con gli stati $(\ket 0, \ket 1)$ dei fotoni, cosicché lo spazio di Hilbert totale contenga gli stati seguenti
\begin{equation*}
    \mathcal{H}=\left\{\ket{\gamma 0}, \ket{\gamma 1}, \ket{g0}, \ket{g1}\right\} \, .
\end{equation*}
L'interazione sarà descritta dall'hamiltoniana \eqref{eq:effective-hamiltonian}: lo stato $\ket{e}$ è irrilevante in questa descrizione, mentre l'accoppiamento qubit-cavità è dato dagli ultimi due termini interagenti (quelli con $g$). Consideriamo l'evoluzione temporale del sistema del qubit: gli operatori $\hat{\sigma}_3$ e $\mathbb{I}$ agiscono solamente su $\ket{e}$ e $\ket{g}$ perché $\ket{\gamma}$ è disaccoppiato. Per tale motivo, dal momento che $\hat \sigma_3\ket{\gamma}=\mathbb{I}\ket{\gamma}=0$ e $\hat \sigma_3\ket{g}=\ket{g}$, la parte interagente di $\hat{H}$ agisce come
\begin{equation*}
    e^{-i \hat H_It}=e^{i\frac{g^2}{\Delta}\hat a^\dagger \hat a \hat \sigma_3 t + i\frac{g^2}{2\Delta}(\hat \sigma_3 - \mathbb{I}_{2\times2})t}=\begin{pmatrix}
        1 & & & \\
        & 1 & & \\
        & & 1 & \\
        & & & e^{i\frac{g^2}{\Delta}t}
    \end{pmatrix}
    \begin{matrix}
        \ket{\gamma 0} \\ \ket{\gamma 1} \\ \ket{g 0} \\ \ket{g1}
    \end{matrix} \, .
\end{equation*}
(come in precedenza sono scritti gli stati sulla destra per ricordare l'origine degli elementi di matrice). Il gate risultante è detto \textbf{QPG}, ossia \textbf{Quantum Phase Gate}, poiché è della forma
\begin{equation*}
    Q_\eta = \begin{pmatrix}
        1 & & & \\
        & 1 & & \\
        & & 1 & \\
        & & & e^{i\eta}
    \end{pmatrix} \, .
\end{equation*}
Questa tipologia di gate include il caso $\eta = \pi$, ovvero 
\begin{equation*}
    Q_\pi = \begin{pmatrix}
        1 & & & \\
        & 1 & & \\
        & & 1 & \\
        & & & -1
    \end{pmatrix} \, ;
\end{equation*}
il gate precedente potrebbe sembrare banale ma non lo è perché, usando operazioni a singolo gate su $\ket{\gamma}$, $\ket{g}$, può essere convertito in un \texttt{CZ-gate} (Controlled-$Z$)
\begin{center}
    \mbox{
        \Qcircuit @C=1em @R=1em {
            & \qw & \ctrl{1} & \qw & \qw & \\
            & \qw & \gate{Z} & \qw & \qw &
        }
    }
\end{center}
ma sappiamo che quest'ultimo è legato al \texttt{CNOT-gate} attraverso l'applicazione di due \texttt{H-gate}

\begin{center}
    \mbox{
        \Qcircuit @C=1em @R=1.2em {
            & \qw & \ctrl{1} & \qw & \qw & \\
            & \qw & \targ & \qw & \qw &
        }
    }
    \raisebox{-1em}{=}
    \mbox{
        \Qcircuit @C=1em @R=1em {
            & \qw & \ctrl{1} & \qw & \qw & \\
            & \gate{H} & \gate{Z} & \gate{H} & \qw &
        }
    }
\end{center}
Questo significa che con delle singole operazioni possiamo trasformare un QPG in un \texttt{CNOT-gate}, che sappiamo molto bene che agisce su due qubit contemporaneamente. 

\noindent Cosa succede invece ai termini non interagenti in \eqref{eq:effective-hamiltonian}? Anche loro permettono di implementare delle trasformazioni sui qubit: l'evoluzione temporale dell'hamiltoniana libera è disaccoppiata poiché $\hat{H}^{(2)}_0$ è somma delle hamiltoniane del qubit e della cavità. Perciò questa evoluzione può sempre essere fattorizzata come
\begin{equation*}
    e^{-i\omega_c\left(\hat a^\dagger \hat a + \frac 12\right)t + i\frac{\omega_q}{2}\hat \sigma_3t} = e^{-i\omega_c\left(\hat a^\dagger \hat a + \frac 12\right)t}e^{i\frac{\omega_q}{2}\hat \sigma_3t}=
    e^{-i\frac{\omega_c}{2}t}
    \underbrace{\begin{pmatrix}
        1 & 0\\
        0 & e^{i\frac{\omega_q}{2}t}
    \end{pmatrix}
    }_{\{\ket \gamma, \ket g\}}
    \otimes
    \underbrace{
    \begin{pmatrix}
        1 & 0\\
        0 & e^{i\omega_c t}
    \end{pmatrix}
    }_{\{\ket 0, \ket 1\}} \, .
\end{equation*}
Questo è un fatto generale: l'evoluzione libera descrive sempre l'evoluzione del singolo qubit perché $\hat{H}^{(2)}_0$ è fattorizzata. 

\noindent Nelle prossime sezioni vedremo alcuni esempi espliciti, come qubit superconduttivi e trappole ioniche, che metteranno in pratica i concetti che abbiamo studiato nel corso delle ultime due sezioni. In generale non è così difficile creare i gate, ma la difficoltà spesso risiede nel controllarli; spesso le idee funzionanti sono frutto di intuizioni creative e geniali legate al trovare la corretta evoluzione temporale del sistema. 
    %%%%%%%%%%%%%%
% LECTURE 19 %
%%%%%%%%%%%%%%
\vspace{1cm}

\noindent\lecture{19}{13/12/2021}

\section{Sistemi a trappola ionica}
A partire da questa sezione focalizzeremo la nostra attenzione sullo studio del controllo e della realizzazione fisica (pratica) di sistemi costituiti da qubit e gate. 

\noindent I sistemi basati sulle cosiddette \textbf{trappole di ioni} sono una tecnologia sperimentale sviluppatasi nel corso degli anni '80. Come già introdotto all'inizio del capitolo, queste apparecchiature sono costituite da un campo elettromagnetico generato da una serie di elettrodi cilindrici che intrappola al proprio interno un gruppo di ioni (solitamente ioni di berillio). Si faccia riferimento alla Figura \ref{fig:ion-trap1} di Pagina \pageref{fig:ion-trap1} per una rappresentazione schematica. A causa della loro particolare disposizione, gli elettrodi generano un potenziale\footnote{I pedici \textbf{dc} e \textbf{rf} significano rispettivamente \textit{direct current} e \textit{radio frequency}.} indipendente dal tempo e uno variabile:
\begin{align*}
    \phi_{\text{dc}} &= k U_0 \left( z^2 - x^2 - y^2 \right) \, , \\
    \phi_{\text{rf}} &= \left( V_0 \cos(\Omega t) + U_0 \right) \left( 1-\frac{x^2-y^2}{R^2} \right) \, .
\end{align*}
È possibile mostrare che l'effetto di questi potenziali è quello di creare un'hamiltoniana con il seguente potenziale armonico
\begin{equation*}
    H = \sum_{i=1}^N \frac{M}{2} \left( \omega_x^2 x_i^2 + \omega_y^2 y_i^2 + \omega_z^2 z_i^2 \right) + \sum_{j>i} \frac{e^2}{4 \pi \varepsilon_0 \abs{\vec{x}_i - \vec{x}_j}} \, ,
\end{equation*}
dove $N$ è il numero di ioni intrappolati dal campo e $M$ la loro massa (il secondo termine è repulsivo). Per realizzare un tale setup si sceglie una direzione privilegiata (per convenzione $z$) tale per cui $\omega_x, \omega_y \gg \omega_z$, in questo modo l'effetto che si ottiene è che gli ioni cercano di allinearsi solamente lungo $z$, ossia la direzione in cui sono "accesi" i modi vibrazionali (lungo $x$ e $y$ sono soppressi). Diagonalizzando esplicitamente un'hamiltoniana della forma precedente si ricava che la frequenza minore è associata al moto del centro di massa del sistema: la frequenza minima corrisponde quindi al movimento rigido degli ioni lungo $z$. 

\noindent Per gli scopi del QC vorremmo essere in grado di indirizzare e manipolare a piacimento gli stati quantistici del sistema. Innanzitutto è necessario lavorare a temperature molto basse, ossia $k T \ll \hbar \omega_z$ (molto più piccole del primo stato eccitato), perché nessuno dei modi vibrazionali lungo $x$ o $y$ deve essere eccitato. Non entriamo nei dettagli\footnote{Ad esempio viene effettuato il cosiddetto \textbf{Doppler cooling}, il quale sfrutta il fatto che, per effetto Doppler, la frequenza degli ioni in movimento rispetto al laser cambi a seconda del verso del moto. In questo modo solamente gli ioni che si dirigono verso il laser possono assorbire fotoni, al contrario invece di quelli che si muovono in direzione contraria.}, tuttavia ci limitiamo a sottolineare che questa procedura di raffreddamento viene attuata per mezzo di opportuni laser. In aggiunta ai laser è necessario costruire le cosiddette \textbf{sidebands}: si eccitano i livelli energetici interni degli stati degli ioni in modo tale che si possano etichettare gli stati utilizzando anche i modi vibrazionali. In generale entrambe queste procedure possono essere realizzate sperimentalmente con il seguente risultato: il primo stato eccitato quantistico corrisponde all'oscillazione del centro di massa del sistema lungo $z$ con frequenza $\omega_z$. 

\noindent Spesso i modi vibrazionali che possono essere eccitati sono comunemente detti \textbf{fononi}. Il moto (oscillazione) del centro di massa è il primo elemento su cui si basano i sistemi a trappola ionica e può essere descritto in maniera del tutto analoga ad un oscillatore armonico: come nella \eqref{x_a_adag}, la quantizzazione è effettuata promuovendo la coordinata spaziale ad operatore
\begin{equation*}
    \hat{z} = z_0 (\hat{a} + \hat{a}^\dag) \, , \quad \text{dove} \quad z_0 = \sqrt{\frac{\hbar}{2 \omega M N}} \, .
\end{equation*}

\noindent Il secondo ingrediente che si aggiunge ai modi vibrazionali sono gli stati atomici degli ioni: solitamente si scelgono opportunamente gli ioni in maniera tale che si riescano a isolare esplicitamente due livelli energetici dello spettro rispetto a tutti gli altri; in questo modo è possibile codificare in questi livelli un qubit, ma soprattutto, essendo gli stati separati dal resto dello spettro, sono facilmente controllabili dai laser.  Per ridurre al minimo la probabilità di transizione $\ket{1} \to \ket{0} $ a seguito dell'emissione spontanea si cercano degli stati eccitati che presentano una vita media molto lunga, come ad esempio alcuni ioni con stati eccitati metastabili. Un'altra scelta è quella di sfruttare la struttura iperfine dei livelli energetici degli ioni (dovuta all'interazione tra gli spin dei nucleoni e degli elettroni esterni): visto che l'ampiezza di decadimento per emissione spontanea \`e $\Gamma \sim \omega^3$, questi livelli molto più stabili di altri livelli energetici atomici.  

\noindent Come già mostrato nella Figura \ref{fig:ion-trap2}, per gli scopi del QC i qubit sono codificati in ciascuno degli ioni (si ottiene un array di qubit) utilizzando entrambi gli stati precedenti: un generico stato è quindi descritto da entrambi i modi, quelli vibrazionali (fononi) e quelli energetici. Per quanto riguarda i fononi si utilizzano due livelli: $\ket{0}$, ossia nessun fonone, e il suo stato eccitato $\ket{1}$, un fonone; vedremo a breve che utilizzando questi stati è possibile codificare un qubit extra, detto \textbf{bus qubit}. 

\noindent Per interagire con un sistema così generato si utilizzano delle opportune radiazioni elettromagnetiche generate da laser. Queste radiazioni presentano un comportamento classico, quindi consideriamo l'accoppiamento tra qubit e campo esterno classico oscillante della Sezione \ref{sec:qubit_campo_em_classico} (notare che gli operatori $\hat{a}$ e $\hat{a}^\dag$ fanno riferimento ai modi vibrazionali, non al campo quantizzato). Consideriamo un solo ione: l'interazione con il campo è descritta dall'accoppiamento $\vec{d} \cdot \vec{E}$, quindi l'hamiltoniana non è altro che la \eqref{H_couple}, ossia
\begin{equation}\label{da_espandere_kz}
    \hat{H} = \Omega \hat{\sigma}_1 \cos \left( k z - \omega t + \phi \right) \, .
\end{equation}
In questo contesto $kz \simeq k z_0 = 2\pi \frac{z_0}{\lambda}$ e $k z_0 \equiv \eta$ è detto \textbf{parametro di Lamb-Dicke}, il quale misura il rapporto tra l'ampiezza dell'oscillazione dei modi vibrazionali del qubit e la lunghezza d'onda della radiazione. Per assicurare che $kz$ sia circa costante sul qubit dobbiamo stare attenti alle due scale del problema:
\begin{enumerate}
    \item Dato che la grandezza dei qubit è quella degli ioni allora vorremmo che la lunghezza d'onda dei laser ($\lambda$) sia molto più grande delle scale atomiche. 
    
    \item Per la presenza dei modi vibrazionali lungo $z$ dobbiamo richiedere che $\lambda$ sia molto più grande delle oscillazioni lungo $z$. 
\end{enumerate}
\noindent Quando le precedenti sono verificate possiamo assumere che $kz \ll 1$, ma non completamente trascurabile, in modo tale da poter effettuare un'espansione perturbativa della \eqref{da_espandere_kz}:
\begin{equation*}
    \hat{H} = \Omega \hat{\sigma}_1 \cos \left( -\omega t + \phi \right) - \Omega k z \hat{\sigma}_1 \sin \left( -\omega t + \phi \right) + \order{(kz)^2} \, ;
\end{equation*}
inserendo gli operatori e trascurando i termini di ordine superiore avremo
\begin{equation}\label{hams_I_II}
    \begin{aligned}
        \hat{H} &= \underbrace{\frac{\Omega}{2} (\hat{\sigma}_+ + \hat{\sigma}_-) \left( e^{-i(\omega t - \phi)} + e^{i(\omega t - \phi)}  \right)}_{\hat{H}_{(I)}} + \\
        &\quad + \underbrace{i \frac{\Omega}{2} \eta (\hat{\sigma}_+ + \hat{\sigma}_-) (\hat{a} + \hat{a}^\dag) \left( e^{-i(\omega t - \phi)} - e^{i(\omega t - \phi)}  \right)}_{\hat{H}_{(II)}} \, ,
    \end{aligned}
\end{equation}
dove abbiamo distinto i due termini di $\hat{H}$ perché tra poco vedremo che contribuiranno in modo differente. Assumiamo di lavorare in un regime in cui la frequenza del laser sia sintonizzata su entrambe le frequenze in gioco, ossia quella della differenza in energia dei livelli del qubit e la frequenza dei modi vibrazionali. Per questo possiamo utilizzare nuovamente la RWA perché sappiamo che solo i termini in risonanza contribuiscono. Innanzitutto, l'hamiltoniana libera e l'operatore $\hat{U}$ con cui effettuiamo la rotazione saranno ($\hbar = 1$)
\begin{equation*}
    \hat{H}_0 = -\frac{\omega_q}{2} \hat{\sigma}_3 + \omega_z (\hat{a}^\dag \hat{a}) \, , \quad \Rightarrow \quad \hat{U} = e^{i \hat{H}_0 t} \, ;
\end{equation*}
facendo uso delle \eqref{RWA_ops_trans} (chiaramente in questo contesto $\omega_c \equiv \omega_z$) possiamo facilmente scrivere l'hamiltoniana ruotata della \eqref{S_eq_rotated_state}: come sappiamo ogni operatore ottiene un fattore di fase e l'idea è quella di tenere solamente i termini in risonanza e trascurare tutti gli altri per tempi molto lunghi. 

\begin{figure}[H]
	\centering	
	\subfloat[][\label{subfig:levels_ion_trap1} ]{\includegraphics[scale=.45,keepaspectratio]{images/levels_ion_trap1}} \\
	\subfloat[][\label{subfig:levels_ion_trap2} ]{\includegraphics[scale=.45,keepaspectratio]{images/levels_ion_trap2}}
	\caption{(\ref{subfig:levels_ion_trap1}) Suddivisione dei livelli energetici degli ioni in un sistema a trappola ionica. Le frequenze $\omega_q - \omega_z$ e $\omega_q + \omega_z$ sono dette \textbf{red sideband} e \textbf{blue sideband} rispettivamente.  (\ref{subfig:levels_ion_trap2}) Oscillazioni di Rabi prodotte dai termini delle hamiltoniane $\hat{H}_{(I)}$ e $\hat{H}_{(II)}$. Notare che per la red sideband contribuiscono gli operatori $\hat{\sigma}_- \hat{a}$ ($\ket{01} \to \ket{10}$) e $\hat{\sigma}_+ \hat{a}^\dag$ ($\ket{10} \to \ket{01}$); mentre per la blue sideband contribuiscono $\hat{\sigma}_+ \hat{a}$ ($\ket{11} \to \ket{00}$) e $\hat{\sigma}_- \hat{a}^\dag$ ($\ket{00} \to \ket{11}$).}
    \label{fig:levels_ion_trap}
\end{figure}

\noindent Ci sono molti termini che possono contribuire a seconda della frequenza del laser. L'idea nei sistemi a trappola ionica è quella di utilizzare una radiazione che possa accomodare 4 differenti frequenze. Innanzitutto etichettiamo gli stati del sistema con la notazione $\ket{nm}$, dove $n$ è il livello energetico del qubit e $m$ il modo di oscillazione vibrazionale (fonone). Tenendo conto della Figura \ref{subfig:levels_ion_trap1} vorremmo utilizzare le 4 frequenze $\pm \omega_q \pm \omega_z$ e $\pm \omega_q$ (quest'ultimo permette le oscillazioni $\ket{0m} \leftrightarrow \ket{1m}$).

\noindent Scriviamo i termini che "sopravvivono" dalla RWA nelle due hamiltoniane $\hat{H}_{(I)}$ e $\hat{H}_{(II)}$:
\begin{align}
    \hat{H}_{(I)} &= \frac{\Omega}{2} \left( \hat{\sigma}_+ e^{i(\omega t - \phi)} + \hat{\sigma}_- e^{-i(\omega t - \phi)} \right) \label{H_I_Rabi} \\
    \hat{H}_{(II)} &= i \frac{\eta \Omega}{2} \left( -\hat{\sigma}_+ \hat{a}^\dag e^{i(\omega t - \phi)} + \hat{\sigma}_- \hat{a} e^{-i(\omega t - \phi)} \right) + \label{H_II_sideband} \\
    &\quad + i \frac{\eta \Omega}{2} \left( -\hat{\sigma}_+ \hat{a} e^{i(\omega t - \phi)} + \hat{\sigma}_- \hat{a}^\dag e^{-i(\omega t - \phi)} \right) \, , \notag
\end{align}
dove nella \eqref{H_I_Rabi} abbiamo assunto $\omega \sim \omega_q$ e nella \eqref{H_II_sideband} si ha $\omega \sim \omega_q - \omega_z$ nella prima riga e $\omega \sim \omega_q + \omega_z$ nella seconda (notare che $\omega$ in queste due righe è scelto appositamente per annullare le fasi derivanti dalla \eqref{RWA_ops_trans}). Come evidente dal disegno in Figura \ref{subfig:levels_ion_trap2}, $\hat{H}_{(I)}$ genera oscillazioni di Rabi tra $\ket{0 m} \leftrightarrow \ket{1m}$ se si utilizza un laser con frequenza $\omega \sim \omega_q$; similmente, le due righe di $\hat{H}_{(II)}$ hanno un comportamento analogo perché la blue sideband (seconda riga) genera oscillazioni di Rabi tra $\ket{11} \leftrightarrow \ket{00}$ e la red sideband (prima riga) produce oscillazioni di Rabi tra $\ket{01} \leftrightarrow \ket{10}$ (si faccia sempre riferimento alla Figura \ref{subfig:levels_ion_trap2} tenendo presente l'azione di $\hat{\sigma}_\pm$ sui livelli del qubit).  

\noindent La logica è quindi quella di utilizzare le oscillazioni di $\hat{H}_{(I)}$ per muovere il qubit lungo la sfera di Bloch (implementare gate agenti su singoli qubit) e le oscillazioni di $\hat{H}_{(II)}$ per codificare delle operazioni agenti contemporaneamente su due tipi differenti di qubit. Vediamo il più semplice esempio di costruzione di un tale gate.

\subsection{Cirac-Zoller gate}
Il \textbf{Cirac-Zoller gate} costituisce un esempio di realizzazione pratica di un \texttt{CNOT-gate}. Immaginiamo di considerare, in aggiunta ai livelli energetici nella Figura \ref{fig:levels_ion_trap}, un livello extra, che chiamiamo $\ket{2m}$, del sistema atomico: 
\begin{center}
    \mbox{
        $
        \begin{matrix}
        &\Qcircuit @C=2em @R=1.3em {
            \lstick{\ket{11}} & \qw & \qw & \qw \\
            \lstick{\ket{10}} & \qw & \qw & \qw
        }
        \\ \\ \\
        &\Qcircuit @C=2em @R=1.3em {
            \lstick{\ket{01}} & \qw & \qw & \qw \\
            \lstick{\ket{00}} & \qw & \qw & \qw
        }
        \end{matrix}
        $
    }
    \raisebox{0.6em}{\mbox{
        \Qcircuit @C=2em @R=1.3em {
            & \qw & \qw & \qw & \rstick{\ket{21}} \\
            & \qw & \qw & \qw & \rstick{\ket{20}}
        }
    }}
\end{center}
Chiamiamo $E_{10} - E_{20} = \omega_{\text{aux}}$ (aux per ausiliaria) e chiaramente poniamo come prima $E_{10} - E_{00} = \omega_q$. L'idea è quella di utilizzare un laser sintonizzato ad una frequenza $\omega = \omega_{\text{aux}} + \omega_z$ per produrre transizioni tra gli stati $\ket{20} \leftrightarrow \ket{11}$. Le oscillazioni di Rabi così prodotte, regolando opportunamente l'ampiezza, la fase e la frequenza del laser, possono essere parametrizzate come al solito
dall'operatore in \eqref{formula_for_Rabi}
\begin{equation*}
    R_{\vec{n}}(\gamma) = e^{-\frac{i}{2} \gamma (\vec{\sigma} \cdot \vec{n})} = \mathbb{I} \cos \! \left( \frac{\gamma}{2} \right) - i (\vec{\sigma} \cdot \vec{n}) \sin \! \left( \frac{\gamma}{2} \right) \, ;
\end{equation*}
se scegliamo di ruotare con angolo $2 \pi$ attorno ad $x$ allora 
\begin{equation*}
    R_x(2 \pi) = e^{-\frac{i}{2} 2 \pi \sigma_x} = \mathbb{I} \cos \pi - i \sin \pi \sigma_x = -\mathbb{I} \, ,
\end{equation*}
quindi una rotazione spaziale di 360° agisce in maniera non banale sui fermioni! (Si noti che questo è vero per qualsiasi direzione $\vec{n}$). Quindi se si scelgono dei laser opportuni che implementano trasformazioni $R_x(2\pi)$ e si aspetta del tempo a sufficienza, allora possiamo realizzare questa operazione sugli stati precedenti: $\ket{11} \to - \ket{11}$ e $\ket{20} \to -\ket{20}$. Se ci dimentichiamo dello stato ausiliario $\ket{20}$ (l'informazione è codificata negli stati $\ket{nm}$), allora l'effetto netto sul sistema non è altro che un \texttt{CZ-gate}
\begin{equation*}
    \begin{pmatrix}
        1 & & & \\ & 1 & & \\ & & 1 & \\ & & & -1
    \end{pmatrix}
    \begin{pmatrix}
        \ket{00} \\ \ket{01} \\ \ket{10} \\ \ket{11}
    \end{pmatrix}
    =
    \begin{cases}
        \ket{00} \to \ket{00} \\
        \ket{01} \to \ket{01} \\
        \ket{10} \to \ket{10} \\
        \ket{11} \to -\ket{11}
    \end{cases}
    \! \! = \; 
    \raisebox{1.4em}{\mbox{
        \Qcircuit @C=1em @R=1.2em {
            & \qw & \ctrl{1} & \qw & \qw \\
            & \qw & \gate{Z} & \qw & \qw
        }
    }}
\end{equation*}
perché l'operatore $Z$ viene applicato sul secondo qubit solamente quando il primo si trova in $\ket{1}$. Come già anticipato al termine della Sezione \ref{sec:int_qubit_CQED}, è molto semplice passare da un \texttt{CZ-gate} ad un \texttt{CNOT-gate}:
\begin{center}
    \mbox{
        \Qcircuit @C=1em @R=1.2em {
            & \qw & \ctrl{1} & \qw & \qw & \\
            & \qw & \targ & \qw & \qw &
        }
    }
    \raisebox{-1em}{= \;}
    \mbox{
        \Qcircuit @C=1em @R=1em {
            & \qw & \ctrl{1} & \qw & \qw & \\
            & \gate{H} & \gate{Z} & \gate{H} & \qw &
        }
    }
\end{center}
In generale è abbastanza semplice realizzare un \texttt{CZ-gate} sui qubit, ma è invece meno banale realizzare questa operazione sugli stati costruiti con i modi vibrazionali. Nonostante ciò, si può ovviare a questo problema ricordando che questo gate ha la proprietà
\begin{center}
    \mbox{
        \Qcircuit @C=1em @R=1.2em {
            & \qw & \ctrl{1} & \qw & \qw \\
            & \qw & \gate{Z} & \qw & \qw
        }
    }
    \raisebox{-1em}{\; = \;}
    \mbox{
        \Qcircuit @C=1em @R=1.2em {
            & \qw & \gate{Z} & \qw & \qw \\
            & \qw & \ctrl{-1} & \qw & \qw
        }
    }
\end{center}
perché il risultato è analogo a quello sopra anche se $Z$ agisce sul primo qubit quando il secondo è in $\ket{1}$: non importa dove è posto $Z$ perché si può equivalentemente applicare questo gate sul qubit (più semplice) o sui modi vibrazionali!

\noindent Questi risultati relativi alla realizzazione di operazioni su un insieme di due qubit furono un grande traguardo negli anni '90, tuttavia al giorno d'oggi si vorrebbe costruire un QC con $\sim 100$ qubit, quindi sarebbe veramente poco pratico creare un sistema entangled tra ioni e modi vibrazionali. Dato che nella trappole ioniche si allinea facilmente un array di qubit in cui essi sono creati separatamente, si vorrebbe codificare l'informazione solamente nei qubit realizzati dagli ioni e non in quelli ottenuti dai modi vibrazionali. Questo scopo può essere raggiunto sfruttando i modi vibrazionali come \textbf{bus}, ossia modi ausiliari, che muovono l'informazione da ione a ione.

\noindent Immaginiamo un generico array di qubit: vorremmo poter indirizzare operazioni a due qubit su due qubit ben precisi dell'array utilizzando in qualche modo i modi vibrazionali come step intermedio. Ciò può essere fatto per mezzo dei fononi e utilizzando il cosiddetto \texttt{SWAP-gate}, ossia un gate che scambia informazioni da un qubit ai modi vibrazionali e successivamente da questi ultimi ad un altro qubit. 

\noindent Immaginiamo ad esempio di voler scambiare gli stati  $\ket{01} \leftrightarrow \ket{10}$: questo può essere fatto per mezzo della matrice
\begin{equation*}
    \begin{pmatrix}
        1 & & & \\ & 0 & 1 & \\ & -1 & 0 & \\ & & & 1
    \end{pmatrix}
    \begin{pmatrix}
        \ket{00} \\ \ket{01} \\ \ket{10} \\ \ket{11}
    \end{pmatrix}
    = 
    \begin{pmatrix}
        \ket{00} \\ \ket{10} \\ \ket{01} \\ \ket{11}
    \end{pmatrix} \, ,
\end{equation*}
ma il blocco interno non è altro che una rotazione di Rabi di angolo $\pi$ lungo $y$:
\begin{equation*}
    \begin{pmatrix}
        0 & 1 \\ -1 & 0
    \end{pmatrix}
    = i \sigma_2 =
    \cos \! \left( \frac{\pi}{2} \right) + i \sin \! \left( \frac{\pi}{2} \right) \sigma_2 = e^{\frac{i}{2} \pi \sigma_2} = R_y(-\pi) \, .
\end{equation*}
Riusciamo a codificare in qualche modo un'oscillazione di Rabi che operi con $R_y(-\pi)$ su $\ket{01}$ e $\ket{10}$? La risposta è affermativa perché possiamo utilizzare una delle frequenze di sideband dell'hamiltoniana in  \eqref{H_II_sideband}: ad esempio possiamo impiegare la red sideband (prima riga) per implementare tutti i possibili operatori $R_{\vec{n}}(\gamma)$ agenti sul sottospazio $\{ \ket{01}, \ket{10} \}$. Come funziona questo \texttt{SWAP-gate}? Immaginiamo di partire in uno stato dato dal prodotto tensoriale di un modo senza fononi e un qubit arbitrario dell'array di ioni: 
\begin{equation*}
    \left( a \ket{0} + b \ket{1} \right) \otimes \ket{0} = a \ket{00} + b \ket{10} \overset{\texttt{SWAP}}{\longrightarrow} a \ket{00} + b \ket{01} = \ket{0} \otimes \left( a \ket{0} + b \ket{1} \right) \, ,
\end{equation*}
è quindi possibile scambiare informazioni che erano codificate nel qubit con lo stato del modo vibrazionale. Successivamente si agisce con un altro \texttt{SWAP-gate} e si sceglie un altro ione nel quale trasferire l'informazione acquisita. 

\noindent Più esplicitamente, etichettiamo gli ioni dell'array con $j,k = 1, 2, 3, 4, \ldots$; ora mostriamo che è possibile costruire un gate $\texttt{CZ}_{(i)}$ per ogni ione (gate che coinvolge il sistema combinato fononi-ioni) e poi agire con un \texttt{SWAP-gate} su ciascun ione per trasferire l'informazione e utilizzare quindi i modi vibrazionali come \textbf{bus}. 

\begin{esempio}[\textbf{\texttt{CZ-gate} e \texttt{CNOT-gate} tra ioni}]
    Supponiamo di voler realizzare un \texttt{CZ-gate} tra due ioni generici $j$ e $k$, dove quest'ultimo agisce come control-qubit sul primo. Possiamo agire con $\texttt{SWAP}_k$ per muovere l'informazione da $k$ ai fononi, applicare $\texttt{CZ}_j$ tra i fononi e lo ione $j$ e infine ritornare allo ione $k$ con $\texttt{SWAP}^{-1}_k$: in ordine significa applicare le operazioni
    \begin{equation*}
        \texttt{SWAP}^{-1}_k \, \texttt{CZ}_j \, \texttt{SWAP}_k \, .
    \end{equation*}
    
    \noindent La stessa procedura può essere applicata per l'implementazione di un \texttt{CNOT-gate} con l'unica differenza che è necessario aggiungere due \texttt{H-gate} prima e dopo: 
    \begin{equation*}
        H_k \texttt{SWAP}^{-1}_k \, \texttt{CZ}_j \, \texttt{SWAP}_k H_k \, .
    \end{equation*}
\end{esempio}

\noindent Storicamente questa procedura per costruire un 2-qubit gate fu importante perché fu il primo esempio di implementazione di operazioni agenti su due qubit contemporaneamente in una trappola ionica. Oggigiorno è considerato un esempio per lo più di importanza storica e non pratica, perché per far sì che lo scambio dell'informazione con lo \texttt{SWAP-gate} avvenga è necessario partire con un sistema senza fononi, cosa che è raggiunta senza non poche difficoltà raffreddandolo con dei laser.   

\noindent Sarebbe decisamente più conveniente poter lavorare con dei gate che funzionino con un numero arbitrario di fononi. Questo è il caso del seguente esempio.

\subsection{M\o lmer-S\o rensen gate}
Non entriamo nel dettaglio della discussione di questo gate, tuttavia ci limitiamo a notare che si tratta di un'altra situazione in cui si necessita risolvere un ingegnoso esercizio in QM. L'idea peculiare è quella di considerare un laser bicromatico (irradia con 2 frequenze differenti) che possa indirizzare 2 o più ioni simultaneamente, i quali possono avere stati intermedi con $n-1$, $n$ e $n+1$ fononi. Denotando come al solito con $\ket{e}, \, \ket{g}$ gli stati del qubit e con $\ket{n}$ i modi vibrazionali dei fononi, facciamo riferimento alla Figura \ref{subfig:molmer_sorensen1}. Supponiamo che esistano alcuni stati intermedi tra $\ket{ggn}$ e $\ket{een}$: è possibile utilizzare un laser con una frequenza leggermente desintonizzata di un fattore $\delta$ per mandare lo stato $\ket{ggn}$ alla sovrapposizione di stati immediatamente prima di $\ket{eg \, n+1}$; successivamente si sceglie la seconda frequenza del laser bicromatico in maniera tale che permetta poi la transizione fino a $\ket{een}$ (vedi frecce rosse). Ovviamente la stessa cosa può essere fatta invertendo le frequenze del laser (vedi frecce blu). Esplicitamente le due frequenze del laser bicromatico sono $\omega = \omega_q \pm (\omega_z - \delta)$. 

\begin{figure}[!h]
	\centering	
	\subfloat[][\label{subfig:molmer_sorensen1} ]{\includegraphics[scale=.31,keepaspectratio]{images/molmer_sorensen1}} \\
	\subfloat[][\label{subfig:molmer_sorensen2} ]{\includegraphics[scale=.31,keepaspectratio]{images/molmer_sorensen2}}
	\caption{(\ref{subfig:molmer_sorensen1}) Oscillazioni di Rabi che connettono gli stati $\ket{ggn} \leftrightarrow \ket{een}$ nel M\o lmer-S\o rensen gate. (\ref{subfig:molmer_sorensen2}) Caso analogo al precedente in cui sono connessi gli stati $\ket{egn} \leftrightarrow \ket{gen}$.}
    \label{fig:molmer_sorensen}
\end{figure}

\noindent Come mostrato in Figura \ref{subfig:molmer_sorensen2}, lo stesso setup può essere utilizzato per connettere gli stati $\ket{egn} \leftrightarrow \ket{gen}$: è quindi possibile, in generale, creare oscillazioni di Rabi tra gli stati $\ket{ggn} \leftrightarrow \ket{een}$ e $\ket{egn} \leftrightarrow \ket{gen}$. 

\noindent Questa tipologia di trasformazioni sono al secondo ordine in teoria delle perturbazioni: gli stati intermedi non sono mai realmente popolati perché per poter effettuare operazioni sui qubit per mezzo delle oscillazioni di Rabi è necessario lavorare alla risonanza. In generale questo non è del tutto ovvio, tuttavia la cosa molto ingegnosa sta nel fatto che le oscillazioni di Rabi risultanti sono del tutto indipendenti dal numero $n$ di fononi! Dunque non è più necessario raffreddare il sistema. 

\noindent Per mostrarlo rigorosamente è necessario svolgere un conto completo in teoria delle perturbazioni dipendenti dal tempo. Intuitivamente si può notare che, dato che $\hat{H}_{\text{int}} \sim (\hat{a} + \hat{a}^\dag)$, allora negli stati dei modi vibrazionali si possono avere solamente un fonone in più o un fonone in meno rispetto allo stato di partenza, quindi $\ket{m} = \ket{n+1}$ oppure $\ket{m} = \ket{n-1}$; ciò è dato dal fatto che solamente gli elementi di matrice di questi stati intermedi sono diversi da zero, ossia $\mel{n}{\hat{a}}{n+1} = \sqrt{n+1}$ e $\mel{n}{a^\dag}{n-1} = \sqrt{n}$. Per questo motivo, un calcolo in teoria delle perturbazioni per campo debole produce  un'ampiezza di Rabi  indipendente da $n$:
\begin{equation*}
    \begin{large}\substack{\text{Ampiezza} \\
    \text{di Rabi}}\end{large} \sim \sum_m \frac{\mel{een}{\hat{H}_{\text{int}}}{m} \mel{m}{\hat{H}_{\text{int}}}{ggn}}{E_m - E_{ggn} - \omega} \sim \frac{(n+1)}{\delta} + \frac{n}{-\delta} \sim \frac{1}{\delta} \, .
\end{equation*}
Un conto pi\`u completo  in approssimazione RWA consente di calcolare esattamente l'operatore di evoluzione temporale che ha la forma
\begin{equation*}
    \hat{U}(t) \sim e^{\alpha(t) \hat{a} + \alpha^\ast(t) \hat{a}^\dag + i S^2_y \beta(t)} \, , \quad \text{dove} \quad S_y = \sigma_1^{(1)} + \sigma_y^{(2)} \, ,
\end{equation*}
dove $\alpha(t)$ e $\beta(t)$ sono due funzioni del tempo calcolabili. Scegliendo i valori del tempo che risolvono l'equazione $\alpha(t) = 0$, l'operatore di evoluzione temporale diventa indipendente dagli oscillatori associati ai fononi e produce un gate universale che realizza l'entanglement tra due qubit.
    %%%%%%%%%%%%%%
% LECTURE 20 %
%%%%%%%%%%%%%%
\vspace{1cm}

\noindent\lecture{20}{17/12/2021}
\section{Sistemi superconduttivi}
Dopo aver studiato e analizzato i sistemi a trappola ionica, diamo uno sguardo, in maniera del tutto generale, a un altro modo, molto diffuso e utilizzato, di andare a realizzare sistemi a due livelli: i \textbf{sistemi superconduttivi}. Spesso vengono anche definiti come \textbf{circuiti QED} (\textbf{cQED}) in analogia proprio con le cavità QED (CQED). Negli esempi analizzati nel caso della CQED si sfruttava il fatto che un semplice modello possa essere utilizzato per descrivere l'interazione di un atomo con una cavità ottica oppure anche per spiegare l'accoppiamento di un qubit con un risonatore a microonde: questo modello include il numero di fotoni nella cavità/risonatore, lo stato dell'atomo/qubit e l'interazione del dipolo elettrico tra l'atomo/qubit e la cavità/risonatore.

\noindent Invece, come suggerisce il nome, la fisica che sta dietro ai sistemi superconduttivi sfrutta il fenomeno della superconduttività. Per una trattazione completa sarebbe richiesto un corso intero, per cui, nel nostro studio, ci limitiamo a riportare i risultati generali che serviranno a descrivere questo tipo di sistemi.

\subsection{Cenni di superconduttività}

L'idea alla base della realizzazione fisica dei qubit è abbastanza semplice, tuttavia la fisica che ci sta dietro è abbastanza complessa perché coinvolge il concetto della \textbf{superconduttività}. Prima di vedere come si realizza la costruzione dei qubit e dei gate, facciamo alcuni cenni su questo importante argomento. 

\noindent Che cos'è la superconduttività? In corrispondenza di temperature sufficientemente basse, elementi metallici ed alcuni semiconduttori vanno incontro ad una transizione di fase. Al di sotto di una particolare temperatura, detta temperatura critica $T_C$, essi acquistano notevoli proprietà fisiche; la loro resistività
diventa bruscamente nulla, perciò in questi materiali diventa possibile, in assenza di campi esterni, misurare correnti che non decadono nel tempo. L’assenza di effetti dissipativi nel meccanismo di conduzione, assieme ad altri fenomeni correlati, sono indicati sinteticamente con il termine \textbf{superconduttività}. Nel 1957 J. Bardeen, L. N. Cooper e J. R. Schrieffer, formularono la prima teoria microscopica della superconduttività (\textbf{teoria BCS}) utilizzando la meccanica quantistica, che valse loro il Nobel nel 1972. Tale teoria è in grado di dare una spiegazione del fenomeno della superconduzione (capacità predittiva e base per le applicazioni). 

\noindent Il fenomeno della superconduttività consiste nella creazione, per mezzo dello scambio di fononi nel metallo e a temperatura sufficientemente bassa, di stati legati costituiti da coppie di elettroni $(e^-,e^-)$, chiamate \textbf{coppie di Cooper}. Questi oggetti sono ovviamente bosoni e costituiscono un particolare condensato di Bose-Einstein.

\noindent In questa configurazione, mentre i fermioni, a causa del principio di esclusione di Pauli, sono disposti in maniera tale da non avere lo stesso set di numeri quantici, i bosoni, a basse temperature, sono tutti situati nello stato fondamentale. In un metallo regolare se gli elettroni vengono messi in moto, tipicamente, per via della presenza di impurezze o reticoli di cristallo con cui gli elettroni fanno scattering, vi è una resistenza. La stessa situazione vale per i bosoni, ma vi è una interazione di scambio: è energicamente favorevole mettere i bosoni nello stesso stato (stesso set di numeri quantici), in particolare nello stato fondamentale perché per spostare uno di questi bosoni è richiesta una grande quantità di energia. La corrente che si viene a originare è un flusso di bosoni, tutti con la stessa velocità e dato che è richiesta energia per rimuovere un bosone dallo stato in cui si trovano tutti gli altri, hanno tutti una bassa resistenza. Un esempio di spettro energetico a basse temperature di un materiale superconduttivo è dato dalla Figura \ref{fig:super-spectrum}.

\begin{figure}[!ht]
    \centering
    \includegraphics[scale=0.45]{images/super-spectrum.jpg}
    \caption{Spettro energetico di un materiale superconduttivo. La differenza tra lo stato fondamentale e gli stati eccitati è noto come \textbf{mass gap} $\Delta$.}
    \label{fig:super-spectrum}
\end{figure}


\subsection{cQED}

L'idea è quella di realizzare dei qubit con dei piccoli circuiti superconduttivi (non si tratta più di un sistema atomico o di spin). Un tale sistema sembrerebbe a prima vista macroscopico, tuttavia, come vedremo, grazie alla presenza del fenomeno della superconduttività è davvero quantistico. 

\begin{figure}[H]
    \centering
    \begin{circuitikz}
        \draw
        (0,0)   to[C=$C$] ++ (0, 2) -- ++ ( 2,0) 
                to[L=$L$] ++ (0,-2) -- ++ (-2,0)
                (1,0)node[ground]{};
    \end{circuitikz}
    \caption{Circuito LC.}
    \label{fig:lc-circuit}
\end{figure}

\noindent Ancora una volta si utilizza un oscillatore armonico: affinché si possa utilizzare come un sistema a due livelli è necessario introdurre dell'anarmonicità nei livelli energetici così da poter distinguere lo stato fondamentale dal primo eccitato senza preoccuparsi degli altri livelli. Il nostro punto di partenza è un semplice \textbf{circuito LC}, come mostrato in Figura \ref{fig:lc-circuit}.

\noindent Vediamo il motivo per cui un tale circuito presenta un comportamento oscillatorio. Innanzitutto, ricordiamo che ciascun elemento mette in relazione la carica o la corrente con il potenziale; in particolare le relazioni costitutive della capacità e dell'induttanza risultano:
\begin{align*}
    &\text{Capacità:} &Q & =CV\, , \\
    &\text{Induttanza:} &V &= L\dv{I}{t}\, .
\end{align*}
Per svolgere questo tipo di trattazione può tornare utile riscrivere la corrente come $I=\dv{Q}{t}$ e introdurre il \textit{flusso} definito come
\begin{equation*}
    \Phi(t)=\int_{-\infty}^{t} \dd{t'} V(t') \, ;
\end{equation*}
a questo punto, la relazione sull'induttanza può, ad esempio, essere riscritta in termini di flusso integrando entrambi i membri 
\begin{equation*}
    \Phi = LI \, .
\end{equation*}
Note le relazioni tra carica/corrente e potenziale, possiamo andare a valutare le energie associate a ciascun elemento del circuito. A partire da
\begin{equation}\label{eq:general-energy}
    E(t)=\int_{-\infty}^{t} \dd{t'} V(t')I(t') \quad (\text{da } \delta E = V \delta Q) \, ,
\end{equation}
assumendo che tutte le quantità in gioco si annullino a $t=-\infty$, ricaviamo
\begin{align*}
    &\text{Capacità:} &E&=\int_{-\infty}^{t} \dd{t'} VC\derivative{V}{t'}=\frac C 2 V^2 = \frac{Q^2}{2C} \, , \\
    &\text{Induttanza:} &E&=\int_{-\infty}^{t} \dd{t'} L\derivative{I}{t'}I=\frac L 2 I^2 \equiv \frac{\Phi^2}{2L}\, .
\end{align*}
In una trattazione classica possiamo interpretare il termine relativo alla capacità come un'\textbf{energia cinetica} mentre quello relativo all'induttanza come \textbf{energia potenziale}. In questo contesto possiamo andare a scrivere la lagrangiana classica del sistema nel seguente modo
\begin{align*}
    \mathcal{L}&= E_k - E_p = \frac{Q^2}{2C}-\frac{\Phi^2}{2L} = \frac{C}{2}V^2-\frac{\Phi^2}{2L} = \frac{C}{2}\dot{\Phi}^2-\frac{\Phi^2}{2L} \, ,
\end{align*}
dove abbiamo inserito il fatto che $V = \dv{\Phi}{t}$. Applicando le equazioni di Eulero-Lagrange
\begin{equation*}
    \dv{t} \left(\pdv{\mathcal{L}}{\dot{\Phi}}\right)-\pdv{\mathcal{L}}{\Phi}=0 \, ,
\end{equation*}
si ottiene l'equazione del moto
\begin{equation*}
    C\ddot{\Phi}+\frac{\Phi}{L}=0 \, ,
\end{equation*}
che possiamo riscrivere sotto forma di equazione del moto di un oscillatore armonico nel seguente modo
\begin{equation*}
    \ddot{\Phi}+\omega^2\Phi = 0 \, , \quad \text{con} \quad \omega=\frac{1}{\sqrt{LC}} \, .
\end{equation*}
Per passare ad una descrizione quantistica del sistema dobbiamo riscrivere la fisica nel formalismo hamiltoniano. Calcoliamo il \textit{momento coniugato}
\begin{equation*}
    \Pi = \fdv{\mathcal{L}}{\dot{\Phi}} = C\dot{\Phi} = CV = Q\, ,
\end{equation*}
cosicché possiamo calcolarci la \textit{trasformata di Legendre} della lagrangiana precedente
\begin{equation}\label{eq:ham-LC}
        H = \Pi\dot{\Phi} - \mathcal{L} = Q\frac QC - \left(\frac{Q^2}{2C}-\frac{\Phi^2}{2L}\right) = \frac{Q^2}{2C} + \frac{\Phi^2}{2L} \, ;
\end{equation}
è evidente che l'hamiltoniana corrisponde proprio alla somma dell'energia cinetica e dell'energia potenziale.

\noindent Supponiamo ora di considerare il medesimo sistema, ma di volerne dare una descrizione quantistica. In Tabella \ref{tab:oa-lc} sono riportate le relative identificazioni con il nuovo sistema quantistico.
\begin{table}[H]
	\centering
    \begin{tabular}{c|c}
        \toprule
        Oscillatore armonico & Circuito LC \\
        \midrule
        $H=\frac{\hat{p}^2}{2m}+\frac{m}{2}\omega^2\hat{x}$ & $H=\frac{Q^2}{2C}+\frac C2 \omega^2\Phi^2$ \\
        \midrule
        $\hat{x}$ & $\Phi$ \\
        \hline
        $\hat{p}$ & $Q$ \\
        \hline
        $m$ & $C$ \\
        \bottomrule
    \end{tabular}\\
    \caption{Identificazioni tra quantità fisiche e operatori dell'oscillatore armonico quantistico e le grandezze fisiche di un circuito LC. Si ricordi che per il circuito LC la frequenza è $\omega = \frac{1}{\sqrt{LC}}$.}
    \label{tab:oa-lc}
\end{table}
\noindent Possiamo quindi mappare un circuito LC con un oscillatore armonico e imporre la quantizzazione canonica per quantizzare un tale sistema. Possiamo promuovere $(\Phi, Q) \to (\hat{\Phi}, \hat{Q})$ ad operatori e scriverli in funzioni degli operatori di creazione e distruzione
\begin{equation}\label{Phi_Q_oscill}
    \hat \Phi = \sqrt{\frac{\hbar}{2C\omega}}\left(\hat a + \hat a^\dagger\right) \, , \qquad \hat Q = \frac{\sqrt{2C \omega \hbar}}{2i}\left(\hat a - \hat a^\dagger\right) \, ;
\end{equation}
in questo modo imponiamo le seguenti regole di commutazione
\begin{equation*}
    \comm{\hat \Phi}{\hat Q}=i\hbar \, , \quad \Leftrightarrow \quad \comm{\hat a}{\hat a^\dagger}=1 \, .
\end{equation*}
È dunque evidente che la procedura di quantizzazione è immediata, tuttavia è abbastanza insolito pensare di svolgere una trattazione quantistica di un sistema macroscopico. 
\noindent Se supponiamo invece di considerare un sistema superconduttivo, possiamo utilizzare alcuni risultati della teoria BCS per riscrivere l'hamiltoniana della relazione \eqref{eq:ham-LC} nel seguente modo
\begin{equation}\label{eq:ham-oa-cQED}
    \hat H = 4E_C\hat n^2 + \frac{E_L}{2}\hat \phi^2 \, , \quad \text{con} \quad \hbar\omega=\sqrt{8E_LE_C} \, ,
\end{equation}
dove
\begin{itemize}
    \item $\hat n = \frac{Q}{2e}$ è l'operatore che indica il numero di coppie di Cooper;
    \item $\hat \phi = \frac{2\pi}{\Phi_0}\Phi$ è l'operatore del flusso ridotto ($\Phi_0=\frac{h}{2e}$ è il \textit{quanto di flusso});
    \item $E_C=\frac{e^2}{2C}$ è l'energia necessaria per aggiungere un elettrone extra alla capacità del circuito;
    \item $E_L=\left(\frac{\Phi_0}{2\pi}\right)^2\frac 1 L$ è l'energia associata all'induttanza.
\end{itemize}
L'altro fatto fondamentale, accanto all'utilizzo di materiali superconduttivi, che rende il sistema adatto ad una descrizione quantistica è che in realtà non andiamo a utilizzare un circuito LC generico (il quale presenta infiniti livelli energetici equidistanti), ma uno in cui si introduce una cosiddetta \textbf{giunzione Josephson} (Figura \ref{subfig:JJ-circuit}). Quest'ultima produce il cosiddetto \textbf{effetto Josephson} (Figura \ref{subfig:J-effect}),  introducendo quindi dell'anarmonicità. 

\begin{figure}[!ht]
	\centering	
	\subfloat[][\label{subfig:JJ-circuit}]{
	    \begin{circuitikz}
        \draw
        (0,0)   to[C=$C$] ++ (0, 2) -- ++ ( 2,0) 
                to[josephson=$L$] ++ (0,-2) -- ++ (-2,0)
                (1,0)node[ground]{};
    \end{circuitikz}
	} \qquad
	\subfloat[][\label{subfig:J-effect}]{\includegraphics[scale=.4,keepaspectratio]{images/J-effect.jpg}}
	\caption{(\ref{subfig:JJ-circuit}) Circuito LC con giunzione Josephson (la giunzione è caratterizzata da una propria capacità $C_J$ e induttanza $L_J$). D'ora in avanti indicheremo nei circuiti una giunzione Josephson con la notazione di questo circuito. (\ref{subfig:J-effect}) Effetto tunnel della coppia di Cooper tra due elettrodi metallici superconduttivi sufficientemente vicini e separati da una sottile barriera di ossido.}
\end{figure}
\noindent L'effetto Josephson può essere riassunto nei seguenti due risultati:
\begin{enumerate}
    \item La corrente prodotta per effetto tunnel delle coppie di Cooper segue la \textbf{prima legge di Josephson}
    \begin{equation}\label{1_Josephson_law}
        I = I_C \sin \phi \, ,
    \end{equation}
    dove $I_C$ è detta \textbf{corrente critica};
    
    \item Applicando una differenza di potenziale ai capi degli elettrodi scorrerà un'altra corrente, la quale segue la \textbf{seconda legge di Josephson}
    \begin{equation}\label{2_Josephson_law}
        V = \dv{\Phi}{t} = \frac{\hbar}{2 e} \dv{\phi}{t} \, .
    \end{equation}
\end{enumerate}
Se si costruisce un circuito LC inserendo questa giunzione è possibile ricalcolare l'energia potenziale associata al circuito sfruttando la relazione in \eqref{eq:general-energy} 
\begin{equation}\label{eq:E_Josephson}
    E=\int_{-\infty}^{t} \dd{t'} V(t')I(t')=\frac{\hbar}{2e}I_C\int_{-\infty}^{t} \dd{t'} \derivative{\phi}{t'}\sin\phi =-E_J\cos\phi \, ,
\end{equation}
dove abbiamo indicato l'\textbf{energia Josephson} con $E_J = \frac{\hbar I_C}{2e}$. In questo modo l'hamiltoniana di un sistema di questo tipo risulta in
\begin{equation}\label{eq:ham-jj}
    \hat H = 4E_C\hat n^2-E_J\cos\hat \phi \, ,
\end{equation}
che differisce dalla \eqref{eq:ham-oa-cQED} per la presenza di un potenziale periodico anarmonico. In Figura \ref{fig:jj-spectrum} si può osservare lo spettro energetico dell'hamiltoniana che descrive un circuito LC con giunzione Josephson. L'obiettivo è quello di utilizzare per la costruzione di un qubit i livelli energetici evidenziati in figura: il punto fondamentale è stato l'introduzione della giunzione Josephson, la quale non solo rende il sistema veramente quantistico, ma soprattutto introduce dell'anarmonicità nello spettro dell'oscillatore. 

\begin{figure}[!ht]
    \centering
    \includegraphics[scale=0.35]{images/jj-spectrum.jpg}
    \caption{Spettro energetico dell'hamiltoniana in \eqref{eq:ham-jj}. Per piccoli $\phi$ possiamo approssimare lo spettro come dei livelli energetici dell'oscillatore armonico più piccole correzioni anarmoniche.}
    \label{fig:jj-spectrum}
\end{figure}


\subsection{Interpretazione dell'effetto Josephson}
Per parlare in maniera del tutto completa ed esaustiva dell'effetto Josephson sarebbe necessaria una lezione completa, pertanto per approfondimenti e chiarimenti rimandiamo a corsi sulla superconduttività. 

\noindent L'idea che seguiamo è quella di procedere attraverso una trattazione schematica dell'effetto Josephson dovuta a Feynman. Il punto di partenza è la differenza nello spettro energetico che c'è tra un metallo normale, dove possiamo avere bande che sono riempite da un mare di elettroni (Figura \ref{subfig:metal-energy-spectrum}) e un superconduttore, in cui vi è una netta separazione tra lo stato fondamentale e gli stati eccitati (Figura \ref{subfig:superconductor-energy-spectrum}). 

\noindent Nel momento in cui andiamo a considerare una giunzione metallica come in basso in Figura \ref{subfig:metal-energy-spectrum}, costituita da due elettrodi ciascuno con il proprio mare di elettroni riempito fino all'energia di Fermi $\varepsilon_F$, e supponiamo che tra i due vi sia una differenza di potenziale $V$, allora possiamo osservare una corrente $I$ che scorre tra i due terminali dovuta al fatto che gli elettroni possano fare tunneling da un elettrodo a un altro. Questa corrente $I$ sarà proporzionale alla tensione applicata $V$ e il coefficiente di proporzionalità è dato dall'inverso della resistenza $R$ del sistema. Supponiamo invece di considerare due metalli superconduttivi, che indichiamo con (L) left e (R) right, ciascuno caratterizzato da un mass-gap $\Delta$: a basse temperature, dove la fisica della superconduttività è rilevante, quello che succede è che gli stati fondamentali di entrambi saranno occupati da un certo numero di coppie di Cooper $N_L$ e $N_R$; in tale situazione il tunneling è dovuto alle coppie di Cooper, da sinistra verso destra e viceversa.

\begin{figure}[!ht]
	\centering	
	\subfloat[][\label{subfig:metal-energy-spectrum}]{\includegraphics[scale=.3,keepaspectratio]{images/metal-energy-spectrum}} \qquad \qquad
	\subfloat[][\label{subfig:superconductor-energy-spectrum}]{\includegraphics[scale=.3,keepaspectratio]{images/superconductor-energy-spectrum}} \qquad
	\caption{(\ref{subfig:metal-energy-spectrum}) Livelli energetici di un metallo caratterizzato da un mare di elettroni e sistema di due elettrodi in cui avviene il fenomeno del tunneling di elettroni da un elettrodo all'altro. (\ref{subfig:superconductor-energy-spectrum}) Livelli energetici di un metallo superconduttivo caratterizzato, a basse temperature, da uno stato fondamentale in cui si trovano tutte le coppie di Cooper e i livelli eccitati separati da un mass-gap $\Delta$ e sistema di due elettrodi superconduttivi in cui avviene il fenomeno del tunneling di coppie di Cooper da un elettrodo all'altro.}
\end{figure}

\noindent Dal punto di vista della QM, se la temperatura è sufficientemente bassa a tal punto da non considerare gli stati eccitati, ciò che conta è il numero quantico che identifica quanti bosoni sono nello stato fondamentale. Nel nostro caso, il sistema dei due elettrodi superconduttivi è descritto da due interi $N_L$ e $N_R$ (numero di coppie di Cooper nello stato fondamentale presenti rispettivamente nell'elettrodo di sinistra e nell'elettrodo di destra). Passare da un elettrodo all'altro significa quindi andare a incrementare e diminuire il numero di bosoni di entrambi gli elettrodi: ad esempio
\begin{equation*}
    \substack{\text{Tunneling di} \\ \text{un bosone da} \\ \text{(L) a (R)}}: \quad \ket{N_L, N_R} \longrightarrow \ket{N_L-1, N_R+1}
\end{equation*}
Dal momento che il numero totale di bosoni dei due elettrodi è fissato, è sufficiente analizzare solamente uno di questi due numeri, ad esempio $N_L$ e lo andiamo a identificare con $m$ e il corrispondente stato con $\ket m$, cioè il numero di coppie di Cooper nel primo metallo. L'idea di Feynman è quella di andare a modellizzare il tunneling tramite un'hamiltoniana di interazione che non è altro che una matrice $2 \times 2$:
\begin{equation*}
    \hat H = -\frac{E_J}{2}\sum_m\big(\underbrace{\op{m}{m+1}}_{\substack{\text{Operatore che} \\ \text{riduce il numero} \\ \text{di coppie} \\ \text{di Cooper.}}}+\underbrace{\op{m+1}{m}}_{\substack{\text{Operatore che} \\ \text{incrementa il} \\ \text{numero di coppie} \\ \text{di Cooper.}}}\big) \, .
\end{equation*}
Questa hamiltoniana è caratterizzata da due operatori che non fanno altro che eseguire il tunneling di una coppia di Cooper da un metallo superconduttivo all'altro. Da questa definizione si possono ottenere vari risultati che ci limitiamo a citare, ma che possono essere verificati con semplici conti di QM:
\begin{enumerate}
    \item Gli autostati dell'hamiltoniana precedente sono del tipo
    \begin{equation*}
        \ket{\varphi}=\sum_{m=-\infty}^{+\infty}e^{im\varphi}\ket{m} \, , \quad \text{con} \quad \varphi \in [0,2\pi) \, ,
    \end{equation*}
    e gli autovalori sono dati dall'energia nella \eqref{eq:E_Josephson} 
    \begin{equation*}
        \hat H \ket{\varphi} = -E_J \cos\varphi\ket{\varphi} \, ,
    \end{equation*}
    dove si vede che il flusso ridotto $\phi$ pu\`o essere identificato con la fase $\varphi$ degli autostati.
    \item Definendo un operatore numero $\hat n$ che indica il numero di coppie di Cooper trasferite attraverso la giunzione
    \begin{equation*}
        \hat n = \sum_m m\op{m}{m} \, ,
    \end{equation*}
    cioè tale che
    \begin{equation*}
        \hat n \ket m = m \ket m \, ;
    \end{equation*}
    e definendo inoltre un operatore corrente $\hat{I}$ tale che
    \begin{equation*}
        \hat{I} = 2 e \dv{\hat{n}}{t} = \frac{2 i e}{\hbar} \comm{\hat{H}}{\hat{n}} =  -i\frac{e}{\hbar}E_J\sum_m \left( \op{m}{m+1}-\op{m+1}{m} \right) \, ,
    \end{equation*}
    allora è possibile ottenere la prima legge di Josephson della \eqref{1_Josephson_law}
    \begin{equation*}
        \hat I \ket{\varphi} = \underbrace{\frac{2eE_J}{\hbar}}_{I_C} \sin\varphi \ket\varphi = I_C \sin\varphi \ket{\varphi} \, .
    \end{equation*}
    
    
    \item Cosa succede quando si applica una differenza di potenziale tra gli elettrodi dei metalli? Pensiamo alla situazione in cui un campo elettrico esterno viene applicato e mantenuto in modo tale che vi sia una caduta di tensione fissa $V$ attraverso il tunnel della giunzione. Questo modifica l'hamiltoniana nel seguente modo
    \begin{equation*}
        \hat H \longrightarrow \hat H -(2e)V\hat n \, ;
    \end{equation*}
    si tratta di un esercizio di QM dimostrare che l'evoluzione temporale dello stato a seguito dell'equazione di Schr\"odinger non è altro che 
    \begin{equation*}
        \ket{\psi(t)}=\exp{\frac{i}{\hbar}E_J\int_0^t\dd{\tau}\cos\left(\varphi(0)+\frac{2e}{\hbar}V\tau\right)}\ket{\varphi_0+\frac{2e}{\hbar}Vt} \, .
    \end{equation*}
    Dunque partendo al tempo $t=0$ dallo stato $\ket{\psi(0)} = \ket{\varphi_0}$, allora al tempo $t$ si ottiene una fase globale e lo stato evolve linearmente nel tempo risultando shiftato di $\varphi_0 \to \varphi_0 + \frac{2 e V}{\hbar} t$. Questo significa che riotteniamo la seconda legge di Josephson della \eqref{2_Josephson_law} usando ancora $\varphi=\phi$
    \begin{equation*}
        \dv{\phi}{t} = \frac{2 e V}{\hbar} \, , \quad \Rightarrow \quad V = \frac{\hbar}{2 e} \dv{\phi}{t} \, . 
    \end{equation*}
\end{enumerate}

\noindent Per riassumere, ciò di cui abbiamo bisogno sono le seguenti informazioni: la natura quantistica del sistema è data dallo stato fondamentale, il quale è l'unico che importa nella superconduttività; la corrente che si origina è dovuta al tunneling delle coppie di Cooper, le quali sono frutto di una particolare condensazione di bosoni; l'oscillatore risultante è quantistico, ma non armonico perché l'hamiltoniana associata è data dalla relazione \eqref{eq:ham-jj}. 

\subsection{Transmon qubit}
Discutiamo come codificare un qubit in un circuito come quello di Figura \ref{subfig:JJ-circuit}. Tra tutte le varie tipologie di qubit superconduttivi, quello più popolare e largamente utilizzato è il regime del cosiddetto \textbf{transmon qubit}. Consideriamo l'hamiltoniana ottenuta in \eqref{eq:ham-jj} a cui aggiungiamo un termine costante $E_J$ (questo porta a una variazione nei livelli energetici, ma non sulla dinamica del sistema)
\begin{equation*}
    \hat H = 4E_C\hat n^2 + E_J(1-\cos\hat\phi) \, .
\end{equation*}

\noindent I transmon qubit si hanno nel limite in cui $E_J/E_C \gg 1$, ossia quando $\phi \sim 0$. Questo limite ci permette di sviluppare il potenziale attraverso un'espansione con il polinomio di Taylor
\begin{equation*}
    \hat V(\hat \phi)=E_J(1-\cos\hat \phi)=\frac{E_J}{2!}\hat \phi^2-\frac{E_J}{4!}\hat \phi^4 + \frac{E_J}{6!}\hat \phi^6 + \order{\hat\phi^8} \, ;
\end{equation*}
si noti che il primo termine genera il potenziale armonico. Per capire il peso relativo di ciascun termine è utile normalizzare il primo introducendo il cambio di variabili $\hat x=\sqrt{E_J}\hat \phi$:
\begin{equation*}
    \hat V(\hat \phi)= \frac{\hat x^2}{2}-\frac{\hat x^4}{4!E_J}+\frac{\hat x^6}{6!E_J^2} + \order{\hat{x}^8} \, ;
\end{equation*}
i vari termini sono soppressi da potenze di $E_J$. Se tronchiamo l'espansione al termine di ordine $\hat{x}^4$, l'hamiltoniana del transmon qubit risultante sarà
\begin{equation}\label{eq:ham-transmon}
    \hat H = \underbrace{4E_C\hat n^2 + \frac{E_J\hat \phi^2}{2}}_{\hat H_0}\underbrace{-\frac{E_J\hat\phi^4}{24}}_{\text{correzione}} \, ,
\end{equation}
dove si è indicato con $\hat{H}_0$ la corrispondente hamiltoniana di oscillatore armonico. In Figura \ref{fig:ho-transmon} sono riportate le differenze nello spettro di un oscillatore armonico (Figura \ref{subfig:ho-transmon1}) e un oscillatore anarmonico con \textbf{accoppiamento quartico} (Figura \ref{subfig:ho-transmon2}), dovuto al termine chiamato \textit{correzione}.

\begin{figure}[H]
	\centering	
	\subfloat[][\label{subfig:ho-transmon1}]{\includegraphics[scale=.8,keepaspectratio]{images/ho-transmon1.jpg}} \qquad \qquad
	\subfloat[][\label{subfig:ho-transmon2}]{\includegraphics[scale=.8,keepaspectratio]{images/ho-transmon2.jpg}}
	\caption{(\ref{subfig:ho-transmon1}) Spettro energetico del QHO, dove i livelli di energia sono equidistanti l'uno dall'altro. (\ref{subfig:ho-transmon2}) La giunzione Josephson rimodella il potenziale energetico quadratico (tratteggiato in rosso) in sinusoidale (blu fisso), il quale non presenta più livelli energetici equispaziati. Si noti che questo potenziale è una buona approssimazione solamente nel limite $\phi \simeq 0$.}
	\label{fig:ho-transmon}
\end{figure}

\noindent In questa nuova configurazione gli autovalori non sono più equidistanziati: in Figura \ref{fig:energy-difference-transmon} possiamo notare che gli stati $\ket 0$ e $\ket 1$ sono separati da un'energia $\hbar\omega$, mentre gli stati $\ket 1$ e $\ket 2$ da un'energia $\hbar\omega+\alpha$ con $\alpha<0$. (Si veda la discussione che segue per capire perché in figura è indicata la frequenza "$\tilde{\omega}$", non "$\omega$"). 

\begin{figure}[H]
	\centering	
	\includegraphics[scale=.45, keepaspectratio]{images/energy-difference-transmon.jpg}
	\caption{Differenza energetica tra i livelli $\ket 0$, $\ket 1$ e $\ket 2$.}
	\label{fig:energy-difference-transmon}
\end{figure}

\noindent Se $\hbar \omega$ è ragionevolmente grande possiamo pensare di utilizzare gli stati $\ket{0}$ e $\ket{1}$ per codificare il qubit e usare degli impulsi laser esterni di frequenza $\sim \omega$ per mandare solamente $\ket{0} \leftrightarrow \ket{1}$, dato che questi livelli non sono in risonanza con tutti gli altri. In particolare, nel caso dei transmon qubit, si lavora con $\omega \sim 3-4 \text{ GHz}$ mentre $\alpha \sim 300 \text{ MHz}$, quindi $E_J/E_C \sim 40$. Tuttavia, a volte, per misure di precisione, si necessita l'utilizzo del livello $\ket 2$, quindi non lo si tiene molto distante energeticamente dagli altri livelli.

\noindent Per vedere esplicitamente come avviene la codifica del qubit, usiamo la definizione di $\hat \phi$ e esplicitiamo gli oscillatori delle relazioni \eqref{Phi_Q_oscill}
\begin{align*}
    \hat \phi = \frac{2\pi}{\Phi_0}\Phi = \frac{2e}{\hbar}\Phi = \frac{2e}{\hbar}\sqrt{\frac{\hbar}{2C\omega}}\left(\hat a + \hat a^\dagger\right) = 2\left(\frac{E_C}{8E_J}\right)^{\frac 14}\left(\hat a + \hat a^\dagger\right) \, ;
\end{align*}
(abbiamo utilizzato le definizioni di $E_C$ e $E_J$). Inserendo questo risultato all'interno dell'hamiltoniana in \eqref{eq:ham-transmon} e esplicitando l'operatore $\hat{n}$ in termini di $\hat{Q}$ otteniamo
\begin{equation}\label{H_sviluppare_quartico}
    \hat H = \underbrace{\hbar\omega\left(\hat a^\dagger\hat a+\frac 12\right)}_{\hat{H}_0}-\frac{E_C}{12}\left(\hat a + \hat a^\dagger\right)^4 \, .
\end{equation}

\noindent Questa è un'hamiltoniana di un oscillatore anarmonico con potenziale quartico che non presenta alcuna soluzione analitica. Come al solito, per semplificare la situazione, consideriamo l'hamiltoniana libera, ci mettiamo in un sistema ruotato e teniamo solamente i termini dell'interazione che evolvono lentamente nel tempo. L'operatore che effettua la rotazione è quindi 
\begin{equation*}
    \hat U = e^{i\hat H_0t} \, ,
\end{equation*}
che applicato agli operatori di creazione e distruzione genera una fase dipendente dal tempo (si vedano le relazioni in \eqref{RWA_ops_trans})
\begin{equation*}
    \hat a \longrightarrow e^{-i\omega t}\hat a \qquad \hat a^\dagger \longrightarrow e^{i\omega t}\hat a^\dagger \, .
\end{equation*}
Questo significa che tutti i termini che hanno un differente numero di $\hat{a}$ e $\hat{a}^\dag$ in $\left(\hat a + \hat a^\dagger\right)^4$ sono mediamente nulli; soltanto i termini con un ugual numero sopravvivono, ovvero
\begin{equation*}
    \left(\hat a + \hat a^\dagger\right)^4 = \hat a\hat a\hat a^\dagger\hat a^\dagger+\hat a\hat a^\dagger\hat a\hat a^\dagger+\hat a\hat a^\dagger\hat a^\dagger\hat a+\hat a^\dagger\hat a^\dagger\hat a\hat a+\hat a^\dagger\hat a\hat a\hat a^\dagger+\hat a^\dagger\hat a\hat a^\dagger\hat a \, ;
\end{equation*}
ricordando che $\comm{\hat a}{\hat a^\dagger}=1$ possiamo portare tutti gli $\hat{a}^\dag$ sulla sinistra e scrivere i termini in \textit{normal ordering}
\begin{equation*}
    \left(\hat a + \hat a^\dagger\right)^4 = c_1\hat a^\dagger \hat a^\dagger\hat a \hat a + c_2\hat a^\dagger \hat a + c_3 \, ,
\end{equation*}
dove $c_1=6$, $c_2=12$ e $c_3=3$. Nella nostra trattazione non consideriamo $c_3$ perché comporta solamente una ridefinizione dell'energia di punto zero; il fatto che $c_2$ sia uguale a $12$, come vedremo ora, è importante. Tenendo conto di questo risultato e non considerando i termini costanti, l'hamiltoniana in \eqref{H_sviluppare_quartico} si scrive
\begin{align}
    \hat H &= \hbar\omega\hat a^\dagger \hat a - \frac{E_C}{2}\hat a^\dagger\hat a^\dagger \hat a \hat a - E_C\hat a^\dagger\hat a \notag \\
    \Rightarrow \quad \hat{H} &= \hbar\tilde{\omega}\hat a^\dagger \hat a + \frac{\alpha}{2}\hat a^\dagger \hat a^\dagger \hat a \hat a \, , \label{final_H_transmon}
\end{align}
dove abbiamo posto $\hbar\tilde{\omega}=\hbar\omega-E_C$ lo shift in frequenza del qubit e $\alpha = -E_C$ l'energia del termine quartico. Ricordando come al solito l'azione di $\hat{a}$ e $\hat{a}^\dag$ su $\ket{n}$ dalle relazioni in \eqref{a_adag_action_states}, questa nuova hamiltoniana presenta i seguenti livelli energetici
\begin{align*}
    \hat H \ket 0 &= 0 \, , \\
    \hat H \ket 1 &= \hbar\tilde{\omega} \ket{1} \, , \\
    \hat H \ket 2 &= \left( 2\hbar\tilde{\omega}+\alpha \right) \ket{2} \, ;
\end{align*}
è evidente quindi, come nella Figura \ref{fig:energy-difference-transmon}, che $E_1-E_0 = \hbar \tilde{\omega}$ e $E_2 - E_1 = \hbar \tilde{\omega} + \alpha$. Si noti, in particolare, che $\ket 2$ è ancora autostato, infatti
\begin{align*}
    \hat a^\dagger \hat a^\dagger \hat a \hat a\ket 2 &= \hat a^\dagger \hat a^\dagger \hat a \sqrt 2 \ket 1 = \hat a^\dagger \hat a^\dagger \sqrt 1\sqrt 2 \ket 0 \\
    &= \hat a^\dagger \sqrt 1 \sqrt 1 \sqrt 2 \ket 1 = \sqrt 2 \sqrt 1 \sqrt 1 \sqrt 2 \ket 2 = 2\ket 2 \, .
\end{align*}
Vediamo alcune grandezze tipiche: se il rapporto $E_J/E_C \sim 40$ e $\hbar\omega=\sqrt{8E_CE_J}\sim 10E_C$ allora la differenza energetica $\alpha$ è circa del $10$\%. 

\noindent Ricapitolando, per codificare il qubit l'idea è quella di considerare gli stati $\ket 0$ e $\ket 1$ e trascurare $\ket 2$ (se non utilizzarlo per le considerazioni viste precedentemente). Consideriamo i seguenti stati
\begin{equation*}
    \ket 0 = \begin{pmatrix}1 \\ 0\end{pmatrix} \qquad \ket 1 = \begin{pmatrix}0 \\ 1 \end{pmatrix} \, ,
\end{equation*}
allora in forma matriciale
\begin{equation*}
    \hat a^\dagger \hat a = \begin{pmatrix}0 & 0 \\ 0 & 1\end{pmatrix}=-\frac{\hat \sigma_3}{2}+\frac{\mathbb{I}}{2} \, ;
\end{equation*}
trascurando nuovamente il termine costante si ottiene l'hamiltoniana standard di un sistema a due livelli:
\begin{equation}\label{usual_H_qubit}
    \hat H = -\frac{\hbar}{2}\tilde{\omega}\hat \sigma_3 \, .
\end{equation}
Il termine quartico $\hat a^\dagger \hat a^\dagger \hat a \hat a$ è nullo su $\ket 0$ e $\ket 1$, ma è necessario tenerlo in considerazione quando si vuole includere correzioni dovute allo stato eccitato $\ket 2$.
In un contesto di oscillatore armonico lo spazio di Hilbert considera solamente
\begin{align*}
    \hat a \ket 0 &= 0 \quad &\hat a^\dagger\ket 0 &= \ket 1 \\
    \hat a \ket 1 &= \ket 0 \quad &\hat a^\dagger \ket 1 &= \sqrt 2 \ket 2 \overset{!}{\equiv}0 \, ,
\end{align*}
perché nell'ultimo caso abbiamo troncato lo spazio di Hilbert. Le definizioni matriciali degli operatori di creazione e distruzione possono essere facilmente dedotte dalla forma vettoriale di $\ket{0}$ e $\ket{1}$:
\begin{equation}\label{a_adag_matrices}
    \hat a =\begin{pmatrix}0 & 1 \\ 0 & 0\end{pmatrix} \equiv \hat \sigma_+ \, , \qquad \hat a^\dagger = \begin{pmatrix}0 & 0 \\ 1 & 0\end{pmatrix} \equiv \hat \sigma_-
\end{equation}
In maniera del tutto simile possiamo anche notare che 
\begin{equation}\label{sigma12_aadag}
    \hat \sigma_1=\left(\hat a + \hat a^\dagger\right) \, , \qquad \hat \sigma_2=-i\left(\hat a - \hat a^\dagger\right) \, .
\end{equation}

\begin{figure}[H]
    \centering
    \begin{circuitikz}
        \draw
        (0,0)   to[C=$C$] ++ (0, 3) -- ++ (4,0);
        \draw 
        (2,3)        to[josephson=$L$] ++ (0,-3) -- ++ (-2,0);
        \draw 
        (4,3)        to[josephson=$L$] ++ (0,-3) -- ++ (-2,0)
        (1,0)node[ground]{};
    \end{circuitikz}
    \caption{Transmon qubit caratterizzato da due giunzioni Josephson. La frequenza del qubit è regolata attraverso la variazione di un campo magnetico esterno $\phi_e$.}
    \label{fig:split-transmon-qubit}
\end{figure}

\noindent Precisiamo che vi sono altre tipologie di transmon qubit: ad esempio, molto diffusi, sono quelli in cui si aggiungono nel circuito di Figura \ref{subfig:JJ-circuit} due giunzioni Josephson in maniera tale che $\tilde{\omega}$ possa essere regolata sperimentalmente mediante una variazione di un flusso magnetico esterno $\phi_e$. Una rappresentazione schematica di questo tipo di qubit è data dalla Figura \ref{fig:split-transmon-qubit}.


    %%%%%%%%%%%%%%
% LECTURE 21 %
%%%%%%%%%%%%%%
\vspace{0.5cm}

\noindent \lecture{21}{20/12/2021}

\vspace{0.5cm}

\noindent Vediamo alcuni esempi di gate che possono essere realizzati a partire da sistemi superconduttivi. Prima di farlo, riassumiamo velocemente i risultati ottenuti nel corso di questa Sezione.

\noindent L'idea è quella di codificare un qubit nei livelli energetici $(\ket 0, \ket{1})$ di un circuito superconduttivo che presenta una giunzione Josephson (Figura \ref{subfig:JJ-circuit}): quest'ultima permette di dare una descrizione quantistica al sistema e introduce dell'anarmonicità nello spettro energetico (vedi Figura \ref{fig:energy-difference-transmon}); l'hamiltoniana del sistema risultante è riportata nella \eqref{final_H_transmon}. Trascurando il livello energetico $\ket{2}$ ci si riduce alla nota all'hamiltoniana in \eqref{usual_H_qubit}, tipica di un sistema a due livelli; la rappresentazione matriciale degli operatori sugli stati del sistema è riportata nelle relazioni in \eqref{a_adag_matrices} e \eqref{sigma12_aadag}. 

\noindent Vediamo come possiamo esplicitamente realizzare su un sistema superconduttivo le tipiche operazioni di un qubit: \textit{misura del sistema}, \textit{codifica di gate agenti su singoli qubit con le oscillazioni di Rabi} e \textit{codifica di gate agenti su più qubit}. 

\noindent Per effettuare questo tipo di operazioni è necessario in generale applicare al circuito dei segnali esterni $V(t)$ (tipicamente dell'ordine del GHz) e dei risonatori del tipo CQED, come nel circuito seguente
\begin{figure}[H]
    \centering
    \begin{circuitikz}
        \draw
        (0,0)   to[C=$C$] ++ (0, 2) -- ++ ( 2,0) 
                to[josephson=$L$] ++ (0,-2) -- ++ (-2,0)
        %
        (0.5,2) |- ++ (-1.5,0.5) node[left, draw] {V(t)}
        (1.5,2) |- ++ ( 1.5,0.5) node[right,draw, align=center] {Cavity/Resonator}
        (1,0)node[ground]{};
    \end{circuitikz}
\end{figure}
\subsubsection{Misura su un qubit}
\noindent Supponiamo ad esempio di partire con un transmon qubit di frequenza $\omega_q$: una tipica misurazione viene effettuata con un risonatore di frequenza $\omega_r$ in un apparato della forma seguente
\begin{figure}[H]
    \centering
    \begin{circuitikz}
        \draw
        (0,0)   to[C=$C$] ++ (0, 2) -- ++ ( 2,0) 
                to[josephson=$L$] ++ (0,-2) -- ++ (-2,0)
        %
        (1,2) |- ++ (1,0.5) to[C=$C$] ++ (2,0)
        (0,2) ++ (4,0.5) node[right,draw, align=center] {Cavity/Resonator}
        (1,0)node[ground]{};
    \end{circuitikz}
\end{figure}

\noindent Dato che la fisica del qubit risulta in questo caso accoppiata alla CQED, sappiamo che l'hamiltoniana del sistema risultante è descritta dall'hamiltoniana di Jaynes-Cummings in \eqref{eq:ham-jaynes-cummings}
\begin{equation*}
    \hat{H} = - \frac{\omega_q}{2} \hat{\sigma}_3 + \omega_r \left( \hat{a}^\dag_r \hat{a}_r + \frac{1}{2} \right) + g \left( \hat{a}_r^\dag \hat{\sigma}_+ + \hat{a} \hat{\sigma}_- \right) \, ;
\end{equation*}
come già visto nella Sezione \ref{sec:int_qubit_CQED}, la misura può essere effettuata nel regime dispersivo dove $\frac{g}{\Delta} \ll 1$ ($\Delta = \omega_q - \omega_r$): in tal caso l'hamiltoniana precedente viene modificata in
\begin{equation}\label{H_measure_transmon}
    \hat{H} = -\frac{\tilde{\omega}_q}{2} \hat{\sigma}_3 + \left( \omega_r - \chi \hat{\sigma}_3 \right) \hat{a}^\dag_r \hat{a}_r \, ,
\end{equation}
dove la frequenza del qubit è shiftata in $\tilde{\omega}_q$ a seguito del \textbf{Lamb shift} e $\chi = \frac{g^2}{\Delta}$. Cosa succede nel caso di un qubit superconduttivo? Nell'approssimazione che abbiamo considerato si trascurava il livello energetico $\ket{2}$: dato che il suo peso relativo rispetto a $\hbar \tilde{\omega}_q$ è del 10\% circa, per svolgere una trattazione più precisa e completa è necessario considerare l'intera hamiltoniana in \eqref{final_H_transmon} accoppiata alla CQED. L'inclusione di queste correzioni si traduce nella seguente modifica di $\chi$
\begin{equation*}
    \chi = \frac{g^2}{\Delta \left( 1 + \frac{\Delta}{\alpha} \right)} \, ;
\end{equation*}
dato che vi è un esplicita dipendenza da $\alpha$, se quest'ultimo non è grande a tal punto da rendere la correzione trascurabile allora è necessario includerlo nella trattazione. Dalla \eqref{H_measure_transmon} è evidente che la frequenza della cavità viene modificata dallo stato in cui si trova il qubit! Per misurare il qubit basta quindi misurare la frequenza del risonatore. 


\subsubsection{Controllare lo stato di un singolo qubit}
Per ottenere questo scopo è necessario accoppiare il qubit ad un campo elettromagnetico esterno oscillante nel tempo e "guidarlo" mediante le oscillazioni di Rabi (si veda la Sezione \ref{sec:qubit_campo_em_classico}). Esistono diversi modi: ad esempio si può costruire il circuito 
\begin{figure}[H]
    \centering
    \begin{circuitikz}
        \draw
        (0,0)   to[C=$C$] ++ (0, 2) -- ++ ( 2,0) 
                to[josephson=$L$] ++ (0,-2) -- ++ (-2,0)
        %
        (1,2) |- ++ (1,0.5) to[C=$C_g$] ++ (2,0)
        (0,2) ++ (4,0.5) node[right,draw, align=center] {$V(t)$}
        (1,0)node[ground]{};
    \end{circuitikz}
\end{figure}
\noindent dove la sorgente $V(t)$ fornisce un segnale e.m.  oscillante dipendente dal tempo e con frequenza circa uguale a quella del qubit. Come funziona questo apparato? I gradi di libertà del circuito (carica, corrente, ecc.) sono legati al comportamento dell'oscillatore (vedi Tabella \ref{tab:oa-lc}), il quale è controllato dagli operatori $\hat{a}$, $\hat{a}^\dag$ e $\hat{\sigma}_i$. Questo significa che l'energia del circuito è regolata dalle differenze di potenziale delle due differenti componenti del circuito: $E_C = \frac{1}{2} C V^2 \to C_g V_1 V_2$ dove $V_1 = \frac{Q}{C}$ e $V_2 = V(t)$; ma allora
\begin{equation*}
    E_C = \frac{C_g}{C} Q V(t) \, ;
\end{equation*}
dal punto di vista del qubit le identificazioni da fare sono
\begin{equation}\label{ID_xp_PhiQ}
\begin{aligned}
    \hat{x} &\longleftrightarrow \hat{\Phi} \sim \left( \hat{a} + \hat{a}^\dag \right) \sim \hat{\sigma}_1 \, , \\
    \hat{p} &\longleftrightarrow \hat{Q} \sim -i \left( \hat{a} - \hat{a}^\dag \right) \sim \hat{\sigma}_2 \, ;
\end{aligned}
\end{equation}
quindi l'hamiltoniana del sistema diventa 
\begin{equation*}
    \hat{H} = -\frac{\omega_q}{2} \hat{\sigma}_3 - i g \left( \hat{a} - \hat{a}^\dag \right) V(t) = -\frac{\omega_q}{2} \hat{\sigma}_3 +  g \hat{\sigma}_2 V(t) \, ,
\end{equation*}
dove il potenziale oscillante controlla il qubit con una frequenza esterna $\omega_d$, ed è quindi della forma
\begin{equation*}
    V(t) = V_0 \cos \! \left( \omega_d t + \phi_0 \right) \, .
\end{equation*}
Questo non è altro che il problema delle oscillazioni di Rabi affrontato nella Sezione \ref{sec:qubit_campo_em_classico}. Come ben sappiamo, dopo la rotazione l'hamiltoniana è scrivibile come
\begin{equation*}
    \hat{\tilde{H}} = -\frac{1}{2} \left( \Delta \hat{\sigma}_3 + A \cos(\phi) \hat{\sigma}_1 - A \sin (\phi) \hat{\sigma}_2 \right) \, ,
\end{equation*}
dove $\Delta = \omega_q - \omega_d$ e $\phi = \phi_0 + \phi_1$. Se poniamo $A e^{-i \phi_1} \equiv - i g V_0$ allora otteniamo che $A = g V_0$ e $\phi_1 = \frac{\pi}{2}$. Con una tale hamiltoniana possiamo agire con l'evoluzione temporale
\begin{equation*}
    \hat{U}_{\text{ev}}(t) = e^{-i \hat{\tilde{H}} t} \, ;
\end{equation*}
la miglior situazione possibile accade in corrispondenza della risonanza ($\Delta = 0$): scegliendo $\phi = 0$ sopravvive solamente il contributo di $\hat{\sigma}_1$, perciò l'evoluzione temporale non è altro che una tipica trasformazione agente sulla sfera di Bloch
\begin{equation*}
    \hat{U}_{\text{ev}}(t) = e^{\frac{i}{2} A \hat{\sigma}_1 t} = R_x(-At) \, ;
\end{equation*}
similmente per $\phi = \frac{\pi}{2}$
\begin{equation*}
    \hat{U}_{\text{ev}}(t) = e^{-\frac{i}{2} A \hat{\sigma}_2 t} = R_y(At) \, .
\end{equation*}
Scegliendo opportunamente $t$ è quindi possibile ottenere qualsiasi matrice di $SU(2)$, ossia qualsiasi gate agente su singoli qubit. 

\noindent In realtà bisogna prestare attenzione: come già mostrato nella Figura \ref{subfig:Bloch_Rabi2}, partendo da $\ket{0}$, se si vuole avere la certezza che ad un certo istante ci si ritroverà in $\ket{1}$ è necessario effettuare una rotazione lungo l'asse $x$; in realtà questo è vero solamente se l'hamiltoniana del sistema superconduttivo non è approssimata e considera anche lo stato $\ket{2}$. Il motivo è dato dal fatto che nelle realizzazioni pratiche la differenza tra $E_2 - E_1$ non è così più piccola rispetto a $E_1-E_0$! Questo vuol dire che sperimentalmente è necessario ricalcolare l'oscillazione di Rabi includendo l'effetto di $\ket{2}$ e in seguito rimpiazzare l'ampiezza $A$ della perturbazione con un'opportuna funzione dipendente dal tempo scelta appositamente per avere la certezza di raggiungere lo stato $\ket{1}$. 


\subsubsection{Creare entanglement tra due qubit}
Nella Sezione \ref{sec:gate_properties} abbiamo visto che è necessario possedere almeno un gate agente su due qubit (che produce entanglement) per poter costruire un qualsiasi gate. Utilizzare un \texttt{CNOT-gate} sarebbe perfetto, tuttavia non è l'unica scelta possibile perché, come vedremo tra poco, utilizzando un sistema superconduttivo è possibile realizzare un cosiddetto \texttt{iSWAP-gate}. 

\noindent Innanzitutto, per mettere in contatto tra loro differenti qubit superconduttivi esistono diversi modi:
\begin{itemize}
    \item Il primo modo è quello di combinare due circuiti superconduttivi con un extra capacità $C_g$, come ad esempio nel circuito
    \begin{figure}[H]
        \centering
        \begin{circuitikz}
            \draw
            (0,0)   to[C=$C_1$] ++ (0, 2) -\- ++ ( 2,0) 
                    to[josephson=$L$] ++ (0,-2) -- ++ (-2,0)
            %
            (1,2) |- ++ (1,0.5) to[C=$C_g$] ++ (2,0) -- ++ (1,0) -- ++ (0,-0.5)
            (4,0)   to[C=$C_2$] ++ (0, 2) -- ++ ( 2,0) 
                    to[josephson=$L$] ++ (0,-2) -- ++ (-2,0)
            (1,0)node[ground]{}
            (5,0)node[ground]{};
        \end{circuitikz}
    \end{figure}
    
    \noindent In tal caso notiamo che le due componenti del circuito presentano due capacità $C_1$ e $C_2$: utilizzando l'identificazione in \eqref{ID_xp_PhiQ} e gli operatori in \eqref{sigma12_aadag}, possiamo scrivere l'hamiltoniana di un tale sistema come
    \begin{equation}\label{C_coupling}
        \hat{H}_{\text{int}} = C_g V_1 V_2 = \frac{C_g}{C_1 C_2} \hat{Q}_1 \hat{Q}_2 \sim \left( \hat{a}_1 - \hat{a}^\dag_1 \right) \left( \hat{a}_2 - \hat{a}^\dag_2 \right) \sim \hat{\sigma}_2^{(1)} \otimes \hat{\sigma}_2^{(2)} \, ,
    \end{equation}
    dove gli apici "$^{(j)}$" indicano il qubit $j$-esimo. 
    
    \item Nel secondo caso possiamo combinare due circuiti tramite l'induttanza, ossia utilizzando due giunzioni Josephson tali che si possano scambiare la corrente prodotta dalle coppie di Cooper:
    \begin{figure}[H]
    \centering
    \begin{circuitikz}
        \draw
        (0,0)   to[C=$C$] ++ (0, 2) -- ++ ( 2,0) 
                to[josephson=$L_1$] ++ (0,-2) -- ++ (-2,0)
        %
        (4,0)   to[josephson=$L_2$] ++ (0, 2) -- ++ ( 2,0) 
                to[C=$C$] ++ (0,-2) -- ++ (-2,0)
        (1,0)node[ground]{}
        (5,0)node[ground]{};
    \end{circuitikz}
    \end{figure}
    
    \noindent È un caso molto simile al precedente in cui bisogna tenere in considerazione il fatto che questa volta l'hamiltoniana si scrive a partire dalle correnti nei due circuiti:
    \begin{equation*}
        \hat{H}_{\text{int}} = M_{12} I_1 I_2 = \frac{M_{12}}{L_1 L_2} \hat{\Phi}_1 \hat{\Phi}_2 \sim \left( \hat{a}_1 + \hat{a}^\dag_1 \right) \left( \hat{a}_2 + \hat{a}^\dag_2 \right) \sim \hat{\sigma}_1^{(1)} \otimes \hat{\sigma}_1^{(2)} \, ,
    \end{equation*}
    dove $M_{12}$ è il coefficiente di mutua induzione. 
    
    \item L'ultima opzione è quella di mediare l'interazione dei circuiti con una cavità/risonatore:
    \begin{figure}[H]
    \centering
    \begin{circuitikz}
        \draw
        (0,0)   to[C=$C$] ++ (0, 2) -- ++ ( 2,0) 
                to[josephson=$L$] ++ (0,-2) -- ++ (-2,0)
        %
        (1,2) |- ++ (1,0.5) node[right,draw, align=center] {Resonator} (4.10,2.5) -- ++ (1.07,0) -- ++ (0,-0.5)
        (4.15,0)   to[C=$C$] ++ (0, 2) -- ++ (2,0) 
                to[josephson=$L$] ++ (0,-2) -- ++ (-2,0)
        (1,0)node[ground]{}
        (5.15,0)node[ground]{};
    \end{circuitikz}
    \end{figure}
    
    \noindent A differenza dei casi precedenti, questa situazione è più complessa perché l'accoppiamento coinvolge un'hamiltoniana di Jaynes-Cummings (i pedici "$_r$" stanno per "risonatore")
    \begin{equation*}
        \hat{H}_{\text{int}} = \omega_r \left( \hat{a}_r^\dag \hat{a}_r + \frac{1}{2} \right) + g_1 \left( \hat{a}_r^\dag \hat{\sigma}_+^{(1)} + \hat{a}_r \hat{\sigma}_-^{(1)} \right) + g_2 \left( \hat{a}_r^\dag \hat{\sigma}_+^{(2)} + \hat{a}_r \hat{\sigma}_-^{(2)} \right) \, ;
    \end{equation*}
    si noti che i due qubit interagiscono l'uno con l'altro per mezzo dei modi di oscillazione nel risonatore.
\end{itemize}

\noindent I diversi gate che si trovano in letteratura possono essere costruiti con delle opportune combinazioni dei circuiti precedenti. Discutiamo brevemente i casi più semplici, ossia le situazioni rappresentate nei primi due circuiti: gli operatori $\hat{\sigma}_1$ e $\hat{\sigma}_2$ sono detti \textbf{trasversi} perché al contrario $\hat{\sigma}_3$ è utilizzato per fissare la frequenza del qubit. 

\noindent Consideriamo il circuito del primo caso (accoppiamento dato dalla capacità) e vediamo come può essere esplicitamente costruito un \texttt{iSWAP-gate} (la $i$ perché l'output è moltiplicato per l'unità immaginaria). Chiamiamo $\omega_q^{(1)}$ e $\omega_q^{(2)}$ le frequenze proprie dei due qubit: assumiamo che possiamo arbitrariamente variare queste frequenze usando transmon a frequenza regolabile\footnote{Si ricordi che questo scopo può essere raggiunto grazie all'inserimento di un'ulteriore giunzione Josephson (vedi Figura \ref{fig:split-transmon-qubit}) e alla variazione di un opportuno flusso magnetico esterno.}. Innanzitutto è bene notare che dobbiamo stare attenti alla forma dell'interazione in \eqref{C_coupling} perché abbiamo omesso numerosi termini: infatti in un sistema di riferimento ruotato mediante l'operatore $\hat{U}(t) = e^{i \hat{H}_0 t}$, sono gli operatori $\hat{\sigma}_{\pm}^{(j)}$ a trasformare prendendo una singola "fase", e non $\hat{\sigma}_{1,2}^{(j)}$:
\begin{equation*}
    \hat{\sigma}_{\pm}^{(j)} \longrightarrow \hat{\sigma}_{\pm}^{(j)} e^{\mp i \omega_q^{(j)} t} \, ;
\end{equation*}
scriviamo quindi esplicitamente l'interazione \eqref{C_coupling} in un frame ruotato:
\begin{align*}
    \hat{\sigma}_2^{(1)} \otimes \hat{\sigma}_2^{(2)} &= - \left( \hat{\sigma}_+^{(1)} - \hat{\sigma}_-^{(1)} \right) \otimes \left( \hat{\sigma}_+^{(2)} - \hat{\sigma}_-^{(2)} \right) \\
    &\overset{\hat{U}(t)}{\longrightarrow} \hat{\sigma}_+^{(1)} \hat{\sigma}_-^{(2)} e^{-i \left( \omega_q^{(1)} - \omega_q^{(2)} \right) t} + \hat{\sigma}_-^{(1)} \hat{\sigma}_+^{(2)} e^{i \left( \omega_q^{(1)} - \omega_q^{(2)} \right) t} - \\
    &\qquad - \hat{\sigma}_+^{(1)} \hat{\sigma}_+^{(2)} e^{-i \left( \omega_q^{(1)} + \omega_q^{(2)} \right) t} - \hat{\sigma}_-^{(1)} \hat{\sigma}_-^{(2)} e^{i \left( \omega_q^{(1)} + \omega_q^{(2)} \right) t} \, ;
\end{align*}
di solito nelle applicazioni sperimentali si preferisce lavorare in regime di risonanza dei due circuiti, quindi assumiamo che $\omega_q^{(1)} = \omega_q^{(2)}$ e utilizziamo la RWA:
\begin{equation*}
    \hat{\sigma}_2^{(1)} \otimes \hat{\sigma}_2^{(2)} \overset{\hat{U}(t)}{\longrightarrow} \hat{\sigma}_+^{(1)} \hat{\sigma}_-^{(2)} + \hat{\sigma}_-^{(1)} \hat{\sigma}_+^{(2)} \, ;
\end{equation*}
quest'ultima è proprio l'interazione che darà origine al gate, detta \textbf{accoppiamento trasverso in RWA}. Con questo accoppiamento possiamo scrivere, nel frame ruotato, la seguente hamiltoniana di interazione efficace 
\begin{equation*}
    \hat{\tilde{H}}_{\text{int}} = g \left( \hat{\sigma}_+^{(1)} \hat{\sigma}_-^{(2)} + \hat{\sigma}_-^{(1)} \hat{\sigma}_+^{(2)} \right) = g
    \begin{pmatrix}
        0 & 0 & 0 & 0 \\
        0 & 0 & 1 & 0 \\
        0 & 1 & 0 & 0 \\
        0 & 0 & 0 & 0
    \end{pmatrix} \, ,
\end{equation*}
dove nell'ultimo passaggio abbiamo fatto uso del prodotto di Kronecker della definizione \ref{def:Kronecker} (è omesso il simbolo "$\otimes$" tra le due matrici dei due termini in parentesi). Il blocco centrale non è altro che la matrice $\sigma_1$, ricordando quindi che 
\begin{equation}\label{matr_sigma1}
    e^{-i g t \sigma_1} = R_x(2 g t) = \cos (gt)  - i \sigma_1 \sin (gt) =
    \begin{pmatrix}
        \cos (gt) & - i \sin (gt) \\
        - i \sin (gt) & \cos (gt)
    \end{pmatrix}
     \, ,
\end{equation}
possiamo facilmente scrivere l'operatore di evoluzione temporale
\begin{equation*}
    \hat{U}_{\text{ev}}(t) = e^{-i \hat{\tilde{H}}_\text{int} t} = 
    \begin{pmatrix}
        1 & 0 & 0 & 0 \\
        0 & \cos(gt) & -i\sin(gt) & 0 \\
        0 & -i\sin(gt) & \cos(gt) & 0 \\
        0 & 0 & 0 & 1
    \end{pmatrix} 
    \begin{matrix}
        \ket{00} \\ \ket{01} \\ \ket{10} \\ \ket{11}
    \end{matrix} \, .
\end{equation*}
Perciò l'evoluzione temporale del sistema ruota automaticamente gli stati $\ket{01}$ e $\ket{10}$. Scegliendo il tempo $t = \frac{3 \pi}{2 g}$ otteniamo esattamente un \texttt{iSWAP-gate}
\begin{equation*}
    \hat{U}_{\text{ev}} \! \left( \frac{3 \pi}{2 g} \right) = \begin{pmatrix}
        1 & 0 & 0 & 0 \\
        0 & 0 & i & 0 \\
        0 & i & 0 & 0 \\
        0 & 0 & 0 & 1
    \end{pmatrix} \equiv \texttt{iSWAP} \, .
\end{equation*}

\noindent Come accennato sopra, è possibile dimostrare che l'\texttt{iSWAP-gate} e i gate agenti sui singoli qubit formano un insieme universale di gate: la seguente combinazione permette infatti di ricostruire, a meno di una fase globale, un \texttt{CNOT-gate}
\begin{center}
    \mbox{
        \Qcircuit @C=1em @R=2.2em {
            & \ctrl{1} & \qw \\
            & \targ & \qw
        }
    }
    \raisebox{-1.6em}{$=e^{i\frac \pi 4}$}
    \mbox{
        \Qcircuit @C=1em @R=1em {
            & \gate{R_z(-\frac \pi 2)} & \qw & \multigate{1}{\texttt{iSWAP}} & \gate{R_x(\frac \pi 2)} & \multigate{1}{\texttt{iSWAP}} & \qw & \qw\\
            & \gate{R_x(\frac \pi 2)} & \gate{R_z(\frac \pi 2)} & \ghost{\texttt{iSWAP}} & \qw & \ghost{\texttt{iSWAP}} & \gate{R_z(\frac \pi 2)} & \qw
        }
    }
\end{center}
È importante sottolineare che per la presenza di due \texttt{iSWAP-gate}, un \texttt{CNOT-gate} così costruito non risulta in realtà molto efficiente: si preferirebbe lavorare con un singolo \texttt{iSWAP-gate}. 

\noindent Per costruire gate agenti su più qubit in maniera più efficiente si pu\`o anche  sfruttare il fatto che esistano degli stati eccitati nel transmon: una volta costruito un qubit superconduttivo con frequenza regolabile e una volta identificati i livelli energetici $(\ket{0}, \ket{1}, \ket{2})$, è possibile dimostrare che esiste una situazione in cui gli stati eccitati $(\ket{11}, \ket{20})$ sono indistinguibili (coincidenti) per un opportuno valore di $\Delta = \omega_q^{(1)} - \omega_q^{(2)}$ (per $\Delta = 0$ sono distinguibili). Una volta trovato tale valore si possono effettuare delle oscillazioni di Rabi per mandare $\ket{11} \to -\ket{11}$ e $\ket{20} \to \ket{20}$.  

\noindent Nella situazione appena analizzata abbiamo impiegato un accoppiamento trasverso della forma in \eqref{C_coupling}, la cui evoluzione temporale produceva un \texttt{iSWAP-gate}. In realtà sarebbe meglio se potessimo usare un'interazione della forma
\begin{equation}\label{H_for_CRgate}
    \hat{\tilde{H}}_{\text{int}} = \frac{g}{2} \hat{\sigma}_3^{(1)} \otimes \hat{\sigma}_1^{(2)} = \frac{g}{2}
    \begin{pmatrix}
        \sigma_1 & 0 \\ 0 & -\sigma_1
    \end{pmatrix}
    = \frac{g}{2}
    \begin{pmatrix}
        0 & 1 & 0 & 0 \\
        1 & 0 & 0 & 0 \\
        0 & 0 & 0 & -1 \\
        0 & 0 & -1 & 0
    \end{pmatrix}
    \, ;
\end{equation}
il motivo è dato dal fatto che la sua evoluzione temporale produce il cosiddetto \textbf{Cross-Resonant gate} (\texttt{CR-gate})
\begin{equation*}
    \hat{U}_{\text{ev}}(t) = e^{-i \hat{\tilde{H}}_\text{int} t} = 
    \begin{pmatrix}
        \cos \! \left( \frac{gt}{2} \right) & -i \sin \! \left( \frac{gt}{2} \right) & 0 & 0 \\
        -i \sin \! \left( \frac{gt}{2} \right) & \cos \! \left( \frac{gt}{2} \right) & 0 & 0 \\
        0 & 0 & \cos \! \left( \frac{gt}{2} \right) & i \sin \! \left( \frac{gt}{2} \right) \\
        0 & 0 & i \sin \! \left( \frac{gt}{2} \right) & \cos \! \left( \frac{gt}{2} \right)
    \end{pmatrix} \equiv \texttt{CR-gate} \, ,
\end{equation*}
dove abbiamo fatto uso due volte della \eqref{matr_sigma1} nell'esponenziazione. Il motivo per cui tale gate è in un certo senso migliore dell'\texttt{iSWAP-gate} è dato dal fatto che permetta di costruire un \texttt{CNOT-gate} più efficientemente mediante un solo impiego:
\begin{center}
    \mbox{
        \Qcircuit @C=1em @R=2.2em {
            & \ctrl{1} & \qw \\
            & \targ & \qw
        }
    }
    \raisebox{-1.6em}{$=e^{i\frac \pi 4}$}
    \mbox{
        \Qcircuit @C=1em @R=1em {
            & \gate{R_z(\frac \pi 2)} & \multigate{1}{\text{CR}(t=-\frac \pi 2)} & \qw\\
            & \gate{R_z(\frac \pi 2)} & \ghost{\text{CR}(t=-\frac \pi 2)} & \qw
        }
    }
\end{center}

\noindent Chiaramente può sorgere spontanea la domanda: come si realizza un'interazione della forma in \eqref{H_for_CRgate}? Si tratta di un modo con cui IBM costruisce i propri computer quantistici superconduttivi. Questo può essere fatto mediante il seguente trucco: si realizza un circuito 
\begin{figure}[H]
    \centering
    \begin{circuitikz}
        \draw
        (0,0)   to[C=$C$] ++ (0, 2) -- ++ ( 2,0) 
                to[josephson=$L$] ++ (0,-2) -- ++ (-2,0)
        %
        (1,2) |- ++ (1,0.5) to[C=$C_g$] ++ (2,0) -- ++ (1,0) -- ++ (0,-0.5)
        (0,0.4) -- ++ (-1,0) node[left,draw, align=center] {$V_1(t)$}
        (4,0)   to[C=$C$] ++ (0, 2) -- ++ ( 2,0) 
                to[josephson=$L$] ++ (0,-2) -- ++ (-2,0)
        (1,0)node[ground]{}
        (5,0)node[ground]{};
    \end{circuitikz}
\end{figure}

\noindent dove si regola la frequenza di drive del primo circuito (mediante $V_1(t)$) utilizzando esattamente la frequenza propria del secondo qubit, ossia $\omega_d = \omega_q^{(2)}$. In una tale situazione è possibile dimostrare che per $\Delta = \omega_q^{(1)} - \omega_q^{(2)} \gg g$ l'unico contributo che sopravvive nell'interazione è della forma desiderata, ossia come in \eqref{H_for_CRgate}.
    \begin{thebibliography}{12}
\section*{Corso}
\bibitem{zaffaroni}
Zaffaroni, A. (2021). \href{https://elearning.unimib.it/course/view.php?id=39146}{Teoria della informazione e della computazione quantistica}. Università degli Studi di Milano - Bicocca.
\section*{Libri}
\bibitem{nielsen}
Nielsen, M. A., Chuang, I. L. (2000). Quantum Computation and Quantum Information. Cambridge University Press.
\bibitem{mermin}
Mermin, N. (2007). Quantum Computer Science: An Introduction. Cambridge University Press.
\section*{Lezioni}
\bibitem{scottaaronson}
Aaronson, S. (2018). \href{https://www.scottaaronson.com/qclec.pdf}{Introduction to Quantum Information Science Lecture Notes}. University of Texas at Austin.
\bibitem{preskill}
Preskill, J. (1997-2020). \href{http://theory.caltech.edu/~preskill/ph219/}{Course Information for Physics and Computer Science on Quantum Computation}. California Institute of Technology.
\section*{Articoli}
\bibitem{surface-code}
Fowler, A., Mariantoni, M., Martinis, J., \& Cleland, A. (2012). Surface codes: Towards practical large-scale quantum computation. Physical Review A, 86(3).
\bibitem{ion-trap1}
Haffner, H., Roos, C., \& Blatt, R. (2008). Quantum computing with trapped ions. Physics Reports, 469(4), 155–203.
\bibitem{ion-trap2}
Bruzewicz, C., Chiaverini, J., McConnell, R., \& Sage, J. (2019). Trapped-ion quantum computing: Progress and challenges. Applied Physics Reviews, 6(2), 021314.
\bibitem{ion-trap3}
Sørensen, A., \& Mølmer, K. (2000). Entanglement and quantum computation with ions in thermal motion. Physical Review A, 62(2).
\bibitem{ion-trap4}
Mølmer, K., \& Sørensen, A. (1999). Multiparticle Entanglement of Hot Trapped Ions. Physical Review Letters, 82(9), 1835–1838.
\bibitem{sup-qubit1}
Krantz, P., Kjaergaard, M., Yan, F., Orlando, T., Gustavsson, S., \& Oliver, W. (2019). A quantum engineer’s guide to superconducting qubits. Applied Physics Reviews, 6(2), 021318.
\bibitem{sup-qubit2}
Kwon, S., Tomonaga, A., Lakshmi Bhai, G., Devitt, S., \& Tsai, J.S. (2021). Gate-based superconducting quantum computing. Journal of Applied Physics, 129(4), 041102.
\bibitem{sup-qubit3}
Girvin, S. M. (2012). Circuit QED: Superconducting Qubits Coupled to Microwave Photons. Oxford University Press.
\end{thebibliography}
    \addcontentsline{toc}{chapter}{Bibliografia}
\end{document}
