%%%%%%%%%%%%%
% LECTURE 6 %
%%%%%%%%%%%%%
\vspace{1cm}

\noindent \lecture{6}{22/10/2021}

\section{Quantum Parallelism}

\begin{definizione}[\textbf{Quantum Parallelism}]
    Il \textbf{quantum parallelism} è una delle caratteristiche fondamentali di molti algoritmi quantistici. Consente ai computer quantistici di valutare una funzione $f(x)$ per molti valori diversi di $x$ contemporaneamente.
\end{definizione}
\noindent Supponiamo di considerare la più semplice funzione possibile $f(x):\{0,1\}^{\otimes n} \rightarrow \{0,1\}$ definita su un dominio (insieme di numeri costruiti con $n$ cifre di 0 e 1) e a elementi in un intervallo di bit. Assumiamo inoltre di saper calcolare efficientemente nel nostro computer tale funzione. Ciò che calcoliamo, dal punto di vista della computazione classica, lo possiamo valutare nella computazione quantistica, pertanto tutte le operazioni aritmetiche possono essere svolte dal calcolo quantistico. Un modo quindi di calcolare questa funzione su un computer quantistico è quello di considerare due differenti stati: immaginiamo un qubit $\ket{y}$ e uno stato che può essere un prodotto tensoriale di qubit, come ad esempio $\ket{0}^{\otimes n}$. Spesso considereremo lo stato $\ket{0}^{\otimes n}$ come stato iniziale in cui il computer quantistico viene preparato mediante una misurazione nella base computazionale perché è facilmente costruibile: ad esempio nel caso $n=3$ se, a seguito di una misurazione, lo stato nel QC collassa in $\ket{\psi} \rightarrow \ket{1} \otimes \ket{0} \otimes \ket{1}$, basterà applicare un \texttt{X-gate} al primo e al terzo qubit per costruire lo stato voluto $\ket{0}^{\otimes 3}$.

\noindent Chiamiamo lo stato iniziale totale $\ket{x,y}$, dove $x$ contiene l'informazione iniziale data in input e $y$ conterrà, dopo delle opportune operazioni, il risultato cercato. Con un'appropriata sequenza di gate è possibile effettuare la trasformazione
\begin{equation}\label{black_box_U_f}
    \ket{x,y} \overset{U_f}{\longrightarrow} \ket{x,y \oplus f(x)} \, ,
\end{equation}
dove $U_f$ è un opportuno gate unitario che implementa l'operazione desiderata. Il circuito che implementa la \eqref{black_box_U_f} è 
\begin{center}
    \mbox{
        \Qcircuit @C=1em @R=1em {
            \lstick{\ket{x}} & \multigate{1}{U_f} & \rstick{\ket{x}} \qw \\
            \lstick{\ket{y}} & \ghost{U_f} & \rstick{\ket{y\oplus f(x)}} \qw
        }
    }
\end{center}
dove $\ket{x}$ prende il nome di \textbf{data register} e $\ket{y}$ prende il nome di \textbf{target register}. Questa rappresentazione è utile perché quando $\ket{y} = \ket{0}$ l'output del target register è esattamente l'oggetto che si vuole calcolare
\begin{center}
    \mbox{
        \Qcircuit @C=1em @R=1em {
            \lstick{\ket{x}} & \multigate{1}{U_f} & \rstick{\ket{x}} \qw \\
            \lstick{\ket{0}} & \ghost{U_f} & \rstick{\ket{0\oplus f(x)}=\ket{f(x)}} \qw
        }
    }
\end{center}
Notiamo che la \eqref{black_box_U_f} è invertibile: se applichiamo $U_f$ due volte, otteniamo:
\begin{equation*}
    \ket{x,y} \rightarrow \ket{x, y \oplus f(x)} \rightarrow \ket{x,y \oplus f(x) \oplus f(x)} = \ket{x,y} \, ,
\end{equation*}
siccome $f(x) \oplus f(x) = 0$ indipendentemente dai valori di $f$. Fino ad ora avremmo potuto effettuare tutte queste operazioni in CC. L'importanza del QC risiede nel fatto che si possano considerare sovrapposizioni di stati appartenenti ad una base. Consideriamo il caso $n = 1$ (il data register è un qubit) e assumiamo il seguente stato iniziale
\begin{equation*}
    \ket{x,y} \equiv \underbrace{\frac{1}{\sqrt 2} (\ket{0}+\ket{1})}_{\ket{x}} \otimes \underbrace{\ket 0}_{\ket{y}} = \frac{1}{\sqrt 2} \left( \ket{00}+\ket{10} \right) \, ;
\end{equation*}
Se assumiamo che il computer sia preparato in $\ket{0} \otimes \ket{0}$ come possiamo rappresentare $\ket{x,y}$ in un circuito? Possiamo sfruttare l'\texttt{H-gate} in questo modo:
\begin{center}
    \mbox{
        \Qcircuit @C=1em @R=1em {
            \lstick{\ket{0}} & \gate{H} & \multigate{1}{U_f} & \qw \\
            \lstick{\ket{0}} & \qw & \ghost{U_f} & \qw
            %\gategroup{1}{4}{2}{4}{0.8em}{\}}
        }
    }
\end{center}
infatti, utilizzando la \eqref{black_box_U_f}, avremo
\begin{equation*}
    \ket{0,0} \overset{H}{\longrightarrow} \frac{1}{\sqrt{2}} \left( \ket{00}+\ket{10} \right) \overset{U_f}{\longrightarrow} \frac{1}{\sqrt{2}} \left( \ket{0, f(0)} + \ket{1, f(1)} \right) \, .
\end{equation*}
Questo circuito è particolarmente interessante perché l'output è una sovrapposizione di differenti stati contenenti informazioni riguardo la funzione: $f(0)$ e $f(1)$ appaiono simultaneamente nel medesimo stato. È come se avessimo valutato $f(x)$ per due valori di $x$ contemporaneamente, parallelamente! A differenza del classic parallelism, in cui più circuiti vengono costruiti per calcolare $f(x)$ ed eseguiti simultaneamente, qui viene impiegato un singolo circuito per valutare la funzione $f(x)$ per più valori di $x$ nello stesso momento: si sta sfruttando la capacità di un computer quantistico di essere in sovrapposizioni di stati diversi. Qui risiede il \textbf{quantum parallelism}.

\noindent Questo discorso può essere facilmente generalizzato al caso di $n$-qubit. Supponiamo che il data register si trovi in $\ket{0}^{\otimes n}$. Usiamo il fatto che l'azione dell'\texttt{H-gate} su $n$-qubit possa essere scritta nel seguente modo:
\begin{align}
    H^{\otimes n}\ket{0}^{\otimes n} &= \underbrace{H\otimes \cdots \otimes H}_{n\text{-volte}} \underbrace{\ket{0} \otimes \cdots \otimes \ket{0}}_{n\text{-volte}} = \frac{1}{\sqrt 2} (\ket 0 + \ket 1) \otimes \cdots \otimes \frac{1}{\sqrt 2} (\ket 0 + \ket 1) \notag \\
    &= \frac{1}{\sqrt{2^n}}(\ket{000 \ldots 0} + \ket{010 \ldots 0} + \ldots + \ket{111 \dots 1}) = \frac{1}{\sqrt{2^n}}\sum_{x=0}^{2^n-1}\ket{x} \, , \label{n_H_gates}
\end{align}
dove $x$ rappresenta tutte le possibili stringhe di $n$-volte $0$ e $1$. Se il target si trova in $\ket{y} = \ket{0}$ e applichiamo ora $U_f$, il risultato è:
\begin{equation*}
    \frac{1}{\sqrt{2^n}}\sum_{x=0}^{2^n-1}\ket{x} \otimes \ket{0} \overset{U_f}{\longrightarrow} \frac{1}{\sqrt{2^n}}\sum_{x=0}^{2^n-1}\ket{x,f(x)} \, ,
\end{equation*}
dove si è fatto uso della \eqref{black_box_U_f} con $\ket{y} = \ket{0}$. In termini di circuiti avremo
\begin{center}
    \mbox{
        \Qcircuit @C=1em @R=1em {
            \lstick{\ket{0}^{\otimes n}} & \gate{H^{\otimes n}} & \multigate{1}{U_f} & \qw & \qw \\
            \lstick{\ket{y} = \ket{0}} & \qw & \ghost{U_f}   & \qw      & \qw
        }
    }
\end{center}
In un certo senso, il quantum parallelism consente di valutare simultaneamente tutti i possibili valori della funzione $f(x)$, anche se apparentemente abbiamo valutato $f(x)$ in una singola volta. Precisiamo che la misura dello stato nel caso del qubit singolo ci darà solamente $\ket{0, f(0)}$ oppure $\ket{1, f(1)}$. In maniera analoga per il caso generale, la misura dello stato $\sum_x\ket{x,f(x)}$ ci darà un solo $f(x_0)$ per un singolo valore casuale $x_0$. Ovviamente un computer classico può farlo più facilmente! La computazione quantistica richiede qualcosa di più del semplice quantum parallelism per essere utile; richiede cioè la capacità di estrarre informazioni su più di un valore di $f(x)$ da stati di sovrapposizione, come $\sum_x\ket{x,f(x)}$. Come vedremo nella prossima sezione, il trucco di considerare una sovrapposizione lineare ci permetterà di estrarre alcune informazioni su $f$ in un modo più efficiente del CC.

\section{Algoritmo di Deutsch}
Una semplice modifica del circuito precedente dimostra come i circuiti quantistici possano essere più performanti rispetto a quelli classici. Nelle ultime righe del paragrafo precedente abbiamo detto che la computazione quantistica richiede qualcosa di più oltre al quantum parallelism per essere utilizzabile. L'\textbf{algoritmo di Deutsch} combina il meccanismo del \textbf{quantum parallelism} con la proprietà della meccanica quantistica dell'\textbf{interferenza}. 

\noindent Si tratta di un algoritmo un po' accademico (le funzioni sono banali), tuttavia utile per illustrare l'idea di algoritmo quantistico. Lasciamo che entrambi input e output register contengano ciascuno un solo qubit, quindi stiamo esplorando le funzioni $f(x)$ che convertono un singolo bit in un singolo bit: $f(x): \; \{ 0,1 \} \rightarrow \{ 0,1 \}$. Ci sono due modi piuttosto diversi di pensare a tali funzioni. Il primo modo è notare che ci sono solo quattro di queste funzioni, come mostrato nella Tabella \ref{tab:Deutsch_Fnct}.

\begin{table}[!ht]
	\centering
    \begin{tabular}{ccc}
        \toprule
        & $x = 0$ & $x=1$ \\
        \midrule
        $f_0$ & $0$ & $0$ \\
        $f_1$ & $0$ & $1$ \\
        $f_2$ & $1$ & $0$ \\
        $f_3$ & $1$ & $1$ \\
        \bottomrule
    \end{tabular} \\
    \caption{Possibili output delle sole quattro funzioni distinte $f_j(x)$ che convertono un bit in un bit. Esse sono tutte facilmente implementabili sia in un computer classico che quantistico.}
    \label{tab:Deutsch_Fnct}
\end{table}

\noindent Supponiamo che ci venga data una black-box (ossia un gate ignoto che indicheremo con \texttt{U-gate}) che calcola una di queste quattro funzioni eseguendo la seguente trasformazione unitaria:
\begin{equation*}
    U_{f_j} \ket{x,y} = \ket{x, y \oplus f_j(x)} \, .
\end{equation*}
In questo caso, se implementiamo in circuiti la Tabella \ref{tab:Deutsch_Fnct} avremo:

\begin{center}
    \mbox{
        $
        \begin{matrix}
             \\
             \\
            f_0: \\
        \end{matrix}
        $
        \Qcircuit @C=1em @R=1em {
            & \multigate{1}{U_{f_0}} & \qw \\
            & \ghost{U_{f_0}}& \qw \\
        }
        $
        \begin{matrix}
             \\
             \\
            \ = \\
        \end{matrix}
        $
        \Qcircuit @C=1em @R=1.9em {
            & \qw & \qw & \qw & \qw & \qw \\
            & \qw & \qw & \qw & \qw & \qw \\
        }
    }
    \qquad \qquad
    \mbox{
        $
        \begin{matrix}
             \\
             \\
            f_1: \\
        \end{matrix}
        $
        \Qcircuit @C=1em @R=1em {
            & \multigate{1}{U_{f_1}} & \qw \\
            & \ghost{U_{f_1}}& \qw \\
        }
        $
        \begin{matrix}
             \\
             \\
            \ = \\
        \end{matrix}
        $
        \Qcircuit @C=1em @R=1.35em {
            & \ctrl{1} & \qw & \qw \\
            & \targ & \qw & \qw  \\
        }
    }
\end{center}
\begin{center}
    \mbox{
        $
        \begin{matrix}
             \\
             \\
            f_2: \\
        \end{matrix}
        $
        \Qcircuit @C=1em @R=1em {
            & \multigate{1}{U_{f_2}} & \qw \\
            & \ghost{U_{f_2}}& \qw \\
        }
        $
        \begin{matrix}
             \\
             \\
            \ = \\
        \end{matrix}
        $
        \Qcircuit @C=1em @R=1.15em {
            & \qw & \ctrl{1} & \qw \\
            & \gate{X} & \targ & \qw \\
        }
    }
    \qquad \qquad
    \mbox{
        $
        \begin{matrix}
             \\
             \\
            f_3: \\
        \end{matrix}
        $
        \Qcircuit @C=1em @R=1em {
            & \multigate{1}{U_{f_3}} & \qw \\
            & \ghost{U_{f_3}}& \qw \\
        }
        $
        \begin{matrix}
             \\
             \\
            \ = \\
        \end{matrix}
        $
        \Qcircuit @C=1em @R=1.25em {
            & \qw & \qw \\
            & \gate{X} & \qw \\
        }
    }
\end{center}
Dato che la regola che vogliamo implementare è $\ket{x,0} \rightarrow \ket{x, f(x)}$ ($\ket{y}$ inizializzato a $\ket{0}$), in termini matematici questo significa scrivere:
\begin{align*}
    &f_0: &\ket{x,0} &\longrightarrow \ket{x,0} \, , \\
    &f_1: &\ket{x,0} &\overset{\texttt{CNOT}}{\longrightarrow} 
    \begin{cases}
        \ket{0,0} \, , &\text{per } x = 0 \\
        \ket{1,1} \, , &\text{per } x = 1
    \end{cases} \, , \\
    &f_2: &\ket{x,0} &\overset{X}{\longrightarrow} \ket{x,1} \overset{\texttt{CNOT}}{\longrightarrow}
    \begin{cases}
        \ket{0,1} \, , &\text{per } x = 0 \\
        \ket{1,0} \, , &\text{per } x = 1
    \end{cases} \, , \\
    &f_3: &\ket{x,0} &\overset{X}{\longrightarrow} \ket{x,1} \, , 
\end{align*}

\noindent Supponiamo che ci venga data una black-box che esegua $U_f$ per una delle quattro funzioni, ma non ci venga detto quale delle quattro operazioni. Ovviamente possiamo scoprirlo lasciando agire due volte la black-box, prima su $\ket0 \otimes \ket0$ e poi su $\ket 1 \otimes \ket 0$. Ma supponiamo di poter far agire la black-box solo una volta. Cosa possiamo conoscere di $f(x)$?

\noindent In un computer classico, dove siamo effettivamente limitati a lasciare che la black-box agisca sui qubit in uno dei quattro stati di base computazionale, possiamo conoscere il valore di:
\begin{itemize}
    \item $f(0)$, lasciando che $U_f$ agisca su uno dei due $\ket0 \otimes \ket0$ o $\ket0 \otimes \ket1$;
    \begin{itemize}
        \item In tal caso possiamo limitare la scelta a $f_0$ o $f_1$ (se $f(0) = 0$) oppure $f_2$ o $f_3$ (se $f(0) = 1$).
    \end{itemize}
    \item $f(1)$, lasciando che $U_f$ agisca su $\ket1 \otimes \ket0$ o $\ket1 \otimes \ket1$;
    \begin{itemize}
        \item In questa situazione abbiamo ristretto la funzione ad essere $f_0$ o $f_2$ (se $f(1) = 0$) oppure $f_1$ o $f_3$ (se $f(1) = 1$).
    \end{itemize}
\end{itemize}
In definitiva, un computer classico necessita di due esecuzioni per determinare se $f$ sia costante o meno. Sorprendentemente, risulta che con un computer quantistico questo non è necessario perché il problema può essere risolto con una singola esecuzione. Il punto interessante è che l'algoritmo non riguarda il calcolo preciso della funzione, ma piuttosto la comprensione di una o più sue proprietà: quando l'algoritmo viene lanciato non impariamo nulla sui valori individuali di $f(0)$ e $f(1)$, ma siamo comunque in grado di rispondere alla domanda sui loro valori relativi. Chiaramente otteniamo meno informazioni di quelle che otterremmo rispondendo alla domanda con un computer classico, ma, rinunciando alla possibilità di acquisire quella parte dell'informazione che è irrilevante per la domanda a cui vogliamo rispondere, possiamo ottenere la risposta con una sola applicazione di $U_f$.

\noindent Come sottolineato in precedenza l'algoritmo combina il quantum parallelism e l'interferenza: possiamo preparare il computer nello stato $\ket 0 \otimes \ket 1$ della base canonica e applicare l'\texttt{H-gate} a entrambi i qubit: 
\begin{equation}\label{eq:Deutsch_1}
    (H\otimes H) \ket{0} \otimes \ket{1} = \underbrace{\frac{\ket0 + \ket1}{\sqrt 2}}_{\substack{\text{quantum} \\ \text{parallelism}}} \otimes \underbrace{\frac{\ket0-\ket1}{\sqrt 2}}_{\text{interferenza}} \, ;
\end{equation}
in un circuito significa scrivere
\begin{center}
    \mbox{
        \Qcircuit @C=1em @R=1em {
            \lstick{\ket{0}} & \gate{H} & \multigate{1}{U_f} & \qw \\
            \lstick{\ket{1}} & \gate{H} & \ghost{U_f} & \qw
        }
    }
\end{center}
\vspace{0.2cm}
Chiamando per semplicità $\ket{x} \equiv \{ \ket{0} ,  \ket{1} \}$ e applicando $U_f$ alla \eqref{eq:Deutsch_1} tramite \eqref{black_box_U_f}, possiamo esplicitamente vedere che cosa implica il termine di interferenza:
\begin{align*}
    &\ket{x} \otimes \frac{1}{\sqrt 2} (\ket 0 - \ket 1) \overset{U_f}{\longrightarrow} \frac{1}{\sqrt 2} \left( \ket{x, 0 \oplus f(x)} - \ket{x, 1 \oplus f(x)} \right) \\
    &=
    \begin{cases}
        \frac{1}{\sqrt 2} \left( \ket{x, 0 \oplus 0} - \ket{x, 1 \oplus 0} \right) = \ket x \otimes \frac{1}{\sqrt 2} (\ket 0 - \ket 1) \, , \quad &\text{per } f(x) = 0 \\
        \frac{1}{\sqrt 2} \left( \ket{x, 0 \oplus 1} - \ket{x, 1 \oplus 1} \right) = - \ket x \otimes \frac{1}{\sqrt 2} (\ket 0 - \ket 1) \, , \quad &\text{per } f(x) = 1
    \end{cases} \, .
\end{align*}
Combinando i due casi in un'unica espressione compatta abbiamo ottenuto
\begin{equation}\label{black_box_action_U_f_on_x}
    \ket{x} \otimes \frac{1}{\sqrt 2} (\ket 0 - \ket 1) \overset{U_f}{\longrightarrow} (-1)^{f(x)} \ket x \otimes \frac{1}{\sqrt 2} (\ket 0 - \ket 1) \, ,
\end{equation}
Sostituendo $\ket{x}$ con lo stato iniziale che implementava il quantum parallelism avremo
\begin{equation*}
    \frac{\ket{0} + \ket{1}}{\sqrt{2}} \otimes \frac{\ket 0 - \ket 1}{\sqrt 2} \overset{U_f}{\longrightarrow} \frac{1}{\sqrt{2}} \left[ (-1)^{f(0)} \ket 0 + (-1)^{f(1)} \ket 1 \right] \otimes \frac{\ket 0 - \ket 1}{\sqrt 2} \, ;
\end{equation*}
dato che il segno relativo nella parentesi quadra dipende dal fatto che $f(0)$ e $f(1)$ siano uguali o meno, possiamo riscrivere quest'ultima espressione come
\begin{equation*}
    \begin{cases}
        (-1)^{f(0)}\frac{\ket 0 + \ket 1}{\sqrt 2}\otimes\frac{\ket 0-\ket 1}{\sqrt 2} \, , &\text{per }f(0) = f(1) \\
        (-1)^{f(0)}\frac{\ket 0 - \ket 1}{\sqrt 2}\otimes\frac{\ket 0-\ket 1}{\sqrt 2} \, , &\text{per }f(0) \neq f(1) 
    \end{cases} \, .
\end{equation*}
Come ultimo passaggio si applica l'\texttt{H-gate} al primo qubit in maniera tale che il circuito totale diventi:
\begin{center}
    \mbox{
        \Qcircuit @C=1em @R=1em {
            \lstick{\ket{0}} & \gate{H} & \multigate{1}{U_f} & \gate{H} & \qw \\
            \lstick{\ket{1}} & \gate{H} & \ghost{U_f} & \qw & \qw
        }
    }
\end{center}
Questa modifica trasforma il risultato precedente in 
\begin{equation*}
    \begin{cases}
        (-1)^{f(0)}\frac{\ket 0 + \ket 1}{\sqrt 2}\otimes\frac{\ket 0-\ket 1}{\sqrt 2} \overset{H}{\longrightarrow} (-1)^{f(0)}\ket 0\otimes\frac{\ket 0-\ket 1}{\sqrt 2} \, , &\text{per }f(0) = f(1) \\
        (-1)^{f(0)}\frac{\ket 0 - \ket 1}{\sqrt 2}\otimes\frac{\ket 0-\ket 1}{\sqrt 2} \overset{H}{\longrightarrow} (-1)^{f(0)}\ket 1\otimes\frac{\ket 0-\ket 1}{\sqrt 2} \, , &\text{per }f(0) \neq f(1) 
    \end{cases} \, .
\end{equation*}
Il risultato finale ci suggerisce che possiamo effettuare solamente una misurazione sul primo qubit: ottenendo $\ket{0}$ o $\ket{1}$ siamo in grado, con una singola misura, di stabilire se $f(0) = f(1)$ oppure $f(0) \neq f(1)$. Questo significa che siamo in grado di escludere 2 delle 4 funzioni con una singola esecuzione dell'algoritmo. 

\noindent Questo esempio permette di evidenziare quale sia la differenza tra il quantum parallelism e gli algoritmi randomizzati classici. Ingenuamente, si potrebbe pensare che lo stato finale corrisponda piuttosto a un calcolatore classico probabilistico che valuta $f(0)$ con probabilità $\frac 12$, o $f(1)$ con probabilità $\frac 12$. La differenza è che in un computer classico queste due alternative si escludono sempre mentre in un computer quantistico è possibile che le due alternative interferiscano l'una con l'altra per ottenere alcune proprietà globali della funzione $f(x)$. Utilizzando un opportuno gate (nel nostro caso l'\texttt{H-gate}) siamo in grado di ricombinare le diverse alternative.

\section{Algoritmo di Deutsch-Jozsa}
L'algoritmo di Deutsch è un semplice caso di un algoritmo quantistico più generale, noto come \textbf{algoritmo di Deutsch-Jozsa}, che evidenzia esplicitamente come il QC offra un grosso miglioramento rispetto ai metodi del CC. Supponiamo di avere una black-box che calcoli una funzione booleana $f(x): \; \{0,1\}^{\otimes n}\rightarrow \{0,1\}$ e supponiamo di sapere per certo che $f(x)$ sia solamente una delle seguenti alternative:
\begin{itemize}
    \item \textbf{Funzione costante} (\textit{constant}): l'output è sempre $0$ oppure $1$ indipendentemente dall'input.
    \item \textbf{Funzione bilanciata} (\textit{balanced}): l'output è costituito per metà dal valore $0$ e metà dal valore $1$.
\end{itemize}
Lo scopo dell'algoritmo è quello di capire quale delle due sia l'alternativa corretta con il minor numero di esecuzioni. Classicamente potremmo risolvere questo problema calcolando $2^{n-1}+1$ valori della funzione perché è necessario calcolare almeno una metà dei valori più un valore aggiuntivo. Chiaramente si tratta di un numero esponenzialmente grande. Quello che fa l'algoritmo di Deutsch-Jozsa è risolvere il problema perfettamente con una sola query quantistica. Cominciamo scrivendo il circuito che descrive tale algoritmo, il quale è molto simile a quello di Deutsch con la sola differenza che il data register non è un singolo qubit, ma piuttosto un prodotto tensoriale di $n$-qubit:
\begin{center}
    \mbox{
        \Qcircuit @C=1em @R=1em {
            \lstick{\ket{0}^{\otimes n}} & \gate{H^{\otimes n}} & \multigate{1}{U_f} & \gate{H^{\otimes n}} & \qw \\
            \lstick{\ket{1}} & \gate{H} & \ghost{U_f} & \qw & \qw
        }
    }
\end{center}
Vediamo nello specifico cosa succede all'interno del circuito:
\begin{enumerate}
    \item Viene inizializzato (preparato) lo stato in $\ket{0}^{\otimes n} \otimes \ket{1}$;
    \item Creiamo una sovrapposizione di stati usando l'\texttt{H-gate} su tutti gli $n+1$ qubit:
        \begin{equation*}
            \ket{0}^{\otimes n} \otimes \ket{1} \overset{H}{\longrightarrow} \frac{1}{\sqrt {2^n}}\sum_{x=0}^{2^n-1}\ket x \otimes \frac{\ket 0 - \ket 1}{\sqrt 2} \, ,
        \end{equation*}
        dove si è fatto uso della \eqref{n_H_gates}. Notiamo che ora nell'output register è presente lo stato che nella sezione precedente avevamo visto essere associato all'interferenza. 
    \item Valutiamo la funzione $f(x)$ usando la black-box di $U_f$
        \begin{equation}\label{dopo_U_f}
            \frac{1}{\sqrt {2^n}}\sum_{x=0}^{2^n-1}\ket x \otimes \frac{\ket 0 - \ket 1}{\sqrt 2} \overset{U_f}{\longrightarrow} \sum_{x=0}^{2^n-1}\frac{(-1)^{f(x)}}{\sqrt{2^n}}\ket x \otimes \frac{\ket 0 - \ket 1}{\sqrt 2} \, ,
        \end{equation}
        dove, essendo $\ket{x}$ arbitrario, abbiamo fatto uso della \eqref{black_box_action_U_f_on_x}. 
    \item Applichiamo nuovamente l'\texttt{H-gate} ai primi $n$ qubit. Per capire il risultato di $H^{\otimes n} \ket{x}$ consideriamo per semplicità il caso $n=1$: formalmente avremo 
    \begin{equation*}
        H \ket{x} = \sum_{z = 0}^1 \frac{(-1)^{xz}}{\sqrt{2}} \ket{z} \, , \; \text{ dove } x = 0 \text{ oppure } 1 \, .
    \end{equation*}
    Per $n$ generico possiamo generalizzare scrivendo
    \begin{align*}
        H^{\otimes n} \ket{x} &= (H \otimes \ldots \otimes H) \ket{x_0} \otimes \ket{x_1} \otimes \ldots \otimes \ket{x_{n-1}} \\
        &= \sum_{z_0=0}^1 \ldots \sum_{z_{n-1}=0}^1 \frac{(-1)^{x_0 z_0} (-1)^{x_1 z_1} \cdots (-1)^{x_{n-1} z_{n-1}}}{\sqrt{2^n}} \ket{z} \, ,
    \end{align*}
    dove $\ket{z} \equiv \ket{z_0, z_1, \ldots, z_{n-1}}$. In maniera più compatta possiamo scrivere quindi l'azione dell'\texttt{H-gate} sugli $n$ qubit (nonché risultato finale del circuito) come
        \begin{equation}\label{output_Deutsch_Jozsa}
            \sum_{z = 0}^{2^n-1} \sum_{x = 0}^{2^n-1} \frac{(-1)^{f(x) + x \cdot z}}{2^n}\ket z \otimes \frac{\ket 0 - \ket 1}{\sqrt 2} \, ,
        \end{equation}
        dove abbiamo indicato con $x\cdot z$ il \textbf{prodotto bit a bit modulo 2}:
        \begin{equation*}
            x\cdot z = (x_0z_0 + \ldots + x_{n-1}z_{n-1}) \mod{2} \, .
        \end{equation*}
    \item Infine misuriamo per ottenere lo stato finale $\ket{z}$. 
\end{enumerate}

\noindent Ricordiamo che il problema è quello di determinare se $f$ sia constant o balanced. Notiamo dal risultato in \eqref{output_Deutsch_Jozsa} che il data register ora contiene una sovrapposizione lineare di tutti i possibili stati che si scrivono come stringhe contenenti $n$ volte 0 e 1. In $\ket{z}$ è presente un caso particolare: consideriamo la situazione in cui $\ket z = \ket{00\ldots0} = \ket0^{\otimes n}$ e cerchiamo la probabilità di ottenere tale stato guardando il modulo quadro del coefficiente:
\begin{equation*}
    P \left( \ket{z} = \ket0^{\otimes n} \right) = \abs{\sum_{x=0}^{2^n-1}\frac{(-1)^{f(x)}}{2^n}}^2 = 
    \begin{cases}
    1 \, , &\text{se } f(x) \text{ è constant} \\
    0 \, , &\text{se } f(x) \text{ è balanced} 
    \end{cases} \, .
\end{equation*}
Notiamo che quando la probabilità è 1 a numeratore si hanno $2^n$ termini tutti uguali ($(-1)^1$ oppure $(-1)^0$) che si semplificano con il fattore $1/2^n$; quando invece la probabilità è nulla a numeratore si ha uno stesso numero di $(-1)^1$ e $(-1)^0$ che si cancellano esattamente. Come abbiamo detto $\ket z=\ket0^{\otimes n}$ è un caso particolare molto importante perché permette di risolvere il problema mediante la misura dello stato. Se misurando $z$ otteniamo $\ket{0}^{\otimes n}$ allora, con probabilità 1 (quindi sempre), lo stato è $\ket{0}^{\otimes n}$ e la funzione è constant; al contrario quando la misura di $z$ produce un qualsiasi stato differente da $\ket{0}^{\otimes n}$ allora, necessariamente $P \left( \ket{z} = \ket0^{\otimes n} \right) = 0$, lo stato $\ket{0}^{\otimes n}$ non è nemmeno presente in $z$ e possiamo stabilire con assoluta certezza che la funzione è balanced. Il fatto importante è che essendo queste misure mutuamente esclusive, possiamo determinare se $f$ sia constant o balanced con una singola misurazione. Quindi si tratta di effettuare una sola misurazione in QC contro $\mathcal{O}(2^n)$ misure in CC.

\noindent Osserviamo che il confronto tra algoritmi classici e quantistici è in qualche modo un confronto delicato, poiché il metodo per valutare la funzione è abbastanza diverso nei due casi. Se fosse consentito utilizzare un computer probabilistico classico, per valutare $f(x)$ per pochi $x$ scelti a caso, si può determinare molto rapidamente con alta probabilità se $f(x)$ è \textit{constant} o \textit{balanced}. Questo scenario probabilistico è forse più realistico dello scenario deterministico che abbiamo considerato.

\noindent Ribadiamo nuovamente che questo algoritmo è un esempio molto accademico in quanto non esistono problemi fisici o matematici reali che necessitano di sapere se una funzione sia constant o balanced. Nonostante ciò il fatto importante è che grazie a questo algoritmo quantistico non è più necessario aspettare un tempo esponenzialmente\footnote{Talvolta non si vuole sapere con precisione assoluta se $f$ sia constant o balanced, ma è sufficiente stabilirlo entro un errore dato $\varepsilon$. Un ipotetico algoritmo classico e probabilistico di questo tipo diventa di ordine polinomiale in $n$: passare da $\mathcal{O}(\text{polinomio in }n)$ a $\mathcal{O}(1)$ mediante la controparte quantistica non è più un miglioramento così estremo come passare da $\mathcal{O}(2^n)$ ad $\mathcal{O}(1)$!} crescente nel numero di bit per sapere il risultato. 

\section{Algoritmo di Bernstein-Vazirani}
Consideriamo un altro algoritmo di black-box per il quale gli algoritmi quantistici forniscono un vantaggio: l'\textbf{algoritmo di Bernstein-Vazirani}. Qui, a differenza dei due casi precedenti, abbiamo accesso alla funzione della black-box $f: \{0, 1\}^n \rightarrow \{0, 1\}$. Supponiamo che la funzioni sia data da\footnote{Come prima il simbolo "$\cdot$" indica il prodotto bit a bit modulo 2}:
\begin{equation*}
    f(x) = a\cdot x = (a_0 x_0 + \ldots + a_{n-1} x_{n-1})\mod{2} \, , \; \text{ dove } a \geq 0 \text{ e } x < 2^n \, .
\end{equation*}
Sappiamo che la funzione è lineare, tuttavia l'obiettivo di questo algoritmo è trovare il valore di $a$. Classicamente, questo problema potrebbe richiedere $n$ query poiché ogni query può fornire solo un nuovo bit di informazioni su $a$, ma $a$ possiede $n$ bit: dobbiamo valutare $f(1000\ldots) = a_0$, $f(0100\ldots) = a_1$ e così via con $n$ valutazioni fino a $f(111\ldots1) = a_{n-1}$. L'algoritmo di Bernstein-Vazirani, invece, risolve il problema quantisticamente utilizzando una sola query!

\noindent Consideriamo il medesimo circuito dell'algoritmo di Deutsch-Josza e il suo output in \eqref{output_Deutsch_Jozsa}: nel caso in cui $f(x) = a \cdot x$ esso diventa 
\begin{equation*}
    \sum_{z=0}^{2^n-1}\sum_{x=0}^{2^n-1}\frac{(-1)^{x\cdot (a+z)}}{2^n}\ket z \otimes \frac{\ket 0 - \ket 1}{\sqrt 2} \, .
\end{equation*}
Come nell'algoritmo precedente guardiamo il coefficiente di $\ket{z}$:
\begin{equation*}
        \frac{1}{2^n} \sum_{x=0}^{2^n-1}(-1)^{x\cdot (a+z)} = \frac{1}{2^n} \sum_{x=0}^{2^n-1}(-1)^{x_0(a_0+z_0) + \ldots + x_{n-1}(a_{n-1}+z_{n-1})} = \frac{1}{2^n} \prod_{j=0}^{n-1} \left( \sum_{x_j=0}^{1}(-1)^{x_j(a_j+z_j)} \right) \, ,
\end{equation*}
ma ogni termine nella parentesi tonda è la somma di termini che possono essere $\pm 1$ a seconda dell'esponente. Distinguiamo i due casi:
\begin{itemize}
    \item Se $(a_j+z_j=0)\mod2 $ allora il coefficiente è 
    \begin{equation*}
        \frac{1}{2^n} \prod_{j=0}^{n-1}(2) = 1 \, , \quad \Rightarrow \quad \text{Probabilità } 1 \, .
    \end{equation*}
    \item Al contrario quando $(a_j+z_j=1)\mod2$ allora il coefficiente diventa 
    \begin{equation*}
        \frac{1}{2^n} \prod_{j=0}^{n-1} \left[ (-1)^{0\cdot 1} + (-1)^{1 \cdot 1} \right] = 0 \, , \quad \Rightarrow \quad \text{Probabilità } 0 \, .
    \end{equation*}
\end{itemize}
Ancora una volta, i due casi della probabilità sono mutuamente esclusivi e quindi avremo
\begin{align*}
    &(a_j+z_j=0)\mod2 \, , \; \forall \, j \, ,  &\Rightarrow& \quad a = z \, , &\Rightarrow& \quad \text{Probabilità } 1 \, , \\
    &(a_j+z_j=1)\mod2 \, , \;  \text{per qualche } j \, , &\Rightarrow& \quad a \neq z \, , &\Rightarrow& \quad \text{Probabilità } 0 \, .
\end{align*}
Questo significa che il nostro stato, in realtà, non è una sovrapposizione lineare, ma contiene bensì solamente lo stato
\begin{equation*}
    \ket a \otimes \frac{\ket 0 - \ket 1}{\sqrt 2} \, ;
\end{equation*}
e quindi attraverso un'unica operazione di misura sui primi $n$-qubit, otteniamo $a$, la nostra incognita.