\documentclass[a4paper, 12pt]{book}

% General Settings
\usepackage[paper = a4paper, margin = 1in]{geometry}
\usepackage[italian]{babel}
\usepackage[utf8]{inputenc}
\usepackage[T1]{fontenc}
\usepackage{hyperref}

% Math packages
\usepackage{amsmath,amssymb,amsthm}
\usepackage{mathtools}
\usepackage{mathrsfs}
\usepackage{physics}

% New commands or theorem
\newcommand{\lecture}[2]{{\scshape{Lezione #1 - #2}} \par}
\newtheorem{definition}{Definizione}[chapter]
\newtheorem{theorem}{Teorema}[chapter]
\theoremstyle{remark}
\newtheorem*{prf}{Dimostrazione}

% Documents
\begin{document}
    \begin{titlepage}
        \begin{center}
            \vspace*{5cm}
            {\scshape\LARGE Università degli Studi di Milano Bicocca \par}
            \vspace{1.0cm}
            \line(1,0){425} \\
            {\huge\bfseries Teoria della materia condensata I \par}
            \line(1,0){425} \\
 	        \vspace{0.5cm}
            {\Large Raccolta di appunti, dispense e libri \par}
            \vspace{1.0cm}
            {Anno accademico 2021/2022 \par}
            \vspace{0.5cm}
            {\bfseries Marco Gobbo \par}
            \vspace{0.5cm}
            {\url{https://github.com/marcogobbo/tecnologie-quantistiche} \par}
            \vspace*{\fill}
            {\large \today \par}
        \end{center}
    \end{titlepage}
    \tableofcontents
    %%%%%%%%%%%%%
% LECTURE 1 %
%%%%%%%%%%%%%

\chapter{Meccanica quantistica}

\lecture{1}{07/10/2021}
\section{Stati e qubit}
Prima di addentrarci nello studio delle tecnologie quantistiche, risulta opportuno fare alcuni richiami di meccanica quantistica implementando alcuni concetti che ci saranno poi utili in futuro. In particolare iniziamo velocemente ricordando il primo postulato della meccanica quantistica
\begin{itemize}
    \item \textbf{I Postulato} (\textbf{Stato}): Che cos'è uno stato? Utilizziamo la notazione di Dirac per rappresentare un vettore $\ket{\psi}$ di uno spazio di Hilbert $\mathcal{H}$ (molto spesso uno spazio vettoriale finito dimensionale) e diremo che $\ket{\psi} \in \mathcal{H}$. Uno stato è un \textbf{raggio} tale che $\norm{\ket{\psi}} = 1$ (per la conservazione della probabilità) e $\ket{\psi} \cong e^{i \alpha} \ket{\psi}$ \footnote{La notazione $\cong$ significa "equivalente a".} con $\alpha \in \mathbb{R}$. Dato che la fase globale è irrilevante, quando due stati differiscono per una fase hanno il medesimo effetto fisico. 
\end{itemize}
Procediamo ora con il definire cosa sia un qubit
\begin{definizione}[\textbf{Qubit}]
    Un qubit è un qualsiasi sistema a due livelli. Ogni sistema quantomeccanico può essere un qubit, ad esempio si può creare utilizzando le due differenti polarizzazioni del fotone, utilizzando l’allineamento dello spin di un nucleo immerso in un campo magnetico uniforme, utilizzando la tecnica della trappola ionica, sistemi superconduttivi, \dots
\end{definizione}
\noindent Davide di Vincenzo, nel 2000, ha indicato cinque criteri necessari per la scelta di un sistema fisico adatto per la computazione quantistica:
\begin{enumerate}
    \item Un sistema fisico scalabile con qubit ben caratterizzati;
    \item La capacità di inizializzare lo stato dei qubit a un semplice stato fiduciale;
    \item Tempi di decoerenza lunghi e rilevanti;
    \item Un insieme "universale" di porte quantistiche;
    \item Una capacità di misurazione specifica per qubit.
\end{enumerate}
La meccanica quantistica si occupa di descrivere il comportamento del nostro sistema a due livelli mediante una hamiltoniana. Per fare ciò lavoriamo in spazi di Hilbert bidimensionali $\mathcal{H}=\mathbb{C}^2$, quindi le hamiltoniane di questi sistemi sono degli operatori definiti su $\mathbb{C}^2 \rightarrow \mathbb{C}^2$. Gli stati in cui si trova il nostro sistema sono descritti da funzioni d'onda generiche $\psi \in \mathbb{C}^2$, in particolar modo possono essere decomposte sulla base computazionale $\{\ket 0, \ket 1\}$. Avremo quindi che 
\begin{equation*}
    \begin{array}{l}
        \hat H \ket 0 = E_0 \ket 0 \\
        \hat H \ket 1 = E_1 \ket 1 \, ,
    \end{array}
\end{equation*}
dove
\begin{equation*}
    \begin{array}{l}
        \ip{0}{0}=\ip{1}{1}=1 \\
        \ip{0}{1}=\ip{1}{0}=0 \, .
    \end{array}
\end{equation*}
Per cui ogni stato generico $\ket \psi$ può essere scritto come combinazione lineare di $\{\ket 0, \ket 1\}$
\begin{equation*}
    \ket \psi = a \ket 0 + b \ket 1 \, ,
\end{equation*}
con $a,b \in \mathbb{C}$ e soddisfacenti la condizione di conservazione di probabilità
\begin{equation*}
    \abs{a}^2+\abs{b}^2=1 \, .
\end{equation*}
Osserviamo che, per come è definito, $\ket \psi$ è uno \textbf{stato puro}, ci dà la massima conoscenza che possiamo ottenere da questo sistema. Infatti abbiamo una probabilità pari a $\abs{a}^2$ di ottenere $\ket 0$ e una probabilità pari a $\abs{b}^2$ di ottenere $\ket 1$. Dobbiamo misurare un numero infinito di volte per poter ottenere queste distribuzioni di probabilità, tuttavia non possiamo eseguire una misura successiva per estrarre ulteriori informazioni sul nostro stato $\ket \psi$ poiché quest'ultimo sarà collassato in $\ket 0$ oppure $\ket 1$. Per determinare univocamente $\alpha$ e $\beta$ si necessiterebbe un'infinità di esperimenti su un'infinità di stati tutti preparati nel medesimo stato $\ket \psi$. La massima conoscenza che possiamo estrarre non è molta, questo fatto è stato oggetto di discussione per molti anni. In particolar modo ci si è chiesti se la teoria meccanica quantistica fosse una teoria completa o meno\footnote{Einstein, A., Podolsky, B., \& Rosen, N. (1935). Can Quantum-Mechanical Description of Physical Reality Be Considered Complete?. Phys. Rev., 47, 777–780.}.\\
Come abbiamo già accennato, $a$ e $b$ sono coefficienti complessi, attraverso la notazione esponenziale possiamo scriverli come
\begin{equation*}
    a=\abs{a}e^{i\theta_0} \qquad b=\abs{b}e^{i\theta_1}\, ,
\end{equation*}
in questo modo
\begin{equation*}
    \begin{aligned}
        \ket \psi &= \abs{a}e^{i\theta_0}\ket 0 + \abs{b}e^{i\theta_1}\ket 1 \\
                  &= \underbrace{e^{i\theta_0}}_{\mathclap{\text{Fase globale}}}\Big(\abs{a}\ket 0 + \abs{b}\underbrace{e^{i\left(\theta_1-\theta_0\right)}}_{\mathclap{\text{Fase relativa}}}\ket 1\Big) \, .
    \end{aligned}
\end{equation*}
Quando misuriamo uno stato, la \textit{fase globale} risulta essere irrilevante, ciò che conta è la \textit{fase relativa} perché può dar luogo a fenomeni come l'interferenza.
\begin{esempio}[Fase relativa]
    Consideriamo gli stati $\ket 0 e \ket 1$, per scrivere i seguenti stati
    \begin{equation*}
        \ket{\psi_1}=\frac{\ket 0 + \ket 1}{\sqrt 2}\, \qquad \ket{\psi_2}=\frac{\ket 0 - \ket 1}{\sqrt 2} 
    \end{equation*}
    In questo caso il segno meno proviene dalla fase relativa. $\ket{\psi_1}$ e $\ket{\psi_2}$ forniscono lo stesso risultato per una misura di energia (lo si può verificare calcolando $\mel{\psi_i}{\hat H}{\psi_i}$), tuttavia riusciamo a distinguerli se facciamo una misura diversa. Ad esempio possiamo considerare la matrice di Pauli
    \begin{equation*}
        \sigma_x = \begin{pmatrix}
                    0 & 1 \\
                    1 & 0
                   \end{pmatrix}\, ,
    \end{equation*}
    $\ket{\psi_1}$ e $\ket{\psi_2}$ sono autostati di $\sigma_x$ con autovalori, rispettivamente, $1$ e $-1$.
\end{esempio}
\noindent Uno dei problemi principali nell'aver a che fare con sistemi quantistici è trovare l'evoluto temporale di un certo stato, perché abbiamo delle hamiltoniane che descrivono ad esempio il rumore degli strumenti, la temperatura dell'ambiente, \dots L'equazione di Schrödinger si comporta bene nel descrivere l'evoluzione di \textbf{sistemi chiusi}, ma un qubit è, in generale, un \textbf{sistema aperto} che si lega a sistemi esterni e quindi la conoscenza sul suo stato tende a diminuire, finché non perdiamo completamente l'informazione che possedeva all'inizio. Questo fatto è legato al \textbf{tempo di coerenza}. Ci sono vari modi per tenere conto di queste interazioni così da poter descrivere al meglio il nostro sistema a due livelli.\\
Supponiamo di avere un sistema chiuso che evolve secondo l'equazione di Schrödinger
\begin{equation*}
    \hat H \ket{\psi(t)}=i\hbar \partialderivative{t}\ket{\psi(t)}\, ,
\end{equation*}
dove $\ket{\psi(t)}=\hat U(t)\ket{\psi(0)}$. $\hat U$ in questo caso è un operatore unitario che può essere espresso, se l'hamiltoniana è costante nel tempo, come
\begin{equation*}
    \hat U(t)=e^{-\frac{i}{\hbar}\hat H t}\, .
\end{equation*}
Pertanto, considerando gli autostati dell'hamiltoniana 
\begin{equation*}
    \hat H \ket{i}=E_i\ket{i}\, ,
\end{equation*}
e riscrivendo il nostro stato iniziale in termini di autostati dell'hamiltoniana
\begin{equation*}
    \ket{\psi(0)}=\sum_i a_i\ket i\, ,
\end{equation*}
possiamo valutare il nostro stato al tempo generico $t$ come
\begin{equation*}
    \ket{\psi(t)}=\sum_i a_ie^{-\frac i \hbar \hat H t}\ket i=\sum_i a_i e^{-\frac i \hbar E_i t}\ket i \qquad \text{dove} \quad a_i(t)=a_i(0)e^{-\frac i \hbar E_i t}\, .
\end{equation*}
Da questo caso generale possiamo trattare il nostro sistema a due livelli, in questo caso l'hamiltoniana sarà
\begin{equation*}
    \hat H = \begin{pmatrix}
        E_0 & 0 \\
        0 & E_1
       \end{pmatrix}\, ,
\end{equation*}
applicando l'equazione di Schrödinger sui coefficienti
\begin{equation*}
    i\hbar\derivative{a_0(t)}{t}=E_0a_0(t)\, ,
\end{equation*}
\begin{equation*}
    i\hbar\derivative{a_1(t)}{t}=E_1a_1(t)\, ,
\end{equation*}
troviamo che il nostro stato finale al tempo generico $t$ sarà
\begin{equation*}
    \ket{\psi(t)}=\underbrace{e^{-\frac{i}{\hbar}E_0t}}_{\text{Fase globale}}\Big(a_0(0)\ket 0 +\underbrace{e^{-\frac{i}{\hbar}(E_1-E_0)t}a_1(0)}_{\text{Fase relativa}}\ket 1\Big)\, .
\end{equation*}
Ancora una volta, la fase globale non produce alcun effetto, ciò che notiamo è che l'evoluzione temporale cambia la fase relativa tra gli stati $\ket 0$ e $\ket 1$. Questo spiega perché se abbiamo una interazione che disturba il nostro sistema possiamo avere un cambio nella fase relativa, questo è dato dal fatto che abbiamo una variazione in termini energetici. Questo disturbo è generato da tutto ciò che è esterno al sistema a due livelli. Se perdiamo il controllo su questa fase, perdiamo tutta l'informazione che abbiamo su $\ket{\psi(t)}$, e se questo accade, non abbiamo più uno stato puro. Per questo motivo necessitiamo qualcosa che vada oltre al concetto di funzione d'onda generica $\psi$.

\section{Matrice densità}
Vogliamo realizzare uno stato puro $\ket \psi$ che sia una combinazione pura di stati $\ket 0$ e $\ket 1$:
\begin{equation*}
    \ket \psi = a\ket 0 + b \ket 1\, ,
\end{equation*}
nella realtà quando cerchiamo di realizzare questo stato, abbiamo un'indeterminazione classica rappresentata da una distribuzione di probabilità di ottenere lo stato esatto oppure uno stato simile. Supponiamo di avere un insieme di stati che indichiamo con $\{p_i, \ket{\psi_i}\}$, dove $p_i$ è la probabilità classica di ottenere un generico stato. Questi stati $\ket{\psi_i}$ sono tutti stati puri, ma non sappiamo quale sia quello giusto e la sua conoscenza è persa. Tutte queste informazioni sono contenute nella \textbf{matrice densità} che rappresenta una distribuzione classica di probabilità.\\
Dal punto di vista della teoria della meccanica quantistica, esiste un altro modo per introdurre la teoria anziché sfruttare gli stati $\psi$. Quello che si fa è sfruttare la matrice densità che è un operatore che agisce nel seguente modo
\begin{equation*}
    \hat \rho \ket{\psi_i}=p_i\ket{\psi_i}\, ,
\end{equation*}
dove $p_i$ rappresenta la probabilità di ottenere lo stato i-esimo. La matrice densità è ora una miscela di stati puri
\begin{equation*}
    \hat \rho = \sum_i p_i \op{\psi_i}{\psi_i}
\end{equation*}
e descrive la mancanza di conoscenza sui sistemi quantistici che avevamo precedentemente. Se utilizzassimo lo stesso operatore $\hat U$ per descrivere l'evoluto temporale di $\ket{\psi_i} \overset{t}{\longrightarrow} \hat U\ket{\psi_i}$, come possiamo applicarlo a $\hat \rho$?
\begin{equation*}
    \hat \rho = \sum_i p_i \op{\psi_i}{\psi_i} \longrightarrow \sum_i p_i \hat U\op{\psi_i}{\psi_i}\hat U^\dagger \,
\end{equation*}
\begin{equation*}
    \hat U \hat \rho \hat U^\dagger = \hat \rho ' \, .
\end{equation*}
Vediamo se le distribuzioni di probabilità classiche, nel caso di stati ortonormali, vengono conservate:
\begin{proof}\mbox{}\\*
    \noindent A $t=0$ :
    \begin{equation*}
          \hat \rho \ket{\psi_i (0)} = p_i \ket{\psi_i (0)} \\
    \end{equation*}
    A $t>0$ :
    \begin{equation*}
        \begin{aligned}
            \hat \rho' \ket{\psi_i (t)} &= \hat U \hat \rho \hat U^\dagger \hat U \ket{\psi_i (0)} \\      
                                        &=\hat U \hat \rho \ket{\psi_i (0)} \\
                                        &=\hat U p_i \ket{\psi_i (0)} \\
                                        &=p_i U \ket{\psi_i (0)} \\
                                        &=p_i\ket{\psi_i (t)}
        \end{aligned}
    \end{equation*}

    \noindent La probabilità $p_i$ non è cambiata nel tempo, ma lo stato sì perché ora è $\ket{\psi_i (t)}$ che non è uguale a $\ket{\psi_i (0)}$.
\end{proof}
    %%%%%%%%%%%%%
% LECTURE 2 %
%%%%%%%%%%%%%
\newpage
\noindent \lecture{2}{08/10/2021}

\section{Osservabili}\label{sec:osservabili}

\begin{itemize}
    \item \textbf{II Postulato} (\textbf{Osservabili}): Che cosa si può misurare in QM? Vengono misurate le \textbf{osservabili}, ossia \textbf{operatori autoaggiunti} (o \textbf{hermitiani}) $\hat{A}$ tali che
    \begin{equation*}
        \hat{A}: \mathcal{H}\rightarrow \mathcal{H} \, \text{ con } \, \hat{A}^\dagger = \hat{A} \, ,
    \end{equation*}
    dove più precisamente $\hat{A}^\dagger \equiv (\hat{A}^t)^\ast$. Dal punto di vista degli elementi di matrice, calcolare l'aggiunto di $A_{ij}$ significa $A^\dagger_{ij} = A^\ast_{ji}$. Dunque le matrici autoaggiunte (hermitiane) sono tali che $A^\dagger \equiv (A^t)^\ast = A$.  
\end{itemize}

\noindent In base a ciò che abbiamo visto sulla notazione braket  ($\bra{\phi} = \ket{\phi}^\dagger$) abbiamo necessariamente che

\begin{equation*}
    \ket{\psi} = B \ket{\phi} \, , \quad \Rightarrow \quad  \bra{\psi} = \bra{\phi} B^\dagger \, .
\end{equation*}

\noindent Focalizzando la nostra attenzione sugli operatori autoaggiunti, richiamiamo un importante teorema di algebra lineare:
\begin{teorema}[\textbf{Teorema Spettrale}]
    Sia $\hat{A}$ un operatore autoaggiunto su uno spazio di Hilbert $\mathcal{H}$. Allora esiste una base ortonormale di $\mathcal{H}$ composta da autovettori di $\hat{A}$, ossia $\exists$ $\lbrace \ket{n} \rbrace \in \mathcal{H}$ tale che $\hat{A} \ket{n} = a_n \ket{n}$ dove gli autovalori $a_n \in \mathbb{R}$.
\end{teorema}

\noindent Si noti dal teorema che $\braket{n}{m} = \delta_{nm}$ dove $n,m = 1, \ldots, N$ con $N \equiv \dim \mathcal{H}$. Trattandosi di una base, qualsiasi vettore dello spazio di Hilbert può essere scritto come combinazione lineare di tali vettori: 

\begin{equation*}
    \ket{\psi} = \sum_{n=1}^N \alpha_n \ket n \, , \, \text{ dove } \, \alpha_n \equiv \braket{n}{\psi} \in \mathbb{C} \, .
\end{equation*}

\noindent Ritornando al nostro caso del sistema a due livelli, lo spazio di Hilbert in esame è $\mathbb{C}^2$, dove consideriamo la \textbf{base canonica} (o \textbf{base computazionale}) data dagli stati $\ket 0$ e $\ket 1$ (si vedano le definizioni in \eqref{computational_basis}). In questo spazio vettoriale gli operatori sono rappresentati da matrici $2 \times 2$. La più generale matrice $2 \times 2$ hermitiana contenente 4 parametri reali è

\begin{equation*}
    A = 
    \begin{pmatrix}
        a+b & c-id \\ 
        c+id & a-b
    \end{pmatrix} \, ,
\end{equation*}

\noindent dove $a, b, c, d \in \mathbb{R}$. Si noti che sulla diagonale le entrate sono puramente reali. Così come abbiamo decomposto uno stato generico $\ket{\psi}$ mediante combinazione lineare di autovettori $\ket{n}$, possiamo decomporre il generico operatore hermitiano di $\mathbb{C}^2$ come 

\begin{equation}\label{generical_matrix_C2}
    A = a \mathbb{I} + c \sigma_1 + d \sigma_2 + b \sigma_3 \, ,
\end{equation}

\noindent dove $\mathbb{I}$ è la matrice \textbf{identità} e $\sigma_1, \sigma_2, \sigma_3$ sono le \textbf{matrici Pauli}:

\begin{equation}\label{Pauli_matrices}
    \sigma_1=
    \begin{pmatrix}
        0 & 1 \\
        1 & 0
    \end{pmatrix} \, , \ \ \ \ \
    \sigma_2=
    \begin{pmatrix}
        0 & -i \\
        i & 0
    \end{pmatrix} \, , \ \ \ \ \
    \sigma_3=
    \begin{pmatrix}
        1 & 0 \\
        0 & -1
    \end{pmatrix} \, . \ \ \ \ \
\end{equation}

\noindent Si ricordi che le matrici di Pauli sono i generatori del momento angolare in QM e sono infatti utilizzate per descrivere l'operatore di spin $\hat{\vec{S}} = \frac{\hbar}{2} \hat{\vec{\sigma}}$. I relativi autovalori e autovettori sono mostrati nella Tabella \ref{tab:Pauli_eig}.

\begin{table}[!ht]
	\centering
    \begin{tabular}{ccc}
        \toprule
        \textbf{Matrice di Pauli} & \textbf{Autovettori} & \textbf{Autovalori}  \\
        \midrule
        $\sigma_1$ & $\ket + = \frac{1}{\sqrt 2} \begin{pmatrix} 1 \\ 1 \end{pmatrix}, \qquad \ket - = \frac{1}{\sqrt 2} \begin{pmatrix} 1 \\ -1 \end{pmatrix} $ & $\lbrace 1, -1 \rbrace$ \\
        $\sigma_2$ & $\ket i = \frac{1}{\sqrt 2} \begin{pmatrix} 1 \\ i \end{pmatrix}, \qquad \ket{-i} = \frac{1}{\sqrt 2} \begin{pmatrix} 1 \\ -i \end{pmatrix} $ & $\lbrace 1, -1 \rbrace$ \\
        $\sigma_3$ & $\ket 0 = \begin{pmatrix} 1 \\ 0 \end{pmatrix}, \qquad \ket 1 = \begin{pmatrix} 0 \\ 1 \end{pmatrix} $ & $\lbrace 1,-1 \rbrace$ \\
        \bottomrule
    \end{tabular}\\
    \caption{Autovettori e autovalori delle matrici di Pauli.}
    \label{tab:Pauli_eig}
\end{table}

\noindent Dato che in futuro ci tornerà utile, osserviamo che gli autovettori di $\sigma_1$ e $\sigma_2$ possono essere espressi mediante base computazionale come

\begin{equation}\label{basi_di_sigma_12}
    \ket + = \frac{\ket 0 + \ket 1}{\sqrt 2} \, , \quad \ket - = \frac{\ket 0 - \ket 1}{\sqrt 2} \, , \quad \ket i = \frac{\ket 0 + i \ket 1}{\sqrt 2} \, , \quad \ket{-i} = \frac{\ket 0 - i \ket 1}{\sqrt 2} \, ,
\end{equation}

\noindent Utilizzando la rappresentazione dei qubit tramite sfera di Bloch, questi autovettori sono mostrati in Figura \ref{fig:BlochSphere2}. 

\begin{figure}[!ht]
    \centering
    \includegraphics[scale=0.6]{images/bloch-hdr-440.png}
    \caption{Rappresentazione degli autovettori delle matrici di Pauli sulla sfera di Bloch. Il punto indicato dalla freccia rossa indica un generico qubit.}
    \label{fig:BlochSphere2}
\end{figure}

\noindent Come detto in precedenza, le 3 matrici di Pauli parametrizzano lo spin e i 3 assi della sfera di Bloch possono essere associati allo spin. Considerando lo stato generico $\ket{\psi}$ della \eqref{generic qubit}, possiamo definire lo spin lungo una direzione generica $\vec{\sigma} \cdot \vec{n}$ dove $\vec n = (\cos\phi\sin\theta, \sin\phi\sin\theta, \cos\theta)$:

\begin{equation*}
    \vec{\sigma} \cdot \vec n = \cos\phi\sin\theta \, \sigma_1 + \sin \phi \sin \theta \, \sigma_2 + \cos \theta \, \sigma_3 \, ;
\end{equation*}
così facendo è un semplice esercizio di QM dimostrare che $\ket{\psi}$ è autostato di $\vec{\sigma} \cdot \vec n$ con autovalore 1, ossia $\vec{\sigma} \cdot \vec n \ket{\psi} = \ket{\psi}$. Questo significa che dato uno stato sulla sfera di Bloch, allora esso è anche autostato di spin nella direzione individuata da tale qubit: infatti l'idea fisica alla base della sfera di Bloch è che la direzione arbitraria scelta non è altro che la direzione della quantizzazione dello spin. 

\section{Misurazioni}

\begin{itemize}
    \item \textbf{III Postulato} (\textbf{Regola di Born}):
    \begin{enumerate}
        \item \textbf{Misurazione}: sia $\hat{A}$ un osservabile con autostati $\ket{n}$, ossia $\hat{A} \ket{n} = a_n \ket{n}$. Prendiamo per semplicità $a_n \neq a_m \ \forall \, n \neq m$ (osservabile con autovalori distinti). Consideriamo uno stato generico espanso sugli autostati precedenti: $\ket \psi = \sum_n \alpha_n \ket n$. Allora una misura dell'osservabile $\hat{A}$ produce il valore $a_n$ con probabilità data da $\abs{\alpha_n}^2$ (assumendo lo stato correttamente normalizzato).
        
        \item \textbf{Collasso dello stato}: cosa succede allo stato del sistema dopo la misurazione? Istantaneamente lo stato $\ket \psi$ collassa sull'autostato associato all'autovalore risultante dalla misura. Ad esempio se misurando otteniamo $a_n$ allora $\ket{\psi} \rightarrow \ket{n}$. Effettuando delle misure successive sullo stato si ottiene sempre $\ket{n}$ con probabilità esattamente uguale a 1.  
    \end{enumerate}
\end{itemize}

\begin{esempio}
    Consideriamo per esempio il generico qubit in \eqref{generic qubit} e immaginiamo di voler effettuare delle misurazioni in differenti basi. Supponiamo di voler misurare lo spin lungo $z$ (base $\lbrace \ket{0}, \ket{1} \rbrace$ di $\sigma_3$) e lungo $x$ (base $\lbrace \ket{+}, \ket{-} \rbrace$ di $\sigma_1$). Essendo il qubit già decomposto sulla base computazionale, una misurazione lungo $z$ produrrà
    
    \begin{equation*}
        P(\ket{0}) = \abs{\cos \! \left( \frac{\theta}{2} \right)}^2 \, , \qquad P(\ket{1}) = \abs{\sin \! \left( \frac{\theta}{2} \right)}^2 \, .
    \end{equation*}
    
    \noindent Per capire il risultato della misurazione lungo $x$, invece, dobbiamo espandere $\ket{\psi}$ sulla base $\lbrace \ket{+}, \ket{-} \rbrace$: usando le \eqref{basi_di_sigma_12} per esprimere $\lbrace \ket{0}, \ket{1} \rbrace$ in termini di $\lbrace \ket{+}, \ket{-} \rbrace$ ricaviamo
    
    \begin{equation*}
        P(\ket{+}) = \frac{1}{2} \abs{\cos \! \left( \frac{\theta}{2} \right) + e^{i \phi} \sin \! \left( \frac{\theta}{2} \right)}^2 \, , \qquad P(\ket{-}) = \frac{1}{2} \abs{\cos \! \left( \frac{\theta}{2} \right) - e^{i \phi} \sin \! \left( \frac{\theta}{2} \right)}^2 \, .
    \end{equation*}
    
    \noindent Si noti come in entrambe le situazioni la probabilità risulta correttamente normalizzata: $P(\ket{0}) + P(\ket{1}) = P(\ket{+}) + P(\ket{-}) = 1$. 
\end{esempio}

\begin{esempio}
    Consideriamo lo stato $\ket{+}$ delle \eqref{basi_di_sigma_12}. Qual è l'interpretazione fisica di tale stato? Supponiamo che rappresenti lo spin di una particella: quando lo spin si trova in $\ket{+}$ allora sappiamo con certezza che punta lungo la direzione $x$, ossia $P(\ket{+}) = 1$. Al contrario, per una misurazione lungo $z$ sappiamo che $P(\ket{0}) = 1/2$ e $P(\ket{1}) = 1/2$: abbiamo la certezza del risultato lungo l'asse $x$, ma lungo l'asse $z$ si ha totale incertezza. Questo fenomeno è dovuto alla non commutatività degli operatori di spin nelle 3 direzioni:
    
    \begin{equation*}
        \comm{\hat{S}_i}{\hat{S}_j} = i \hbar \varepsilon_{ijk} \hat{S}_k \, .
    \end{equation*}
    
    \noindent Se consideriamo infatti il sistema preparato in $\ket{+}$ e supponiamo di effettuare una misura lungo $z$ ottenendo $\ket{0}$ allora lo stato collasserà in $\ket{0}$ e, d'ora in avanti, qualsiasi misurazione lungo $z$ produrrà sempre $\ket{0}$ con $P(\ket{0}) = 1$. Nonostante ciò, il fatto che $\hat{S}_z$ non commuti con $\hat{S}_x$ fa sì che una misura successiva lungo $x$ "rigeneri" dell'incertezza: $P(\ket{+}) = 1/2$ e $P(\ket{-}) = 1/2$ (si veda $\ket{0}$ espresso in termini di $\lbrace \ket{+}, \ket{-} \rbrace$ dalle \eqref{basi_di_sigma_12}). 
\end{esempio}

\noindent Discutiamo la generalizzazione del III postulato nel caso in cui alcuni autovalori associati ad autostati differenti siano uguali, cioè siamo in presenza di \textbf{degenerazione}. Per esempio supponiamo il caso $N = \dim \mathcal{H} = 6$: 

\begin{equation*}
    \ket \psi = \alpha_1 \ket 1 + \alpha_2 \ket 2 + \alpha_3 \ket 3 + \alpha_4 \ket 4 + \alpha_5 \ket 5 + \alpha_6 \ket 6 \, , 
\end{equation*}

\noindent dove supponiamo la degenerazione su $a_1 = a_2$ e $a_4 = a_5 = a_6$. Introduciamo gli operatori $\hat{P}_{a_i}$ che considerano solamente la parte di $\ket{\psi}$ corrispondente all'autospazio associato ad $a_i$:

\begin{equation*}
    \ket \psi = \underbrace{\alpha_1 \ket 1 + \alpha_2 \ket 2 }_{\hat P_{a_1} \ket \psi} + \underbrace{\alpha_3 \ket 3}_{\hat P_{a_3} \ket \psi} + \underbrace{\alpha_4 \ket 4 + \alpha_5 \ket 5 + \alpha_6 \ket 6}_{\hat P_{a_4 \ket \psi}} \, ;
\end{equation*}

\noindent tali operatori prendono il nome di \textbf{proiettori} e soddisfano le proprietà seguenti:  
\begin{enumerate}
    \item $\hat P_{a_i}^\dagger = \hat P_{a_i}$;
    \item $\hat P_{a_i}^2 = \hat P_{a_i}$;
    \item $\sum_i \hat P_{a_i} = \mathbb{I}$. 
\end{enumerate}

\noindent I proiettori sono utili per scrivere la \textbf{regola di Born} (III postulato) nel caso generale: dato uno stato $\ket{\psi}$ con degenerazione sugli autovalori $a_i$, la probabilità di ottenere il risultato $a_n$ è

\begin{equation*}
    P(a_n) = \norm{\hat{P}_{a_n} \ket{\psi}}^2 \, ;
\end{equation*}

\noindent dopo la misura, la funzione d'onda collassa nel seguente stato normalizzato:

\begin{equation*}
    \ket{\psi} \rightarrow \frac{\hat{P}_{a_n} \ket{\psi}}{\norm{\hat{P}_{a_n} \ket{\psi}}} \, .
\end{equation*}

\noindent Ad esempio, nel caso dello stato sopra scritto, la probabilità di ottenere $a_1 = a_2$ non è altro che 

\begin{equation*}
    P(a_1) = \norm{\hat{P}_{a_1} \ket{\psi}}^2 = \norm{\alpha_1 \ket{1} + \alpha_2 \ket{2}}^2 = \abs{\alpha_1}^2 + \abs{\alpha_2}^2 \, ,
\end{equation*}

\noindent e lo stato collassa in

\begin{equation*}
    \ket{\psi} \rightarrow \frac{\alpha_1 \ket{1} + \alpha_2 \ket{2}}{\sqrt{\abs{\alpha_1}^2 + \abs{\alpha_2}^2}} \, ;
\end{equation*}

\noindent si noti che si ha ancora incertezza su in quale stato si trovi $\ket{\psi}$, ma con un esperimento successivo \textbf{diverso} siamo in grado di risolvere la degenerazione e ottenere $\ket{1}$ o $\ket{2}$. 

\section{Evoluzione temporale}
Il postulato successivo riguarda l'evoluzione temporale degli stati:

\begin{itemize}
    \item \textbf{IV Postulato} (\textbf{Evoluzione temporale}): L'evoluzione temporale di uno stato generico $\ket{\psi(0)}$ è descritta dall'equazione di Schrödinger:
    
    \begin{equation*}
        i \hbar \dv{t} \ket{\psi(t)} = \hat H \ket{\psi(t)} \, ,
    \end{equation*}
    
    \noindent dove $\hat{H}$ è l'operatore (hermitiano) \textbf{hamiltoniana} del sistema. L'equazione di Schrödinger conserva le probabilità: $\braket{\psi(t)} = \braket{\psi(0)} = 1$. 
\end{itemize}

\noindent Solitamente si risolve questa equazione introducendo l'\textbf{operatore di evoluzione temporale} $\hat{U}(t)$:

\begin{equation*}
    \ket{\psi(t)} = \hat{U}(t) \ket{\psi(0)} \, ;
\end{equation*}

\noindent quando l'hamiltoniana è indipendente dal tempo, $\hat{U}(t)$ diventa semplicemente

\begin{equation*}
    \hat{U} (t) = e^{-\frac{i}{\hbar} \hat H t} \, ;
\end{equation*}

\noindent se invece $\hat H = \hat H(t)$, è necessario distinguere i casi di hamiltoniane commutanti o non commutanti a tempi differenti.\\
\noindent Come detto sopra, l'evoluzione temporale preserva le probabilità e ciò è una diretta conseguenza del fatto che $\hat{U}(t)$ sia \textbf{unitario}: 
\begin{itemize}
    \item $\hat U \hat U^\dagger = \hat U^\dagger U = \mathbb{I} \quad \Rightarrow \quad \hat{U}^\dagger = \hat{U}^{-1}$.
    \item Il prodotto scalare è conservato: $\braket{\hat U\phi}{\hat U\psi} = \braket{\phi}{\hat U^\dagger \hat U\psi} = \braket{\phi}{\psi}$. 
\end{itemize}
\noindent Notiamo che $\hat{U}(t)$ per hamiltoniane indipendenti da $t$ è effettivamente unitario:

\begin{equation*}
    \hat U^\dagger \hat U = \left( e^{-\frac{i}{\hbar} \hat H t} \right)^\dagger e^{-\frac{i}{\hbar} \hat H t} = e^{\frac{i}{\hbar} \hat H t} e^{-\frac{i}{\hbar} \hat H t} = \mathbb{I} \, .
\end{equation*}

\section{Gate}\label{sec:gate}

\begin{definizione}[\textbf{Porte quantistiche}]
    L'analogo quantistico delle porte (o gate) logiche classiche sono le \textbf{porte quantistiche} (o \textbf{gate quantistici}). Un gate quantistico è un operatore unitario che cambia lo stato del sistema.
\end{definizione}

\noindent Notiamo che una delle principali differenze che rendono di difficile implementazione le porte quantistiche risiede nel fatto che non possiamo implementare direttamente le più semplici operazioni classiche come \texttt{AND}, \texttt{OR} o \texttt{XOR}. 

\begin{definizione}[\textbf{Circuito Quantistico}]
    Un \textbf{circuito quantistico} è un modello di computazione quantistica in cui una sequenza ordinata di gate quantistici è applicata ai qubit.
\end{definizione}

\noindent In un circuito classico l'uso dei gate logici è banale. Supponiamo di considerare un bit che si trova in 0 o 1: un gate costituisce l'implementazione di un agente esterno che cambia lo stato del bit. Si pensi ad esempio al gate \texttt{NOT} per il quale $a \rightarrow$ \texttt{NOT} $a$:
\begin{center}
    \begin{circuitikz}
        \draw
        (0,4.5) node[not port] (mynot) {}
        (mynot.in) node[left = .4cm, anchor=east] (a) {$0$}
        (mynot.out) node[right = .4cm,anchor=west] (b) {$1 \, ,$}
        (mynot.in) -- (a)
        (mynot.out) -- (b);
    \end{circuitikz}
    $\qquad$
    \begin{circuitikz}
        \draw
        (0,4.5) node[not port] (mynot) {}
        (mynot.in) node[left = .4cm, anchor=east] (a) {$1$}
        (mynot.out) node[right = .4cm,anchor=west] (b) {$0 \, ,$}
        (mynot.in) -- (a)
        (mynot.out) -- (b);
    \end{circuitikz}
\end{center}

\noindent Nel caso invece di un qubit, i circuiti funzionano diversamente perché le porte agiscono su sistemi a due livelli. Immaginiamo che a causa di un agente esterno il qubit $\ket{\psi}$ subisca un evoluzione temporale $\hat{U}$: rappresentiamo questo fatto mediante il circuito seguente
\begin{center}
    \mbox{
        \Qcircuit @C=2em @R=2em {
            \lstick{\ket{\psi}} & \gate{\hat{U}} & \rstick{\hat{U} \ket{\psi} \, ,} \qw \\
        }
    }
\end{center}

\noindent Si ricordi che $\hat{U}$ è sempre un operatore unitario: ad esempio per un'hamiltoniana indipendente dal tempo si ha semplicemente $\hat{U}(t) = e^{-\frac{i}{\hbar}\hat H t}$. \\
\noindent Consideriamo le matrici di Pauli: sappiamo che sono hermitiane ($\sigma^\dagger_i = \sigma_i$) e che soddisfano la proprietà $\sigma_i^2 = \mathbb{I}$, ma questo significa che sono anche matrici unitarie. Questo fatto ci permette di costruire\footnote{Un tale sistema in natura è abbastanza semplice da implementare poiché, essendo $\hat{H} = \vec{\sigma} \cdot \vec B$ l'accoppiamento tra spin e campo magnetico, è facile costruire una tale evoluzione temporale.} dei gate in cui $\hat{U} = \hat{\sigma}_i$.  Ad esempio è possibile implementare dei gate come $\mathbb{I}$, $\sigma_1 \equiv X$, $\sigma_2 \equiv Y$ e $\sigma_3 \equiv Z$. Ricordando che $\sigma_i \sigma_j = 2i \varepsilon_{ijk} \sigma_k$, notiamo che $XZ = - i Y$ e inoltre anche la matrice $-iY$ è unitaria. Per tale ragione molto spesso, al posto di considerare i gate $\lbrace \mathbb{I}, X, Y, Z \rbrace$ si sceglie la base $\lbrace \mathbb{I}, X, Z, XZ \rbrace$: questo significa che possiamo implementare i gate seguenti
\begin{center}
    \mbox{
        \Qcircuit @C=2em @R=2em {
            & \gate{X} & \qw \\
        }
    } 
    , \ \ \ \ 
    \mbox{
        \Qcircuit @C=2em @R=2em {
            & \gate{Z} & \qw \\
        }
    }
    , \ \ \ \ 
    \mbox{
        \Qcircuit @C=2em @R=2em {
            & \gate{XZ} & \qw \\
        }
    }
    ,
\end{center}

\noindent Consideriamo l'\texttt{X-gate}: dalle \eqref{Pauli_matrices} è evidente che $X$ rappresenta una sorta di "quantum" \texttt{NOT} perché inverte semplicemente lo stato della base computazionale:
\begin{center}
    \mbox{
        \Qcircuit @C=2em @R=2em {
            \lstick{\ket{0}} & \gate{X} & \rstick{\ket{1} \, ,} \qw \\
        }
    } 
    \\
    \mbox{
        \Qcircuit @C=2em @R=2em {
            \lstick{\ket{1}} & \gate{X} & \rstick{\ket{0} \, ,} \qw \\
        }
    }
\end{center}

\noindent Consideriamo ora lo \texttt{Z-gate}: gli stati della base computazionale sono autovettori con autovalori 0 e 1 di $\sigma_3$, quindi questo gate inverte semplicemente il segno
\begin{center}
    \mbox{
        \Qcircuit @C=2em @R=2em {
            \lstick{\ket{0}} & \gate{Z} & \rstick{\ket{0} \, ,} \qw \\
        }
    } 
    \\
    \mbox{
        \Qcircuit @C=2em @R=2em {
            \lstick{\ket{1}} & \gate{Z} & \rstick{-\ket{1} \, ,} \qw \\
        }
    }
\end{center}
L'azione dello \texttt{Z-gate} su un generico qubit risulterà quindi in
\begin{center}
    \mbox{
        \Qcircuit @C=2em @R=2em {
            \lstick{a \ket{0} + b \ket{1}} & \gate{Z} & \rstick{a \ket{0} - b \ket{1} \, ,} \qw \\
        }
    } 
\end{center}
e questo significa che $Z$ aggiunge semplicemente una fase $e^{i \pi} = -1$ allo stato. Ricapitolando: l'\texttt{X-gate} implementa un'interferenza dall'esterno che inverte lo stato (ad esempio cambia segno dello spin lungo $z$) e lo \texttt{Z-gate} implementa l'introduzione di una fase. \\
\noindent Una matrice particolarmente importante per i nostri scopi è 
\begin{equation}\label{Hadamard_matrix}
    H = \frac{1}{\sqrt{2}} 
    \begin{pmatrix}
        1 & 1 \\ 1 & -1 
    \end{pmatrix} \, , 
\end{equation}
chiamata \textbf{matrice di Hadamard}. Notiamo che è unitaria in quanto $H^\dagger H = \mathbb{I}$. Essa può essere implementata nel cosiddetto \texttt{H-gate} o \textbf{gate di Hadamard}: si tratta di un gate particolarmente importante (lo useremo largamente durante tutto il corso) in quanto permette di cambiare base $\lbrace \ket{0}, \ket{1} \rbrace \leftrightarrow \lbrace \ket{+}, \ket{-} \rbrace$ 

\begin{center}
    \mbox{
        \Qcircuit @C=2em @R=2em {
            \lstick{\ket{0}} & \gate{H} & \rstick{\ket{+} \, ,} \qw \\
        }
        \ \ \ \ \ \ \ \ \ \ \ \ \ \ \ \ \ \ \ \ 
        \Qcircuit @C=2em @R=2em {
            \lstick{\ket{+}} & \gate{H} & \rstick{\ket{0}\, ,} \qw \\
        }
    }
    \\
    \mbox{
        \Qcircuit @C=2em @R=2em {
            \lstick{\ket{1}} & \gate{H} & \rstick{\ket{-} \, ,} \qw \\
        } 
        \ \ \ \ \ \ \ \ \ \ \ \ \ \ \ \ \ \ \ \ 
        \Qcircuit @C=2em @R=2em {
            \lstick{\ket{-}} & \gate{H} & \rstick{\ket{1} \, ,} \qw \\
        }
    }
\end{center}

\noindent Possiamo introdurre anche le matrici seguenti (ci serviranno più avanti)

\begin{equation}\label{S_T_matrices}
    S \equiv \sqrt{Z} =
\begin{pmatrix}
    1 & 0 \\ 0 & i
\end{pmatrix} \, , \qquad 
T \equiv \sqrt{S} =
\begin{pmatrix}
    1 & 0 \\ 0 & e^{i \frac{\pi}{4}}
\end{pmatrix} \, ,
\end{equation}

\noindent Le matrici introdotte in precedenza costituiscono gli oggetti base con cui andremo a implementare diversi gate e circuiti durante tutto il corso. Per costruire il gate più generale possiamo esponenziare scrivendo $U = e^{-\frac{i}{\hbar} H t}$ dove $H = a \mathbb{I} + b_i \sigma_i$ e $a, b_i \in \mathbb{R}$ con $i = 1,2,3$. In particolare esiste una particolare classe di operatori che utilizzeremo molto
\begin{equation*}
    R_{\vec{n}} = e^{-i \frac{\lambda}{2} (\vec n \cdot \vec \sigma)} \, ;
\end{equation*}
si tratta di un caso particolare dell'esponenziazione precedente in cui $a = 0$ e i coefficienti $b_i$ sono scelti lungo un particolare versore $\vec n$. Questo operatore unitario implementa una rotazione di angolo $\lambda$ attorno ad una direzione particolare individuata da $\vec n$:

\begin{equation}\label{rotation_n_lambda}
    R_{\vec{n}}(\lambda) = e^{-i \frac{\lambda}{2} (\vec n \cdot \vec \sigma)} = \cos \! \left( \frac{\lambda}{2} \right) \mathbb{I} - i \sin \! \left( \frac{\lambda}{2} \right) \vec \sigma \cdot \vec n \, ;
\end{equation}
(si espanda il LHS con la serie di Taylor dell'esponenziale e si usi $(\vec \sigma \cdot \vec n)^2 = \mathbb{I}$ per dimostrare l'uguaglianza con il RHS). È possibile dimostrare, inoltre, che qualsiasi matrice unitaria $2 \times 2$ può essere scritta nella forma seguente
\begin{equation}\label{general_2by2_matrix}
    U = e^{i \alpha}
    \begin{pmatrix}
        e^{-i \frac{\beta}{2}} & 0 \\ 0 & e^{i \frac{\beta}{2}}
    \end{pmatrix}
    \begin{pmatrix}
        \cos \frac{\gamma}{2} & - \sin \frac{\gamma}{2} \\ \sin  \frac{\gamma}{2} & \cos \frac{\gamma}{2}
    \end{pmatrix}
    \begin{pmatrix}
        e^{-i \frac{\delta}{2}} & 0 \\ 0 & e^{i \frac{\delta}{2}}
    \end{pmatrix}
    = e^{i \alpha} R_z(\beta) R_x(\gamma) R_z(\delta) \, ; 
\end{equation}
perciò il più generale operatore unitario presenta 4 parametri reali $\alpha, \beta, \gamma, \delta \in \mathbb{R}$ e può implementare un possibile gate in un computer quantistico. Appare subito evidente come la scelta di 4 possibili parametri reali (quindi continui) consenta di realizzare un numero nettamente maggiore di gate logici quantistici rispetto al caso dei gate logici classici. 
    %%%%%%%%%%%%%%%%%%%%%%%
%%%%%% Lezione 3 %%%%%%
%%%%%%%%%%%%%%%%%%%%%%%
\vspace{1.0cm}
\newline
\lecture{3}{11/10/2021}
\vspace{1.0cm}

\noindent Nel caso dei funzionali come $E\big[u_\alpha\big]$ con vincolo $\ip{u_\alpha}{u_\beta}-\delta_{\alpha\beta}=0$ utilizziamo $\Lambda_{\alpha\beta}$ come moltiplicatore di Lagrange:
\begin{equation*}
    \functionalderivative{u_\alpha^*(\overline x)}\bigg(E\big[\{u_\alpha\},\{u^*_\alpha\}\big]-\Lambda_{\alpha'\beta'}(\ip{u_{\alpha'}}{u_{\beta'}}-\delta_{\alpha'\beta'})\bigg)=0
\end{equation*}
\textbf{Termine} $\ip{u_{\alpha'}}{u_{\beta'}}$:
\begin{equation*}
    \functionalderivative{u_\alpha^*(\overline x)}\int d^3x'u_{\alpha'}^*(\overline{x}')u_{\beta'}(\overline{x}')=u_{\beta'}(\overline x)
\end{equation*}
\textbf{Termine} $\sum_{\mu}\mel{\mu}{\hat h}{\mu}$:
\begin{equation*}
    \functionalderivative{u_\alpha^*(\overline x)}\int d^3x'u_\mu^*(\overline{x}')\hat h u_\mu(\overline{x}')=\delta_{\alpha\mu}\hat h u_\alpha(\overline x)
\end{equation*}
\textbf{Termine} $\frac 12\sum_{\mu\nu}J_{\mu\nu}$:
\begin{equation*}
    \functionalderivative{u_\alpha^*(\overline x)}\frac 12 \sum_{\mu\nu}\int d^3x' d^3x''u_\mu^*(\overline{x}')u_\nu^*(\overline{x}'')\frac{e_0^2}{4\pi\varepsilon_0|\overline{x}'-\overline{x}''|}u_\mu({\overline{x}'})u_\nu(\overline{x}'')
\end{equation*}
Se $\mu=\alpha \vee \nu=\alpha$ ho due contributi, prendiamo $\mu=\alpha$ e semplifichiamo il fattore $\frac 12$
\begin{equation*}
    \functionalderivative{u_\alpha^*(\overline x)} \sum_{\nu}\int d^3x' u_\alpha^*(\overline{x}')u_\alpha(\overline{x}') \int d^3x''u_\nu^*(\overline{x}'')\frac{e_0^2}{4\pi\varepsilon_0|\overline{x}'-\overline{x}''|}u_\nu(\overline{x}'') =
\end{equation*}
\begin{equation*}
    = \underbrace{\sum_{\nu}\int d^3x''u_\nu^*(\overline{x}'')u_\nu(\overline{x}'')\frac{e_0^2}{4\pi\varepsilon_0|\overline{x}'-\overline{x}''|}}_{\text{Potenziale di Hartree}}u_\alpha(\overline x)=V_{\text{Hartree}}(\overline x)u_\alpha(\overline x)
\end{equation*}
\textbf{Termine} -$\frac 12\sum_{\mu\nu}K_{\mu\nu}$:
\begin{equation*}
    -\functionalderivative{u_\alpha^*(\overline x)}\frac 12 \sum_{\mu\nu}\int d^3x' d^3x''u_\mu^*(\overline{x}')u_\nu^*(\overline{x}'')\frac{e_0^2}{4\pi\varepsilon_0|\overline{x}'-\overline{x}''|}u_\nu({\overline{x}'})u_\mu(\overline{x}'')
\end{equation*}
Come prima, se $\mu=\alpha \vee \nu=\alpha$ ho due contributi, prendiamo $\mu=\alpha$ e semplifichiamo il fattore $\frac 12$
\begin{equation*}
    - \functionalderivative{u_\alpha^*(\overline x)} \sum_{\nu}\int d^3x' u_{\alpha}^*(\overline{x}')\int d^3x''u_\nu^*(\overline{x}')u_\nu(\overline{x}'')\frac{e_0^2}{4\pi\varepsilon_0|\overline{x}'-\overline{x}''|}u_\alpha(\overline{x}') =
\end{equation*}
\begin{equation*}
    = -\sum_\nu \int d^3x''u_\nu^*(\overline{x}'')u_\alpha(\overline{x}'')\frac{e_0^2}{4\pi\varepsilon_0|\overline{x}-\overline{x}''|}u_\nu(\overline{x})=-\hat V_{\text{Fock}}(\overline x)u_\alpha(\overline x)
\end{equation*}
Mettendo insieme tutti i termini abbiamo:
\begin{equation*}
    \hat h(\overline x) u_\alpha(\overline x)+\hat V_H(\overline x)u_\alpha(\overline x)-\hat V_{\text F}(\overline x) u_\alpha(\overline x) = \Lambda_{\alpha\beta'}u_{\beta'}(\overline x)
\end{equation*}
Specifichiamo meglio questi contributi:
\begin{itemize}
    \item $J_{\mu\nu}=\mel{\mu\nu}{V_{12}}{\mu\nu}$: senza perdere in generalità, distinguiamo la componente spaziale da quella di spin come: $\mu=\mu\sigma$, $\sigma=\pm\frac 12$, quindi
    \begin{equation*}
        J_{\mu\nu}=\mel{\mu\nu}{V_{12}}{\mu\nu}=\mel{\mu\sigma\nu\sigma'}{V_{12}}{\mu\sigma\nu\sigma'}=\mel{\mu\nu}{V_12}{\mu\nu}\ip{\sigma\sigma'}{\sigma\sigma'}
    \end{equation*}
    dal momento che $V_{12}$ non dipende dallo spin, rimane solo la componente spaziale. Pertanto nel valutare l'energia di Hartree, l'integrale è sì un integrale spaziale, ma la somma viene eseguita anche sullo spin:
    \begin{equation*}
        E_{\text H}=\frac 12\sum_{\mu\nu\sigma\sigma'}\mel{\mu\nu}{V_{12}}{\mu\nu}
    \end{equation*}
    \item $K_{\mu\nu}=\mel{\mu\nu}{V_{12}}{\nu\mu}$: stesso discorso vale per l'integrale di scambio, tuttavia, il suo valore è non nullo se gli spin hanno lo stesso verso:
    \begin{equation*}
        K_{\mu\nu}=\mel{\mu\nu}{V_{12}}{\nu\mu}=\mel{\mu\sigma\nu\sigma'}{V_{12}}{\nu\sigma'\mu\sigma}=-\frac 12 \sum_{\mu\nu\sigma}K_{\mu\sigma\nu\sigma}
    \end{equation*}
\end{itemize}
Possiamo fare un passo in più considerando una \textbf{proprietà del determinante di Slater}: esso è invariante rispetto a una transformazione unitaria del set di numeri quantici delle funzioni d'onda a singola particella $\{u_\alpha\}$. Questo significa che $E_{\text H}$ è invariante. Osserviamo inoltre che la \textbf{matrice dei moltiplicatori di Lagrange} $\Lambda_{\alpha\beta}$ è hermitiana (autovalori lineari) e simmetrica. Mettendo insieme queste informazioni abbiamo che l'equazione dei vincoli è invariante rispetto allo scambio tra particelle, quindi i moltplicatori di Lagrange non dipendono dalle funzioni d'onda e $\Lambda_{\alpha\beta}$ è una matrice hermitiana. Possiamo quindi scegliere $u_\alpha$ affinché $\Lambda_{\alpha\beta}$ sia \textbf{diagonale}. In questo caso avremo:
\begin{equation*}
    \hat h u_\alpha(\overline x)+\hat V_{\text H}u_\alpha(\overline x)-\hat V_{\text F}u_\alpha(\overline x)=\varepsilon_\alpha u_\alpha(\overline x)
\end{equation*}
dove i $\varepsilon_\alpha$ sono numeri reali e dipendono dalla funzione d'onda.
\begin{equation*}
    \underbrace{(\hat h + \hat V_{\text H} - \hat V_{\text F})}_{\hat h_{\text{HF}}}u_\alpha(\overline x)=\varepsilon_\alpha u_\alpha(\overline x)
\end{equation*}
se rimuoviamo il termine $\hat V_{\text F}$, rifiutiando quindi di considerare il termine di scambio, troviamo il risultato del \textbf{metodo di Hartree}. \\
Alla fine quello che dobbiamo risolvere è questa equazione in modo da ottenere soluzioni autoconsistenti $\{u_\alpha\}$. Possiamo calcolare:
\begin{equation*}
    E=\sum_{\mu\sigma}\mel{\mu\sigma}{\hat h}{\mu\sigma}+\frac 12 \sum_{\mu\nu\sigma\sigma'}J_{\mu\sigma\nu\sigma'}-\frac 12 \sum_{\mu\nu\sigma}K_{\mu\sigma\nu\sigma}
\end{equation*}
Un altro modo di esprimere l'\textbf{equazione di Hartree-Fock}:
\begin{equation*}
    \mel{u_\alpha}{\hat h + \hat V_{\text H} - \hat V_{\text F}}{u_\alpha}=\mel{u_\alpha}{\varepsilon_\alpha}{u_\alpha}
\end{equation*}
\begin{equation*}
    \sum_\alpha\mel{\alpha}{\hat h}{\alpha}+\sum_\alpha\mel{u_\alpha}{\hat V_{\text H}}{u_\alpha}-\sum_\alpha\mel{u_\alpha}{\hat V_{\text F}}{u_\alpha}=\sum_\alpha\varepsilon_\alpha
\end{equation*}
\begin{equation*}
    \sum_\mu\mel{\mu}{\hat h}{\mu}+\sum_{\mu\nu}(J_{\mu\nu}-K_{\mu\nu})=\sum_\alpha \varepsilon_\alpha
\end{equation*}
è simile all'energia esatta, ma non è esatta poiché stiamo contando due volte l'interazione, vista in un altro modo:
\begin{equation*}
    E_{\text{HF}}=\sum_\alpha\varepsilon_\alpha-\frac 12\sum_{\mu\nu}(J_{\mu\nu}-K_{\mu\nu})=\sum_\alpha\varepsilon_\alpha-E_{\text H}-E_{\text X}
\end{equation*}
Gli \textbf{autovalori} dell'energia di Hartree-Fock sono i \textbf{moltiplicatori di Lagrange} $\varepsilon_\alpha$, ma qual è il loro significato fisico? Come possono essere usati? La risposta viene dal seguente teorema
\begin{theorem}[\textbf{Teorema di Koopmans}]
    Si consideri $N$ elettroni con funzioni d'onda $\{u_\gamma\}$ e il determinante di Slater della soluzione dell'\textbf{equazione di Hartree-Fock}. Supponiamo di considerare N-1 elettroni, rimuovendo quindi un elettrone: un partifolare stato a energia $\varepsilon_\alpha$ descritto da $u_\alpha$. Il nuovo set di funzioni d'onda sarà $\{u_\beta\}\not\owns u_\alpha$. Il sistema a N particella è: $E_N=E_{N-1}+\varepsilon_\alpha$. Ciò significa che $\varepsilon_\alpha$ è l'energia da fornire al sistema per rimuovere l'elettrone. Se per esempio abbiamo un atomo e $\varepsilon_\alpha$ dell'elettrone è negativa, $\varepsilon_\alpha$ corrisponde all'energia da fornire per ionizzare il sistema.
\end{theorem}
\begin{prf}
    \begin{equation*}
        E_N=\sum_\mu\mel{\mu}{\hat h}{\mu}+\frac 12\sum_{\mu\nu}(J_{\mu\nu}-K_{\mu\nu})
    \end{equation*}
    \begin{equation*}
        E_{N-1}=\sum_{\mu\neq\alpha}\mel{\mu}{\hat h}{\mu}+\frac 12\sum_{\mu\nu\neq \alpha}(J_{\mu\nu}-K_{\mu\nu})
    \end{equation*}
    \begin{equation*}
        E_N=E_{N-1}+\underbrace{\mel{\alpha}{\hat h}{\alpha}+\sum_\mu(J_{\alpha\mu}-K_{\alpha\mu})}_{\varepsilon_\alpha}
    \end{equation*}
\end{prf}
\subsection*{Osservazioni}
Consideriamo $N$ elettroni e un sistema che passa dallo stato fondamentale a uno eccitato: $\varepsilon_\alpha \rightarrow \varepsilon_{\alpha'}$. Abbiamo un nuovo determinante di Slater e la differenza di energia sarà:
\begin{equation*}
    \Delta E = E_{\alpha'}-E_\alpha=\varepsilon_{\alpha'}-\varepsilon_\alpha - (J_{\alpha\alpha'}-K_{\alpha\alpha'})
\end{equation*}
Questo valore non è trascurabile quando parliamo di atomi, tuttavia quando parliamo di materiali, se eseguiamo gli integrali, questi valori sono molto piccoli soprattutto tra due elettroni separati da una grande distanza. Questi valori possono essere usati nell'equazione di Hartree-Fock per calcolare lo stato eccitato. Supponiamo di rimuovere in $\alpha$ un elettrone e di metterlo nello stato $\alpha'$, l'\textbf{energia di rilassamento}, $J_{\alpha\alpha}$ è trascurabile in questo caso.\\
Il \textbf{metodo di Hartree-Fock} fornisce una soluzione per sistemi a molti elettroni con un singolo determinante di Slater ed è molto utile per i \textbf{sistemi a shell chiusa}.\\
Consideriamo:
\begin{itemize}
    \item Atomo di He $1s^2$: è un sistema a shell chiusa ed è descritto bene dal modello di Hartree poiché non c'è energia di scambio.
    \item Atomo di Be $1s^22s^2$: è un sistema a shell chiusa e abbiamo un contributo da parte dell'energia di scambio.
    \item Atomo di He eccitato $1s2s$: in questo caso applicando la teoria delle perturbazioni abbiamo:
    \begin{itemize}
        \item Stato di singoletto: $S=0, m_s=0$
        \item Stato di tripletto: $S=1, m_s=\pm 1$ (di cui ne prendiamo il determinante a singola particella del Slater) e $S=1, m_s= 0$
    \end{itemize}
    L'idea è quindi di risolvere in maniera autoconsistente il \textbf{metodo di Hartree-Fock} con una funzione d'onda a singola particella per ottenere tutti gli stati. Per gli stati a singola particella l'energia sarà esatta, altrimenti sarà differente.
    \item Atomo di C $1s^22s^22p^2$: abbiamo differenti termini per questo atomo ${}^3P, {}^2D, {}^1S$, ma soltanto il secondo possiede un determinante di Slater a singola particella, gli altri no.
\end{itemize}

\section{Modello di Thomas-Fermi}
Un altro modello che possiamo prendere in considerazione è il \textbf{modello di Thomas-Fermi}, a differenza del \textbf{metodo di Hartree-Fock}, l'energia è scritta come funzionale della densità di elettroni:
\begin{equation*}
    E\big[n(\overline x)\big]=\int \dd[3]{x}V_{\text ext}(\overline x)n(\overline x)+E_{\text H}\big[n(\overline x)\big]+E_{\text K}\big[n(\overline x)\big]
\end{equation*}
dove
\begin{equation*}
    E_{\text H}\big[n(\overline x)\big]=\frac 12 \frac{e_0^2}{4\pi\varepsilon_0}\int \dd[3]{x'}\dd[3]{x''}\frac{n(\overline{x}')n(\overline{x}'')}{|\overline{x}'-\overline{x}''|}
\end{equation*}
e supponiamo che il funzionale dell'energia cinetica sia funzione di $n(\overline x)$.\\
L'approssimazione che si fa in questo modello consiste nel considerare un sistema di un grande numero N di elettroni liberi e confinati in una certa regione dello spazio con le opportune condizioni periodiche al contorno. Supponiamo di avere delle particelle indipendenti in una scatola di volume $V=L^3$:
\begin{equation*}
    u_{\overline k}(\overline x)=\frac{e^{i\overline k \cdot \overline x}}{\sqrt V} \ \ \ \ \ \ \ \ \ \  \overline k = \frac{2\pi}{L}(n_x,n_y,n_z) \ \ \ \ \ \ \ \ \ \ \varepsilon_{\overline k}=\frac{\hbar^2\overline{k}^2}{2m}
\end{equation*}
Nello stato fondamentale, possiamo riempire tutti i livelli energetici fino all'energia più alta: \textbf{energia di Fermi}, $\varepsilon_{\text F}$.
Per N particelle e un volume V, possiamo definire la \textbf{densità di particelle}: $n=\frac NV$. Cosicché l'energia di Fermi risulti definita come:
\begin{equation*}
    \varepsilon_{\text F}=\frac{\hbar^2k_{\text F}^2}{2m} \ \ \ \ \ k_{\text F}=(3\pi^2n)^{\frac 13}
\end{equation*}
Alla luce di ciò possiamo definire la densità di particelle come:
\begin{equation*}
    n=\int_0^{\varepsilon_F}d\varepsilon D(\varepsilon)
\end{equation*}
dove $D(\varepsilon)$ rappresenta la densità di stati, cioè stati per unità di volume ed energia, $D(\varepsilon)\propto\sqrt\varepsilon$.
\begin{equation*}
    D(\varepsilon)=\frac{1}{2\pi^2}\bigg(\frac{2m}{\hbar^2}\bigg)^{\frac 32}\sqrt\varepsilon
\end{equation*}
L'energia media per particella sarà:
\begin{equation*}
    \expval{\varepsilon}=\frac EN=\frac VN \frac EV = \frac 1 n \frac EV=\frac 1 n \int_{0}^{\varepsilon_{\text F}}d\varepsilon D(\varepsilon)\varepsilon = \frac 3 5 \varepsilon_{\text F}(n)
\end{equation*}
L'idea nel \textbf{modello di Thomas-Fermi} è di considerare che localmente non abbiamo più un sistema omogeneo, abbiamo in principio un sistema non omogeneo perché la densità di elettroni dipende da $\overline x$. Più precisamente l'approssimazione riguarda l'energia cinetica quantistica in cui si assume che localmente gli elettroni abbiano una energia cinetica media che corrisponde all'energia cinetica media del sistema omogeneo di particelle indipendenti a quella densità locale.
\begin{equation*}
    E_{\text K}=\int \dd[3]{x}n(\overline x)\underbrace{\frac 35 \varepsilon_{\text F}(n(\overline x))}_{\mathclap{\expval{\varepsilon} \text{ al punto } \overline x}}
\end{equation*}
Abbiamo usato quindi il risultato di particelle indipendenti in una scatola per costruire un'approssimazione per l'energia cinetica di un sistema non omogeneo di elettroni interagenti. A questo punto possiamo scrivere il funzionale dell'energia come:
\begin{equation*}
    E_{\text H}\big[n(\overline x)\big]=\int \dd[3]{x} V_{\text{ext}}(\overline x)n(\overline x)+E_{\text H}\big[n(\overline x)\big]+\int \dd[3]{x}n(\overline x)\frac 35\varepsilon_{\text F}(n(\overline x))
\end{equation*}
Ancora una volta vogliamo minimizzare questo funzionale con il seguente vincolo:
\begin{equation*}
    \int_V \dd[3]{x} n(\overline x)=N
\end{equation*}
A differenza del \textbf{metodo di Hartree-Fock}, qui abbiamo un solo moltiplicatore di Lagrange:
\begin{equation*}
    \functionalderivative{n(\overline x)}\Bigg(E\big[n(\overline x)\big]-\mu\bigg(\underbrace{\int d^3x'n(x')}_{=1}-\underbrace{N}_{=0}\bigg)\Bigg)=0
\end{equation*}
\begin{equation*}
    \functionalderivative{n(\overline x)}E\big[n(\overline x)\big]=\mu
\end{equation*}
Questa è l'equazione che dobbiamo risolvere e ci dà la densità dello stato fondamentale. Scriviamo esplicitamente:
\begin{equation*}
    V_{\text {ext}}(\overline x)+V_{\text H}(\overline x) + \functionalderivative{n(\overline x)}E_{\text K}=\mu
\end{equation*}
\begin{equation*}
    \begin{aligned}
        \functionalderivative{n(\overline x)}E_{K}
        & =\functionalderivative{n(\overline x)}\int d^3x n(\overline x)\frac 35 \frac{\hbar^2}{2m}(3\pi^2)^{\frac 23}n(\overline x)^{\frac 23} \\
        & = \functionalderivative{n(\overline x)}\int d^3x n(\overline x)^{\frac 53}\frac 35 \frac{\hbar^2}{2m}(3\pi^2)^{\frac 23} \\
        & = \frac 35\frac{\hbar^2}{2m}(3\pi^2)^{\frac 23}\frac 53n^{\frac 23}(\overline x)\\
        & = \varepsilon_{\text F}(n(\overline x))
    \end{aligned}
\end{equation*}
Ottenendo così l'equazione di Thomas-Fermi:
\begin{equation*}
    V_{\text {ext}}(\overline x)+V_{\text H}(\overline x) + \varepsilon_{\text F}(n(\overline x))=\mu
\end{equation*}
Risolvendo questa equazione si ottiene la densità dello stato fondamentale, da cui poi possiamo ricavare l'energia dello stato fondamentale.
    %%%%%%%%%%%%%
% LECTURE 4 %
%%%%%%%%%%%%%
\vspace{1cm}

\noindent \lecture{4}{15/10/2021}

\vspace{0.5cm}

\noindent Continuiamo la discussione riguardante il concetto di entanglement. Dato che questo fenomeno si manifesta in sistemi con almeno due qubit, possiamo utilizzare due differenti basi:
\begin{itemize}
    \item \textbf{Base computazionale} (o \textbf{standard}): formata dagli stati $\lbrace \ket{00}, \ket{01}, \ket{10}, \ket{11} \rbrace$ (la due entrate indicano gli stati del primo e secondo qubit rispettivamente).
    
    \item \textbf{Base di Bell} (o \textbf{EPR}): costituita dagli stati 
    \begin{align*}
        \ket{\beta_{00}} &= \frac{1}{\sqrt{2}} (\ket{00} + \ket{11}) \, , &\ket{\beta_{01}} &= \frac{1}{\sqrt{2}} (\ket{01} + \ket{10}) \, , \\
        \ket{\beta_{10}} &= \frac{1}{\sqrt{2}} (\ket{00} - \ket{11}) \, , &\ket{\beta_{11}} &= \frac{1}{\sqrt{2}} (\ket{01} - \ket{10}) \, ;
    \end{align*}
    notiamo che si trattano di stati entangled costruiti a partire da combinazioni lineari indipendenti degli stati della base computazionale.
\end{itemize}

\noindent Chiaramente, trattandosi di una base, possiamo espandere qualsiasi stato $\ket{\psi}$ nella base EPR, scrivendo
\begin{equation*}
    \ket{\psi} = \sum_{n,m=0}^1 \alpha_{nm} \ket{\beta_{nm}} \, .
\end{equation*}
Conseguentemente, se si cerca la probabilità di trovarsi in $\ket{\beta_{nm}}$ si può effettuare una misurazione nella base di Bell e ottenere $P(\ket{\beta_{nm}}) = \abs{\alpha_{nm}}^2$. 

\noindent Gli stati della base di Bell non sono difficili da costruire utilizzando i gate che abbiamo visto precedentemente. Supponiamo di poter utilizzare un computer quantistico i cui qubit si trovano nella base standard $\lbrace \ket{0}, \ket{1} \rbrace$. Utilizziamo l'\texttt{H-gate} e il \texttt{CNOT-gate}: 
\begin{itemize}
    \item Ricordiamo che $H\ket{0} = \ket{+}$ e $H \ket{1} = \ket{-}$, quindi il gate di Hadamard permette di passare da un qubit nella base computazionale ad un qubit in una combinazione lineare di elementi di questa base (si ricordi la matrice in \eqref{Hadamard_matrix} e le definizioni in \eqref{basi_di_sigma_12}). 
    
    \item Il \texttt{CNOT-gate}, invece, scambia il secondo qubit solamente se il primo si trova in $\ket{1}$:
    \begin{align*}
    \ket{00} &\overset{\texttt{CNOT}}{\longrightarrow} \ket{00} \, , &\ket{01} &\overset{\texttt{CNOT}}{\longrightarrow} \ket{01} \, , \\
    \ket{10} &\overset{\texttt{CNOT}}{\longrightarrow} \ket{11} \, , &\ket{11} &\overset{\texttt{CNOT}}{\longrightarrow} \ket{10} \, .
\end{align*}
\end{itemize}

\noindent Utilizzando questi due gate possiamo facilmente implementare il circuito seguente 
\begin{center}
    \mbox{
        \Qcircuit @C=2em @R=1.5em {
            \lstick{\ket{x}} & \gate{H} & \ctrl{1} & \qw \\
            \lstick{\ket{y}} & \qw & \targ & \qw
        }
    }
\end{center}
i cui output sono esattamente gli stati $\ket{\beta_{xy}}$ della base di Bell. Verifichiamolo:
\begin{align*}
    \ket{00} &\overset{H}{\rightarrow} \frac{1}{\sqrt{2}} (\ket{0} + \ket{1}) \otimes \ket{0} = \frac{1}{\sqrt{2}} \left( \ket{00} + \ket{10} \right) \overset{\texttt{CNOT}}{\rightarrow} \frac{1}{\sqrt{2}} \left( \ket{00} + \ket{11} \right) \equiv \ket{\beta_{00}} \, , \\
    \ket{01} &\overset{H}{\rightarrow} \frac{1}{\sqrt{2}} (\ket{0} + \ket{1}) \otimes \ket{1} = \frac{1}{\sqrt{2}} \left( \ket{01} + \ket{11} \right) \overset{\texttt{CNOT}}{\rightarrow} \frac{1}{\sqrt{2}} \left( \ket{01} + \ket{10} \right) \equiv \ket{\beta_{01}} \, , \\
    \ket{10} &\overset{H}{\rightarrow} \frac{1}{\sqrt{2}} (\ket{0} - \ket{1}) \otimes \ket{0} = \frac{1}{\sqrt{2}} \left( \ket{00} - \ket{10} \right) \overset{\texttt{CNOT}}{\rightarrow} \frac{1}{\sqrt{2}} \left( \ket{00} - \ket{11} \right) \equiv \ket{\beta_{10}} \, , \\
    \ket{11} &\overset{H}{\rightarrow} \frac{1}{\sqrt{2}} (\ket{0} - \ket{1}) \otimes \ket{1} = \frac{1}{\sqrt{2}} \left( \ket{01} - \ket{11} \right) \overset{\texttt{CNOT}}{\rightarrow} \frac{1}{\sqrt{2}} \left( \ket{01} - \ket{10} \right) \equiv \ket{\beta_{11}} \, ;
\end{align*}
si noti come l'azione di un gate (in questo caso l'\texttt{H-gate}) su un singolo qubit non basti per produrre uno stato entangled. Al contrario il \texttt{CNOT-gate}, invece, crea stati entangled poiché agisce su coppie di qubit. 

\noindent Vediamo ora due esplicite applicazioni dell'entanglement. 

\section{Superdense coding}
Si tratta del primo esempio esplicito delle potenzialità dei metodi quantistici contro i metodi classici. Il problema riguarda il come inviare informazioni di due bit classici (00, 01, 10, 11) ad un generico sperimentatore. Consideriamo la sperimentatrice Alice e supponiamo che abbia due bit di informazione (due numeri $xy$) e che voglia inviarli allo sperimentatore Bob. Dal punto di vista classico, Alice semplicemente utilizza un canale classico (un telefono ad esempio) per comunicare direttamente a Bob quale coppia di numeri possiede. Nel caso in cui Alice possieda due qubit, invece, può inviare solamente uno dei due sfruttando il fatto che siano entangled. Supponiamo che Alice e Bob condividano due qubit entangled, ad esempio
\begin{equation*}
    \ket{\psi} = \frac{1}{\sqrt{2}} \left( \ket{00} + \ket{11} \right) = \ket{\beta_{00}} \, ,
\end{equation*}
dove Alice possiede il primo qubit (prima entrata del ket) e Bob il secondo. Cosa deve fare Alice per inviare solamente un singolo "pezzo" di informazione? Ad esempio Alice può effettuare una qualche operazione sul suo qubit e, sfruttando l'entanglement, Bob sarà in grado di leggere l'informazione desiderata (la coppia di numeri $xy$ che Alice vuole inviare) facendo una singola misurazione. Più precisamente, supponiamo che Alice voglia inviare delle informazioni effettuando le seguenti operazioni sul proprio qubit di $\ket{\psi}$:
\begin{equation*}
    \text{Alice invia:} \; 
    \begin{cases}
        00 \, , &\text{non fa niente} \\
        10 \, , &\text{applica } Z \\
        01 \, , &\text{applica } X \\
        11 \, , &\text{applica } ZX
    \end{cases} \, .
\end{equation*}
Cosa succede allo stato condiviso quando applica queste operazioni? 
\begin{itemize}
    \item Quando vuole inviare 00 non effettua alcuna operazione quindi $\ket{\psi} \rightarrow \ket{\psi} = \ket{\beta_{00}}$. 
    
    \item Nel caso in cui decide di inviare 10 applica $Z$:
    \begin{equation*}
        \ket{\psi} \rightarrow Z \frac{1}{\sqrt{2}} \left( \ket{00} + \ket{11} \right) = \frac{1}{\sqrt{2}} \left( \ket{00} - \ket{11} \right) = \ket{\beta_{10}} \, .
    \end{equation*}
    
    \item Quando invece vuole inviare 01 applica $X$:
    \begin{equation*}
        \ket{\psi} \rightarrow X \frac{1}{\sqrt{2}} \left( \ket{00} + \ket{11} \right) = \frac{1}{\sqrt{2}} \left( \ket{10} + \ket{01} \right) = \ket{\beta_{01}} \, .
    \end{equation*}
    
    \item Infine se vuole inviare 11 applica $ZX$:
    \begin{equation*}
        \ket{\psi} \rightarrow ZX \frac{1}{\sqrt{2}} \left( \ket{00} + \ket{11} \right) = Z \frac{1}{\sqrt{2}} \left( \ket{10} + \ket{01} \right) = \frac{1}{\sqrt{2}} \left( \ket{01} - \ket{10} \right) = \ket{\beta_{11}} \, .
    \end{equation*}
\end{itemize}

\noindent Effettuando queste operazioni, Alice è in grado di spedire ciò che vuole: $xy \rightarrow \ket{\beta_{xy}}$. Bob può effettuare una (singola) misura nella base di Bell, stabilire quale dei 4 stati possiede, %Qui risiede l'idea di \textbf{non-località} della QM: sebbene Bob possa trovarsi molto lontano da Alice, il suo qubit è cambiato a seguito delle operazioni di lei e può 
e leggere quindi i bit corrispondenti $xy$. L'informazione classica corrispondente a due bit \`e stata inviata attraverso un solo qubit.

\noindent Questo esempio, nonostante sia un po' accademico, risulta particolarmente interessante perché mette in risalto come, grazie all'entanglement, sia possibile ridurre il numero di operazioni necessarie per inviare un'informazione rispetto al caso classico. 

\section{Teleportation}
Innanzitutto che cosa intendiamo con il termine \textit{teletrasporto}? In questo contesto viene inteso con il significato di ricostruire un qubit molto lontano da dove si trovava in origine: il qubit originale sparisce e una sua nuova copia viene creata altrove. L'idea è quella di effettuare questa particolare ricostruzione usando solamente operazioni classiche sui qubit.

\noindent Supponiamo che Alice (d'ora in avanti chiameremo sempre in questo modo i nostri due sperimentatori) abbia un generico qubit $\ket{\psi} = \alpha \ket{0} + \beta \ket{1}$ e che voglia inviarlo a Bob senza utilizzare alcun canale quantistico. Dalle leggi della QM sappiamo che non possiamo estrarre sia $\alpha$ che $\beta$ con una singola misura e inoltre non è possibile clonare questo stato generico. Inoltre, se volesse inviare direttamente questo stato con assoluta precisione (assumiamo $\alpha , \beta \in \mathbb{R}$) mediante un canale classico, allora necessiterebbe due stringhe infinite di bit e quindi del tempo infinito per inviarle. Come nel caso precedente, assumiamo che Alice e Bob condividano lo stato entangled $\ket{\beta_{00}} = \frac{1}{\sqrt{2}} \left( \ket{00} + \ket{11} \right)$, dove il primo qubit è di Alice e il secondo di Bob. Notiamo che Alice possiede due qubit: il qubit generico $\ket{\psi}$ che vuole teletrasportare e il qubit entangled con quello di Bob. Lo stato iniziale non è quindi altro che
\begin{equation}\label{teleportation_initial_state}
    \left( \alpha \ket{0} + \beta \ket{1} \right) \ket{\beta_{00}} = \frac{\alpha}{\sqrt{2}} \ket{0} \left( \ket{00} + \ket{11} \right) + \frac{\beta}{\sqrt{2}} \ket{1} \left( \ket{00} + \ket{11} \right) \, .
\end{equation}
Alice sottopone gli stati in suo possesso al seguente circuito:
\begin{center}
    \mbox{
        \Qcircuit @C=1em @R=1em {
            \lstick{\ket{\psi}} & \ctrl{1} & \gate{H} & \qw & \rstick{\text{Alice}} \\
            \lstick{} & \targ & \qw & \qw & \rstick{\text{Alice}} \\
            \lstick{} & \qw & \qw & \qw & \rstick{\text{Bob}}
            \inputgroupv{2}{3}{1.4em}{1em}{\ket{\beta_{00}}\; \; \; \;}{2em}
        }
    }
\end{center}
dove si è indicato in output a chi appartiene quel determinato qubit. Esplicitamente, si applica \texttt{CNOT-gate} ai due qubit di Alice in \eqref{teleportation_initial_state}, ottenendo
\begin{equation*}
    \frac{\alpha}{\sqrt{2}} \ket{0} \left( \ket{00} + \ket{11} \right) + \frac{\beta}{\sqrt{2}} \ket{1} \left( \ket{10} + \ket{01} \right)  \, ;
\end{equation*}
dopodiché viene applicato \texttt{H-gate} al primo qubit di Alice:
\begin{equation*}
    \frac{\alpha}{2} (\ket{0} + \ket{1}) \left( \ket{00} + \ket{11} \right) + \frac{\beta}{2} (\ket{0} - \ket{1}) \left( \ket{10} + \ket{01} \right) \, ;
\end{equation*}
infine possiamo riscrivere l'espressione nel modo seguente
\begin{equation*}
    \frac{1}{2} \bigg[ \ket{00} (\alpha \ket{0} + \beta \ket{1}) + \ket{01} (\alpha \ket{1} + \beta \ket{0}) + \ket{10} (\alpha \ket{0} - \beta \ket{1}) + \ket{11} (\alpha \ket{1} - \beta \ket{0}) \bigg] \, ,
\end{equation*}
dove in questo ultimo passaggio abbiamo svolto i conti e riordinato l'espressione focalizzandoci su ciò che è posseduto da Alice (i due qubit di fronte alle 4 parentesi tonde) e da Bob (stato nella parentesi tonda). Consideriamo ora la Tabella \ref{tab:teleportation}: Alice può effettuare una misura nella base computazionale e dire a Bob (mediante un canale classico) ciò che ha ottenuto; dopo la misura lo stato collassa e Bob, a seconda del risultato, può effettuare o meno un'opportuna operazione sul proprio stato per ricostruire precisamente ciò che si voleva teletrasportare. 

\begin{table}[!ht]
	\centering
    \begin{tabular}{ccc}
        \toprule
        \text{Alice misura} & \text{Bob trova} & \text{Bob applica}  \\
        \midrule
        $\ket{00}$ & $\alpha \ket{0} + \beta \ket{1}$ & \text{Nulla} \\
        $\ket{01}$ & $\alpha \ket{1} + \beta \ket{0}$ & $X$ \\
        $\ket{10}$ & $\alpha \ket{0} - \beta \ket{1}$ & $Z$ \\
        $\ket{11}$ & $\alpha \ket{1} - \beta \ket{0}$ & $ZX$ \\        \bottomrule
    \end{tabular}\\
    \caption{Una volta che Alice effettua la propria misura nella base computazionale e dice a Bob ciò che ha ottenuto, quest'ultimo può applicare una precisa operazione per ricostruire lo stato $\ket{\psi}$ che Alice voleva teletrasportare. Si noti che, in tutti e quattro i casi, lo stato finale che ha Bob è sempre $\ket{\psi}$ indipendentemente dal risultato di Alice.}
    \label{tab:teleportation}
\end{table}

\noindent Un fatto fondamentale da evidenziare è che solamente informazioni classiche sono state trasferite tra Bob e Alice poiché tutto il resto (misurazioni e operazioni sugli stati) viene svolto localmente nel laboratorio: non c'è né violazione della relatività speciale in quanto non avviene alcun trasferimento di informazioni più veloce della luce, né violazione del teorema di no-cloning perché, una volta che Bob ottiene $\ket{\psi}$, Alice non possiede più lo stato che voleva teletrasportare. Si tratta solamente di un modo ingegnoso per sfruttare l'entanglement. 

\section{Disuguaglianze di Bell}
L'argomento delle disuguaglianze di Bell è un tema molto vasto che comprende moltissime disuguaglianze testabili sperimentalmente: ciò che accomuna tutte le misurazioni è la profonda differenza tra il concetto di probabilità \textit{classica} e \textit{quantistica}. Nella prima metà del '900, dopo la nascita della QM e i conseguenti trionfi che tale teoria era in grado di riportare, molti fisici, tra cui lo stesso Einstein, erano profondamente insoddisfatti del concetto intrinseco ed inevitabile di probabilità che permea tale teoria. In particolare, coloro che non accettavano la QM come teoria completa, credevano che il suo comportamento fosse in realtà dovuto alla nostra ignoranza su teorie ancora più fondamentali. Questo gruppo di persone credevano che le \textbf{osservabili}, in fisica, dovessero sempre soddisfare 2 requisiti base:
\begin{itemize}
    \item \textbf{Realismo}: un'osservabile deve avere un valore definito anche prima che la misura sia effettuata.
    
    \item \textbf{Località}: un esperimento effettuato in un ben preciso luogo ha solamente un effetto locale perché non può in alcun modo modificare risultati e comportamenti di altri esperimenti effettuati in regioni causalmente disconnesse. L'entanglement, ad esempio, è in profondo contrasto con il concetto di località. 
\end{itemize}

\noindent Nel corso di quegli anni furono svolti numerosi tentativi di riscrivere la QM in maniera tale che soddisfacesse i requisiti precedenti. Ad esempio, furono utilizzate le cosiddette \textbf{teorie delle variabili nascoste}. Tali teorie si basano sull'assunto secondo cui quando si misura un valore $a$ di un'osservabile $A$, in realtà il risultato della misurazione è incompleto perché l'intera teoria prevede l'esistenza di un'altra variabile $\lambda$ \textit{nascosta} ed inaccessibile. Se si potesse conoscere $\lambda$ allora si potrebbe predire qualsiasi cosa in maniera del tutto deterministica. Nonostante ciò, il concetto di probabilità in QM vìola queste regole e, come vedremo tra poco, utilizzando le disuguaglianze di Bell è possibile rilevare sperimentalmente tale violazione. 

\begin{esempio}[Singolo qubit]
    Consideriamo nuovamente il caso di un qubit e immaginiamo di trovarci nello stato $\ket{0}$. Nella Sezione \ref{sec:osservabili} abbiamo visto che il più generale operatore hermitiano che agisce su un singolo qubit è dato da una combinazione lineare di matrici di Pauli (non consideriamo l'identità), ossia $\vec{\sigma} \cdot \vec{n}$ dove $\abs{\vec{n}} = 1$. Supponiamo che a seguito di una misurazione possiamo ottenere $\vec{\sigma} \cdot \vec{n} \ket{\vec{n}} = \ket{\vec{n}}$ e $\vec{\sigma} \cdot \vec{n} \ket{-\vec{n}} = - \ket{-\vec{n}}$, quindi il risultato è $\pm 1$. Perciò, ricordando la decomposizione $\ket{0} = c_1 \ket{\vec{n}} + c_2 \ket{-\vec{n}}$, la probabilità di misurare 1 è $P(\vec{\sigma} \cdot \vec{n} = 1) = \abs{c_1}^2 = \abs{\braket{0}{\vec{n}}}^2$. Al tempo stesso sappiamo anche che il generico qubit si scrive come in \eqref{generic qubit}, quindi $\vec{n}$ può essere specificato scegliendo gli angoli $\theta$ e $\phi$: dato che avevamo sottolineato che $\vec{\sigma} \cdot \vec{n} \ket{\psi} = \ket{\psi}$ allora la soluzione che cerchiamo è $\ket{\vec{n}} = \ket{\psi}$, quindi
    \begin{equation*}
        P(\vec{\sigma} \cdot \vec{n} = 1) = \abs{\braket{0}{\vec{n}}}^2 = \abs{\braket{0}{\psi}}^2 = \cos^2 \! \left( \frac{\theta}{2} \right) \, .
    \end{equation*}
    Vediamo se riusciamo a riprodurre la medesima distribuzione di probabilità utilizzando una teoria classica basata sulle variabili nascoste. Supponiamo che, oltre allo spin, la particella sia in realtà descritta da un'extra variabile $\lambda$: tutte le particelle sono specificate dalla coppia fissata $(a = \pm 1, \lambda)$, ossia hanno spin $a = \pm 1$ e un preciso valore di $\lambda$ persino prima di effettuare la misurazione. Per semplicità assumiamo $\lambda \in [0,1]$. Supponiamo di voler effettuare una misurazione dello spin in una particolare direzione $\vec{n}$: la misurazione, in questa teoria, corrisponde a particolari valori di spin e $\lambda$ con l'idea che una misura effettuata con angolo $\theta$ abbia risultato dipendente dal valore assunto da $\lambda$ nell'intervallo $[0,1]$. Più precisamente, consideriamo una teoria con variabili nascoste in cui  il risultato delle misure sia quello indicato nella formula seguente: assumendo di non poter rilevare il valore $\lambda$ e richiedendo che le particelle abbiano dei valori di tale variabile uniformemente distribuiti, un esperimento di questo tipo produce 
    \begin{equation*}
        \begin{cases}
            \text{spin }\ket{\uparrow} \, , & \text{per} \, \, \, 0 \leq \lambda \leq \cos^2 \theta/2 \\
            \text{spin }\ket{\downarrow} \, , &\text{per} \,\,  \cos^2 \theta/2 \leq \lambda \leq 1
        \end{cases} \, , \quad \Rightarrow \quad P(a = 1) = \cos^2 \! \left( \frac{\theta}{2} \right) \, .
    \end{equation*}
\end{esempio}

\noindent Nell'esempio precedente è stato analizzato il caso di un singolo qubit, che \`e riproducibile da una teoria con variabili nascoste. Prendiamo ora in esame il sistema di 2 qubit, dove sappiamo che l'entanglement gioca un ruolo centrale e dove ci sar\`a  differenza tra probabilità \textit{classica} e \textit{quantistica}. Esistono diversi modi per scrivere delle disuguaglianze che testino sperimentalmente questa profonda differenza. Uno dei più famosi è il seguente

\subsection{Disuguaglianza CHSH}
Si tratta di una semplice generalizzazione delle  disuguaglianze scritte originariamente da Bell e utilizzata per le verifiche sperimentali. Ancora una volta, consideriamo i due sperimentatori Alice e Bob situati in città differenti. Supponiamo che entrambi abbiano a disposizione un apparato identico su cui possano effettuare misure e che vengano riforniti (ad esempio da un terzo sperimentatore) di infinite copie costituite da coppie di particelle correlate sulle quali possono compiere dei test, e ciascuno possieda una particella della coppia. Alice misura le osservabili $a,a'$ per la sua particella, mentre Bob misura $b,b'$ per la sua, dove $a,a',b,b' = \pm 1$. Entrambi scelgono di fare misurazioni simultanee di una sola delle due osservabili, scelta ogni volta in maniera casuale.

\noindent In un'ipotetica teoria basata sulle variabili nascoste, dato questo sistema è impossibile stabilire immediatamente quale sia il risultato di una misura poiché ci sarà necessariamente una distribuzione di probabilità classica, che chiamiamo $P(a,a',b,b')$, associata alla nostra ignoranza. Consideriamo ora l'osservabile $ C = (a+a') b + (a-a') b'$; per costruzione sappiamo che
\begin{equation*}
    \begin{cases}
        a+a' = 0 \, , \; a-a' = \pm 2 \, , \quad &\text{se } a \neq a' \\
        a+a' = \pm 2 \, , \; a - a' = 0 \, , \quad &\text{se } a = a'
    \end{cases} \, , 
\end{equation*}
ma questo significa allora che per qualsiasi valore delle 4 osservabili in gioco si ha sempre $C = \pm 2$. Dalla teoria della probabilità classica sappiamo che $\abs{\expval{C}} \leqslant \expval{\abs{C}}$ dato che $\abs{\sum_c c p(c)} \leqslant \sum_c \abs{c} p(c)$. Siccome assumiamo che $a,a',b,b',C$ esistano indipendentemente dalla nostra misura, possiamo applicare questa disuguaglianza: in tutte le possibili configurazioni $\abs{C} = 2$ quindi $\expval{\abs{C}} = 2$ e allora
\begin{equation}\label{CHSH_inequality}
    \abs{\expval{C}} \leqslant 2 \, .
\end{equation}
La disuguaglianza precedente prende il nome di \textbf{disuguaglianza CHSH}\footnote{\textit{Clauser, J., Horne, M., Shimony, A., \& Holt, R. (1969). Proposed Experiment to Test Local Hidden-Variable Theories. Phys. Rev. Lett., 23, 880–884.}}. Notiamo che si tratta di un risultato classico derivante dalla teoria della probabilità. 

\noindent In QM è facile trovare un esempio nel quale questa disuguaglianza è violata. Supponiamo che Alice e Bob condividano lo stato entangled $\ket{\psi} = \frac{1}{\sqrt{2}} \left( \ket{01} - \ket{10} \right)$ e che entrambi decidano di misurare qualcosa che in QM abbia 2 possibili valori. In particolare misurano
\begin{align*}
    a &= \vec{\sigma} \cdot \hat{a} = \pm 1 \, , &a' &= \vec{\sigma} \cdot \hat{a}' = \pm 1 \, , \\
    b &= \vec{\sigma} \cdot \hat{b} = \pm 1 \, , &b' &= \vec{\sigma} \cdot \hat{b}' = \pm 1 \, ,
\end{align*}
dove il simbolo "$\hat{\,}$" indica un vettore di modulo unitario. È possibile dimostrare in QM che
\begin{equation}\label{formula_CHSH}
    \expval{(\vec{\sigma} \cdot \hat{c}) \otimes (\vec{\sigma} \cdot \hat{d})}{\psi} = - \hat{c} \cdot \hat{d} = - \cos \theta \, ,
\end{equation}
dove $\theta$ è l'angolo tra $\hat{c}$ e $\hat{d}$. Supponiamo che Alice e Bob decidano di misurare nelle direzioni indicate dagli angoli di Figura \ref{fig:CHSH}. Utilizzando la \eqref{formula_CHSH} possiamo calcolare il valore di aspettazione di $C$:
\begin{equation*}
    \expval{C} = \expval{ab + a'b + a b' - a'b'}{\psi} = - \left[ \frac{1}{\sqrt{2}} + \frac{1}{\sqrt{2}} + \frac{1}{\sqrt{2}} - \left( -\frac{1}{\sqrt{2}} \right) \right] = -2\sqrt{2} \, .
\end{equation*}
Abbiamo quindi ricavato che, secondo la QM, $\abs{\expval{C}} = 2\sqrt{2}$, in disaccordo\footnote{In realtà esiste un teorema che stabilisce che $2\sqrt{2}$ è il più grande valore che può essere ottenuto.} con il risultato classico in \eqref{CHSH_inequality}: la probabilità \textit{quantistica} è intrinsecamente differente dalla probabilità \textit{classica}! 

\begin{figure}[!t]
    \centering
    \includegraphics[scale=0.45]{images/CHSH}
    \caption{Direzioni spaziali delle 4 osservabili misurate da Alice e Bob. Si noti che l'apparato di uno è ruotato di 45° rispetto a quello dell'altro perciò $\theta_{a'b} = \theta_{ab} = \theta_{ab'} = 45$° e $\theta_{a'b'} = 135$°.}
    \label{fig:CHSH}
\end{figure}
\noindent Altri esperimenti degni di nota sono quelli condotti da Freedman e Clauser nel 1972, la serie di esperimenti condotti da Aspect negli anni 1981 e 1982, da Tittel e il gruppo Geneva nel 1988 e da Weihs sotto condizioni di località "strettamente einsteniane" nel 1998. La serie di esperimenti sulle disuguaglianze di Bell, di crescente sofisticazione, ha ridotto i critici, che mettono in discussione i risultati, a indicare falle in tale esperimenti, alcune delle quali distorcerebbero i risultati sperimentali in favore della meccanica quantistica. Nel 2015 è stato pubblicato il primo esperimento dichiarato totalmente privo di falle (loopholes-free), che ha confermato i risultati degli esperimenti precedenti.
    %%%%%%%%%%%%%
% LECTURE 5 %
%%%%%%%%%%%%%

\chapter{Algoritmi quantistici}

\vspace{1cm}

\lecture{5}{18/10/2021}

\vspace{0.5cm}

\noindent Prima di cominciare la vera discussione riguardante gli algoritmi più importanti e conosciuti della computazione quantistica, affrontiamo l'analisi della crittografia quantistica, la quale mostra ancora una volta la potenzialità dei metodi quantistici rispetto a quelli classici. 

\section{Crittografia quantistica}
Molti anni prima che fu introdotta la crittografia RSA che utilizziamo oggigiorno, molte persone pensavano che gli stati quantistici e il "bizzarro" comportamento della QM potessero essere utilizzati per scopi crittografici. Il nome del protocollo quantistico che fu pensato per trasmettere dati criptati è \textbf{protocollo BB84}. Consideriamo, come al solito, Alice e Bob in differenti città e supponiamo che vogliano comunicare tra loro tramite una linea criptata. Vediamo come viene affrontato questo problemi sia dal punto di vista classico che quantistico. 

\subsection{Esempio di crittorafia classica}
Classicamente entrambi possiedono una sequenza $S$ di bit casuali, chiamata \textbf{codepad}. Immaginiamo che Alice voglia inviare a Bob un messaggio $M$: un modo classico di inviare il messaggio criptato è quello di inviare la sequenza $M \oplus S$, dove il simbolo "$\oplus$" indica l'\textbf{addizione bit a bit modulo 2}. Ad esempio supponiamo che il codepad sia $S = 0110$ e il messaggio sia $M = 1111$: la sequenza $M \oplus S$ è data da 
\begin{table}[!ht]
	\centering
    \begin{tabular}{c|cccc|c}
        \toprule
        $S$ & 0 & 1 & 1 & 0 & $= 6$ \\
        $M$ & 1 & 1 & 1 & 1 & $=15$ \\
        \midrule
        $M \oplus S$ & 1 & 0 & 0 & 1 & $=9$ \\
        \bottomrule
    \end{tabular}
\end{table}

\noindent dove nell'ultima colonna abbiamo inserito il numero associato a quel messaggio (ad esempio, leggendo da destra verso sinistra, per $S$ si ha $0 \times 2^0 + 1 \times 2^1 + 1 \times 2^2 + 0 \times 2^3 = 6$). Il vantaggio di questo modo di crittografare messaggi risiede nel fatto che anche se si parte con una stringa $M$ sensata, l'operazione $M \oplus S$ la trasforma in una sequenza apparentemente casuale di 0 e 1 che può essere decifrata solamente se si possiede il codepad. Infatti, una volta che Bob riceve $M \oplus S$, si può facilmente ricostruire il messaggio originale calcolando $(S \oplus M) \oplus S = M \oplus (S \oplus S) = M$ dato che 
\begin{equation*}
    x \oplus x =
    \begin{cases}
        0 \oplus 0 = 0 \\
        1 \oplus 1 = 0 
    \end{cases} \, , \quad \forall \, x \, .
\end{equation*}
Il problema di un tale protocollo crittografico, oltre al fatto che il codepad non debba essere scoperto da nessun altro al di fuori di Alice e Bob, risiede nel fatto che non sia molto efficiente a seguito del cosiddetto "one-time codepad", dato che la stringa $S$ può essere utilizzata una volta sola. Per capirne il motivo supponiamo che Alice voglia inviare entrambi i messaggi $M_1$ e $M_2$: il messaggio ricevuto da Bob è $(M_1 \oplus S) \oplus (M_2 \oplus S) = M_1 \oplus M_2$, il quale è costituito da una sequenza di 0 e 1 abbastanza randomica. Se Bob è abbastanza abile e conosce almeno una parte del messaggio che Alice voleva inviare allora ci sono possibilità che riesca a decifrare $M_1$ e $M_2$ separatamente riconoscendo degli opportuni schemi in $M_1 \oplus M_2$; tuttavia non è detto che chi riceva il messaggio sia sempre così abile !


\subsection{Il protocollo BB84}
La versione quantistica viene chiamata \textbf{protocollo BB84} ed è abbastanza simile al caso classico ma molto più potente: lo scopo è quello di creare un codepad $S$ che non possa essere in alcun modo (o quasi\footnote{Si sfrutta la natura probabilistica della QM quindi se  i qubit inviati da Alice sono in gran numero, è solo una questione di tempo prima che una terza persona venga scoperta intercettare i messaggi. Si veda la discussione di seguito per chiarimenti più espliciti.}) intercettato da una terza persona, che chiameremo Eve, la quale vuole rovinare i piani di Alice e Bob. Il protocollo funziona come segue: Alice possiede una serie di qubit che vorrebbe inviare a Bob tramite un canale sicuro; invia allora casualmente dei qubit che sono preparati nella base computazionale $C = \{ \ket{0}, \ket{1} \}$ oppure nella base di Hadamard $H = \{ \ket{+}, \ket{-} \}$ (si vedano le \eqref{basi_di_sigma_12}). Quindi Alice, prima di inviare i qubit, effettua due scelte: sceglie la base e poi sceglie uno stato di quella base da inviare. Nel frattempo, Bob riceve i qubit inviati e tiene attentamente conto dell'ordine di ricezione di questi qubit, dopodiché effettua una misurazione scegliendo randomicamente la base $C$ oppure\footnote{Ad esempio, se Alice invia delle particelle dotate di spin, Bob può scegliere, mediante un apparato simile a quello dell'esperimento di Stern e Gerlach, di misurare lo spin lungo la direzione $z$ (base $C$) oppure lungo la direzione $x$ (base $H$).} o la base $H$. Dato che Bob sceglie una delle due basi, per ogni qubit che riceve ci sono due possibilità:
\begin{itemize}
    \item Sceglie la stessa base di Alice. Ad esempio se Alice avesse scelto $(C, \ket{0})$ allora necessariamente, dai postulati della QM, sappiamo che Bob misura obbligatoriamente $(C, \ket{0})$ con probabilità 1 (nel caso in cui nessuno abbia intercettato il messaggio).
    
    \item Sceglie una base differente da quella di Alice. Ad esempio se Alice invia $(C, \ket{0})$ e Bob sceglie la base $H$ allora sappiamo che ha il 50\% di possibilità di trovare $\ket{0}$ nella propria misurazione. 
\end{itemize}

\noindent Notiamo che i risultati ottenuti da Bob nelle proprie misurazioni non sono in alcun modo correlati con le informazioni che Alice vuole inviare. Riassumendo, gli step necessari sono i seguenti:
\begin{enumerate}
    \item Alice sceglie una base e invia casualmente i qubit.
    
    \item Bob riceve i qubit tenendo conto dell'ordine di arrivo e misura con il proprio apparato scegliendo casualmente una delle due basi. 
    
    \item Alice e Bob comparano \textbf{solamente le basi} di un numero arbitrario di qubit concordato a priori dai due (alcuni e non tutti perché in generale Alice potrebbe inviare un numero altissimo di qubit) tramite una linea non sicura (ossia che può essere intercettata da Eve). Questo significa che per ogni qubit che confontano, i due si scambiano la base in cui è stata effettuata la misura, non lo stato misurato.
    
    \item Infine Bob, usando parte dei qubit inviati da Alice, ossia solamente quelli che ha misurato nella sua stessa base, può costruire un codepad comune e stabilire se qualcuno ha intercettato i qubit inviati (si veda la discussione dopo l'esempio \ref{BB84_example}) confrontando direttamente i qubit delle misure con la stessa base. 
\end{enumerate}

\noindent Per capire al meglio il funzionamento di questo meccanismo illustriamolo con un esempio.

\begin{esempio}\label{BB84_example}
    Immaginiamo che le basi scelte e i risultati delle misurazioni effettuate da Alice e Bob siano quelli mostrati nella Tabella \ref{tab:BB84}. 
    
\begin{table}[!ht]
	\centering
    \begin{tabular}{c | c c c >{\columncolor[gray]{0.8}} c >{\columncolor[gray]{0.8}} c c >{\columncolor[gray]{0.8}} c >{\columncolor[gray]{0.8}} c c c >{\columncolor[gray]{0.8}} c}
        \toprule
        \text{Alice} & \text{base} & $\rightarrow$ & $C$ & $H$ & $H$ & $C$ & $C$ & $H$ & $C$ & $H$ & $C$ \\
        \text{} & \text{qubit} & $\rightarrow$ & 0 & 1 & 0 & 0 & 0 & 0 & 1 & 0 & 1 \\
        \midrule
        \text{Bob} & \text{base} & $\rightarrow$ & $H$ & $H$ & $H$ & $H$ & $C$ & $H$ & $H$ & $C$ & $C$ \\
        \text{} & \text{qubit} & $\rightarrow$ & 1 & 1 & 0 & 0 & 0 & 0 & 1 & 1 & 1 \\
        \bottomrule
    \end{tabular}\\
    \caption{Basi scelte e rispettive misurazioni effettuate da Alice e Bob. Si noti che nelle righe dei qubit misurati si è indicato solamente il bit di informazione inviato da Alice o ottenuto da Bob, ossia: $\ket{0}, \, \ket{+} \rightarrow 0$ e $\ket{1}, \ket{-} \rightarrow 1$. Nella tabella sono state colorate in grigio le colonne corrispondenti alle misurazioni effettuate nella medesima base.}
    \label{tab:BB84}
\end{table}

\noindent Una volta che Alice ha terminato\footnote{In generale esistono varie versioni di questa procedura perché dal punto di vista pratico è molto difficile accumulare una sequenza di qubit mantenendoli tutti inalterati. Una versione alternativa più conveniente e realistica prevede Alice che invia il suo qubit e subito dopo comunica immediatamente la base che ha scelto, in maniera tale che una volta che Bob abbia ricevuto il qubit possa scartare quelli misurati in basi differenti.} la sequenza di qubit che voleva inviare, comunica a Bob, mediante un canale classico, la sequenza di basi scelte, ossia la prima riga della tabella: dalle regole della QM sappiamo che ogniqualvolta che Bob sceglie (per coincidenza) la medesima base di Alice, il risultato della misurazione che ottiene è obbligatoriamente il medesimo qubit che Alice ha scelto di inviare (a tal proposito si vedano infatti le colonne colorate). Notiamo che per una coincidenza fortuita le misurazioni nelle colonne 4 e 7 sono le medesime sebbene la base scelta fosse differente: in questa situazione, ossia quando i due sperimentatori scelgono una base diversa, Bob ottiene casualmente 0 o 1 con probabilità $1/2$. Una volta effettuata la chiamata, i due decidono di tenere solamente i risultati in cui hanno scelto le stesse basi e formano con tali misure un codepad comune: nella situazione della Tabella \ref{tab:BB84}, solo le colonne colorate hanno la stessa base, quindi il codepad non è altro che $S = 10001$. Ovviamente questo codepad è comune perchè Alice e Bob si sono scambiati, oltre alle basi, anche i qubit delle misure effettuate nella stessa base.
\end{esempio}

\noindent Per quale ragione il codepad comune formato da Alice e Bob è più protetto di quello classico ? Come interviene Eve nella trasmissione delle informazioni per capire ciò che è stato inviato ? Eve può semplicemente intercettare il qubit durante il transito: dalla QM sappiamo che è obbligata ad effettuare una misurazione, la quale disturba inevitabilmente il sistema. Dato che Eve non conosce la base in cui il qubit è stato preparato è costretta a fare una scelta ! Nel caso in cui sia fortunata, scegliendo cioè la stessa base di Alice, Eve vede il qubit inviato senza modificare lo stato, tuttavia quando sceglie la base opposta ottiene un numero casuale 0 o 1 con probabilità $1/2$ e causa il collasso dello in uno dei due stati della base utilizzata. 

\noindent Capiamo meglio questo discorso con un esempio.

\begin{esempio}
    Supponiamo che Alice abbia scelto di inviare $(C, \ket{0})$ e che Eve scelga di misurare nella base $H$ ottenendo $\ket{+}$: lo stato è ora collassato in $\ket{+}$, quindi se Bob effettua una misurazione in $C$, egli può ottenere sia $\ket{0}$ sia $\ket{1}$ con probabilità $1/2$, nonostante Alice avesse inviato $(C, \ket{0})$. Se dovesse succedere che Bob misuri $(C,\ket{1})$, allora Alice e Bob concludono che hanno misurato due stati differenti, nonostante abbiano scelto la medesima base, ma questo è impossibile dalla QM se nessuno è intervenuto sullo stato !
\end{esempio}

\noindent A seguito del collasso dello stato in uno stato di una base differente da quella scelta da Alice, può accadere che nelle colonne colorate della Tabella \ref{tab:BB84} (misurazioni con stesse basi) i due sperimentatori ottengano uno stato differente: se nessuno sta intercettando gli stati in transito questo è impossibile per le leggi della QM ! In questo modo, una volta comunicati i qubit misurati nelle stesse basi, Bob capisce che qualcuno ha interferito con i qubit che Alice sta inviando. 

\noindent Statisticamente, quante volte Eve sta ascoltando sistematicamente il messaggio e vi è una possibilità che Alice e Bob non concordino su una misura effettuata nella stessa base ? Tipicamente per $1/4$ delle volte. Il motivo è dato dal fatto che Eve può essere fortunata e misurare nella stessa base di Alice (probabilità $1/2$ per questa scelta) e inoltre anche se Eve sceglie la base sbagliata, Bob deve effettuare una misurazione in cui ottiene lo stesso qubit di Alice il 50\% delle volte: quindi $\frac{1}{2} \times \frac{1}{2} = \frac{1}{4}$, dove il primo $1/2$ deriva dalla scelta di Eve e il secondo dalla misura di Bob.  

\subsection{Quantum non-demolition measures}
Chiaramente ci si potrebbe domandare se Eve possa fare di meglio. Esiste una possibilità in cui possa misurare senza recare alcun disturbo allo stato ? Delle volte queste misure vengono chiamate in letteratura \textbf{quantum non-demolition measures}: si trattano di particolari misure in cui Eve effettua la misurazione senza disturbare lo stato oppure disturba lo stato ma è in grado di resettarlo all'originale inviato da Alice. La risposta alla domanda precedente è no per un motivo simile alla dimostrazione del teorema di No-cloning.

\noindent Supponiamo che Alice stia inviando l'insieme di stati $\ket{\phi_\mu} = \{ \ket{0}, \ket{1}, \ket{+}, \ket{-} \}$, dove $\mu = 0, 1,2,3$, e inoltre assumiamo che Eve possieda un proprio computer quantistico sul quale può effettuare operazioni. Nell'intercettare il messaggio, Eve osserva lo stato $\ket{\phi_\mu} \otimes \ket{\phi}$, dove $\ket{\phi}$ si trova nel suo computer. Supponiamo inoltre che nel suo computer ci sia un altro insieme di stati $\ket{\psi_\mu}$, con $\mu = 0,1,2,3$, tale che possa essere distinto da una misura effettuata da Eve stessa. La domanda é: esiste qualche sorta di processo quantistico (gate unitario $U$) che agisce come
\begin{equation}\label{U_non_demolition_measure}
    U \left( \ket{\phi_\mu} \otimes \ket{\phi} \right) = \ket{\phi_\mu} \otimes \ket{\psi_\mu} \, ,
\end{equation}
ossia tale che quando Eve misura $\ket{\psi_\mu}$ e legge il valore $\mu$ allora con probabilità 1 legge anche lo stesso $\mu$ che Alice sta inviando, senza però disturbare $\ket{\phi_\mu}$? La risposta è no, similmente al teorema di No-cloning. Per dimostrare questo fatto calcoliamo il prodotto scalare di ambo i membri della \eqref{U_non_demolition_measure}, il quale, come sappiamo a seguito dell'unitarietà di $U$, deve rimanere preservato:
\begin{align*}
    \left( \bra{\phi_\mu} \otimes \bra{\phi} \right) \left( \ket{\phi_\nu} \otimes \ket{\phi} \right) &\overset{?}{=} \left( \bra{\phi_\mu} \otimes \bra{\psi_\mu} \right) \left( \ket{\phi_\nu} \otimes \ket{\psi_\mu} \right) \, , \; \text{ con } \mu \neq \nu \, \text{ in generale.} \\
    \Rightarrow \qquad \braket{\phi_\mu}{\phi_\nu} \underbrace{\braket{\phi}}_1 &\overset{?}{=} \braket{\phi_\mu}{\phi_\nu} \braket{\psi_\mu}{\psi_\nu} \, , \; \forall \text{ paia di indici } (\mu,\nu) \, .
\end{align*}
Analizziamo i casi in cui $\mu \neq \nu$. Quando $(\mu = 2, \nu = 3)$ e $(\mu = 0, \nu = 1)$ (o viceversa) si ha l'identità $0 = 0$, che non è interessante (ricordare sopra gli stati $\ket{\phi_\mu}$ di Alice). Nei casi invece $(\mu = 0, \nu = 2)$, $(\mu = 0, \nu = 3)$, $(\mu = 1, \nu = 2)$ oppure $(\mu = 1, \nu = 3)$ (o viceversa) i prodotti scalari $\braket{\phi_\mu}{\phi_\nu}$ non sono nulli e possono essere semplificati ad entrambi i membri. Per queste scelte otteniamo quindi che $\braket{\psi_\mu}{\psi_\nu} = 1$, ossia $\ket{\psi_\mu} = \ket{\psi_\nu}$ a meno di una fase. Ma questo significa allora che $\ket{\psi_0} = \ket{\psi_1} = \ket{\psi_2} = \ket{\psi_3}$ e quindi, non essendo stati differenti, Eve non può in alcun modo distinguere ciò che ha inviato Alice. 

\noindent La conclusione è che non esiste alcun modo di effettuare una misura con operazioni unitarie che distingua il qubit inviato da Alice senza necessariamente disturbare il sistema. 

\section{Proprietà dei gate}
Nella sezione \ref{sec:gate} abbiamo introdotto alcuni concetti preliminari riguardanti i gate, i circuiti e i computer quantistici. In molti casi nei computer si hanno degli algoritmi, ossia una ben precisa sequenza di istruzioni, che permettono di calcolare risultati desiderati. Approfondiamo le analogie e differenze dei gate classici e quantistici.

\subsection{Gate classici: il \texttt{TOFFOLI-gate}}
In CC si hanno i bit 0 e 1 e le funzioni classiche sono tali che $f: \; \{ 0,1 \}^{\otimes n} \rightarrow \{ 0,1 \}^{\otimes m}$, sono cioè mappe da $n$ a $m$ bit classici. Quindi in generale abbiamo
\begin{equation*}
    f_i(x_1, x_2, \ldots, x_n) = \{ 0, 1 \} \, , \; \text{ dove } i = 1, \ldots , n \, , \; \text{ e } x_i = 0, 1 \, .
\end{equation*}
Si vorrebbe che il computer sia in grado di calcolare tali funzioni e che inoltre gli strumenti a disposizione siano sufficientemente efficienti per farlo: quello che uno vorrebbe è poter calcolare funzioni generali con l'ausilio di solamente pochi gate. In CC si ha che con le seguenti operazioni è possibile calcolare quasi tutti i conti di algebra e aritmetica: \texttt{NOT}, $a \rightarrow -a$; \texttt{AND}, indicato con $a \land b$ e \texttt{OR}, indicato con $a \lor b$. Questo insieme di operazioni è detto \textbf{universale} perché utilizzando questi pochi gate è possibile calcolare tutte le operazioni di aritmetica di interesse. 

\noindent Sempre nella sezione \ref{sec:gate} abbiamo osservato che il CC \textbf{non} é \textbf{reversibile}\footnote{Si pensi ad esempio al fatto che le operazioni non reversibili dissipano calore all'interno della macchina. Si tratta di tutte quelle situazioni in cui si parte con molta informazione e si giunge alla fine ad un singolo risultato, creando nel frattempo numerosi risultati di scarto.} (in generale). Si pensi ad esempio all'\texttt{AND-gate}. Nel corso degli anni si è studiato numerosi metodi per implementare operazioni reversibili: questo è possibile mediante il cosiddetto \textbf{Toffoli gate} o \texttt{cont-cont-NOT}. Il circuito classico è
\begin{center}
    \mbox
    {
        \Qcircuit @C=2em @R=1.35em 
        {
            \lstick{x} & \ctrl{1} & \rstick{x} \qw \\
            \lstick{y} & \ctrl{1} & \rstick{y} \qw \\
            \lstick{z} & \targ & \rstick{z \oplus xy} \qw 
        }
    }
\end{center}
Quando uno dei due tra $x$ e $y$ é 0, allora $xy$ è 0 e niente succede all'output di $z$. L'output si modifica solamente quando sono $1$ perché controllano entrambi il risultato di $z$: quando $xy = 1$ allora $z \oplus 1$ inverte il valore iniziale di $z$. 

\begin{esempio}[Azione Toffoli gate]
    In pratica il \texttt{TOFFOLI-gate} agisce come mostrato in Tabella \ref{tab:Toffoli}:
    \begin{table}[!ht]
	    \centering
        \begin{tabular}{ccc|ccc}
            \toprule
            $\qquad$ & \text{Bit iniziali} & $\qquad \quad$ & $\qquad \quad$ & \text{Bit finali} & \text{} \\
            \midrule
            $x$ & $y$ & $z$ & $x$ & $y$ & $z \oplus xy$ \\
            \midrule
            0 & 0 & 0 & 0 & 0 & 0 \\
            0 & 0 & 1 & 0 & 0 & 1 \\
            0 & 1 & 0 & 0 & 1 & 0 \\
            0 & 1 & 1 & 0 & 1 & 1 \\
            1 & 0 & 0 & 1 & 0 & 0 \\
            1 & 0 & 1 & 1 & 0 & 1 \\
            1 & 1 & 0 & 1 & 1 & 1 \\
            1 & 1 & 1 & 1 & 1 & 0 \\
            \bottomrule
        \end{tabular}\\
        \caption{Azione del \texttt{TOFFOLI-gate} su tutti i possibili bit.}
        \label{tab:Toffoli}
    \end{table}
\end{esempio}

\noindent E' possibile dimostrare che mediante l'uso del \texttt{TOFFOLI-gate} si possono realizzare tutte le operazioni base, quindi è universale e reversibile. Notiamo che è reversibile poiché, come evidenziato nelle ultime due righe della Tabella \ref{tab:Toffoli}, esso agisce sulle stringhe facendo una \textbf{permutazione}, la quale è invertibile. 


\subsection{Gate quantistici: reversibili e continui}
Consideriamo ora il caso dei gate quantistici. Per definizione, essendo implementati da operatori unitari, sono sempre dei gate \textbf{reversibili}. Questo significa ad esempio che
\begin{center}
    \mbox
    {
        \Qcircuit @C=2em @R=1.35em 
        {
            \lstick{\ket{\psi}} & \gate{U} & \gate{U^\dag} & \rstick{\ket{\psi}} \qw
        }
    }
\end{center}
dato che $U U^\dag = \mathbb{I}$. E' possibile implementare il \texttt{TOFFOLI-gate} anche in un computer quantistico ? La risposta è sì: consideriamo una base di stati per 3 qubit 
\begin{equation*}
    \{ \ket{000}, \ket{001}, \ket{010}, \ket{100}, \ket{101}, \ket{011}, \ket{110},  \ket{111} \} \, .
\end{equation*}
La matrice unitaria $U_T$ $8 \times 8$ che agisce sul vettore contenenti gli stati della base è
\begin{equation*}
    \begin{pmatrix}
        1 & 0 & 0 & 0 & 0 & 0 & 0 & 0 \\
        0 & 1 & 0 & 0 & 0 & 0 & 0 & 0 \\
        0 & 0 & 1 & 0 & 0 & 0 & 0 & 0 \\
        0 & 0 & 0 & 1 & 0 & 0 & 0 & 0 \\
        0 & 0 & 0 & 0 & 1 & 0 & 0 & 0 \\
        0 & 0 & 0 & 0 & 0 & 1 & 0 & 0 \\
        0 & 0 & 0 & 0 & 0 & 0 & 0 & 1 \\
        0 & 0 & 0 & 0 & 0 & 0 & 1 & 0
    \end{pmatrix}
    \begin{pmatrix}
        \ket{000} \\ \ket{001} \\ \ket{010} \\ \ket{100} \\ \ket{101} \\ \ket{011} \\ \ket{110} \\ \ket{111}
    \end{pmatrix}
    =
    \begin{pmatrix}
        \ket{000} \\ \ket{001} \\ \ket{010} \\ \ket{100} \\ \ket{101} \\ \ket{011} \\ \ket{111} \\ \ket{110}
    \end{pmatrix} \, .
\end{equation*}
Notiamo infatti che $U_T$ è unitaria ($U_T U_T^\dag = \mathbb{I}$). Tramite $U_T$ possiamo realizzare su un computer quantistico le stesse operazioni che faremmo su un computer classico. Fino ad ora non abbiamo ancora detto se effettivamente queste operazioni possano essere eseguite in maniera più efficiente su un QC. 

\noindent Il secondo fatto importante dei gate quantistici è che sono \textbf{continui}: matrici unitarie possono dipendere da parametri reali continui. Ad esempio, per un sistema di 1 qubit abbiamo visto nella \eqref{general_2by2_matrix} come si scrive la più generale matrice $2 \times 2$ unitaria (gruppo $U(2)$) tramite l'implementazione delle rotazioni di angolo $\lambda$ sulla sfera di Bloch e lungo la direzione generica $\vec{n}$ (si veda la \eqref{rotation_n_lambda}). Abbiamo visto che la \eqref{general_2by2_matrix} dipende in generale da 4 parametri reali arbitrari, quindi persino per un singolo qubit si ha un insieme continuo di gate ! Il problema è che le cose si complicano notevolmente se si passa ad un sistema generico di $n$-qubit. In tale situazione lo spazio di Hilbert associato ha dimensione $2^n$, quindi le matrici unitarie che agiscono su tale spazio sono $2^n \times 2^n$, le quali formano il gruppo $U(2^n)$. Ognuna di queste matrici contiene in generale $2^{2n}$ parametri reali ! Il problema è quindi dato dal fatto che il numero di parametri cresce esponenzialmente con il numero di qubit: il numero di gate è estremamente grande e non vogliamo un QC in cui possiamo implementare qualsiasi trasformazione con $2^{2n}$ parametri reali. Una tale situazione è troppo difficile da realizzare, tuttavia preferiamo considerare un numero limitato di gate e costruire le operazioni desiderate tramite composizione. Si noti inoltre che un insieme continuo di operazioni é impossibile per la memoria limitata di un computer. 

\noindent Il meglio che possiamo fare è introdurre una nozione di \textbf{universalità} e approssimare abbastanza bene una generica trasformazione unitaria utilizzando solamente un insieme finito di gates. Qual è il significato di tale approssimazione ? Dobbiamo definire un'opportuna nozione di \textbf{distanza} tra matrici:

\begin{definizione}[\textbf{Distanza tra matrici}]
    Date due matrici $U$ e $V$, definiamo la seguente funzione \textbf{distanza}
    \begin{equation*}
        E(U,V) = \max\limits_{\ket{\psi}} \norm{(U-V) \ket{\psi}} \, ,
    \end{equation*}
    dove $\ket{\psi}$ è un vettore arbitrario.
\end{definizione}

\noindent E' possibile trovare un insieme discreto di matrici tale che tutte le possibili matrici unitarie possano essere realizzate a partire da tale insieme a meno di un errore $\varepsilon$ arbitrario ? La risposta è sì: un possibile insieme di gate che soddisfa la precedente nozione di "approssimazione universale" è dato dai seguenti 4
\begin{center}
    \mbox{
        \Qcircuit @C=2em @R=2em {
            & \gate{H} & \qw \\
        }
    } 
    , \ \ \ \ 
    \mbox{
        \Qcircuit @C=2em @R=2em {
            & \gate{S} & \qw \\
        }
    }
    , \ \ \ \ 
    \mbox{
        \Qcircuit @C=2em @R=2em {
            & \gate{T} & \qw \\
        }
    }
    , \ \ \ \
    \mbox{
        \Qcircuit @C=2em @R=1em {
            & \ctrl{1} & \qw \\
            & \targ & \qw
        }
    }
\end{center}
Si vedano esplicitamente le matrici \eqref{Hadamard_matrix} e \eqref{S_T_matrices}. Questi gate prendono il nome di \texttt{H-gate} $H$ (Hadamard gate), \texttt{Phase-gate} $S$, \texttt{$\pi/4$-gate} $T$ e il \texttt{CNOT-gate}. Si noti che il nome della matrice $T$ deriva dal fatto che
\begin{equation*}
    T = e^{i \frac{\pi}{8}} 
    \begin{pmatrix}
        e^{-i \frac{\pi}{8}} & 0 \\ 0 & e^{i \frac{\pi}{8}}
    \end{pmatrix} \, .
\end{equation*}
Non lo dimostriamo esplicitamente, ma l'insieme di questi 4 gate è universale. E' importante sottolineare che $H, S$ e $T$ sono gate agenti sui singoli qubit, mentre il \texttt{CNOT-gate} agisce sempre su almeno 2 qubit. Dato che avevamo visto che $T^2 = S$ potrebbe sorgere spontanea la domanda: perché è necessario considerare entrambi $T$ e $S$ ? Di solito si preferisce tenere anche $S$ per la cosiddetta \textbf{Fault tolerance computation}, che approfondiremo quando parleremo di propagazione degli errori nei circuiti quantistici. 

\noindent Alcune importanti proprietà che ci servirà sapere sui gate quantistici sono le seguenti:
\begin{enumerate}
    \item \textit{Tutti i gate agenti su $n$ qubit possono essere approssimati come un prodotto di un opportuno numero di gate agenti su $2$ qubit}. 
    
    \noindent Chiaramente il numero di gate agenti su 2 qubit deve essere sufficientemente grande per approssimare una matrice $2^n \times 2^n$ (matrice agente su $n$ qubit). Per capire il significato di questa affermazione si pensi al circuito seguente di 6 qubit:
    \vspace{-1.2cm}
    \begin{center}
        \mbox{
            \Qcircuit @C=2em @R=0.12em {
                & \multigate{5}{U} & \qw \\
                & \ghost{U} & \qw \\
                & \ghost{U} & \qw \\
                & \ghost{U} & \qw \\
                & \ghost{U} & \qw \\
                & \ghost{U} & \qw \\
            }
            $
            \quad
            \begin{matrix}
                \\
                \\
                \\
                \\
                \simeq \\
            \end{matrix}
            \quad
            $
            \Qcircuit @C=2em @R=0.12em {
                & \multigate{1}{U_1} & \qw \\
                & \ghost{U_1} & \qw \\
                & \multigate{1}{U_2} & \qw \\
                & \ghost{U_2} & \qw \\
                & \multigate{1}{U_3} & \qw \\
                & \ghost{U_3} & \qw \\
            }
        }
    \end{center}
    In questo esempio il gate originale $U$ è stato approssimato fattorizzandolo in 3 gate agenti ciascuno localmente solo su 2 qubit: il numero di operazioni per ricostruire la matrice $2^n \times 2^n$ del circuito a LHS è di ordine $\order{2^{2n}}$. Chiaramente si tratta solamente di algebra: questo non è un modo molto efficiente di approssimare un gate agente su $n$ qubit perché tipicamente si hanno comunque $2^{2n}$ fattori da tenere in considerazione. 
    
    \item \textit{I gate agenti su 2 qubit possono essere scritti in termini di un \texttt{CNOT-gate} e di un gate agente su un singolo qubit}.
    
    \noindent Notiamo che, a differenza della proprietà precedente, questa proprietà è esatta e può essere svolta senza alcuna approssimazione. In termini di circuiti stiamo dicendo che 
    \begin{center}
        \mbox{
            \Qcircuit @C=2em @R=0.12em {
                & \multigate{1}{U_4} & \qw \\
                & \ghost{U_4} & \qw \\
            }
            $
            \quad
            \begin{matrix}
                \\
                = \\
            \end{matrix}
            \quad
            $
            \Qcircuit @C=2em @R=0.5em {
                & \ctrl{1} & \qw \\
                & \targ & \qw \\
            }
            $
            \quad
            \begin{matrix}
                \\
                + \\
            \end{matrix}
            \quad
            $
            \Qcircuit @C=2em @R=0.12em {
                & \gate{U_2} & \qw \\
            }
        }
    \end{center}
    dove $U_4 \in U(4)$ e $U_2 \in U(2)$. In generale è possibile dimostrare che per costruire una generica matrice di $U(4)$ si possono utilizzare due opportune matrici $A, B \in U(2)$ tali che $\comm{A}{B} \neq 0$. 
    
    \item \textit{I gate agenti sui singoli qubit possono essere approssimati come prodotto di matrici $H$ e $T$ con un errore, il quale può essere arbitrariamente scelto più piccolo di $\varepsilon$}.
    
    Questo significa che data $V$ la generica matrice unitaria $2 \times 2$ da approssimare (ricordare che contiene $2^2 = 4$ parametri reali) possiamo scrivere un'opportuna sequenza di prodotti tra $H$ e $T$ tali che
    \begin{equation*}
        E(V, \ldots H H T H \ldots T \ldots H \ldots) < \varepsilon \, .
    \end{equation*}
    Chiaramente più lunga è la sequenza più piccolo sarà l'errore entro il quale si può approssimare $V$. In realtà $H$ e $T$ non sono matrici speciali: questo argomento funziona con qualsiasi $A, B \in U(2)$ tali che $\comm{A}{B} \neq 0$. Matematicamente questa proprietà è dovuta al fatto che il sottogruppo dato dai prodotti di $H$ e $T$ è \textbf{denso} in $U(2)$. 
    
    \noindent Ci si può chiedere se un'approssimazione mediante prodotti di $H$ e $T$ possa essere efficiente. Ancora una volta, fortunatamente la risposta è sì: esiste un teorema, chiamato \textbf{teorema di Solovay-Kitaev}, che stabilisce che il numero di prodotti tra $H$ e $T$ per approssimare una generica matrice di $U(2)$ è dell'ordine di $\order{\log_{10}^c(1/\varepsilon)}$ dove $c \sim 2$. 
    
\end{enumerate}

    %%%%%%%%%%%%%
% LECTURE 6 %
%%%%%%%%%%%%%
\vspace{1cm}

\noindent \lecture{6}{22/10/2021}

\section{Quantum Parallelism}

\begin{definizione}[\textbf{Quantum Parallelism}]
    Il \textbf{quantum parallelism} è una delle caratteristiche fondamentali di molti algoritmi quantistici. Consente ai computer quantistici di valutare una funzione $f(x)$ per molti valori diversi di $x$ contemporaneamente.
\end{definizione}
\noindent Supponiamo di considerare la più semplice funzione possibile $f(x):\{0,1\}^{\otimes n} \rightarrow \{0,1\}$ definita su un dominio (insieme di numeri costruiti con $n$ cifre di 0 e 1) e a elementi in un intervallo di bit. Assumiamo inoltre di saper calcolare efficientemente nel nostro computer tale funzione. Ciò che calcoliamo, dal punto di vista della computazione classica, lo possiamo valutare nella computazione quantistica, pertanto tutte le operazioni aritmetiche possono essere svolte dal calcolo quantistico. Un modo quindi di calcolare questa funzione su un computer quantistico è quello di considerare due differenti stati: immaginiamo un qubit $\ket{y}$ e uno stato che può essere un prodotto tensoriale di qubit, come ad esempio $\ket{0}^{\otimes n}$. Spesso considereremo lo stato $\ket{0}^{\otimes n}$ come stato iniziale in cui il computer quantistico viene preparato mediante una misurazione nella base computazionale perché è facilmente costruibile: ad esempio nel caso $n=3$ se, a seguito di una misurazione, lo stato nel QC collassa in $\ket{\psi} \rightarrow \ket{1} \otimes \ket{0} \otimes \ket{1}$, basterà applicare un \texttt{X-gate} al primo e al terzo qubit per costruire lo stato voluto $\ket{0}^{\otimes 3}$.

\noindent Chiamiamo lo stato iniziale totale $\ket{x,y}$, dove $x$ contiene l'informazione iniziale data in input e $y$ conterrà, dopo delle opportune operazioni, il risultato cercato. Con un'appropriata sequenza di gate è possibile effettuare la trasformazione
\begin{equation}\label{black_box_U_f}
    \ket{x,y} \overset{U_f}{\longrightarrow} \ket{x,y \oplus f(x)} \, ,
\end{equation}
dove $U_f$ è un opportuno gate unitario che implementa l'operazione desiderata. Il circuito che implementa la \eqref{black_box_U_f} è 
\begin{center}
    \mbox{
        \Qcircuit @C=1em @R=1em {
            \lstick{\ket{x}} & \multigate{1}{U_f} & \rstick{\ket{x}} \qw \\
            \lstick{\ket{y}} & \ghost{U_f} & \rstick{\ket{y\oplus f(x)}} \qw
        }
    }
\end{center}
dove $\ket{x}$ prende il nome di \textbf{data register} e $\ket{y}$ prende il nome di \textbf{target register}. Questa rappresentazione è utile perché quando $\ket{y} = \ket{0}$ l'output del target register è esattamente l'oggetto che si vuole calcolare
\begin{center}
    \mbox{
        \Qcircuit @C=1em @R=1em {
            \lstick{\ket{x}} & \multigate{1}{U_f} & \rstick{\ket{x}} \qw \\
            \lstick{\ket{0}} & \ghost{U_f} & \rstick{\ket{0\oplus f(x)}=\ket{f(x)}} \qw
        }
    }
\end{center}
Notiamo che la \eqref{black_box_U_f} è invertibile: se applichiamo $U_f$ due volte, otteniamo:
\begin{equation*}
    \ket{x,y} \rightarrow \ket{x, y \oplus f(x)} \rightarrow \ket{x,y \oplus f(x) \oplus f(x)} = \ket{x,y} \, ,
\end{equation*}
siccome $f(x) \oplus f(x) = 0$ indipendentemente dai valori di $f$. Fino ad ora avremmo potuto effettuare tutte queste operazioni in CC. L'importanza del QC risiede nel fatto che si possano considerare sovrapposizioni di stati appartenenti ad una base. Consideriamo il caso $n = 1$ (il data register è un qubit) e assumiamo il seguente stato iniziale
\begin{equation*}
    \ket{x,y} \equiv \underbrace{\frac{1}{\sqrt 2} (\ket{0}+\ket{1})}_{\ket{x}} \otimes \underbrace{\ket 0}_{\ket{y}} = \frac{1}{\sqrt 2} \left( \ket{00}+\ket{10} \right) \, ;
\end{equation*}
Se assumiamo che il computer sia preparato in $\ket{0} \otimes \ket{0}$ come possiamo rappresentare $\ket{x,y}$ in un circuito? Possiamo sfruttare l'\texttt{H-gate} in questo modo:
\begin{center}
    \mbox{
        \Qcircuit @C=1em @R=1em {
            \lstick{\ket{0}} & \gate{H} & \multigate{1}{U_f} & \qw \\
            \lstick{\ket{0}} & \qw & \ghost{U_f} & \qw
            %\gategroup{1}{4}{2}{4}{0.8em}{\}}
        }
    }
\end{center}
infatti, utilizzando la \eqref{black_box_U_f}, avremo
\begin{equation*}
    \ket{0,0} \overset{H}{\longrightarrow} \frac{1}{\sqrt{2}} \left( \ket{00}+\ket{10} \right) \overset{U_f}{\longrightarrow} \frac{1}{\sqrt{2}} \left( \ket{0, f(0)} + \ket{1, f(1)} \right) \, .
\end{equation*}
Questo circuito è particolarmente interessante perché l'output è una sovrapposizione di differenti stati contenenti informazioni riguardo la funzione: $f(0)$ e $f(1)$ appaiono simultaneamente nel medesimo stato. È come se avessimo valutato $f(x)$ per due valori di $x$ contemporaneamente, parallelamente! A differenza del classic parallelism, in cui più circuiti vengono costruiti per calcolare $f(x)$ ed eseguiti simultaneamente, qui viene impiegato un singolo circuito per valutare la funzione $f(x)$ per più valori di $x$ nello stesso momento: si sta sfruttando la capacità di un computer quantistico di essere in sovrapposizioni di stati diversi. Qui risiede il \textbf{quantum parallelism}.

\noindent Questo discorso può essere facilmente generalizzato al caso di $n$-qubit. Supponiamo che il data register si trovi in $\ket{0}^{\otimes n}$. Usiamo il fatto che l'azione dell'\texttt{H-gate} su $n$-qubit possa essere scritta nel seguente modo:
\begin{align}
    H^{\otimes n}\ket{0}^{\otimes n} &= \underbrace{H\otimes \cdots \otimes H}_{n\text{-volte}} \underbrace{\ket{0} \otimes \cdots \otimes \ket{0}}_{n\text{-volte}} = \frac{1}{\sqrt 2} (\ket 0 + \ket 1) \otimes \cdots \otimes \frac{1}{\sqrt 2} (\ket 0 + \ket 1) \notag \\
    &= \frac{1}{\sqrt{2^n}}(\ket{000 \ldots 0} + \ket{010 \ldots 0} + \ldots + \ket{111 \dots 1}) = \frac{1}{\sqrt{2^n}}\sum_{x=0}^{2^n-1}\ket{x} \, , \label{n_H_gates}
\end{align}
dove $x$ rappresenta tutte le possibili stringhe di $n$-volte $0$ e $1$. Se il target si trova in $\ket{y} = \ket{0}$ e applichiamo ora $U_f$, il risultato è:
\begin{equation*}
    \frac{1}{\sqrt{2^n}}\sum_{x=0}^{2^n-1}\ket{x} \otimes \ket{0} \overset{U_f}{\longrightarrow} \frac{1}{\sqrt{2^n}}\sum_{x=0}^{2^n-1}\ket{x,f(x)} \, ,
\end{equation*}
dove si è fatto uso della \eqref{black_box_U_f} con $\ket{y} = \ket{0}$. In termini di circuiti avremo
\begin{center}
    \mbox{
        \Qcircuit @C=1em @R=1em {
            \lstick{\ket{0}^{\otimes n}} & \gate{H^{\otimes n}} & \multigate{1}{U_f} & \qw & \qw \\
            \lstick{\ket{y} = \ket{0}} & \qw & \ghost{U_f}   & \qw      & \qw
        }
    }
\end{center}
In un certo senso, il quantum parallelism consente di valutare simultaneamente tutti i possibili valori della funzione $f(x)$, anche se apparentemente abbiamo valutato $f(x)$ in una singola volta. Precisiamo che la misura dello stato nel caso del qubit singolo ci darà solamente $\ket{0, f(0)}$ oppure $\ket{1, f(1)}$. In maniera analoga per il caso generale, la misura dello stato $\sum_x\ket{x,f(x)}$ ci darà un solo $f(x_0)$ per un singolo valore casuale $x_0$. Ovviamente un computer classico può farlo più facilmente! La computazione quantistica richiede qualcosa di più del semplice quantum parallelism per essere utile; richiede cioè la capacità di estrarre informazioni su più di un valore di $f(x)$ da stati di sovrapposizione, come $\sum_x\ket{x,f(x)}$. Come vedremo nella prossima sezione, il trucco di considerare una sovrapposizione lineare ci permetterà di estrarre alcune informazioni su $f$ in un modo più efficiente del CC.

\section{Algoritmo di Deutsch}
Una semplice modifica del circuito precedente dimostra come i circuiti quantistici possano essere più performanti rispetto a quelli classici. Nelle ultime righe del paragrafo precedente abbiamo detto che la computazione quantistica richiede qualcosa di più oltre al quantum parallelism per essere utilizzabile. L'\textbf{algoritmo di Deutsch} combina il meccanismo del \textbf{quantum parallelism} con la proprietà della meccanica quantistica dell'\textbf{interferenza}. 

\noindent Si tratta di un algoritmo un po' accademico (le funzioni sono banali), tuttavia utile per illustrare l'idea di algoritmo quantistico. Lasciamo che entrambi input e output register contengano ciascuno un solo qubit, quindi stiamo esplorando le funzioni $f(x)$ che convertono un singolo bit in un singolo bit: $f(x): \; \{ 0,1 \} \rightarrow \{ 0,1 \}$. Ci sono due modi piuttosto diversi di pensare a tali funzioni. Il primo modo è notare che ci sono solo quattro di queste funzioni, come mostrato nella Tabella \ref{tab:Deutsch_Fnct}.

\begin{table}[!ht]
	\centering
    \begin{tabular}{ccc}
        \toprule
        & $x = 0$ & $x=1$ \\
        \midrule
        $f_0$ & $0$ & $0$ \\
        $f_1$ & $0$ & $1$ \\
        $f_2$ & $1$ & $0$ \\
        $f_3$ & $1$ & $1$ \\
        \bottomrule
    \end{tabular} \\
    \caption{Possibili output delle sole quattro funzioni distinte $f_j(x)$ che convertono un bit in un bit. Esse sono tutte facilmente implementabili sia in un computer classico che quantistico.}
    \label{tab:Deutsch_Fnct}
\end{table}

\noindent Supponiamo che ci venga data una black-box (ossia un gate ignoto che indicheremo con \texttt{U-gate}) che calcola una di queste quattro funzioni eseguendo la seguente trasformazione unitaria:
\begin{equation*}
    U_{f_j} \ket{x,y} = \ket{x, y \oplus f_j(x)} \, .
\end{equation*}
In questo caso, se implementiamo in circuiti la Tabella \ref{tab:Deutsch_Fnct} avremo:

\begin{center}
    \mbox{
        $
        \begin{matrix}
             \\
             \\
            f_0: \\
        \end{matrix}
        $
        \Qcircuit @C=1em @R=1em {
            & \multigate{1}{U_{f_0}} & \qw \\
            & \ghost{U_{f_0}}& \qw \\
        }
        $
        \begin{matrix}
             \\
             \\
            \ = \\
        \end{matrix}
        $
        \Qcircuit @C=1em @R=1.9em {
            & \qw & \qw & \qw & \qw & \qw \\
            & \qw & \qw & \qw & \qw & \qw \\
        }
    }
    \qquad \qquad
    \mbox{
        $
        \begin{matrix}
             \\
             \\
            f_1: \\
        \end{matrix}
        $
        \Qcircuit @C=1em @R=1em {
            & \multigate{1}{U_{f_1}} & \qw \\
            & \ghost{U_{f_1}}& \qw \\
        }
        $
        \begin{matrix}
             \\
             \\
            \ = \\
        \end{matrix}
        $
        \Qcircuit @C=1em @R=1.35em {
            & \ctrl{1} & \qw & \qw \\
            & \targ & \qw & \qw  \\
        }
    }
\end{center}
\begin{center}
    \mbox{
        $
        \begin{matrix}
             \\
             \\
            f_2: \\
        \end{matrix}
        $
        \Qcircuit @C=1em @R=1em {
            & \multigate{1}{U_{f_2}} & \qw \\
            & \ghost{U_{f_2}}& \qw \\
        }
        $
        \begin{matrix}
             \\
             \\
            \ = \\
        \end{matrix}
        $
        \Qcircuit @C=1em @R=1.15em {
            & \qw & \ctrl{1} & \qw \\
            & \gate{X} & \targ & \qw \\
        }
    }
    \qquad \qquad
    \mbox{
        $
        \begin{matrix}
             \\
             \\
            f_3: \\
        \end{matrix}
        $
        \Qcircuit @C=1em @R=1em {
            & \multigate{1}{U_{f_3}} & \qw \\
            & \ghost{U_{f_3}}& \qw \\
        }
        $
        \begin{matrix}
             \\
             \\
            \ = \\
        \end{matrix}
        $
        \Qcircuit @C=1em @R=1.25em {
            & \qw & \qw \\
            & \gate{X} & \qw \\
        }
    }
\end{center}
Dato che la regola che vogliamo implementare è $\ket{x,0} \rightarrow \ket{x, f(x)}$ ($\ket{y}$ inizializzato a $\ket{0}$), in termini matematici questo significa scrivere:
\begin{align*}
    &f_0: &\ket{x,0} &\longrightarrow \ket{x,0} \, , \\
    &f_1: &\ket{x,0} &\overset{\texttt{CNOT}}{\longrightarrow} 
    \begin{cases}
        \ket{0,0} \, , &\text{per } x = 0 \\
        \ket{1,1} \, , &\text{per } x = 1
    \end{cases} \, , \\
    &f_2: &\ket{x,0} &\overset{X}{\longrightarrow} \ket{x,1} \overset{\texttt{CNOT}}{\longrightarrow}
    \begin{cases}
        \ket{0,1} \, , &\text{per } x = 0 \\
        \ket{1,0} \, , &\text{per } x = 1
    \end{cases} \, , \\
    &f_3: &\ket{x,0} &\overset{X}{\longrightarrow} \ket{x,1} \, , 
\end{align*}

\noindent Supponiamo che ci venga data una black-box che esegua $U_f$ per una delle quattro funzioni, ma non ci venga detto quale delle quattro operazioni. Ovviamente possiamo scoprirlo lasciando agire due volte la black-box, prima su $\ket0 \otimes \ket0$ e poi su $\ket 1 \otimes \ket 0$. Ma supponiamo di poter far agire la black-box solo una volta. Cosa possiamo conoscere di $f(x)$?

\noindent In un computer classico, dove siamo effettivamente limitati a lasciare che la black-box agisca sui qubit in uno dei quattro stati di base computazionale, possiamo conoscere il valore di:
\begin{itemize}
    \item $f(0)$, lasciando che $U_f$ agisca su uno dei due $\ket0 \otimes \ket0$ o $\ket0 \otimes \ket1$;
    \begin{itemize}
        \item In tal caso possiamo limitare la scelta a $f_0$ o $f_1$ (se $f(0) = 0$) oppure $f_2$ o $f_3$ (se $f(0) = 1$).
    \end{itemize}
    \item $f(1)$, lasciando che $U_f$ agisca su $\ket1 \otimes \ket0$ o $\ket1 \otimes \ket1$;
    \begin{itemize}
        \item In questa situazione abbiamo ristretto la funzione ad essere $f_0$ o $f_2$ (se $f(1) = 0$) oppure $f_1$ o $f_3$ (se $f(1) = 1$).
    \end{itemize}
\end{itemize}
In definitiva, un computer classico necessita di due esecuzioni per determinare se $f$ sia costante o meno. Sorprendentemente, risulta che con un computer quantistico questo non è necessario perché il problema può essere risolto con una singola esecuzione. Il punto interessante è che l'algoritmo non riguarda il calcolo preciso della funzione, ma piuttosto la comprensione di una o più sue proprietà: quando l'algoritmo viene lanciato non impariamo nulla sui valori individuali di $f(0)$ e $f(1)$, ma siamo comunque in grado di rispondere alla domanda sui loro valori relativi. Chiaramente otteniamo meno informazioni di quelle che otterremmo rispondendo alla domanda con un computer classico, ma, rinunciando alla possibilità di acquisire quella parte dell'informazione che è irrilevante per la domanda a cui vogliamo rispondere, possiamo ottenere la risposta con una sola applicazione di $U_f$.

\noindent Come sottolineato in precedenza l'algoritmo combina il quantum parallelism e l'interferenza: possiamo preparare il computer nello stato $\ket 0 \otimes \ket 1$ della base canonica e applicare l'\texttt{H-gate} a entrambi i qubit: 
\begin{equation}\label{eq:Deutsch_1}
    (H\otimes H) \ket{0} \otimes \ket{1} = \underbrace{\frac{\ket0 + \ket1}{\sqrt 2}}_{\substack{\text{quantum} \\ \text{parallelism}}} \otimes \underbrace{\frac{\ket0-\ket1}{\sqrt 2}}_{\text{interferenza}} \, ;
\end{equation}
in un circuito significa scrivere
\begin{center}
    \mbox{
        \Qcircuit @C=1em @R=1em {
            \lstick{\ket{0}} & \gate{H} & \multigate{1}{U_f} & \qw \\
            \lstick{\ket{1}} & \gate{H} & \ghost{U_f} & \qw
        }
    }
\end{center}
\vspace{0.2cm}
Chiamando per semplicità $\ket{x} \equiv \{ \ket{0} ,  \ket{1} \}$ e applicando $U_f$ alla \eqref{eq:Deutsch_1} tramite \eqref{black_box_U_f}, possiamo esplicitamente vedere che cosa implica il termine di interferenza:
\begin{align*}
    &\ket{x} \otimes \frac{1}{\sqrt 2} (\ket 0 - \ket 1) \overset{U_f}{\longrightarrow} \frac{1}{\sqrt 2} \left( \ket{x, 0 \oplus f(x)} - \ket{x, 1 \oplus f(x)} \right) \\
    &=
    \begin{cases}
        \frac{1}{\sqrt 2} \left( \ket{x, 0 \oplus 0} - \ket{x, 1 \oplus 0} \right) = \ket x \otimes \frac{1}{\sqrt 2} (\ket 0 - \ket 1) \, , \quad &\text{per } f(x) = 0 \\
        \frac{1}{\sqrt 2} \left( \ket{x, 0 \oplus 1} - \ket{x, 1 \oplus 1} \right) = - \ket x \otimes \frac{1}{\sqrt 2} (\ket 0 - \ket 1) \, , \quad &\text{per } f(x) = 1
    \end{cases} \, .
\end{align*}
Combinando i due casi in un'unica espressione compatta abbiamo ottenuto
\begin{equation}\label{black_box_action_U_f_on_x}
    \ket{x} \otimes \frac{1}{\sqrt 2} (\ket 0 - \ket 1) \overset{U_f}{\longrightarrow} (-1)^{f(x)} \ket x \otimes \frac{1}{\sqrt 2} (\ket 0 - \ket 1) \, ,
\end{equation}
Sostituendo $\ket{x}$ con lo stato iniziale che implementava il quantum parallelism avremo
\begin{equation*}
    \frac{\ket{0} + \ket{1}}{\sqrt{2}} \otimes \frac{\ket 0 - \ket 1}{\sqrt 2} \overset{U_f}{\longrightarrow} \frac{1}{\sqrt{2}} \left[ (-1)^{f(0)} \ket 0 + (-1)^{f(1)} \ket 1 \right] \otimes \frac{\ket 0 - \ket 1}{\sqrt 2} \, ;
\end{equation*}
dato che il segno relativo nella parentesi quadra dipende dal fatto che $f(0)$ e $f(1)$ siano uguali o meno, possiamo riscrivere quest'ultima espressione come
\begin{equation*}
    \begin{cases}
        (-1)^{f(0)}\frac{\ket 0 + \ket 1}{\sqrt 2}\otimes\frac{\ket 0-\ket 1}{\sqrt 2} \, , &\text{per }f(0) = f(1) \\
        (-1)^{f(0)}\frac{\ket 0 - \ket 1}{\sqrt 2}\otimes\frac{\ket 0-\ket 1}{\sqrt 2} \, , &\text{per }f(0) \neq f(1) 
    \end{cases} \, .
\end{equation*}
Come ultimo passaggio si applica l'\texttt{H-gate} al primo qubit in maniera tale che il circuito totale diventi:
\begin{center}
    \mbox{
        \Qcircuit @C=1em @R=1em {
            \lstick{\ket{0}} & \gate{H} & \multigate{1}{U_f} & \gate{H} & \qw \\
            \lstick{\ket{1}} & \gate{H} & \ghost{U_f} & \qw & \qw
        }
    }
\end{center}
Questa modifica trasforma il risultato precedente in 
\begin{equation*}
    \begin{cases}
        (-1)^{f(0)}\frac{\ket 0 + \ket 1}{\sqrt 2}\otimes\frac{\ket 0-\ket 1}{\sqrt 2} \overset{H}{\longrightarrow} (-1)^{f(0)}\ket 0\otimes\frac{\ket 0-\ket 1}{\sqrt 2} \, , &\text{per }f(0) = f(1) \\
        (-1)^{f(0)}\frac{\ket 0 - \ket 1}{\sqrt 2}\otimes\frac{\ket 0-\ket 1}{\sqrt 2} \overset{H}{\longrightarrow} (-1)^{f(0)}\ket 1\otimes\frac{\ket 0-\ket 1}{\sqrt 2} \, , &\text{per }f(0) \neq f(1) 
    \end{cases} \, .
\end{equation*}
Il risultato finale ci suggerisce che possiamo effettuare solamente una misurazione sul primo qubit: ottenendo $\ket{0}$ o $\ket{1}$ siamo in grado, con una singola misura, di stabilire se $f(0) = f(1)$ oppure $f(0) \neq f(1)$. Questo significa che siamo in grado di escludere 2 delle 4 funzioni con una singola esecuzione dell'algoritmo. 

\noindent Questo esempio permette di evidenziare quale sia la differenza tra il quantum parallelism e gli algoritmi randomizzati classici. Ingenuamente, si potrebbe pensare che lo stato finale corrisponda piuttosto a un calcolatore classico probabilistico che valuta $f(0)$ con probabilità $\frac 12$, o $f(1)$ con probabilità $\frac 12$. La differenza è che in un computer classico queste due alternative si escludono sempre mentre in un computer quantistico è possibile che le due alternative interferiscano l'una con l'altra per ottenere alcune proprietà globali della funzione $f(x)$. Utilizzando un opportuno gate (nel nostro caso l'\texttt{H-gate}) siamo in grado di ricombinare le diverse alternative.

\section{Algoritmo di Deutsch-Jozsa}
L'algoritmo di Deutsch è un semplice caso di un algoritmo quantistico più generale, noto come \textbf{algoritmo di Deutsch-Jozsa}, che evidenzia esplicitamente come il QC offra un grosso miglioramento rispetto ai metodi del CC. Supponiamo di avere una black-box che calcoli una funzione booleana $f(x): \; \{0,1\}^{\otimes n}\rightarrow \{0,1\}$ e supponiamo di sapere per certo che $f(x)$ sia solamente una delle seguenti alternative:
\begin{itemize}
    \item \textbf{Funzione costante} (\textit{constant}): l'output è sempre $0$ oppure $1$ indipendentemente dall'input.
    \item \textbf{Funzione bilanciata} (\textit{balanced}): l'output è costituito per metà dal valore $0$ e metà dal valore $1$.
\end{itemize}
Lo scopo dell'algoritmo è quello di capire quale delle due sia l'alternativa corretta con il minor numero di esecuzioni. Classicamente potremmo risolvere questo problema calcolando $2^{n-1}+1$ valori della funzione perché è necessario calcolare almeno una metà dei valori più un valore aggiuntivo. Chiaramente si tratta di un numero esponenzialmente grande. Quello che fa l'algoritmo di Deutsch-Jozsa è risolvere il problema perfettamente con una sola query quantistica. Cominciamo scrivendo il circuito che descrive tale algoritmo, il quale è molto simile a quello di Deutsch con la sola differenza che il data register non è un singolo qubit, ma piuttosto un prodotto tensoriale di $n$-qubit:
\begin{center}
    \mbox{
        \Qcircuit @C=1em @R=1em {
            \lstick{\ket{0}^{\otimes n}} & \gate{H^{\otimes n}} & \multigate{1}{U_f} & \gate{H^{\otimes n}} & \qw \\
            \lstick{\ket{1}} & \gate{H} & \ghost{U_f} & \qw & \qw
        }
    }
\end{center}
Vediamo nello specifico cosa succede all'interno del circuito:
\begin{enumerate}
    \item Viene inizializzato (preparato) lo stato in $\ket{0}^{\otimes n} \otimes \ket{1}$;
    \item Creiamo una sovrapposizione di stati usando l'\texttt{H-gate} su tutti gli $n+1$ qubit:
        \begin{equation*}
            \ket{0}^{\otimes n} \otimes \ket{1} \overset{H}{\longrightarrow} \frac{1}{\sqrt {2^n}}\sum_{x=0}^{2^n-1}\ket x \otimes \frac{\ket 0 - \ket 1}{\sqrt 2} \, ,
        \end{equation*}
        dove si è fatto uso della \eqref{n_H_gates}. Notiamo che ora nell'output register è presente lo stato che nella sezione precedente avevamo visto essere associato all'interferenza. 
    \item Valutiamo la funzione $f(x)$ usando la black-box di $U_f$
        \begin{equation}\label{dopo_U_f}
            \frac{1}{\sqrt {2^n}}\sum_{x=0}^{2^n-1}\ket x \otimes \frac{\ket 0 - \ket 1}{\sqrt 2} \overset{U_f}{\longrightarrow} \sum_{x=0}^{2^n-1}\frac{(-1)^{f(x)}}{\sqrt{2^n}}\ket x \otimes \frac{\ket 0 - \ket 1}{\sqrt 2} \, ,
        \end{equation}
        dove, essendo $\ket{x}$ arbitrario, abbiamo fatto uso della \eqref{black_box_action_U_f_on_x}. 
    \item Applichiamo nuovamente l'\texttt{H-gate} ai primi $n$ qubit. Per capire il risultato di $H^{\otimes n} \ket{x}$ consideriamo per semplicità il caso $n=1$: formalmente avremo 
    \begin{equation*}
        H \ket{x} = \sum_{z = 0}^1 \frac{(-1)^{xz}}{\sqrt{2}} \ket{z} \, , \; \text{ dove } x = 0 \text{ oppure } 1 \, .
    \end{equation*}
    Per $n$ generico possiamo generalizzare scrivendo
    \begin{align*}
        H^{\otimes n} \ket{x} &= (H \otimes \ldots \otimes H) \ket{x_0} \otimes \ket{x_1} \otimes \ldots \otimes \ket{x_{n-1}} \\
        &= \sum_{z_0=0}^1 \ldots \sum_{z_{n-1}=0}^1 \frac{(-1)^{x_0 z_0} (-1)^{x_1 z_1} \cdots (-1)^{x_{n-1} z_{n-1}}}{\sqrt{2^n}} \ket{z} \, ,
    \end{align*}
    dove $\ket{z} \equiv \ket{z_0, z_1, \ldots, z_{n-1}}$. In maniera più compatta possiamo scrivere quindi l'azione dell'\texttt{H-gate} sugli $n$ qubit (nonché risultato finale del circuito) come
        \begin{equation}\label{output_Deutsch_Jozsa}
            \sum_{z = 0}^{2^n-1} \sum_{x = 0}^{2^n-1} \frac{(-1)^{f(x) + x \cdot z}}{2^n}\ket z \otimes \frac{\ket 0 - \ket 1}{\sqrt 2} \, ,
        \end{equation}
        dove abbiamo indicato con $x\cdot z$ il \textbf{prodotto bit a bit modulo 2}:
        \begin{equation*}
            x\cdot z = (x_0z_0 + \ldots + x_{n-1}z_{n-1}) \mod{2} \, .
        \end{equation*}
    \item Infine misuriamo per ottenere lo stato finale $\ket{z}$. 
\end{enumerate}

\noindent Ricordiamo che il problema è quello di determinare se $f$ sia constant o balanced. Notiamo dal risultato in \eqref{output_Deutsch_Jozsa} che il data register ora contiene una sovrapposizione lineare di tutti i possibili stati che si scrivono come stringhe contenenti $n$ volte 0 e 1. In $\ket{z}$ è presente un caso particolare: consideriamo la situazione in cui $\ket z = \ket{00\ldots0} = \ket0^{\otimes n}$ e cerchiamo la probabilità di ottenere tale stato guardando il modulo quadro del coefficiente:
\begin{equation*}
    P \left( \ket{z} = \ket0^{\otimes n} \right) = \abs{\sum_{x=0}^{2^n-1}\frac{(-1)^{f(x)}}{2^n}}^2 = 
    \begin{cases}
    1 \, , &\text{se } f(x) \text{ è constant} \\
    0 \, , &\text{se } f(x) \text{ è balanced} 
    \end{cases} \, .
\end{equation*}
Notiamo che quando la probabilità è 1 a numeratore si hanno $2^n$ termini tutti uguali ($(-1)^1$ oppure $(-1)^0$) che si semplificano con il fattore $1/2^n$; quando invece la probabilità è nulla a numeratore si ha uno stesso numero di $(-1)^1$ e $(-1)^0$ che si cancellano esattamente. Come abbiamo detto $\ket z=\ket0^{\otimes n}$ è un caso particolare molto importante perché permette di risolvere il problema mediante la misura dello stato. Se misurando $z$ otteniamo $\ket{0}^{\otimes n}$ allora, con probabilità 1 (quindi sempre), lo stato è $\ket{0}^{\otimes n}$ e la funzione è constant; al contrario quando la misura di $z$ produce un qualsiasi stato differente da $\ket{0}^{\otimes n}$ allora, necessariamente $P \left( \ket{z} = \ket0^{\otimes n} \right) = 0$, lo stato $\ket{0}^{\otimes n}$ non è nemmeno presente in $z$ e possiamo stabilire con assoluta certezza che la funzione è balanced. Il fatto importante è che essendo queste misure mutuamente esclusive, possiamo determinare se $f$ sia constant o balanced con una singola misurazione. Quindi si tratta di effettuare una sola misurazione in QC contro $\mathcal{O}(2^n)$ misure in CC.

\noindent Osserviamo che il confronto tra algoritmi classici e quantistici è in qualche modo un confronto delicato, poiché il metodo per valutare la funzione è abbastanza diverso nei due casi. Se fosse consentito utilizzare un computer probabilistico classico, per valutare $f(x)$ per pochi $x$ scelti a caso, si può determinare molto rapidamente con alta probabilità se $f(x)$ è \textit{constant} o \textit{balanced}. Questo scenario probabilistico è forse più realistico dello scenario deterministico che abbiamo considerato.

\noindent Ribadiamo nuovamente che questo algoritmo è un esempio molto accademico in quanto non esistono problemi fisici o matematici reali che necessitano di sapere se una funzione sia constant o balanced. Nonostante ciò il fatto importante è che grazie a questo algoritmo quantistico non è più necessario aspettare un tempo esponenzialmente\footnote{Talvolta non si vuole sapere con precisione assoluta se $f$ sia constant o balanced, ma è sufficiente stabilirlo entro un errore dato $\varepsilon$. Un ipotetico algoritmo classico e probabilistico di questo tipo diventa di ordine polinomiale in $n$: passare da $\mathcal{O}(\text{polinomio in }n)$ a $\mathcal{O}(1)$ mediante la controparte quantistica non è più un miglioramento così estremo come passare da $\mathcal{O}(2^n)$ ad $\mathcal{O}(1)$!} crescente nel numero di bit per sapere il risultato. 

\section{Algoritmo di Bernstein-Vazirani}
Consideriamo un altro algoritmo di black-box per il quale gli algoritmi quantistici forniscono un vantaggio: l'\textbf{algoritmo di Bernstein-Vazirani}. Qui, a differenza dei due casi precedenti, abbiamo accesso alla funzione della black-box $f: \{0, 1\}^n \rightarrow \{0, 1\}$. Supponiamo che la funzioni sia data da\footnote{Come prima il simbolo "$\cdot$" indica il prodotto bit a bit modulo 2}:
\begin{equation*}
    f(x) = a\cdot x = (a_0 x_0 + \ldots + a_{n-1} x_{n-1})\mod{2} \, , \; \text{ dove } a \geq 0 \text{ e } x < 2^n \, .
\end{equation*}
Sappiamo che la funzione è lineare, tuttavia l'obiettivo di questo algoritmo è trovare il valore di $a$. Classicamente, questo problema potrebbe richiedere $n$ query poiché ogni query può fornire solo un nuovo bit di informazioni su $a$, ma $a$ possiede $n$ bit: dobbiamo valutare $f(1000\ldots) = a_0$, $f(0100\ldots) = a_1$ e così via con $n$ valutazioni fino a $f(111\ldots1) = a_{n-1}$. L'algoritmo di Bernstein-Vazirani, invece, risolve il problema quantisticamente utilizzando una sola query!

\noindent Consideriamo il medesimo circuito dell'algoritmo di Deutsch-Josza e il suo output in \eqref{output_Deutsch_Jozsa}: nel caso in cui $f(x) = a \cdot x$ esso diventa 
\begin{equation*}
    \sum_{z=0}^{2^n-1}\sum_{x=0}^{2^n-1}\frac{(-1)^{x\cdot (a+z)}}{2^n}\ket z \otimes \frac{\ket 0 - \ket 1}{\sqrt 2} \, .
\end{equation*}
Come nell'algoritmo precedente guardiamo il coefficiente di $\ket{z}$:
\begin{equation*}
        \frac{1}{2^n} \sum_{x=0}^{2^n-1}(-1)^{x\cdot (a+z)} = \frac{1}{2^n} \sum_{x=0}^{2^n-1}(-1)^{x_0(a_0+z_0) + \ldots + x_{n-1}(a_{n-1}+z_{n-1})} = \frac{1}{2^n} \prod_{j=0}^{n-1} \left( \sum_{x_j=0}^{1}(-1)^{x_j(a_j+z_j)} \right) \, ,
\end{equation*}
ma ogni termine nella parentesi tonda è la somma di termini che possono essere $\pm 1$ a seconda dell'esponente. Distinguiamo i due casi:
\begin{itemize}
    \item Se $(a_j+z_j=0)\mod2 $ allora il coefficiente è 
    \begin{equation*}
        \frac{1}{2^n} \prod_{j=0}^{n-1}(2) = 1 \, , \quad \Rightarrow \quad \text{Probabilità } 1 \, .
    \end{equation*}
    \item Al contrario quando $(a_j+z_j=1)\mod2$ allora il coefficiente diventa 
    \begin{equation*}
        \frac{1}{2^n} \prod_{j=0}^{n-1} \left[ (-1)^{0\cdot 1} + (-1)^{1 \cdot 1} \right] = 0 \, , \quad \Rightarrow \quad \text{Probabilità } 0 \, .
    \end{equation*}
\end{itemize}
Ancora una volta, i due casi della probabilità sono mutuamente esclusivi e quindi avremo
\begin{align*}
    &(a_j+z_j=0)\mod2 \, , \; \forall \, j \, ,  &\Rightarrow& \quad a = z \, , &\Rightarrow& \quad \text{Probabilità } 1 \, , \\
    &(a_j+z_j=1)\mod2 \, , \;  \text{per qualche } j \, , &\Rightarrow& \quad a \neq z \, , &\Rightarrow& \quad \text{Probabilità } 0 \, .
\end{align*}
Questo significa che il nostro stato, in realtà, non è una sovrapposizione lineare, ma contiene bensì solamente lo stato
\begin{equation*}
    \ket a \otimes \frac{\ket 0 - \ket 1}{\sqrt 2} \, ;
\end{equation*}
e quindi attraverso un'unica operazione di misura sui primi $n$-qubit, otteniamo $a$, la nostra incognita.
    %%%%%%%%%%%%%%%%%%%%%%%
%%%%%% Lezione 7 %%%%%%
%%%%%%%%%%%%%%%%%%%%%%%

\vspace{1.0cm}
\lecture{7}{25/10/2021}
\vspace{1.0cm}

\section{Screening}

Abbiamo osservato che l'interazione tra due cariche in un gas omogeneo di elettroni o in generale, in un qualque sistema di interazione tra elettroni, l'interazione è schermata dalle altre cariche presenti. Possiamo formalizzare questa idea con la \textbf{teoria della risposta lineare}

\subsection{Teoria della risposta lineare}
Questa teoria assume che la densità è lineare. Consideriamo un cambiamento nel $V_{\text{ext}}$ ed è tale per cui genera un cambiamento, a sua volta, nella densità in una relazione lineare:
\begin{equation*}
    \delta V_{\text{ext}}(\overline{x}') \longrightarrow \delta n(\overline x) = \int \dd[3]{\overline{x}'}\chi(\overline x, \overline{x}')\delta V_{\text{ext}}(\overline{x}')
\end{equation*}
$\chi(\overline x, \overline{x}')$ prende il nome di \textbf{suscettività} e in un gas omogeneo di elettroni dipende solo della distanza tra $\overline x$ e $\overline{x}'$:
\begin{equation*}
    \chi(\overline x, \overline{x}')=\chi(|\overline x - \overline{x}'|)
\end{equation*}
Questa espressione può essere introdotta usando le trasformazioni di Fourier, possiamo scegliere qualunque definizione, noi scegliamo quella con il fattore $(2\pi)^3$ a denominatore:
\begin{equation*}
    \delta n(\overline x) = \frac{1}{(2\pi)^3}\int \dd[3]{q} \delta n(\overline q) e^{i\overline q \cdot \overline x}
\end{equation*}
\begin{equation*}
    \delta n(\overline q) = \frac{1}{(2\pi)^3}\int \dd[3]{x} \delta n(\overline x) e^{-i\overline q \cdot \overline x}
\end{equation*}
Ora possiamo usare questa definizione per scrivere la risposta lineare:
\begin{equation*}
    \begin{aligned}
        \delta n(\overline x) &=\frac{1}{(2\pi)^3}\int \dd[3]q\delta n(\overline q)e^{i\overline q \cdot \overline x} \\
        & = \int \dd[3]{x'}\chi(|\overline x - \overline{x}'|)\frac{1}{(2\pi)^3}\int \dd[3]{q}\delta V_{\text{ext}}(\overline q)e^{i\overline q \cdot \overline x} \\
        & = \int \dd[3]{x'}\chi(|\overline x - \overline{x}'|)\frac{1}{(2\pi)^3}\int \dd[3]{q}\delta V_{\text{ext}}(\overline q)e^{i\overline q \cdot \overline x}e^{i\overline q \cdot \overline x}e^{-i\overline q \cdot \overline x} \\
        & = \frac{1}{(2\pi)^3}\int \dd[3]{q}e^{i\overline q \cdot \overline x}\int\dd[3]{x'}\chi(|\overline x - \overline{x}'|)e^{i\overline q \cdot (\overline{x}'-\overline{x})}\delta V_{\text{ext}}(\overline q)
    \end{aligned}
\end{equation*}
Confrontando la prima e l'ultima riga troviamo che:
\begin{equation*}
    \begin{aligned}
        \delta n(\overline q) &=\underbrace{\int \dd[3]{x'}\chi(|\overline x - \overline{x}'|)e^{i\overline q \cdot (\overline{x}' - \overline x)}}_{\text{Trasformata di Fourier di } \chi(-\overline{q})}\delta V_{\text{ext}}(\overline q) \\
        & = \chi(-\overline{q})\delta V_{\text{ext}}(\overline q)
    \end{aligned}
\end{equation*}
Siccome in un gas omogeneo di elettroni, la suscettività dipende dal modulo di $\overline q$, pertanto:
\begin{equation*}
    \delta n(\overline q) = \chi(\overline q)\delta V_{\text{ext}}(\overline q)
\end{equation*}
Questo è sempre vero, se consideriamo la convoluzione tra le due funzioni, abbiamo:
\begin{equation*}
    \delta n(\overline x)=\int \dd[3]{x'}\chi(\overline x, \overline{x}')\delta V_{\text{ext}}(\overline{x}')
\end{equation*}
Questa $\chi$ è la cosiddetta \textbf{suscettività longitudinale} perché $\dots$
Il cambiamento su $V_{\text{ext}}$ induce, come abbiamo detto, un cambiamento in $n(\overline x)$ che è responsabile del potenziale totale:
\begin{equation*}
    \begin{aligned}
        \delta V_{\text{tot}}(\overline x) &=\delta V_{\text H}+ \delta V_{\text{ext}} \\
        & = \int \dd[3]{x'}\frac{e_0^2}{4\pi\varepsilon_0|\overline{x} - \overline{x}'|}\delta n(\overline{x}') + \delta V_{\text{ext}}
    \end{aligned}
\end{equation*}
Scritto in termini del reticolo reciproco:
\begin{equation*}
    \delta V_{\text{tot}}(\overline q)=V_{\text C}(\overline q)\delta n (\overline q)+\delta V_{\text{ext}}(\overline q)
\end{equation*}
Infatti è possibile applicare la trasformata di Fourier al potenziale coulombiano:
\begin{equation*}
    \frac{e_0^2}{4\pi\varepsilon_0}\frac{1}{|\overline x|} \longrightarrow V_{\text C}(\overline q)
\end{equation*}
Questo potenziale può essere valutato nell'equazione di Poisson, ricordando il caso di una carica posta nell'origine:
\begin{equation*}
    \nabla^2 \frac{e_0^2}{4\pi\varepsilon_0|\overline{x}|}=-\frac{e_0^2}{\varepsilon_0}\delta(\overline x)
\end{equation*}
\begin{equation*}
    \nabla^2 \frac{1}{|\overline x|}= -4\pi\delta(\overline x)
\end{equation*}
Considerando la trasformata di Fourier e prendendone il laplaciano, abbiamo:
\begin{equation*}
    \begin{aligned}
        \nabla^2\frac{1}{(2\pi)^3}\int \dd[3]{q}V_{\text C}(q)e^{i\overline q \cdot \overline x} & =\frac{1}{(2\pi)^3}\int \dd[3]{q}(-q^2)V_{\text C}(q)e^{i\overline q \cdot \overline x} \\
        & = -\frac{e_0^2}{\varepsilon_0}\delta(\overline x) \text{ introduco la T.F. della } \delta\\
        & = -\frac{e_0^2}{\varepsilon_0}\frac{1}{(2\pi)^3}\int \dd[3]{q}\delta(\overline q)e^{i \overline q \cdot \overline x}
    \end{aligned}
\end{equation*}
Confrontando la seconda espressione con quest'ultima, troviamo:
\begin{equation*}
    -q^2V_{\text{C}}(\overline q)=-\frac{e_0^2}{\varepsilon_0}\delta(\overline q)
\end{equation*}
Esplicitando $V_{\text{C}}$
\begin{equation*}
    V_{\text{C}}(\overline q) = \frac{e_0^2}{\varepsilon_0 q^2}\delta(\overline q) = \frac{e_0^2}{\varepsilon_0 q^2}
\end{equation*}
Questo perché:
\begin{equation*}
    \delta q = \int \dd[3]{x} \delta(\overline x) e^{-i\overline q \cdot \overline x}=1
\end{equation*}
Possiamo inserire questo risultato nel potenziale totale:
\begin{equation*}
    \delta V_{\text{tot}}(\overline q)=\delta n(\overline q)\frac{e_0^2}{\varepsilon_0q^2}+\delta V_{\text{tot}}(\overline q)
\end{equation*}
Possiamo introdurre un ulteriore oggetto: la \textbf{funzione dielettrica}. Questo può essere introdotto attraverso lo studio dell'elettromagnetismo, ricordiamo che:
\begin{equation*}
    \overline D = \varepsilon \overline E
\end{equation*}
dove
\begin{equation*}
    \divergence{D}=\rho_{\text{ext}}(\overline x)
\end{equation*}
Se consideriamo un campo longitudinale, $\overline D$ è il campo generato dalle cariche esterne, mentre $\overline E$ rappresenta il campo totale.
\begin{equation*}
    \underbrace{\overline D}_{\mathllap{\text{esterno}}} = \varepsilon \underbrace{\overline E}_{\mathrlap{\text{totale}}}
\end{equation*}
da cui abbiamo:
\begin{equation*}
    \delta V_{\text{ext}}(\overline q) = \varepsilon(\overline q)\delta V_{\text{tot}}(\overline q)
\end{equation*}
cioè:
\begin{equation*}
    \delta V_{\text{tot}}(\overline q) = \frac{\delta V_{\text{ext}}(\overline q)}{\varepsilon(\overline q)}
\end{equation*}
Siccome $\varepsilon < 1$, il potenziale totale è maggiore del potenziale esterno. Torniamo alla definizione di potenziale totale:
\begin{equation*}
    \begin{aligned}
        \delta V_{\text{tot}}(\overline q) & =\delta n(\overline q)V_{\text C}(\overline q)+\delta V_{\text{ext}}(\overline q) \\
        & = \chi(\overline q)\delta V_{\text{ext}}(\overline q)V_{\text C}(\overline q)+\delta V_{\text{ext}}(\overline q) \\
        & = \delta V_{\text{ext}}(\overline q)(1+\chi(\overline q)V_{\text C}(\overline q))
    \end{aligned}
\end{equation*}
Mettendo insieme le due equazioni si ottiene:
\begin{equation*}
    \frac{1}{\varepsilon(\overline q)}=1+\chi(\overline q)V_{\text C}(\overline q)
\end{equation*}

\subsection{Suscettività dielettrica per un gas omogeneo di elettroni}
Possiamo pensare di calcolare la suscettività dielettrica per un gas omogeneo di elettroni, tuttavia non possiamo risolverla in maniera analitica, ma possiamo introdurre alcune approssimazioni. La più semplice delle approssimazione che si può fare è stivare $\varepsilon(\overline q)$ nel \textbf{modello di Thomas-Fermi}. Ricordiamo che l'energia è espressa come:
\begin{equation*}
    E\big[n(\overline x)\big]=\int \dd[3]{x} V_{\text{ext}}(\overline x)n(\overline x)+E_{\text H}\big[n(\overline x)\big]+\int \dd[3]{x}n(\overline x)\frac 35 \varepsilon_{\text F}(n(\overline x))
\end{equation*}
Attraverso la minimizzazione di questo funzionale di energia possiamo trovare la densità che lo minimizza:
\begin{equation*}
    \functionalderivative{E}{n(\overline x)}=\mu
\end{equation*}
Ricordiamo che $\mu$ è un moltiplicatore di Lagrange perché abbiamo questo vincolo $\int \dd[3]{x} n(\overline x)=N$. Pertanto, introducendo un cambiamento nel potenziale esterno, avremo:
\begin{equation*}
    V_{\text{ext}} \rightarrow V_{\text{ext}}+\delta V_{\text{ext}}
\end{equation*}
\begin{equation*}
    n(\overline{x}) \rightarrow n(\overline x)+\delta n(\overline x)
\end{equation*}
L'equazione che minimizza il funzionale energetico porta all'equazione di Thomas-Fermi:
\begin{equation*}
    \varepsilon_{\text F}(n(\overline x))+V_{\text H}(\overline x) + V_{\text{ext}}(\overline x) = \mu
\end{equation*}
Abbiamo però un cambiamento su $V_{\text{ext}}$:
\begin{equation*}
    \varepsilon_{\text{F}}(n(\overline x)+\delta n(\overline x))+V_{\text H}+\delta V_{\text{H}}+V_{\text{ext}}+\delta V_{\text{ext}}=\mu
\end{equation*}
Siamo nella teoria della risposta lineare, pertanto il cambiamento è piccolo e possiamo svilupparlo al primo ordine. Ricordando che:
\begin{equation*}
    \varepsilon_{\text{F}}=\frac{\hbar^2}{2m}(3\pi^2n)^{\frac 13}
\end{equation*}
Possiamo espandere per piccoli cambiamenti:
\begin{equation*}
    \varepsilon_{\text{F}}(n(\overline x))+\frac{\hbar^2}{2m}(3\pi^2)^{\frac 23}\frac 23 \frac{n^{\frac 23}}{n}\delta n
\end{equation*}
Inserendo questo risultato troviamo che tutti i termini imperturbati si semplificano e rimangono solamente:
\begin{equation*}
    \frac 23 \frac{\varepsilon_{\text F}}{n}\delta n(\overline x)+\delta V_{\text H}(\overline x)+\delta V_{\text{ext}}(\overline x)=0
\end{equation*}
Ora vogliamo studiare il cambiamento del potenziale in un gas omogeneo di elettroni. Quando abbiamo discusso del gas omogeneo di elettroni, abbiamo introdotto la densità di stati $D(\varepsilon) \propto \sqrt{\varepsilon}$. In particolar modo abbiamo visto che:
\begin{equation*}
    D(\varepsilon_{\text{F}})=\frac 32 \frac n{\varepsilon_{\text F}}
\end{equation*}
In questo modo possiamo riscrivere l'equazione precedente come:
\begin{equation*}
    \frac{\delta n(\overline x)}{D(\varepsilon_{\text F})}+\delta V_{\text H}+\delta V_{\text{ext}}=0
\end{equation*}
Possiamo scrivere la stessa equazione nel reticolo reciproco utilizzando la trasformata di Fourier:
\begin{equation*}
    \frac{\delta n(\overline q)}{D(\varepsilon_{\text F})}+\delta n(\overline q)V_{\text C}(\overline q)+\delta V_{\text{ext}}(\overline q)=0
\end{equation*}
\begin{equation*}
    \delta n(\overline q)\Bigg[\frac{1}{D(\varepsilon_{\text{F}})}+V_{\text{C}}(\overline q)\Bigg]=-\delta V_{\text{ext}}(\overline q)
\end{equation*}
\begin{equation*}
    \delta n(\overline q)= -\frac{D(\varepsilon_{\text F})}{(1+D(\varepsilon_{\text{F}})V_{\text{C}}(\overline q))}\delta V_{\text{ext}}(\overline q)
\end{equation*}
Da cui ricaviamo:
\begin{equation*}
    \chi(\overline q)=-\frac{D(\varepsilon_{\text F})}{(1+D(\varepsilon_{\text{F}})V_{\text{C}}(\overline q))}
\end{equation*}
Se conosciamo $\chi(\overline q)$ possiamo ottenere:
\begin{equation*}
    \begin{aligned}
        \frac{1}{\varepsilon(\overline q)} &=1+\chi(\overline q)V_{\text{C}}(\overline q) \\
        & = 1-\frac{D(\varepsilon_{\text F})}{1+D(\varepsilon_{\text F})V_{\text C}(\overline q)}V_{\text C}(\overline q) \\
        & = \frac{1+D(\varepsilon_{\text F})V_{\text C}(\overline q)- D(\varepsilon_{\text F})V_{\text{C}}(\overline q)}{1+D(\varepsilon_{\text F})V_{\text{C}}(\overline q)} \\
        & = \frac{1}{1+D(\varepsilon_{\text F})V_{\text{C}}(\overline q)}
    \end{aligned}
\end{equation*}
Invertendo la relazione troviamo:
\begin{equation*}
    \begin{aligned}
        \varepsilon(\overline q) &=1+D(\varepsilon_{\text F})V_{\text{C}}(\overline q) \\
        & = 1+\frac{D(\varepsilon_{\text F})e_0^2}{\varepsilon_0 q^2} \\
        & = 1+ \frac{q_{\text{TF}}^2}{q^2}
    \end{aligned}
\end{equation*}
Dove abbiamo introdotto:
\begin{equation*}
    q_{\text{TF}}^2=\frac{D(\varepsilon_{\text{F}})e_0^2}{\varepsilon_0}=\frac{1}{\lambda_{\text{TF}}}
\end{equation*}
In particolar modo siamo interessati al suo inverso $\lambda_{\text{TF}}$ che prende il nome di \textbf{lunghezza di screening di Thomas-Fermi}. \\
Viene introdotto questo concetto perché possiamo calcolare
\begin{equation*}
    \delta V_{\text{tot}}(\overline q)=\frac{\delta V_{\text{ext}}(\overline q)}{\varepsilon(\overline q)}
\end{equation*}
Se consideriamo come potenziale esterno il potenziale generato da una carica puntiforme: $\delta V_{\text{ext}}(\overline q)=\frac{Ze_0^2}{\varepsilon_0 q^2}$, otteniamo:
\begin{equation*}
    \delta V_{\text{tot}}(\overline q)=\frac{Ze_0^2}{\varepsilon_0q^2\bigg(1+\frac{q_{\text{TF}}^2}{q^2}\bigg)}=\frac{Ze_0^2}{\varepsilon_0(q^2+q_{\text{TF}}^2)}
\end{equation*}
Applicando la trasformata di Fourier e il teorema dei residui (in coordinate sferiche), si trova:
\begin{equation*}
    \delta V_\text{tot}(\overline x)=\frac{1}{(2\pi)^3}\int\dd[3]{q}\frac{Ze_0^2}{\varepsilon_0(q^2+q_{\text{TF}}^2)}e^{i \overline q \cdot \overline x} = \frac{Ze_0^2}{4\pi\varepsilon_0r}e^{-\frac r {\lambda_{\text{TF}}}}
\end{equation*}
Ricordando $2 < r_{\text S} < 6$ e $\varepsilon_\text{F}$ inserendoli in $q_{\text{TF}}^2$ troviamo che:
\begin{equation*}
    \lambda_{\text{TF}}=r_{\text S}^{\frac 12}a_0\frac{2\pi}{\big(\frac{12}{\pi}\big)^\frac 13}\approx 4.01 r_{\text S}^{\frac 12}a_0
\end{equation*}
\end{document}
