\vspace{0.5cm}
\newline 
\noindent \lecture{15}{30/11/2021}
\vspace{0.5cm}
\noindent Nel caso in cui all'estremo dell'elettrodo ci siano due giunzioni, si parla di TRANSMON simmetrico.
\begin{figure}
    \centering
    \begin{circuitikz}
        \draw (0,0)
        to[barrier=$\Phi_e$] (0,2) 
        to[short] (2,2)
        to[C] (2,0) 
        to[short] (-2,0)
        to[barrier] (-2, 2)
        to[short] (0,2);
    \end{circuitikz}    
    
    \caption{}
\end{figure}
Questa configurazione permette di variare la corrente critica a piacimento, tramite l'applicazione del campo esterno $\Phi_e$:
\begin{equation*}
    I_0' = 2I_o \abs{\cos \frac{\pi \phi_e}{\phi_o}}
\end{equation*}
L'hamiltoniana, a questo punto, diventa:
\begin{equation*}
    H=4E_cn^2-2E_J\left|\cos(\pi\frac{\Phi_e}{\Phi_0})\right|\cos\delta
\end{equation*}
Perciò possiamo ora variare sia $E_C$ (progettando il circuito adeguatamente) e $E_J$ (tramite il campo esterno). L'anarmonicità risulta inoltre data da $\pi\frac{\Phi_e}{\Phi_0}=k\pi/2$ (con k interi). Abbiamo, dunque, la capacità di scegliere in modo arbitrario la frequenza di risonanza del qubit (caratteristica chiaramente necessaria, in particolare nel caso di utilizzo di più qubit contemporaneamente).
Si noti che differenti varianti del TRANSMON simmetrico sono state proposte (ad esempio usando due giunzioni differenti si può diminuire la dipendenza dal flusso esterno della frequenza del qubit).
Questo tipo di circuiti, tuttavia, porta con sé alcuni problemi di fabbricazione: in particolare le giunzioni devono avere dimensioni che non possono essere ottenute con litografia ottica (è necessaria la più complessa litografia a raggi ionici). Inoltre la produzione di tutti i componenti deve essere svolta nel medesimo vuoto in modo da avere superfici il più possibile uniformi.

\section{Controllo di un qubit}

\subsection{TRANSMON Qubit}

Il motivo per cui si utilizza la configurazione "a più" dell'XMON è che in questo modo abbiamo 3 punti di accoppiamento fra la giunzione e l'esterno. Tramite questi punti potremo leggere lo stato del qubit e modificarlo.
Il controllo del qubit (e dunque l'applicazione dei gate) viene detto \textit{XY control}: è costituito da una piccola capacità $C_d$ (\textit{d} di \textit{drive}, come viene chiamato il segnale) collegata a una sorgente \textit{RF} (la presenza della capacità serve a tenere separati i due componenti del circuito, in modo da non diminuire i tempi di decoerenza).
L'hamiltoniana del sistema, approssimando il qubit a un semplice oscillatore armonico, risulta (con $Q=\partialderivative{\mathcal L}{\dot{\Phi}}=\dot{\Phi}(C+C_d)$):
\begin{equation*}
    H_0 = \frac{Q^2}{2(C+C_d)^2} +\frac{\Phi^2}{2L}
\end{equation*}
Alla quale bisogna aggiungere un ulteriore termine di drive:
\begin{equation*}
    H_d = C_d \dot \Phi V_d(t)=\frac{C_d}{C+C_d}QV_d(t)
\end{equation*}
A questo punto possiamo scrivere (in funzione della carica di fluttuazione di punto zero $Q_{\text{ZPF}}=\sqrt{\frac{\hbar}{2z}}$, con $z=\sqrt{\frac{L}{C}}$):
\begin{equation*}
    Q= -i Q_{\text{ZPF}}(a - a^\dagger)
\end{equation*}
E infine possiamo scrivere l'hamiltoniana totale ($\omega_q=\frac{\sqrt{8E_JE_C}-E_C}{\hbar}$):
\begin{equation*}
    H = \hbar \omega_q \left(a^\dagger a + \frac{1}{2}\right)- i\frac{C_d}{C_{TOT}}V_d(t)Q_{ZPF}(a-a^\dagger)
\end{equation*}
Modulando $V_d(t)$ dimostreremo che è possibile controllare lo stato del qubit ruotando lo spazio di Bloch.
L'hamiltoniana, scritta con le matrici di Pauli, risulta (concentrandoci solo sui primi due livelli):
\begin{equation*}
    H = - \sigma_z\frac{\hbar\omega_q}{2}+\frac{C_d}{C_{\text{TOT}}}V_d(t)Q_{ZPF}\sigma_y
\end{equation*}

\subsection{Phase Qubit}
In modo simile possiamo ragionare per il qubit di fase. Ricordiamo che il qubit di fase è costituito da un a giunzione Josephson in parallelo a una capacità, il tutto accoppiato con una corrente di bias $I_B$ (non c'è alcun elettrodo isolato per le coppie di Cooper, dunque).
\begin{equation*}
    \begin{aligned}
        \hat H &= E_C \hat{n}^2 - E_J \cos {\hat \delta} - E_J\frac{\hbar}{2e} I_B \hat \delta \\
        &=E_C\hat{n}^2-E_J\left(\cos\hat\delta+\frac{I_B}{I_0}\hat \delta\right) \\
        &= E_C \hat{n}^2 - \frac{I_0}{2\pi}\Phi _0 \cos{\hat \delta}- \frac{I_0}{2\pi}\Phi_0 \hat \delta
    \end{aligned}
\end{equation*}
La forma della curva del potenziale è determinata dal valore scelto per $I_B$. Aggiungendo (in serie) una corrente variabile nel tempo, possiamo controllare il qubit a piacimento.
Dunque abbiamo ora $I_B=I_{B_0}+\Delta I(t)$ (con $I_{B_0}$ che definisce i livelli energetici $E_0$ e $E_1$ e i rispettivi autostati $\ket 0$ e $\ket 1$.
L'hamiltoniana risulta data da:
\begin{equation*}
    \hat H' = \hat H_0 - \Delta I (t) \frac{\Phi_0}{2\pi}\hat \delta
\end{equation*}
Come abbiamo imparato, l'hamiltoniana considera infiniti livelli mentre noi approssimeremo il sistema solo ai primi due livelli.
\begin{equation*}
    \hat H' = \sum_{ij}\ket{i}\bra{i}\hat H'\ket{j}\bra{j}
\end{equation*}
Il primo termine di $\hat H'$ è già diagonale nella base di $\ket 0-\ket 1$, il secondo termine è leggermente più complicato e necessita dell'approssimazione di \textit{rotating wave} (che vedremo tra poco).
Alla fine otteniamo:
\begin{equation*}
    \hat H = - \hat\sigma_z\frac{\hbar\omega_q}{2}+\frac{\Delta I(t)\Phi_0}{2\pi}\delta_{ZPF}\hat\sigma_x
\end{equation*}
Dunque otteniamo un'hamiltoniana dipendente da $\sigma_x$ e un'altra matrice di Pauli ($\sigma_y$). Questi sono gli ingredienti fondamentali per il controllo XY.

\subsection{Oscillazioni di Rabi}

Facciamo un passo indietro. La matrice densità nello spazio di un qubit può essere scritta come:
\begin{equation*}
   \rho=\frac{1}{2}(1+\sum\rho_i\sigma_i)
\end{equation*}
E si può rappresentare come un vettore unitario in 3 dimensioni: $\vec\rho=(\rho_x,\rho_y,\rho_z)$ o, in coordinate sferiche, $\vec\rho=(cos\phi sin\theta,sin\phi sin\theta,cos\theta)$.
In ugual modo l'hamiltoniana può essere scritta come vettore:
\begin{equation*}
   H = H_x \sigma_x +  H_y \sigma_y  +  H_z \sigma_z \rightarrow \vec H = \left( H_x , H_y , H_z \right)
\end{equation*}
In generale si parla del termine con $\sigma_z$ come di termine di stato di qubit, mentre $\sigma_x$ e $\sigma_y$ sono adibiti al controllo del qubit. Questo perché possiamo generalmente scrivere l'hamiltoniana come:
\begin{equation*}
    H = \epsilon\sigma_z+\Delta(t)(\sigma_x+\sigma_y)
\end{equation*}
Abbiamo anche visto che l'evoluzione della matrice densità è descritta dall'equazione di Neumann-Liouville $i\hbar \partialderivative{\rho}{t} = \left[ H, \rho\right]$. Nel caso che stiamo analizzando abbiamo dei vettori e, dunque:

\begin{equation*}
 \partialderivative{\rho}{t}=\frac{1}{\hbar}\vec H\times\vec\rho
\end{equation*}
Chiaramente devo anche considerare i tempi di \textit{dephasing} e \textit{decoherence}, perciò otterrei, tramite i termini di Limblad:
\begin{align*}
    \dot\rho_x=[H\times\rho]_x-\frac{\rho_x}{T_2}  \\
    \dot\rho_y=[H\times\rho]_y-\frac{\rho_y}{T_2}   \\
    \dot\rho_z=[H\times\rho]_z-\frac{\rho_z-\bar{\rho_z}}{T_1}   
\end{align*}
Questi termini descrivono un moto di precessione attorno al vettore $\vec H$. Dunque, variando il termine dipendente dal tempo di $H$ e aspettando un certo tempo, possiamo far variare a piacimento il vettore $\vec \rho$.
Inoltre, dopo un certo tempo e a temperatura bassa (in modo da non eccitare il qubit), il vettore $\vec \rho$ si troverà a coincidere con $\vec H$ (dunque conosceremo lo stato iniziale del sistema). 

\subsection{Approssimazione di rotating wave (RWA)}
L'approssimazione di \textit{Rotating Wave} (indicata generalmente con RWA) consiste nel passare nel sistema di riferimento in rotazione e cancellare i termini dell'hamiltoniana in rotazione con frequenza elevata. Questa approssimazione è valida nel caso in cui stiamo operando intorno alla frequenza di risonanza e ci permette di semplificare i conti in diversi momenti.
Abbiamo appena visto che, se $\vec \rho$ e $\vec H$ non sono paralleli avremo una rotazione di $\vec \rho$. Possiamo cambiare il sistema di riferimento in modo che questa rotazione sia soppressa. 
Se la rotazione ha una frequenza caratteristica $\omega_q=\frac{E_1-E_0}{\hbar}$ la trasformazione unitaria per il cambio di sistema di riferimento sarà:

\begin{equation*}
    U_{RF}= e^{i \frac{H_0 t}{\hbar}}
\end{equation*}
E, chiaramente, siamo in grado di ricavare la trasformazione contraria con facilità.
Immaginiamo, ad esempio, di avere uno stato già in evoluzione:
\begin{equation*}
    \ket{\psi (t)}= U\ket{\psi_0}=e^{-i H_0 t/\hbar}\ket{\psi _0}
\end{equation*}
Passando al sistema di riferimento in rotazione avremo:
\begin{equation*}
    \ket{\psi_{rf}}=U_{rf}\ket{\psi(t)}=\ket{\psi_0}
\end{equation*}
Dunque abbiamo dimostrato che il vettore di stato cessa di ruotare. L'hamiltoniana (scritta inizialmente come $H=H_0+H(t)$), è modificata:
\begin{equation*}
   i\hbar\partialderivative{\ket{\psi_{rf}}}{t}=i\hbar\left(\partialderivative{U}{t}\ket{\psi_0}+U\partialderivative{\ket{\psi_0}}{t}\right)=\\
   i\hbar\dot{U}U^\dagger\ket{\psi_{rf}}+UHU^\dagger\ket{\psi_{rf}}=H_{rf}\ket{\psi_{rf}}
\end{equation*}
Dunque abbiamo:
\begin{equation*}
    H_{rf}= i\hbar \dot U U^\dagger + U H U^\dagger 
\end{equation*}