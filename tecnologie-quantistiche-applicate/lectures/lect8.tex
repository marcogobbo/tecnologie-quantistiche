\vspace{0.5cm}

\noindent  \lecture{8}{2/11/2021}

\section{Misurazioni indirette}\label{sec:mis_ind}

Per ora abbiamo analizzato il problema di \textit{cosa} avviene al sistema successivamente a una misurazione. Tuttavia non abbiamo ancora cercato di spiegare \textit{come} avviene una misurazione.
Generalmente, fra l'osservatore e il sistema quantistico, vi è un cosiddetto dispositivo classico di misura (\textit{classical measuring device})\footnote{Esempi di dispositivi di questo tipo sono le foto-emulsioni, la camera a nebbia di Wilson etc. etc.}. Le misurazioni in cui un dispositivo di misura classico interagisce propriamente con l'oggetto quantistico studiato vengono chiamate \textbf{misure dirette}. 
Chiaramente, in una misura di questo genere, stiamo utilizzando un sistema a molti gradi di libertà che vanno inevitabilmente a perturbare fortemente il sistema (abbiamo perturbazioni ben più grandi di quelle previste dai limiti di Heisenberg). Inoltre è raro che il sistema modifichi unicamente il grado di libertà studiato: con ogni probabilità porterà a una grossa perturbazione su una vasta gamma di osservabili.
Per questo si tende a preferire quelle che vengono chiamate \textbf{misure indirette}. Tali misurazioni sfruttano una sonda quantistica (\textit{quantum probe}) per mediare l'interazione sistema classico - sistema quantistico.
Abbiamo un processo a due fasi:
\begin{enumerate}
    \item la sonda quantistica interagisce con l'oggetto da misurare e fra di essi si stabilisce una correlazione;
    \item la sonda viene rilevata e misurata direttamente da un dispositivo classico.
\end{enumerate}
\noindent La sonda viene preparata in modo adeguato quando è ancora del tutto sconnessa al sistema e ha, in particolare, un osservabile detto \textbf{\textit{pointer observable}} che ha un legame con l'osservabile d'interesse nel sistema quantistico: ovvero c'è un rapporto (idealmente 1:1) fra i valori dei due osservabili.
Durante l'interazione sonda - sistema, d'altro canto, dobbiamo aspettarci che i due sistemi diventino \textit{entangled} e questo limiterà parzialmente la nostra misura finale: benché possiamo pensare di avere alte perturbazioni nella seconda fase del processo (perché non ci interessa conservare la sonda) dovremo limitarci per non perturbare troppo il sistema originale.
Per raggiungere un'alta precisione nel processo di misura cercheremo di rispettare sempre due semplici condizioni:
\begin{itemize}
    \item il secondo step della misurazione deve avvenire solo quando il primo risulta completato;
    \item il secondo step non deve contribuire significativamente all'errore totale della misura.
\end{itemize}
Se seguiamo queste regole, dunque, le uniche fonti di errore nella misura e le sole perturbazioni del sistema saranno quelle relative a incertezze intrinseche al processo di preparazione della sonda\footnote{Non stiamo considerando interazioni coi gradi di libertà ambientali.}.
\vspace{1cm}
\noindent Analizziamo, dunque, il processo di misura.
Abbiamo due differenti sistemi quantistici (la sonda e l'oggetto studiato) descritti dalle rispettive matrici densità $\hat \rho _P$ e $\hat \rho _{INIT}$. L'evoluzione di entrambe può essere descritta introducendo $\hat U (t) = e^{-\frac{i}{\hbar}\hat H t}$ dove $\hat H$ è l'hamiltoniana totale del sistema ($\hat H = \hat H_{probe} + \hat H _{obj} + \hat H _{interaction}$):
\begin{equation*}
    (\hat \rho_P \hat \rho_{init})'=\hat U (t) \hat \rho_P \hat \rho_{init} \hat U ^\dagger (t)
\end{equation*}
\noindent Abbiamo detto che il processo di misurazione non è unitario, ma a questo punto la non-unitarietà è tutta contenuta nella fase di interazione fra sonda e strumentazione di misura.
Lo stato finale della sonda, dopo l'interazione, è dato dalla seguente formula:
\begin{equation*}
    \hat \rho_{probe}' = \Tr_{obj} \left( \hat U \hat \rho_{probe} \hat \rho_{int} \hat U ^\dagger \right)
\end{equation*}
\noindent Solo a questo punto utilizziamo una misura proiettiva sulla sonda. 
Misuriamo il \textit{pointer observable} $P\tilde a$ nello spazio della sonda che è collegato con l'osservabile $A$ nello spazio dell'oggetto studiato. Avremo l'autostato dell'operatore puntatore: $\ket{\tilde a}$ che corrisponderà all'autovalore (e rispettivo autostato) $\tilde a$ per l'oggetto.
La probabilità di avere un certo autovalore sarà:
\begin{equation*}
    P(\tilde a) = \Tr \left( \ket{\tilde a}\bra{\tilde a} \hat \rho_{probe}'\right)
\end{equation*}
E la traccia che scriviamo qui è una traccia totale che opera nello spazio della sonda.
Possiamo riscrivere questa equazione:
\begin{equation*}
    P(\tilde a) = \Tr \left( \hat \Pi (\tilde a) \rho_{init} \right)
\end{equation*}
Dove, a questo punto, la traccia opera solo sullo spazio dell'oggetto e abbiamo introdotto l'operatore $\hat \Pi$ (anch'esso opera sullo spazio dell'oggetto)  che definiamo come:
\begin{equation*}
    \hat \Pi (\tilde a) = \Tr_{probe}\left( \hat U ^\dagger \ket{\tilde a}\bra{\tilde a}\hat U \hat \rho_{probe}  \right)
\end{equation*}
A questo punto l'operatore $\hat \Pi$ contiene lo stato in cui la sonda è preparata ($\hat \rho_{probe}$), lo stato finale della sonda ($\ket{\tilde a}\bra{\tilde a}$) e l'evoluzione temporale (che è l'unico operatore che operava sia sullo spazio della sonda, dipendenza eliminata dalla traccia parziale, che sullo spazio dell'oggetto).
Abbiamo visto la misura dal punto di vista della sonda, ma cosa sta succedendo al nostro oggetto?
Avevamo scritto l'equazione \ref{eq:prob_omega} rimandandone la dimostrazione che, però, vediamo ora.
Possiamo scrivere la matrice finale dell'oggetto (post interazione):
\begin{equation*}
    \hat \rho_{fin} (\tilde a) = \frac{1}{P(\tilde a)}\bra{\tilde a}\hat U \hat \rho_{probe}\hat \rho_{init} \hat U^\dagger \ket{\tilde a}
\end{equation*}
Se possiamo scrivere lo stato iniziale della sonda come: $\hat \rho_{probe} = \sum_i w_i \ket{\psi_i}\bra{\psi_i}$ arriviamo facilmente, per sostituzione, a:
\begin{equation*}
    \hat \rho_{fin} (\tilde a) = \frac{1}{P(\tilde a)} \sum_i \bra{\tilde a}\hat U \ket{\psi_i}\hat \rho_{init} \bra{\psi_i}\hat U ^\dagger \ket{\tilde a}
\end{equation*}
E ricordiamo che $\ket{\tilde a}$ e $\ket{\psi_i}$ sono stati della sonda e $\hat U$ agisce su entrambi gli stati. Dunque gli operatori $\bra{}\hat U \ket{}$ sono operatori dell'oggetto.
Se la sonda si trova, inizialmente, in uno stato puro $\hat \rho_{probe} = \ket{\psi}\bra{\psi}$, allora abbiamo:
\begin{equation*}
    \hat \rho_{fin} (\tilde a) = \frac{1}{P(\tilde a)}\bra{\tilde a}\hat U \ket{\psi}\hat \rho_{init} \bra{\psi}\hat U ^\dagger \ket{\tilde a}
\end{equation*}
E possiamo identificare:
\begin{equation*}
    \bra{\tilde a} \hat U \ket{\psi} = \hat \Omega = \hat U_{obj}\sqrt{\hat M}
\end{equation*}
Con $\hat U_{obj}$ (diverso dall'operatore $\hat U$ precedente) che contiene le informazioni sulla perturbazione dell'oggetto causata dalla sonda.
\vspace{0.5cm}
Abbiamo già visto che abbiamo una misurazione senza demolizione nel caso in cui $[\hat A, \hat \Omega]=0$ che ora possiamo riscrivere:
\begin{equation*}
    \bra{\tilde a}[\hat A, \hat \Omega] \ket{\psi}=0  
\end{equation*}
Tale equazione deve essere vera per ogni autostato $\ket{\tilde a}$, perciò:
\begin{equation*}
    (\hat A \hat U - \hat U \hat A)\ket{\psi}= 0
\end{equation*}
Se moltiplichiamo per $\hat U^\dagger$ otteniamo:
\begin{equation*}
    \hat U ^\dagger (\hat A \hat U - \hat U \hat A)\ket{\psi}= (\hat U^\dagger \hat A \hat U - \hat A ) \ket{\psi}=0
\end{equation*}
Otteniamo, dunque, questa equazione che è la condizione per avere una QND (\textit{Quantum NonDemolition measurement}). Si noti che l'operatore fra parentesi corrisponde alla variazione dell'osservabile $A$ nella rappresentazione di Heisenberg.
Perché questa condizione sia rispettata vi sono due possibilità:
\begin{itemize}
    \item che valga sempre $\hat U ^\dagger \hat A \hat U-\hat A = 0$;
    \item che lo stato iniziale della sonda sia un autostato della differenza $\hat U^\dagger \hat A \hat U - \hat A$.
\end{itemize}
Il secondo caso non è stato particolarmente analizzato né dal punto di vista sperimentale né da quello teorico, mentre ci si è concentrati sulla prima condizione.
Quest'ultima equivale a richiedere $[\hat A, \hat U] = 0$. Tale condizione è necessaria e sufficiente per avere una misura senza demolizione, ma è generalmente complesso verificare che sia verificata poiché è necessario conoscere l'evoluzione di oggetto e sonda.
Per questo si preferisce richiedere una condizione più forte (sufficiente, ma non necessaria): $[\hat A, \hat H]=0$.
Tipicamente avremo l'hamiltoniana scrivibile come:
\begin{equation*}
    \hat H = \hat H_{obj} + \hat H_{probe}+\hat H_{int}
\end{equation*}
Chiaramente abbiamo immediatamente la commutatività con la parte relativa alla sonda: $[\hat A, \hat H_{probe}]$=0 (poiché $\hat A$ è un osservabile dell'oggetto); mentre dovremo esplicitamente richiedere:
\begin{align}
    [\hat A , \hat H_{obj} ] &= 0\\
    [\hat A , \hat H_{int} ] &= 0
\end{align}