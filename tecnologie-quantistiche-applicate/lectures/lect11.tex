\vspace{1cm}
\newline
\lecture{11}{11/11/2021}
\noindent Come possiamo arrivare a una descrizione quantistica di qubit superconduttivi? Le giunzioni Josephson sono degli oggetti che hanno uno strano comportamento e abbiamo trattato questo effetto utilizzando un approccio classico poiché $\delta$ è trattabile solamente in modo classico. Per vedere effetti quantistici necessitiamo di sapere quando gli SQUIDS possono essere utilizzati come qubit. Per vedere effetti quantistici, dove la fase non è ben definita, trattiamo un basso numero di coppie di Cooper. L'energia associata alla capacità è data da
\begin{equation*}
    E_c=\frac{e^2}{2C} \, ,
\end{equation*}
per avere un comportamento quantistico $E_c\gg k_BT$. Procediamo con un esempio numerico
\begin{esempio}
    Supponiamo di essere a una temperatura di $18 \text{ mK}$, questo fornisce un'energia termica pari a $k_BT=10^{-6} \text{ eV}$. Affinché si abbia un comportamento quantistico, la capacità del condensatore deve essere $C \ll 0.1 \text{ pF}$. Cioè l'area della giunzione Josephson dovrà essere $10^{-6} \text{cm}^3$.
\end{esempio}
\noindent In principio noi vorremmo lavorare con temperature più alte, non solo, dobbiamo considerare la scala temporale del \textbf{principio di indeterminazione di Heisenberg}
\begin{equation*}
    \Delta E \Delta t = \frac \hbar 2 \, ,
\end{equation*}
quindi
\begin{equation*}
    E_c > \Delta E =\frac{\hbar}{2\Delta t} \, ,
\end{equation*}
siccome $\Delta t \sim RC$, $R$ deve essere superiore di $6 \text{ k}\Omega$, cioè $R=\frac{2\hbar}{(2e)^2}$. Questo è una possibile considerazione, altrimenti uno può valutare ad esempio l'energia della giunzione $E_J > k_BT$, dove $E_J = \frac{\hbar}{2e}I_0$, dove $I_0$ dipende dall'area. Il comportamento della giunzione utilizzata nei qubit dipende da $E_J$, $E_c$ e $k_BT$. L'idea di base per lavorare con i qubit è quindi avere una giunzione di piccole dimensioni e lavorare a basse temperature.

\section{Quantizzazione di un circuito LC}
Dal momento che vogliamo quantizzare un qubit necessitiamo di una lagrangiana da cui costruire la corrispondente hamiltoniana e, applicando la quantizzazione canonica, otteniamo il nostro sistema quantizzato.
Supponiamo di considerare un semplice circuito LC come mostrato in Figura \ref{fig:lc-circuit}.

\begin{figure}[!ht]
    \centering
    \begin{circuitikz}
        \draw
        (0,0)   to[C=$C$] ++ (0, 2) -- ++ ( 2,0) 
                to[L=$L$] ++ (0,-2) -- ++ (-2,0);
    \end{circuitikz}
    \caption{Circuito LC.}
    \label{fig:lc-circuit}
\end{figure}
\noindent In generale, in ciascun ramo abbiamo una tensione $V_b$ e una corrente $I_b$, ciò che vogliamo andare a calcolare sono $\Phi_b$ e $Q_b$, dove
\begin{equation*}
    \begin{aligned}
        \Phi_b(t)=\int_{-\infty}^t \dd{t'} V_b(t') \\
        Q_b(t)=\int_{-\infty}^t \dd{t'} I_b(t') \, .
    \end{aligned}
\end{equation*}
Ciascun elemento è caratterizzato da una relazione costitutiva che collega le variazioni di corrente e tensione. Dobbiamo distinguere tra elementi capacitivi per cui la relazione è della forma
\begin{equation*}
    V_b = f(Q_b) \, ,
\end{equation*}
ed elementi induttivi per cui la relazione è della forma
\begin{equation*}
    I_b = g(\Phi_b) \, .
\end{equation*}
Capacità e induttanze lineari usuali sono dei casi speciali in cui
\begin{equation*}
    \begin{aligned}
        f(Q_b) = \frac{Q_b}{C} \\
        g(\Phi_b) = \frac{\Phi_b}{L} \, .
    \end{aligned}
\end{equation*}
Un controesempio è dato dalla giunzione tunnel Josephson che è un elemento induttivo e la funzione $g(\Phi_b)$ è un seno. Tuttavia lavorare con i rami non è una scelta ottimale. Le leggi di Kirchhoff da risolvere sono sulla corrente e tensione, ma invece di utilizzare $I_b$ e $V_b$ usiamo la carica $Q_b$ e il flusso $\Phi_b$. Questo perché il flusso e la carica in un nodo sono uguali al flusso e alla carica in un ramo, anche se in generale non sono sempre lo stesso. Siccome
\begin{equation*}
        \delta = \frac{2e}{\hbar}\Phi \mod{2\pi} = 2\pi \frac{\Phi}{\Phi_0}
\end{equation*}
\begin{equation*}
        I_L = \frac{\Phi}{L} = \ddot{\Phi}C = I_C \, ,
\end{equation*}
avremo che la nostra lagrangiana sarà
\begin{equation*}
    L(\dot \Phi, \Phi) = C\frac{\dot{\Phi}^2}{2} - \frac{\Phi^2}{2L} \, .
\end{equation*}
Applicando la \textbf{trasformata di Legendre} con cui calcoliamo il momento coniugato
\begin{equation*}
    Q = \partialderivative{L}{\dot \Phi} = C\dot \Phi \, ,
\end{equation*}
possiamo scrivere ora l'hamiltoniana del nostro sistema:
\begin{equation*}
    H = Q\dot \Phi - L = \frac{Q^2}{2C} + \frac{\Phi^2}{2L} \, .
\end{equation*}
Avendo ora a disposizione l'hamiltoniana possiamo procedere ad una quantizzazione canonica in cui
\begin{align*}
    &\Phi \longrightarrow \hat \Phi \\
    &Q \longrightarrow \hat Q = -i\hbar \partialderivative{\Phi} \\
    &H \longrightarrow \hat H = \frac{\hat Q^2}{2C} + \frac{\hat \Phi^2}{2L}
\end{align*}
Come possiamo notare l'hamiltoniana quantizzata appena scritta assomiglia all'hamiltoniana di un oscillatore armonico, infatti possiamo notare un'analogia con
\begin{equation*}
    \hat H = \frac{\hat p^2}{2m} + \frac 12 m\omega^2\hat x^2 \, ,
\end{equation*}
\begin{align*}
    Q &\longleftrightarrow p \\
    \Phi &\longleftrightarrow x \\
    C &\longleftrightarrow m \\
    L &\longleftrightarrow \frac 1 k
\end{align*}
con pulsazione
\begin{equation*}
    \omega = \sqrt{\frac{k}{m}} = \frac{1}{\sqrt{LC}} \, .
\end{equation*}
In questo contesto, ricordando che
\begin{equation*}
    E_c = \frac{e^2}{2C} \qquad \text{e} \qquad Q = 2en \, ,
\end{equation*}
possiamo scrivere
\begin{equation*}
    \frac{Q^2}{2C} = \frac{4e^2n^2}{2C} = 4E_cn^2
\end{equation*}
Allo stesso modo
\begin{equation*}
    E_L = \left(\frac{\Phi_0}{2\pi}\right)^2\frac 1L \, ,
\end{equation*}
quindi 
\begin{equation*}
    \frac{\Phi^2}{2L} = \frac{\delta^2\Phi_0^2}{(2\pi)^2}\frac{1}{2L} = \frac{1}{2}\delta^2E_L \,
\end{equation*}
Un altro modo dunque di scrivere l'hamiltoniana è il seguente
\begin{equation*}
    \hat H = 4 E_C \hat n^2 + \frac 12 E_L \hat \delta^2 \, .
\end{equation*}
se prima $\comm{\hat \Phi}{\hat Q} = i\hbar$, ora abbiamo che $\comm{\hat \delta}{\hat n} = i$.
Conoscendo l'oscillatore armonico abbiamo che le soluzioni degli autovettori $\ket k$ e degli autovalori $E{k+1} - E_k = \hbar \omega$ sono anche soluzioni del nostro circuito LC. Usando gli operatori di creazione $\hat a^\dagger$ e distruzione $\hat a$ possiamo riscrivere l'hamiltoniana come
\begin{equation*}
    \hat H = \hbar \omega \left(\hat a^\dagger \hat a + \frac 12 \right)
\end{equation*}
con 
\begin{equation*}
    \hat a = \frac{1}{\sqrt{2\hbar z}}\left(\hat \Phi + iz \hat Q\right) \qquad \text{e} \qquad \hat a^\dagger = \frac{1}{\sqrt{2\hbar z}}\left(\hat \Phi - iz \hat Q\right) \qquad \qquad z = \sqrt{\frac{L}{C}}
\end{equation*}
e le relazioni inverse
\begin{equation*}
    \hat Q = i\sqrt{\frac{\hbar}{2z}}\left(\hat a^\dagger - \hat a\right) \qquad \qquad \hat \Phi = \sqrt{\frac{\hbar z}{2}}\left(\hat a^\dagger + \hat a\right)
\end{equation*}
Possiamo inoltre valutare i valori medi di $\hat Q^2$ e $\hat \Phi^2$
\begin{align*}
    \expval{\hat Q^2} = \expval{\hat Q^2}{0} = \frac{\hbar}{2z} \\
    \expval{\hat \Phi^2} = \expval{\hat \Phi^2}{0} = \frac{\hbar z}{2}
\end{align*}
e prendono il nome di \textbf{zero-point fluctuations} perché $\Delta Q \Delta \Phi = \frac{\hbar}{2}$.
Come possiamo notare, il circuito LC non può essere utilizzato come qubit, questo perché il gap energetico tra i vari livelli è lo stesso, inoltre non possiamo vedere un comportamento quantistico perché $\hbar \omega \ll k_BT$. Se prendessimo $T=10 \text{ mK}$, allora $\hbar\omega > 10^{-5}\text{ eV}$, cioè $\omega \gg 10 \text{GHz}$.
Questo fatto è interessante, perché per avere un sistema quantistico, a questa temperatura, necessitiamo solamente di una frequenza dell'ordine di $10 \text{GHz}$. Questo significa che il sistema in questione deve avere una dimensione $l < \frac{\omega}{2\pi c}$, ma questo non è un problema. Il problema risiede nel fatto che dobbiamo accoppiarlo con l'ambiente.
Un altro fatto di cui dobbiamo tenere conto è quello riguardante la possibilità di avere tunneling, necessitiamo quindi di lavorare con livelli meno energetici del gap che c'è tra i vari livelli in un circuito LC. La probabilità di tunneling aumenta più si eccita il sistema. Martinis, Devoret e altri fisici che studiarono questo sistema rimpiazzarono l'induttore $L$ con una giunzione Josephson. La giunzione, comportandosi da elemento non lineare, va ad alterare la distribuzione dei livelli energetici.