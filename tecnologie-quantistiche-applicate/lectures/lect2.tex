\noindent \lecture{2}{12/10/2021}
\vspace{0.5cm}

\noindent Vediamo alcune proprietà generali della matrice densità:
\begin{enumerate}
    \item \textit{$\rho$ è hermitiana e positiva}. 
    \begin{proof}
        Ricordando che $p_i \geq 0 \in \mathbb{R}$ allora
        \begin{equation*}
            \left( \sum_i p_i \ketbra{\psi_i} \right)^\dag = \sum_i p_i \ketbra{\psi_i} \, .
        \end{equation*}
        La positività di un operatore $A$ è data dalla proprietà $\expval{A}{\phi} \geq 0$ per ogni $\ket{\phi}$, quindi
        \begin{equation*}
            \expval{\rho}{\phi} = \bra{\phi} \sum_i p_i \ket{\psi_i} \braket{\psi_i}{\phi} = \sum_i p_i \braket{\phi}{\psi_i} \underbrace{\braket{\psi_i}{\phi}}_{\braket{\phi}{\psi_i}^\ast} = \sum_i p_i \abs{\braket{\phi}{\psi_i}}^2 \geq 0 \, .
        \end{equation*}
    \end{proof}
    
    \item $\Tr \rho = 1$.
    \begin{proof}
        \begin{equation*}
            \Tr \left[ \sum_i p_i \ketbra{\psi_i} \right] = \sum_i p_i \Tr \ketbra{\psi_i} = \sum_i p_i \braket{\psi_i} = 1 \, ,
        \end{equation*}
    \end{proof}
    
    \item \textit{$\Tr \rho^2 \leq 1$ e $\Tr \rho^2 = 1$ solo per gli stati puri}.
    \begin{proof}
        Dato che $\rho$ è hermitiana allora può essere diagonalizzata, quindi $\rho \ket{n} = \rho_n \ket{n}$. In particolare avremo
        \begin{equation}\label{diagonalization_rho}
            \rho = \sum_i p_i \ketbra{\psi_i} \equiv \sum_n \rho_n \ketbra{n} \, ;
        \end{equation}
        quindi la matrice densità è scrivibile come somma del prodotto tra i proiettori in direzione degli autospazi e dei corrispondenti autovalori. Notiamo che la formula precedente consiste di fatto nella diagonalizzazione in notazione di Dirac. Chiaramente in generale $p_i \neq \rho_i$! Dato che $\Tr \rho = 1$ allora dalla precedente si ha che $\sum_n \rho_n = 1$. Cerchiamo di valutare $\Tr \rho^2 = \sum_n \rho_n^2$: dato che la matrice densità è hermitiana allora $\rho_n \geq 0 \in \mathbb{R}$, ma al tempo stesso si deve avere $0 \leq \rho_n \leq 1$ poiché $\sum_n \rho_n = 1$. Ma allora
        \begin{equation*}
            0 \leq \rho_n^2 \leq \rho_n \leq 1 \, , \quad \Rightarrow \quad \sum_n \rho_n^2 \leq \sum_n \rho_n = 1 \, .
        \end{equation*}
        Il caso limite è dato da
        \begin{equation*}
            \sum_n \rho_n^2 = 1 = \sum_n \rho_n \quad \Leftrightarrow \quad \rho_n = \rho_n^2 \, , \; \forall n \, ,
        \end{equation*}
        ma per numeri tra 0 e 1 questo è vero solamente per $\rho_n = 1 \lor \rho_n = 0$: solamente un valore è diverso da 0 (uguale a 1) mentre tutti gli altri sono 0, quindi $\rho = \ketbra{n}$ per un particolare $n$, ossia si tratta di uno stato puro. 
    \end{proof}
    Si noti che la formula $\Tr \rho^2 = 1 \Leftrightarrow $ stato puro è un criterio per stabilire se effettivamente uno stato è puro. Inoltre, essendo $\rho$ hermitiana, può sempre essere diagonalizzata secondo la \eqref{diagonalization_rho} e quindi la sua forma non è unica.
\end{enumerate}

\noindent Alla luce della definizione di matrice densità, cerchiamo di comprendere come alcuni postulati della meccanica quantistica possano essere riscritti in termini di $\hat \rho$. In particolare iniziamo dal
\begin{itemize}
    \item \textbf{IV Postulato (Evoluzione temporale)}: Cosa succede nel momento in cui il nostro sistema quantistico evolve? Se è descritto da una funzione d'onda $\ket \psi$, dopo un certo intervallo di tempo $t$, il nostro stato evolverà nel seguente modo
    \begin{equation*}
        \ket{\psi(t)}=\hat U\ket \psi \, ,
    \end{equation*}
    dove $\hat U$ è l'operatore unitario di evoluzione temporale.\\
    Se invece il nostro sistema è descritto da un insieme di stati, con ciascuno la propria probabilità $\{p_i, \ket{\psi_i}\}$ e per questo insieme possiamo definire una matrice densità $\rho=\sum_i p_i\op{\psi_i}{\psi_i}$, allora considerando sistemi chiusi, la nostra matrice densità evolverà come
    \begin{equation*}
        \hat \rho'=\sum_i p_i \hat U\op{\psi_i}{\psi_i}\hat U^\dagger = \hat U \rho \hat U^\dagger \, ,
    \end{equation*}
    quindi $\hat \rho'$ descriverà un nuovo insieme di stati  $\{p_i, \hat U\ket{\psi_i}\}$ dove però $p_i$ è preservata.\\
\end{itemize}
Osserviamo che se il nostro stato iniziale è uno stato puro, la matrice densità che possiamo costruire è anch'essa uno stato puro. Se il nostro sistema è chiuso, cioè non interagisce con l'ambiente e quindi l'evoluzione dello stato puro rimane uno stato puro, allora anche l'evoluto della matrice densità rimane uno stato puro.

\section{Misurazioni}\label{misurazioni_proiettive}
Prima di addentrarci nella discussione sulle misurazioni, che saranno uno degli argomenti chiave del corso, introduciamo il postulato della meccanica quantistica relativo alle osservabili.
\begin{itemize}
    \item \textbf{II Postulato} (\textbf{Osservabili}): Che cosa si può misurare in meccanica quantistica? Vengono misurate le \textbf{osservabili} rappresentate da \textbf{operatori autoaggiunti} (o \textbf{hermitiani}) $\hat{A}$ tali che
    \begin{equation*}
        \hat{A}: \mathcal{H}\rightarrow \mathcal{H} \, \text{ con } \, \hat{A}^\dagger = \hat{A} \, ,
    \end{equation*}
    dove più precisamente $\hat{A}^\dagger \equiv (\hat{A}^t)^\ast$. Dal punto di vista degli elementi di matrice, calcolare l'aggiunto di $A_{ij}$ significa $A^\dagger_{ij} = A^\ast_{ji}$.
\end{itemize}

\noindent Focalizzando la nostra attenzione sugli operatori hermitiani, richiamiamo un importante teorema dell'algebra lineare:
\begin{teorema}[\textbf{Teorema Spettrale}]
    Sia $\hat{A}$ un operatore autoaggiunto su uno spazio di Hilbert $\mathcal{H}$ (reale o complesso). Allora esiste una base ortonormale di $\mathcal{H}$ composta da autovettori di $\hat{A}$, ossia $\exists$ $\lbrace \ket{a_i} \rbrace \in \mathcal{H}$ tale che $\hat{A} \ket{a_i} = a_i \ket{a_i}$ dove gli autovalori $a_i \in \mathbb{R}$.
\end{teorema}

\begin{itemize}
    \item \textbf{III Postulato} (\textbf{Regola di Born}):
    \begin{enumerate}
        \item \textbf{Misurazione}: Sia $\hat{A}$ un osservabile con autostati $\ket{a_i}$, ossia $\hat{A} \ket{a_i} = a_i \ket{a_i}$. Prendiamo per semplicità $a_i \neq a_j \ \forall i \neq j$ (osservabile con autovalori distinti). Consideriamo uno stato generico espanso sugli autostati precedenti: $\ket \psi = \sum_i \alpha_i \ket a_i$. Allora una misura dell'osservabile $\hat{A}$ produce il valore $a_i$ con probabilità data da $\abs{\alpha_i}^2$ (assumendo lo stato correttamente normalizzato).
        
        \item \textbf{Collasso dello stato}: Cosa succede allo stato del sistema dopo la misurazione? Istantaneamente lo stato $\ket \psi$ collassa sull'autostato associato all'autovalore risultante dalla misura. Ad esempio se misurando otteniamo $a_i$ allora $\ket{\psi} \rightarrow \ket{a_i}$. Effettuando delle misure successive sullo stato si ottiene sempre $\ket{a_i}$ con probabilità esattamente uguale a 1.  
    \end{enumerate}
\end{itemize}
Soffermiamoci su questo postulato e analizziamone bene il contenuto. Questo tipo di misurazione prende il nome di \textbf{Von Neumann measurement} o \textbf{projective measurement}. Il motivo di quest'ultimo nome è che se noi abbiamo un'osservabile $\hat A$ e $a_i$ è l'autovalore associato allo stato $\ket{a_i}$, allora possiamo definire un operatore $\hat P_i$ che prende il nome di \textbf{proiettore} relativo allo stato $\ket{a_i}$, definito come
\begin{equation*}
    \hat P_i=\op{a_i}{a_i} \, .
\end{equation*}
La probabilità associata al fatto che il nostro generico stato $\ket \psi$ collassi nello stato $\ket{a_i}$ è pari a
\begin{equation*}
    \begin{aligned}
        p_i &= \mel{\psi}{\hat P_i}{\psi} \\
            &= \ip{\psi}{a_i}\ip{a_i}{\psi}\\
            &= \abs{\ip{a_i}{\psi}}^2 \, .
    \end{aligned}
\end{equation*}
I \textbf{proiettori} hanno alcune proprietà particolari
\begin{enumerate}
    \item $\hat A = \sum_i a_i \hat P_i$;
    \item $\sum_i \hat P_i = \mathbb{I}$;
    \item $\hat P_i \hat P_j = \hat P_i\delta_{ij}$.
\end{enumerate}
Per via di quest'ultima proprietà chiamiamo questo tipo di misure \textbf{orthogonal measurement} e questo concetto si rifà al punto 2 del postulato relativo alle misure. Alla luce della definizione di proiettore possiamo vedere l'operazione di misura sotto questo punto di vista: se abbiamo uno stato $\ket \psi$ e andiamo a misurare l'osservabile $\hat A$, proiettiamo lo stato iniziale nello stato $a_i$:
\begin{equation*}
    \begin{aligned}
        \ket \psi \longrightarrow \hat P_i\ket{\psi} &= \op{a_i}{a_i}\sum_{k}\ip{a_k}{\psi}\ket{a_k} \\
                                                     &= \op{a_i}{a_i}\sum_{k}\sqrt{p_k}\ket {a_k} \\
                                                     &= \sqrt{p_i}\ket {a_i}
    \end{aligned}
\end{equation*}
Affinché sia uno stato, deve essere necessariamente normalizzato: $\ket \psi'=\frac{P_i\ket{\psi}}{\sqrt{p_i}}$. Da qui in poi è possibile valutare il valore di aspettazione dell'osservabile $\expval{A}$ e l'incertezza associata $\Delta A$:
\begin{equation*}
    \expval{A}=\expval{A}{\psi}=\sum_i a_i \mel{\psi}{\hat P_i}{\psi}=\sum_i p_ia_i
\end{equation*}
\begin{equation*}
    \Delta A = \expval{ A^2}-\expval{A}^2
\end{equation*}
\subsection{Misurazioni della matrice densità}
Consideriamo il nostro insieme di stati $\{p_i, \ket{\psi_i}\}$, dove $\hat \rho = \sum_i p_i \op{\psi_i}{\psi_i}$. Abbiamo un'osservabile $\hat A$ con i suoi autostati $\ket{a_n}$, autovalori $a_n$ e possiamo definire i proiettori $\hat P_n = \op{a_n}{a_n}$.\\
Se prendiamo uno stato $\ket{\psi_i}$ dall'insieme di stati e valutiamo la probabilità di ottenere lo stato $\ket{a_n}$ dato lo stato $\ket{\psi_i}$ otteniamo
\begin{equation*}
    p(a_n|i)=\mel{\psi_i}{P_n}{\psi_i}=\Tr\left(\hat P_n \op{\psi_i}{\psi_i}\right)
\end{equation*}
In generale, per una matrice densità $\rho$, la probabilità di ottenere $a_n$ è
\begin{equation}\label{dm-meas1}
    \begin{aligned}
        p(a_n) &= \sum_i p_i p(a_n|i) \\
               &= \sum_i p_i \Tr\left(\hat P_n \op{\psi_i}{\psi_i}\right) \\
               &= \Tr \left(\hat P_n \sum_i p_i \op{\psi_i}{\psi_i}\right) \\
               &= \Tr \left(\hat P_n \hat \rho\right) \\
               &= \Tr \left(\hat \rho \hat P_n \right)
    \end{aligned}
\end{equation}
Dopo aver eseguito la misura di $a_n$, otteniamo lo stato $\ket{a_n}$ che in termini di matrice densità possiamo scrivere come
\begin{equation}\label{dm-meas2}
    \hat \rho_n = \frac{\hat P_n \hat \rho \hat P_n}{\Tr \left(\hat P_n \hat \rho \right)}
\end{equation}
Possiamo dimostrare che si tratta di uno stato puro.
\begin{proof}
    \begin{equation*}
        \begin{aligned}
            \hat \rho_k &= \frac{\hat P_n \hat \rho \hat P_n}{\Tr \left(\hat P_n \hat \rho \right)} \\
                        &= \frac{\op{a_n}{a_n}\sum_i p_i \op{\psi_i}{\psi_i} \op{a_n}{a_n}}{\Tr \left(\hat P_n \hat \rho\right)} \\
                        &= \frac{\ket{a_n}\left(\sum_i p_i \ip{a_n}{\psi_i}\ip{\psi_i}{a_n}\right)\bra{a_n}}{\Tr \left(\hat P_n \hat \rho\right)} \\
                        &= \frac{\Tr \left(\hat P_n \hat \rho\right)}{\Tr \left(\hat P_n \hat \rho\right)}\op{a_n}{a_n} \\
                        &= \op{a_n}{a_n}
        \end{aligned}
    \end{equation*}
\end{proof}
\noindent I risultati ottenuti dalle equazioni \eqref{dm-meas1} e \eqref{dm-meas2} rappresentano l'equivalente del \textbf{III Postulato} per $\hat \rho$. 