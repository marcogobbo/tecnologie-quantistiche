\documentclass[a4paper, 12pt]{book}

% General Settings
\usepackage[paper = a4paper, margin = 1in]{geometry}
\usepackage[italian]{babel}
\usepackage[utf8]{inputenc}
\usepackage[T1]{fontenc}
\usepackage{hyperref}

% Math packages
\usepackage{amsmath,amssymb,amsthm}
\usepackage{mathrsfs}

% Quantum circuits packages
\usepackage[braket, qm]{qcircuit}

% Documento
\begin{document}
    \begin{titlepage}
        \begin{center}
            \vspace*{5cm}
            {\scshape\LARGE Università degli Studi di Milano Bicocca \par}
            \vspace{1.0cm}
            \line(1,0){400} \\
            {\huge\bfseries Teoria della informazione e \\ della computazione quantistica \par}
            \line(1,0){400} \\
 	        \vspace{0.5cm}
            {\Large Raccolta di appunti, dispense e libri \par}
            \vspace{1.0cm}
            {Anno accademico 2021/2022 \par}
            \vspace{0.5cm}
            {\bfseries Marco Gobbo \par}
            \vspace{0.5cm}
            {\url{https://github.com/marcogobbo/tecnologie-quantistiche} \par}
            \vspace*{\fill}
            {\large \today \par}
        \end{center}
    \end{titlepage}
\end{document}
