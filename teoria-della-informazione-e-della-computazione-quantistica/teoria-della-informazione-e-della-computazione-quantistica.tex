\documentclass[a4paper, 12pt]{book}

% General Settings
\usepackage[paper = a4paper, margin = 1in]{geometry}
\usepackage[english]{babel}
\usepackage[utf8]{inputenc}
\usepackage[T1]{fontenc}
\usepackage{hyperref}
\usepackage{graphicx}

% Math and physic packages
\usepackage{amsmath,amssymb,amsthm}
\usepackage{mathrsfs}

% Quantum and classical circuits packages
\usepackage[braket, qm]{qcircuit}
\usepackage{circuitikz}

% New or renewed commands
\newcommand{\lecture}[2]{{\scshape{Lecture #1 - #2}} \par}
\newtheorem{definition}{Definition}[chapter]
\newtheorem{theorem}{Theorem}[chapter]
\renewcommand{\thefootnote}{\roman{footnote}}

% Documents
\begin{document}
    \begin{titlepage}
        \begin{center}
            \vspace*{5cm}
            {\scshape\LARGE Università degli Studi di Milano Bicocca \par}
            \vspace{1.0cm}
            \line(1,0){400} \\
            {\huge\bfseries Teoria della informazione e \\ della computazione quantistica \par}
            \line(1,0){400} \\
 	        \vspace{0.5cm}
            {\Large Raccolta di appunti, dispense e libri \par}
            \vspace{1.0cm}
            {Anno accademico 2021/2022 \par}
            \vspace{0.5cm}
            {\bfseries Marco Gobbo \par}
            \vspace{0.5cm}
            {\url{https://github.com/marcogobbo/tecnologie-quantistiche} \par}
            \vspace*{\fill}
            {\large \today \par}
        \end{center}
    \end{titlepage}
    \tableofcontents
    \chapter{Introduzione}
\lecture{1}{05/10/2021}

Lorem ipsum dolor sit amet, consectetur adipiscing elit, sed do eiusmod tempor incididunt ut labore et dolore magna aliqua. Ut enim ad minim veniam, quis nostrud exercitation ullamco laboris nisi ut aliquip ex ea commodo consequat. Duis aute irure dolor in reprehenderit in voluptate velit esse cillum dolore eu fugiat nulla pariatur. Excepteur sint occaecat cupidatat non proident, sunt in culpa qui officia deserunt mollit anim id est laborum.
\end{document}
