%%%%%%%%%%%%%
% LECTURE 3 %
%%%%%%%%%%%%%
\vspace{1.0cm}

\noindent\lecture{3}{11/10/2021}

\section{Sistemi a più qubit}

\noindent Enunciamo l'ultimo postulato della QM riguardante i sistemi composti da diversi sottosistemi:
\begin{itemize}
    \item \textbf{V Postulato} (\textbf{Sistemi multi-partiti}): Consideriamo un sistema quantistico composto da 2 sottosistemi $A$ e $B$ con spazi di Hilbert $\mathcal{H}_A$ e $\mathcal{H}_B$ rispettivamente. Lo spazio di Hilbert del sistema totale è dato da\footnote{Il simbolo "$\otimes$" indica il \textbf{prodotto tensoriale} tra spazi.} $\mathcal{H} = \mathcal{H}_A \otimes \mathcal{H}_B$.
\end{itemize}

\begin{esempio}[\textbf{Sistema di 2 qubit}]
    Immaginiamo un sistema quantistico costituito da 2 qubit tale che ogni qubit possa esistere in uno stato differente, ossia $\ket{0}$ o $\ket{1}$. Ovviamente ci sono 4 possibili scelte per il sistema totale: $\ket{00}$, $\ket{01}$, $\ket{10}$ e $\ket{11}$, dove la notazione implica che la prima entrata si riferisca al primo qubit e la seconda al secondo qubit. Il generico stato del sistema è dato da una combinazione lineare di questi ultimi:
    
    \begin{equation*}
        \ket{\psi} = \alpha_{00} \ket{00} + \alpha_{01} \ket{01} + \alpha_{10} \ket{10} + \alpha_{11} \ket{11} \, ,
    \end{equation*}
    
    \noindent con la condizione di normalizzazione $\sum_{i,j=0}^1 \abs{\alpha_{ij}}^2 = 1$. La probabilità che il primo qubit si trovi in $\ket{i}$ e al contempo il secondo si trovi in $\ket{j}$ è data da $P(\ket{ij}) = \abs{\alpha_{ij}}^2$. Se introduciamo i proiettori $\hat{P}_0$ e $\hat{P}_1$ sul risultato del primo qubit, possiamo allora scrivere che
    
    \begin{equation*}
        \ket{\psi} = \underbrace{\alpha_{00} \ket{00} + \alpha_{01} \ket{01}}_{\hat{P}_0 \ket{\psi}} + \underbrace{\alpha_{10} \ket{10} + \alpha_{11} \ket{11}}_{\hat{P}_1 \ket{\psi}} = \hat{P}_0 \ket{\psi} + \hat{P}_1 \ket{\psi} \, ;
    \end{equation*}
    
        \noindent dalla regola di Born generalizzata avremo che
        \begin{align*}
            P_1(\ket{0}) &= \norm{\hat{P}_0 \ket{\psi}}^2 = \abs{\alpha_{00}}^2 + \abs{\alpha_{01}}^2 \, , \quad \Rightarrow \quad \ket{\psi} \rightarrow \frac{\hat{P}_0 \ket{\psi}}{\norm{\hat{P}_0 \ket{\psi}}} \, , \\
            P_1(\ket{1}) &= \norm{\hat{P}_1 \ket{\psi}}^2 = \abs{\alpha_{10}}^2 + \abs{\alpha_{11}}^2 \, , \quad \Rightarrow \quad \ket{\psi} \rightarrow \frac{\hat{P}_1 \ket{\psi}}{\norm{\hat{P}_1 \ket{\psi}}} \, .
        \end{align*}
\end{esempio}

\noindent Più in generale, consideriamo il caso di due spazi di Hilbert generici $\mathcal{H}_A$ e $\mathcal{H}_B$ con basi $\ket{n}_A$ e $\ket{n}_B$ rispettivamente. Possiamo scrivere lo spazio di Hilbert totale come 

\begin{equation*}
    \mathcal{H}_A \otimes \mathcal{H}_B = \left\lbrace \sum_{n,m} \alpha_{nm} \ket{nm} \right\rbrace \, ,
\end{equation*}

\noindent dove $\dim (\mathcal{H}_A \otimes \mathcal{H}_B) = (\dim \mathcal{H}_A) \times (\dim \mathcal{H}_B)$. Questo spazio possiede la seguente operazione

\begin{definizione}[\textbf{Prodotto Tensoriale}]
    Siano $\lbrace \ket{n}_A \rbrace$ e $\lbrace \ket{n}_B \rbrace$ i due set di basi di due spazi di Hilbert $\mathcal{H}_A$ e $\mathcal{H}_B$. Definiamo \textbf{prodotto tensoriale} la funzione $\otimes: \; \mathcal{H}_A \times \mathcal{H}_B \rightarrow \mathcal{H}_A \otimes \mathcal{H}_B$ con la regola
    
    \begin{equation*}
        \lbrace \ket{n}_A, \ket{n}_B \rbrace \rightarrow \ket{n}_A \otimes \ket{n}_B \equiv \ket{nm} \, .     
    \end{equation*}
\end{definizione}

\noindent Questa operazione può essere estesa per linearità su tutto lo spazio di Hilbert: consideriamo due stati $\ket{\psi}_A \in \mathcal{H}_A$ e $\ket{\phi}_B \in \mathcal{H}_B$, allora
\begin{equation}\label{separable_state}
    \ket{\psi}_A \otimes \ket{\phi}_B = \left( \sum_n \alpha_n \ket{n}_A \right) \otimes \left( \sum_m \beta_m \ket{m}_B \right) = \sum_{n,m} \underbrace{\alpha_n \beta_m}_{\alpha_{nm}} \ket{nm} \, .
\end{equation}

\noindent Vettori di $\mathcal{H}_A \otimes \mathcal{H}_B$ che possono essere scritti come la decomposizione in \eqref{separable_state} sono detti \textbf{separabili}. Solamente alcuni vettori di $\mathcal{H}_A \otimes \mathcal{H}_B$ sono separabili perché $\alpha_{nm}$ è una matrice particolare, risultante dal prodotto dei due vettori contenenti i coefficienti $\alpha_n$ e $\beta_m$. È possibile dimostrare che tali matrici che si scrivono come $A_{nm} = \alpha_n \beta_m$ hanno $\det A_{nm} = 0$ (sono di rango 1). Gli stati particolarmente interessanti che tratteremo a lungo durante il corso sono quelli che non soddisfano la decomposizione \eqref{separable_state}, detti stati \textbf{entangled}. 

\begin{esempio}
    Il seguente stato è separabile:
    
    \begin{equation*}
        \frac{1}{\sqrt{2}} (\ket{00} + \ket{01}) \equiv \frac{1}{\sqrt{2}} (\ket{0} \otimes \ket{0} + \ket{0} \otimes \ket{1}) = \ket{0} \otimes \frac{1}{\sqrt{2}} (\ket{0} + \ket{1}) \equiv \ket{\psi}_A \otimes \ket{\phi}_B \, .
    \end{equation*}
    
    \noindent Al contrario lo stato $\frac{1}{\sqrt{2}} (\ket{00} + \ket{11})$ è entangled poiché non può essere decomposto come in \eqref{separable_state}. 
\end{esempio}

\noindent Analizziamo ciò che abbiamo appena visto nel contesto di qubit e circuiti. In qualità di portatori di informazioni denoteremo due qubit con due linee in questo modo

\begin{center}
    \mbox{
        \Qcircuit @C=2em @R=1.5em {
            \lstick{\ket{0}, \ket{1}} & \qw & \qw \\
            \lstick{\ket{0}, \ket{1}} & \qw & \qw 
        }
    }
\end{center}

\noindent mentre il risulto di tali qubit è la combinazione lineare $\sum_{n,m} \alpha_{nm} \ket{nm}$. Dato che $\dim (\mathcal{H}_A \otimes \mathcal{H}_B) = 2 \times 2 = 4$ allora $\mathcal{H}_A \otimes \mathcal{H}_B \simeq \mathbb{C}^4$ e su tale spazio di Hilbert gli operatori unitari corrispondenti ai gate quantistici sono le matrici unitarie $4 \times 4$. Alcune di queste matrici derivano dalle operazioni sui qubit singoli: ad esempio se

\begin{center}
    \mbox{
        \Qcircuit @C=2em @R=2em {
            \lstick{\ket{\psi}} & \gate{A} & \rstick{A \ket{\psi} \, ,} \qw \\
        }
    } 
    \\
    \mbox{
        \Qcircuit @C=2em @R=2em {
            \lstick{\ket{\phi}} & \gate{B} & \rstick{B \ket{\phi} \, ,} \qw \\
        }
    }
\end{center}

\noindent la matrice risultante agente se entrambi i qubit è $U = A \otimes B$, la quale agisce naturalmente come $U (\ket{\psi} \otimes \ket{\phi}) = A \ket{\psi} \otimes B \ket{\phi}$. Queste tipologie di matrici sono molto speciali e non sono sicuramente le più generali. \\
\noindent Come facciamo per passare dai vettori a 2 componenti di ciascun qubit ad un vettore a 4 componenti del sistema a 2 qubit? Supponiamo che ciascun qubit sia un vettore

\begin{equation*}
    \alpha_0 \ket{0} + \alpha_1 \ket{1} \equiv 
    \begin{pmatrix}
        \alpha_0 \\ \alpha_1
    \end{pmatrix} \in \mathbb{C}^2 \, ,
\end{equation*}
e consideriamo il vettore risultante dal sistema di due qubit
\begin{equation*}
    \alpha_{00} \ket{00} + \alpha_{01} \ket{01} + \alpha_{10} \ket{10} + \alpha_{11} \ket{11} \equiv
    \begin{pmatrix}
        \alpha_{00} \\ \alpha_{01} \\ \alpha_{10} \\ \alpha_{11}
    \end{pmatrix} \in \mathbb{C}^4 \, .
\end{equation*}

\noindent Per passare dall'uno all'altro possiamo utilizzare il prodotto seguente:

\begin{definizione}[\textbf{Prodotto di Kronecker}]
    Siano $A$ e $B$ due matrici. Assumendo che $A$ sia una matrice $q \times p$ allora definiamo il prodotto di Kronecker $A \otimes B$ come
    \begin{equation*}
        A \otimes B =
        \begin{pmatrix}
            A_{11} B & A_{12} B & \cdots & A_{1p} B \\ 
            \vdots & & \ddots & \\
            A_{q1} B & A_{q2} B & \cdots & A_{qp} B
        \end{pmatrix}
    \end{equation*}
\end{definizione}

\begin{esempio}[Vettore sistema di due qubit]
    Utilizzando il prodotto di Kronecker il vettore risultante di un sistema di due qubit può essere scritto come
    \begin{equation*}
        \begin{pmatrix}
            \alpha_0 \\ \alpha_1
        \end{pmatrix}
        \otimes
        \begin{pmatrix}
            \beta_0 \\ \beta_1
        \end{pmatrix}
        =
        \begin{pmatrix}
            \alpha_0 \beta_0 \\ \alpha_0 \beta_1 \\ \alpha_1 \beta_0 \\ \alpha_1 \beta_1
        \end{pmatrix} \, ;
    \end{equation*}
    notiamo infatti che questo vettore di coefficienti è lo stesso risultante dal prodotto tensoriale dei due qubit
    \begin{equation*}
        (\alpha_0 \ket{0} + \alpha_1 \ket{1}) \otimes (\beta_0 \ket{0} + \beta_1 \ket{1}) = \alpha_0 \beta_0 \ket{00} + \alpha_0 \beta_1 \ket{01} + \alpha_1 \beta_0 \ket{10} + \alpha_1 \beta_1 \ket{11} \, .
    \end{equation*}
\end{esempio}

\begin{esempio}
    Chiaramente il prodotto di Kronecker funziona anche per un sistema di due qubit in cui agiscono un \texttt{X-gate} e un \texttt{Z-gate}
    \begin{center}
        \mbox{
            \Qcircuit @C=2em @R=2em {
                \lstick{\ket{\psi}} & \gate{X} & \rstick{X \ket{\psi} \, ,} \qw \\
            }
        } 
        \\
        \mbox{
            \Qcircuit @C=2em @R=2em {
                \lstick{\ket{\phi}} & \gate{Z} & \rstick{Z \ket{\phi} \, ,} \qw \\
            }
        }
    \end{center}
    la matrice $4 \times 4$ risultante sarà
    \begin{equation*}
        X \otimes Z = \sigma_1 \otimes \sigma_3 =
        \begin{pmatrix}
            0 & 0 & 1 & 0 \\ 0 & 0 & 0 & -1 \\ 1 & 0 & 0 & 0 \\ 0 & -1 & 0 & 0 
        \end{pmatrix}
    \end{equation*}
\end{esempio}

\noindent Così come sottolineato in precedenza per gli stati separabili, queste matrici non costituiscono il caso generale perché essendo frutto di un prodotto tensoriale in realtà agiscono separatamente su un singolo qubit alla volta: quello di cui abbiamo bisogno in QC ("quantum computing") è di utilizzare matrici generiche che rendono i qubit entangled. \\
\noindent Cominciamo a vedere qualche esempio nel caso del CC ("classical computing"). Oltre al gate logico \texttt{NOT}, vediamo l'azione di altri gate
\begin{align*}
    x \texttt{ AND } y &= 
    \begin{cases}
        1 \, , \text{ se entrambi } x = y = 1 \\
        0 \, , \text{ altrimenti}
    \end{cases} \, , \\
    x \texttt{ OR } y &= 
    \begin{cases}
        1 \, , \text{ se } x = 1 \text{ oppure } y = 1 \\
        0 \, , \text{ altrimenti}
    \end{cases} \, , \\
    x \texttt{ XOR } y &= (x + y) \! \! \! \! \mod 2 \equiv x \oplus y = 
    \begin{cases}
        0 \oplus 0 = 0 \\ 
        0 \oplus 1 = 1 \\
        1 \oplus 0 = 1 \\
        1 \oplus 1 = 0
    \end{cases} \, ,
\end{align*}
dove l'operazione $(x + y) \! \! \! \! \mod 2$ indica il resto della divisione per 2 della somma $x+y$. Tutte queste operazioni non possono essere fattorizzate perché non sono il prodotto di operazioni sui singoli bit. \\
\noindent Consideriamo ora il caso quantistico. Una delle operazioni più importanti è il \textbf{controlled NOT} o \texttt{CNOT-gate}, il quale è una sorta di generalizzazione al caso quantistico del classico \texttt{XOR}. Si tratta di un gate $U_{\text{CN}}$ (unitario) che agisce su un sistema di 2 qubit (sistema di dimensione 4) come segue: 
\begin{align*}
    U_{\text{CN}}: \quad \ket{00} &\rightarrow \ket{00} \, , &\ket{01} &\rightarrow \ket{01} \, , \\
    \ket{10} &\rightarrow \ket{11} \, , &\ket{11} &\rightarrow \ket{10} \, ,
\end{align*}
Come evidente, il primo qubit è utilizzato come \textbf{target}, mentre il secondo funge da qubit di \textbf{controllo}: a seconda del valore del primo qubit si svolge o meno un'azione sul secondo. In particolare quando il primo qubit è in $\ket{0}$, il secondo non viene toccato; quando invece il primo si trova in $\ket{1}$, il secondo viene scambiato. Dal punto di vista grafico indicheremo il \texttt{CNOT-gate} come 
\begin{center}
    \mbox{
        \Qcircuit @C=2em @R=1.5em {
            \lstick{\textbf{Control: } \ket{x}} & \ctrl{1} & \rstick{\ket{x} \, ,} \qw \\
            \lstick{\textbf{Target: } \ket{y}} & \targ & \rstick{\ket{x \oplus y} \, ,} \qw
        }
    }
\end{center}
È chiaro che per $x=0$ si ha $y \to 0 \oplus y = y$ mentre per $x = 1$ si ha $y \to 1 \oplus y$, il quale risulta 1 per $y = 0$ e 0 per $y = 1$. In forma matriciale questo gate non è altro che 
\begin{equation*}
    U_{\text{CN}} =
    \begin{pmatrix}
        1 & 0 & 0 & 0 \\ 0 & 1 & 0 & 0 \\ 0 & 0 & 0 & 1 \\ 0 & 0 & 1 & 0
    \end{pmatrix} \, ,
\end{equation*}
Si tratta di un esempio di gate (matrice $4 \times 4$) che agisce in maniera non banale sui qubit e che non può essere fattorizzato come azione sui singoli qubit. \\
\noindent Un punto fondamentale che differenzia i gate quantistici da quelli classici è che, essendo unitari, sono sempre \textbf{reversibili}. Infatti 
\begin{center}
    \mbox{
        \Qcircuit @C=2em @R=1.5em {
            \lstick{\ket{\psi}} & \gate{U} & \gate{U^\dagger} & \rstick{U U^\dagger \ket{\psi} = \ket{\psi} \, ,} \qw 
        }
    }
\end{center}
In generale il CC non è reversibile\footnote{In realtà esiste un modo per calcolare solamente operazioni reversibili mediante gate reversibili in un computer classico. Lo vedremo tra qualche lezione. Si tratta di convertire le operazioni base dell'aritmetica in calcoli step by step reversibili.}: ad esempio se si ha che $x \texttt{ AND } y = 0 $ non possiamo dire nulla su $x$ e $y$ separatamente. 

\section{Teorema di no-cloning}
Il teorema di no-cloning è un risultato molto importante in QM perché stabilisce che cosa è permesso fare o meno in un computer quantistico. Discutiamolo in dettaglio. Supponiamo di considerare lo stato $\ket{\psi} = \alpha \ket{0} + \beta \ket{1}$: anche se, a differenza del caso classico, i coefficienti $\alpha$ e $\beta$ sono arbitrari, il problema risiede nell'estrarre informazioni. Per trovare in quale stato preciso si trova il sistema bisogna considerare un numero infinito di qubit tutti preparati nel medesimo stato iniziale, ma in un computer quantistico tipicamente si ha solamente 1 singolo qubit! Il punto è che c'è moltissima informazione in $\ket{\psi}$ ma non sappiamo come estrarla: infatti l'arte del quantum computing consiste nell'estrarre \textit{particolari} informazioni riguardanti $\alpha$ e $\beta$ usando solamente una sola misurazione. \\
\noindent Un problema come questo può essere risolto se potessimo duplicare gli stati: il teorema di no-cloning stabilisce proprio il fatto che ciò non è possibile. Nonostante ciò la questione è sottile quindi vediamo di analizzarla in dettaglio. \\
\noindent Nel CC è possibile clonare un bit utilizzando un \texttt{CNOT-gate} classico: per $y=0$ infatti (indichiamo nel circuito seguente i singoli bit e non gli stati perché è un circuito classico)
\begin{center}
    \mbox{
        \Qcircuit @C=2em @R=1.5em {
            \lstick{x} & \ctrl{1} & \rstick{x \, ,} \qw \\
            \lstick{0} & \targ & \rstick{x \oplus 0 = x \, ,} \qw
        }
    }
\end{center}
siamo quindi riusciti ad ottenere due copie esatte del bit $x$. Nell'analogo caso quantistico abbiamo 
\begin{center}
    \mbox{
        \Qcircuit @C=2em @R=1.5em {
            \lstick{\ket{x}} & \ctrl{1} & \rstick{\ket{x} \, ,} \qw \\
            \lstick{\ket{0}} & \targ & \rstick{\ket{x \oplus 0} = \ket{x} \, ,} \qw
        }
    }
\end{center}
perciò sembrerebbe che siamo riusciti a clonare anche lo stato quantistico. Ad esempio per $x=0$ si ha $\ket{00} \overset{\texttt{CNOT}}{\rightarrow} \ket{00}$ e per $x=1$ si ha $\ket{10} \overset{\texttt{CNOT}}{\rightarrow} \ket{11}$. Da questi esempi si potrebbe ingenuamente concludere che sia possibile clonare uno stato anche nel mondo quantistico. In realtà la questione è più sottile perché è possibile clonare uno stato che appartiene ad una base, come $\lbrace \ket{0}, \ket{1} \rbrace$, ma è molto semplice vedere che non si può clonare uno stato generico. Ad esempio, prendiamo il caso in cui vogliamo clonare il generico qubit $\ket{\psi} = \alpha \ket{0} + \beta \ket{1}$. Applichiamo un \texttt{CNOT-gate} con un target dato da $\ket{0}$:
\begin{equation*}
    \left( \alpha \ket{0} + \beta \ket{1} \right) \otimes \ket{0} = \alpha \ket{00} + \beta \ket{10} \overset{\texttt{CNOT}}{\rightarrow} \alpha \ket{00} + \beta \ket{11} \, ,
\end{equation*}
questo risultato è chiaramente differente da 
\begin{equation*}
    \ket{\psi} \otimes \ket{\psi} = \alpha^2 \ket{00} + \alpha \beta \ket{01} + \beta \alpha \ket{10} + \beta^2 \ket{11} \, ,
\end{equation*}
nel caso generale in cui $\alpha \neq 0$ e $\beta \neq 0$. In generale è possibile clonare solamente stati ortogonali tra loro. Enunciamo il teorema:

\begin{teorema}[\textbf{No-cloning}]
    Dato uno stato generico $\ket{\psi}$ normalizzato, non esiste alcun operatore unitario $U$ tale che
    \begin{equation}\label{no_cloning}
        \ket{\psi} \otimes \ket{\phi} \rightarrow U \left( \ket{\psi} \otimes \ket{\phi} \right) = \ket{\psi} \otimes \ket{\psi} \, , \quad \forall \, \ket{\psi} \, . 
    \end{equation}
\end{teorema}

\begin{proof}
    Supponiamo per assurdo che esista un operatore unitario $U$ che realizza la trasformazione \eqref{no_cloning}. Supponiamo di aver clonato due generici stati $\ket{\psi_1}$ e $\ket{\psi_2}$:
    \begin{align*}
        \ket{\psi_1} \otimes \ket{\phi} &\overset{U}{\rightarrow} \ket{\psi_1} \otimes \ket{\psi_1} \, , \\
        \ket{\psi_2} \otimes \ket{\phi} &\overset{U}{\rightarrow} \ket{\psi_2} \otimes \ket{\psi_2} \, ,
    \end{align*}
    ma dato che l'operatore è unitario deve preservare il prodotto scalare:
    \begin{align*}
        \left( \bra{\psi_2} \otimes \bra{\phi} \right) \left( \ket{\psi_1} \otimes \ket{\phi} \right) &\overset{?}{=} \left( \bra{\psi_2} \otimes \bra{\psi_2} \right) \left( \ket{\psi_1} \otimes \ket{\psi_1} \right) \, , \\
        \Rightarrow \quad \braket{\psi_2}{\psi_1} &= \braket{\psi_2}{\psi_1}^2 \, ,
    \end{align*}
    dove abbiamo assunto $\braket{\phi} = 1$. Questo significa che $\braket{\psi_2}{\psi_1} \left( 1 - \braket{\psi_2}{\psi_1} \right) = 0$, quindi $\braket{\psi_2}{\psi_1} = 0$ (i due stati sono ortogonali) oppure $\braket{\psi_2}{\psi_1} = 1$, il che significa che $\ket{\psi_1} = e^{i \alpha} \ket{\psi_2}$, quindi i due stati sono proporzionali per una fase e quindi di fatto si tratta dello stesso stato. In definitiva non è possibile clonare stati generici, ma solamente stati particolari. 
\end{proof}

\noindent Dobbiamo capire come superare queste limitazioni in un computer quantistico. In realtà vedremo che per la maggior parte delle situazioni fisicamente rilevanti è già abbastanza clonare solamente degli stati che appartengono ad una base, quindi non sarà necessario duplicare stati generici.  


\chapter{Entanglement}

In questo capitolo affronteremo molti argomenti riguardanti il concetto di \textit{teletrasporto}, \textit{crittografia quantistica} e \textit{disuguaglianze di Bell}. Tutti questi temi sono legati dal fenomeno dell'\textbf{entanglement}. Innanzitutto ricordiamo che uno stato è definito \textbf{entangled} se \textbf{non} può essere scritto come stato separabile $\ket{\psi} = \ket{\psi}_A \otimes \ket{\psi}_B$, ovvero non è frutto del prodotto tensoriale di stati appartenenti a differenti spazi di Hilbert. 

\begin{esempio}
    Per un sistema in cui è presente una particella dotata di spin, lo stato di singoletto\footnote{In generale non è difficile realizzare un tale stato in natura. Si vedano ad esempio l'ortoelio e il paraelio.} $\ket{00} = \frac{1}{\sqrt{2}} (\ket{\uparrow \downarrow} - \ket{\downarrow \uparrow})$ (momento angolare totale nullo) è uno stato entangled. 
\end{esempio}

\noindent Gli stati entangled danno origine a numerosi paradossi che sono stati studiati a partire dall'inizio del '900. Il paradosso più famoso è probabilmente il \textbf{paradosso EPR} (dai nomi Einstein-Podolski-Rosen). 
\begin{esempio}[\textbf{Paradosso EPR}]
    Consideriamo un sistema costituito da 2 qubit (o due particelle dotate di spin) che si trova nello stato entangled $\ket{\psi} = \frac{1}{\sqrt{2}} (\ket{01} - \ket{10})$ e immaginiamo di voler effettuare una misurazione sul primo qubit (spin). Secondo le regole della QM, semplicemente avremo che
    \begin{align*}
        P(\ket{0}_1) &= \frac{1}{2} \, , &P(\ket{1}_1) &= \frac{1}{2} \, , \\
        P(\ket{0}_2) &= \frac{1}{2} \, , &P(\ket{1}_2) &= \frac{1}{2} \, ,
    \end{align*}
    dove il pedice numerico sui ket indica il qubit (spin) che è stato misurato. Dalla forma dello stato $\ket{\psi}$ notiamo che le misure sono correlate perché una misura sul sistema causa il collasso dello stato in $\ket{01}$ o $\ket{10}$ e quindi il risultato della misura sull'altro qubit viene direttamente influenzato. La correlazione è sottile perché se supponiamo di separare le due particelle (due qubit) in due città differenti mantenendo lo stato totale entangled allora è possibile determinare il risultato di entrambe le misure effettuando una singola misurazione! Ad esempio, se una misura sul primo qubit produce $\ket{0}_1$ allora $\ket{\psi}$ collassa istantaneamente in $\ket{01}$ (ora lo stato non è più entangled): d'ora in avanti il secondo sperimentatore che si trova nell'altra città trova sempre $P(\ket{1}_2) = 1$, sebbene prima del collasso vedeva le probabilità equiprobabili.
\end{esempio}

\noindent Chiaramente ciò che accade nel paradosso EPR è molto "strano": il fatto che lo stato sia entangled suggerisce una sorta di azione istantanea a distanza. È importante evidenziare che questo paradosso non ha nulla a che fare con la violazione della relatività speciale perché nessuno dei due sperimentatori ha inviato informazioni istantanee. Uno dei due aspetti che il paradosso vuole sottolineare è la violazione del \textbf{principio di località}: una misura effettuata in una regione non può influenzare istantaneamente una misura che viene effettuata in un'altra regione casualmente disconnessa alla precedente. Ma come l'evidenza sperimentale mostra, il risultato è l'esatto contrario.

% Vedi se lasciare o meno "Ma come l'evidenza sperimentale mostra, il risultato è l'esatto contrario." oppure toglierlo, era per rendere chiaro che in realtà avviene la violazione del principio di località.